\documentclass[oneside,numbers,Spanish]{ezthesis}
%% # Opciones disponibles para el documento #
%%
%% Las opciones con un (*) son las opciones predeterminadas.
%%
%% Modo de compilar:
%%   draft            - borrador con marcas de fecha y sin im'agenes
%%   draftmarks       - borrador con marcas de fecha y con im'agenes
%%   final (*)        - version final de la tesis
%%
%% Tama'no de papel:
%%   letterpaper (*)  - tama'no carta (Am'erica)
%%   a4paper          - tama'no A4    (Europa)
%%
%% Formato de impresi'on:
%%   oneside          - hojas impresas por un solo lado
%%   twoside (*)      - hijas impresas por ambos lados
%%
%% Tama'no de letra:
%%   10pt, 11pt, o 12pt (*)
%%
%% Espaciado entre renglones:
%%   singlespace      - espacio sencillo
%%   onehalfspace (*) - espacio de 1.5
%%   doublespace      - a doble espacio
%%
%% Formato de las referencias bibliogr'aficas:
%%   numbers          - numeradas, p.e. [1]
%%   authoryear (*)   - por autor y a'no, p.e. (Newton, 1997)
%%
%% Opciones adicionales:
%%   spanish         - tesis escrita en espa'nol
%%
%% Desactivar opciones especiales:
%%   nobibtoc   - no incluir la bibiolgraf'ia en el 'Indice general
%%   nofancyhdr - no incluir "fancyhdr" para producir los encabezados
%%   nocolors   - no incluir "xcolor" para producir ligas con colores
%%   nographicx - no incluir "graphicx" para insertar gr'aficos
%%   nonatbib   - no incluir "natbib" para administrar la bibliograf'ia

%% Paquetes adicionales requeridos se pueden agregar tambi'en aqu'i.
%% Por ejemplo:
%\usepackage{subfig}
%\usepackage{multirow}
\usepackage{amsmath}
%\usepackage[spanish]{babel}


\newcommand{\addcite}[1][]{\def\tst{#1}\ifx\tst\empty\textcolor{red}{[{\footnotesize REFs}]} \else\textcolor{red}{[{\footnotesize REFs: #1}]} \fi}
\renewcommand{\figurename}{Figura}
\renewcommand{\tablename}{Tabla} 
%% # Datos del documento #
%% Nota que los acentos se deben escribir: \'a, \'e, \'i, etc.
%% La letra n con tilde es: \~n.

\author{N\'estor Ariel  P\'erez Ch\'avez}
\title{Microgeles Polim\'ericos: encapsulado, liberaci\'on de f\'armacos y soluciones coloidales}
\degree{Doctor en Ciencias Qu\'imicas}
\supervisor{Gabriel S. Longo \\ Alberto G. Albesa}
\institution{Universidad Nacional de La Plata}
\faculty{Facultad de Ciencias Exactas}
\department{Departamento de Qu\'imica}

%% # M'argenes del documento #
%% 
%% Quitar el comentario en la siguiente linea para austar los m'argenes del
%% documento. Leer la documentaci'on de "geometry" para m'as informaci'on.

%\geometry{top=40mm,bottom=33mm,inner=40mm,outer=25mm}

%% El siguiente comando agrega ligas activas en el documento para las
%% referencias cruzadas y citas bibliogr'aficas. Tiene que ser *la 'ultima*
%% instrucci'on antes de \begin{document}.
\hyperlinking
\begin{document}

%% En esta secci'on se describe la estructura del documento de la tesis.
%% Consulta los reglamentos de tu universidad para determinar el orden
%% y la cantidad de secciones que debes de incluir.

%% # Portada de la tesis #
%% Mirar el archivo "titlepage.tex" para los detalles.
\include{titlepage}

%% # Prefacios #
%% Por cada prefacio (p.e. agradecimientos, resumen, etc.) crear
%% un nuevo archivo e incluirlo aqu'i.
%% Para m'as detalles y un ejemplo mirar el archivo "gracias.tex".
\textbf{Agradecimientos}\\

Al llegar a este punto del camino, no puedo evitar mirar atr\'as y reflexionar sobre el viaje que ha sido completar esta tesis. Ha sido un per\'iodo lleno de desaf\'ios, alegr\'ias, tritezas, mucho aprendizaje, todo un crecimiento personal dir\'ia y claro acad\'emico. Este logro no habr\'ia sido posible sin el apoyo, la gu\'ia y la inspiraci\'on de muchas personas que me acompa\~naron en diferentes etapas de este proceso.

En primer lugar, deseo expresar mi m\'as profundo agradecimiento a mis directores de tesis, Dr. Gabriel S. Longo y Dr. Alberto G. Albesa (Gaby y Beto con cari\~no) cuya experiencia, conocimiento y sobre todo mucha paciencia fueron fundamentales para realizar este doctorado. Gracias por confiar en m\'i en todo este camino.

No quiero mencionar m\'as nombres porque son demasiadas personas a las que tendr\'ia que agradecerle y me parecer\'ia injusto que alguien se quede afuera. Compa\~neres de militancia, de casa, de oficinas, de cer\'amica, f\'utbol, familia, etc, etc. 

No puedo dejar de mencionar a las insituciones que hicieron posible todo esto es decir a CONICET, AGENCIA, al INIFTA, la Facultad de Ciencias Exactas, la Universidad Nacional de La Plata. Y en particular la Laboratorio de Materia Blanda lugar que fue mi segundo hogar y en donde conoc\'i y conviv\'i con personas maravillosas. 

Finalmente, mi gratitud se extiende a todos aquellos que, de una manera u otra, contribuyeron a este trabajo, ya sea a trav\'es de charlas, intercambios acad\'emicos o simplemente ofreciendo su amistad y un mate. A todes ustedes, mi m\'as sincero agradecimiento.

...y a las Umas claro.


%% # 'Indices y listas de contenido #
%% Quitar los comentarios en las lineas siguientes para obtener listas de
%% figuras y cuadros/tablas.
\tableofcontents
%\listoffigures
%\listoftables

%% # Cap'itulos #
%% Por cada cap'itulo hay que crear un nuevo archivo e incluirlo aqu'i.
%% Mirar el archivo "intro.tex" para un ejemplo y recomendaciones para
%% escribir.
%% Los cap'itulos inician con \chapter{T'itulo}, estos aparecen numerados y
%% se incluyen en el 'indice general.
%%
%% Recuerda que aqu'i ya puedes escribir acentos como: 'a, 'e, 'i, etc.
%% La letra n con tilde es: 'n.

\chapter{Introducci'on}
\label{Chapter1} % Change X to a consecutive number; for referencing this chapter elsewhere, use \ref{ChapterX}

%----------------------------------------------------------------------------------------
%	SECTION 1
%----------------------------------------------------------------------------------------

\section{Materia Blanda}

Durante la decada pasada el uso de materiales blandos ha sido de mucho interes debido a la cantidad de materiales e innovaciones que se optienen con ellos.

La versatilidad de estos los han convertido en materiales  importantes en una amplia variedad de aplicaciones tecnológicas.
Se han utilizado como  espumas y adhesivos, son excelente detergentes, est\'an presentes en la industria de los  cosméticos y pinturas, adem\'as de ser ampliamente usados en aditivos alimentarios . El campo de la medicina, la industria farmaceutica, ha encontrado en estos materiales una oportunidad para la creaci\'on de de trasnportadores de drogas m\'as eficientes biocompatibles y biodegradables.

En estas ultimas aplicaciones los  films y geles  polim\'ericos han sido pioneros en su uso y han tenido un creciente inter\'es.
La capacidad de adsorber solventes y la estructura de cadenas entrelazadas proporciona condiciones adecuadas para la adsorcion y proteccion de adsorbatos de interés. \addcite 
Estas características permiten que las drogas/proteinas adsorbidas interactuen con el solvente en el cual son solubles. La proteccion es debido a la restricción de entrada de otros agentes, además de evitar el movieminto libre de la droga. Aspecto muy importante si se consideran proteinas y se quiere evitar su desplegamiento.
Sin embargo la caracteris más notoria de estos materiales es su capacidad de responder a diversos estimulos. 
Muchos estudios han reportado la respuesta a la temperatura de estos materiales, entre los m\'as conocidos se encuentra PNIPAm, Pluronics, elastin-like polypeptides (ELP) and poly(N- vinylcaprolactam) (PNVCL).

Interest in thermoresponsive polymers has steadily grown over many decades, and a great deal of work has been dedicated to developing temperature sensitive macromolecules that can be crafted into new smart materials. 

However, the overwhelming majority of previously reported temperature-responsive polymers are based on poly(N-isopropylacrylamide) (PNIPAM), despite the fact that a wide range of other thermoresponsive polymers have demonstrated similar promise for the preparation of adaptive materials. Herein, we aim to highlight recent results that involve thermoresponsive systems that have not yet been as fully considered. Many of these (co)polymers represent clear opportunities for advancements in emerging biomedical and materials fields due to their increased biocompatibility and tuneable response. By highlighting recent examples of newly developed thermoresponsive polymer systems, we hope to promote the development of new generations of smart materials.


Este comportamiento responsivo hace de estos materiales excelentes candidados en el secuestro de particulas, intercambios ionicos lo cual permite el dise\~no de transportadores. 
\textcolor{green}{here}

El diseño de dispositivos inteligentes optimizados para tareas específicas basados en estos sitemas blandos demanda un estricto control sobre sus propiedades fisicoquímicas y la respuesta de estos geles.
Por lo tanto, su diseño  requiere el desarrollo de modelos que permitan entender y predecir cómo el comportamiento observado resulta de la interrelación entre la estructura química del polímero que lo componen, la organización molecular bajo condiciones de confinamiento,  las interacciones entre especies y las propiedades del medio en el que se encuentren.


Por ejemplo, en la actualidad desarrollar materiales que respondan a múltiples estímulos (pH y temperatura) de manera predecible y controlable representa un desafío enorme. \addcite
\begin{enumerate}
    \item matariales con respuesta a Temperaruta y como cambia su respuesta al modifiar la arquitectura... film y gel.
    \item para materiales con solo respuesta a pH, como el cambio en la concentraci\'on salina del medio afecta la respuesta de los geles.
    \item trabajos experimentales de copolimeros, como se afectan entre ellos la respuesta de cada uno.
\end{enumerate}

...fin de la tesis, estudio te\'orico haciendo uso de simulaciones computaciones para el desarrollo de nuevos materiales capaces de responder a estimulo, una caracterizaci\'on exhaustiva de sus propiedades fisicoqu\'imcas.

Descripci\'on generica de los obsetivos... uso de la TM y familiarizaci\'on de la misma con hidrogeles de homopolimeros, como responden y pueden servir para la captura de moleculas. Uso de estos materiales para fines ambientales...
Del mismo modo estudiar microgeles en un modelo sencillo como primera aproximaci\'on. Fisicoquimca de geles homogeneos y con respuesta a multi estimulo. Adsorci\'on de drogas terapeuticas, condiciones optimcas de encapsulamiento.
Complejizaci\'on de la teor\'ia para modificar parametros de estructura, es decir como afectan los cambios de topologia de estos geles.
Adsorci\'on de biomoleculas. Nueva inormaci\'on sobre la arquitectura de los geles y su relaci\'on con la adsorci\'on de proteinas. 
Finalmente, m\'etodo estocastico para el estudio de soluciones de geles polimericos. En un modelo sencillo.
C\'omo la concentraci\'on afecta la distribuci\'on de tama\~no, adsorci\'on, etc.
%-----------------------------------
%	SUBSECTION 1
%-----------------------------------
\subsection{Objetivos}

Los {\bf objetivos específicos} del presente plan de trabajo son los siguientes:
%
\begin{enumerate}
\item Desarrollar un modelo mecano-estadístico utilizando TM para describir la respuesta a cambios de pH, temperatura y concentración de sal en microgeles formados por homopolímeros.%\label{objetivo_1}
\item Extender dicho modelo para investigar el comportamiento de microgeles de copolímeros con respuesta a múltiples estímulos.%\label{objetivo_2}
\item Estudiar los mecanismos de adsorción de diferentes biomoléculas en los microgeles en función de las condiciones del medio y la estructura/composición química de las cadenas poliméricas.%\label{objetivo_3}
\item Desarrollar un modelo combinando simulaciones de TM y Dinámica Molecular (DM) para estudiar el comportamiento de estos microgeles en soluciones relativamente concentradas.%\label{objetivo_4}
\end{enumerate}
%


\subsection{Films poliméricos}

Los hidrogeles de cadenas de polímeros reticulados se consideran actualmente para diversas aplicaciones en la investigación biomédica.%\cite{Wang2019}
Por ejemplo, los biomateriales basados ...en hidrogeles sensibles al pH se han explorado como vehículos de administración oral de fármacos que tienen el potencial de encapsular y transportar un agente terapéutico a través del tracto gastrointestinal, protegiendo la carga del medio ácido del estómago y liberándola en el ambiente neutral del intestino delgado.
%\cite{Lowman1999,Zhao2019,Qindeel2019,Li2019}
Estos hidrogeles responden a los cambios de pH porque contienen una cantidad significativa de grupos ácidos débiles.
%-----------------------------------
%	SUBSECTION 2
%-----------------------------------

\subsection{Subsection 2}
Morbi rutrum odio eget arcu adipiscing sodales. Aenean et purus a est pulvinar pellentesque. Cras in elit neque, quis varius elit. Phasellus fringilla, nibh eu tempus venenatis, dolor elit posuere quam, quis adipiscing urna leo nec orci. Sed nec nulla auctor odio aliquet consequat. Ut nec nulla in ante ullamcorper aliquam at sed dolor. Phasellus fermentum magna in augue gravida cursus. Cras sed pretium lorem. Pellentesque eget ornare odio. Proin accumsan, massa viverra cursus pharetra, ipsum nisi lobortis velit, a malesuada dolor lorem eu neque.

%----------------------------------------------------------------------------------------
%	SECTION 2
%----------------------------------------------------------------------------------------

\section{Main Section 2}

Sed ullamcorper quam eu nisl interdum at interdum enim egestas. Aliquam placerat justo sed lectus lobortis ut porta nisl porttitor. Vestibulum mi dolor, lacinia molestie gravida at, tempus vitae ligula. Donec eget quam sapien, in viverra eros. Donec pellentesque justo a massa fringilla non vestibulum metus vestibulum. Vestibulum in orci quis felis tempor lacinia. Vivamus ornare ultrices facilisis. Ut hendrerit volutpat vulputate. Morbi condimentum venenatis augue, id porta ipsum vulputate in. Curabitur luctus tempus justo. Vestibulum risus lectus, adipiscing nec condimentum quis, condimentum nec nisl. Aliquam dictum sagittis velit sed iaculis. Morbi tristique augue sit amet nulla pulvinar id facilisis ligula mollis. Nam elit libero, tincidunt ut aliquam at, molestie in quam. Aenean rhoncus vehicula hendrerit.




Existen dos tipos de citas bibliograf'icas: usa \verb|\citep{..}| para
citas en \emph{par'entesis} y \verb|\citet{..}| para citas
en el \emph{texto}. Por ejemplo, estudios reciente han mostrado nuevos e
interesantes modelos que se pueden aplicar para reformular teor'ias
f'isicas~\citep{NewCam97}. Mientras que, el trabajo de \citet{Rofl06} fue
considerado muy divertido por una significativa fracci'on de la comunidad
de investigadores. Tambi'en es posible citar a varios trabajos en una sola
referencia \citep{Lamport86,Knuth84}.

Estos comandos para producir citas bibliograficas son provistos por
el paquete \textsf{natbib}. Para obtener m'as informaci'on, consulta la
documentaci'on de ese paquete~\citep{doc:natbib}. Por su parte, en
la documentaci\'on de \textsf{geometry} puedes encontrar detalles
adicionales sobre el sistema para ajustar los m'argenes del
documento~\citep{doc:geometry}. Lo que sigue
es un mont'on de texto sin sentido en lat'in que utilizaremos para llenar
algunas p'aginas.

Lorem ipsum dolor sit amet, consectetuer adipiscing elit. Integer arcu nisl,
consectetuer ut, vehicula nec, blandit id, nulla. Vestibulum in odio a odio
volutpat sollicitudin. Donec congue porta tellus. Ut quis est sed velit
blandit fringilla. Nunc lobortis dui vitae sapien. In tincidunt magna eget
purus. Nam lorem quam, vehicula in, dictum et, congue eget, odio. Curabitur
gravida mi id dui. Aliquam erat volutpat. Fusce velit turpis, accumsan vel,
tincidunt at, aliquet at, sem. Cras viverra eros ac orci. Aenean vestibulum,
lorem sed luctus congue, arcu pede ultricies libero, at posuere felis nulla
et leo.

Fusce rhoncus lobortis orci. Quisque suscipit dolor. In tincidunt dictum
elit. Cras metus. Donec nibh mi, ornare a, ullamcorper in, gravida non,
augue. Aliquam erat volutpat. Aliquam commodo tellus sed dolor. Sed urna.
Phasellus blandit orci sit amet nulla. Fusce vel eros. Aenean ultrices
sodales mi. Aliquam erat volutpat. Fusce orci sem, sollicitudin convallis,
auctor a, sollicitudin vitae, dui. Sed massa. Duis luctus lectus ut lacus.

Morbi felis tellus, placerat quis, congue pretium, consectetuer at, tortor.
Nunc condimentum mattis urna. Donec dolor erat, fringilla ut, auctor ac,
vestibulum ut, velit. Aliquam convallis magna ac neque. Praesent varius
congue augue. Nulla adipiscing urna faucibus diam. Mauris porta sapien ut
justo. Donec suscipit tortor gravida ligula. Aliquam ac purus et massa
scelerisque vehicula. Maecenas a libero. Class aptent taciti sociosqu ad
litora torquent per conubia nostra, per inceptos himenaeos. Pellentesque
sit amet est eget metus tincidunt semper. Phasellus nec purus. Proin
venenatis lectus vel elit. Pellentesque augue quam, tincidunt sed, pretium
ut, feugiat id, odio. Aenean eu nibh et quam dignissim facilisis.

Suspendisse adipiscing. Maecenas tincidunt placerat justo. Ut mattis nunc ac
orci. Vestibulum quis velit sed massa vulputate posuere. Duis rhoncus lacus.
Quisque non lacus et nibh molestie tincidunt. Nulla tortor pede, auctor id,
eleifend sit amet, ultrices id, risus. Duis et lectus. Suspendisse interdum,
magna ut porta porta, quam tellus suscipit ligula, cursus consectetuer purus
erat et dolor. Phasellus venenatis, risus malesuada lacinia placerat, lectus
tellus lobortis ligula, eu porttitor tellus nibh eu enim. Morbi vel erat in
sem pharetra molestie. Duis tellus. In ipsum. Vivamus ac augue sed dui
hendrerit pulvinar. In dui erat, molestie ut, lacinia at, sagittis sed, nisi.
Maecenas libero. Nam volutpat dictum erat.

Fusce laoreet sapien ut lorem. Mauris sed leo a mi luctus sollicitudin.
Donec ornare nisi id dolor. Ut eros metus, tristique quis, ultrices ac,
accumsan cursus, est. Pellentesque mollis posuere sapien. Morbi nec augue.
Cum sociis natoque penatibus et magnis dis parturient montes, nascetur
ridiculus mus. Duis tristique, ipsum in tincidunt gravida, nunc nulla
vehicula felis, elementum eleifend nunc elit id magna. Cum sociis natoque
penatibus et magnis dis parturient montes, nascetur ridiculus mus. Curabitur
rhoncus dui ut sapien.

Sed orci. Nunc nisi lorem, convallis nec, porttitor at, porttitor et, erat.
Lorem ipsum dolor sit amet, consectetuer adipiscing elit. Donec luctus, velit
quis lacinia pulvinar, risus urna malesuada nisl, vel hendrerit erat enim ac
enim. Aliquam sapien dolor, fringilla quis, consequat auctor, sodales id,
est. In imperdiet est et dui. Cras libero lacus, feugiat a, auctor ut,
vulputate sollicitudin, orci. Ut tellus velit, rutrum tristique, eleifend sit
amet, auctor consectetuer, sapien. Fusce eget justo. Nam auctor lorem at
purus. Vestibulum ante ipsum primis in faucibus orci luctus et ultrices
posuere cubilia Curae; Pellentesque pretium enim sed tortor. Sed luctus velit
at ligula. Nunc id elit. Curabitur lacus. Mauris placerat nibh sit amet
turpis. Fusce varius, justo et ultrices dictum, urna risus rhoncus ipsum, sed
ultricies nunc arcu eu risus. Nam vitae purus.

Proin augue. Duis vehicula mauris sollicitudin sapien. Nam tristique lacus
nec nisl. Praesent quis enim. Vestibulum vel velit in purus luctus mattis.
Mauris ullamcorper tempor lorem. Quisque rutrum. Praesent enim nibh,
pellentesque non, lacinia accumsan, euismod a, lorem. Etiam fringilla
iaculis mauris. Aenean adipiscing purus in lacus. Maecenas quis nibh. Ut
non augue at mauris elementum luctus. Duis varius tincidunt mi. Aliquam
justo massa, auctor nec, malesuada interdum, mollis ac, mauris. Maecenas
ultricies gravida dui. Aliquam arcu elit, pretium eu, gravida ac, molestie
eu, enim. Etiam facilisis orci eget est. Integer eu orci non felis tincidunt
consectetuer. Sed imperdiet ultrices nibh.

\chapter{Hoja de ruta}
\label{ruta}

Este cap\'itulo busca guiar al lector en la distribución de la presente tesis. 


En su primer cap\'itulo, nos adentramos en el  mundo de los materiales blandos, la esencia misma de estos materiales y su relevancia en la industria y la ciencia moderna. Se har\'a enfas\'is en los hidrogeles polim\'ericos dentro de los cuales encotraremos a los films y a los micro y nano geles. Las motivaciones y objetivos son presentados. 
 Adem\'as se mencionar\'a el tipo enfoque y las herramientas que se utilizaron para llevar a cabo los estudios presentados en la tesis.

El primer sistema polim\'erico, los films, se retoma en los antecedentes, \ref{Chapter-film}, en la cual se da primera aproximaci\'on de parte de quien les escribe a la Teor\'ia Molecular. Las ventajas del uso de la termodin\'amica estad\'istica y como su combinaci\'on con las propiedades moleculares de los sistemas nos permite explicar toda la termodin'amica necesaria para la descipci\'on y predicci\'on de su comportamiento en condiciones determinadas.  
Se  revisan conceptos generales del comportamiento de estas cadenas polim\'ericas entrecruzadas en soluci\'on.
En part\'icular se mostrar\'a la respuesta  de los films poliméricos a cambios en el pH y la concentraci\'on de sal.  En este capitulo también encontraremos resultados sobre la aplicabilidad en el secuestro de dos prote\'inas modelos (citocromo y mioglobina), la elección de estos resultados, respuesta a pH, concentraci\'on salida, adsoric\'on de prote\'inas,  es debido a que los mismos son retomados en los cap\'itulos subsecuentes.

Es importante destacar que en  el transcurso de este aprendizaje logró la presentación de trabajos en donde se hace uso de los films polim\'ericos en el secuestro de moleculas, con aplicabilidad en la agroqu\'imcia  \addcite[film - glifosato], y principalmente en su uso de secuestro de f\'armacos \addcite[polimainas] trabajo con el cual nos acercamos  a los objetivos del presente trabajo.


Emocionado por su primer éxito, Carlos se aventuró en el siguiente capítulo: los micro y nano geles poliméricos. Descubrió que estos geles eran como pequeñas partículas con habilidades especiales, capaces de responder a estímulos externos y adaptarse a diferentes entornos. A medida que profundizaba en su investigación, Carlos comprendió la importancia de diferenciar entre micro y nano geles, explorando los métodos utilizados para sintetizarlos y caracterizarlos en el laboratorio. Fascinado por las aplicaciones emergentes de estos geles, desde la entrega de medicamentos hasta la ingeniería de tejidos, Carlos comenzó a trazar conexiones entre los films poliméricos y los geles poliméricos, revelando un mundo de posibilidades interconectadas.

Guiado por su deseo de exploración y conocimiento, Carlos pasó al siguiente capítulo: la metodología experimental. Aquí, descubrió los secretos detrás de los experimentos y análisis rigurosos. Se adentró en los materiales utilizados en su investigación y aprendió los procedimientos precisos para sintetizar tanto los films poliméricos como los micro y nano geles poliméricos. Cada paso, cada reacción química, se convirtió en un viaje hacia la comprensión y la perfección.

Después de meses de trabajo duro y dedicación, llegó el momento más esperado: los resultados y la discusión. Carlos presentó sus hallazgos al mundo, revelando datos reveladores y sorprendentes. A medida que profundizaba en la interpretación de los resultados, Carlos se sintió como un mago de la ciencia, conectando los puntos y desvelando nuevos conocimientos. Compartió sus conclusiones con la comunidad científica, analizando la relevancia de su trabajo y las posibles áreas de mejora.

Finalmente, Carlos escribió las conclusiones de su tesis. En este capítulo final, resumió sus descubrimientos y evaluó el éxito de sus objetivos. Reconociendo el valor de su investigación, Carlos compartió las contribuciones que su trabajo podría hacer al campo de estudio. Con entusiasmo, dejó abierta la puerta para futuras investigaciones, instando a otros a continuar su legado y desafiar los límites del conocimiento científico.

Y así, el viaje de Carlos llegó a su fin. Con su tesis completada, se despidió del mundo de los films poliméricos y los micro y nano geles poliméricos, pero no sin llevar consigo un tesoro invaluable: el conocimiento y la pasión por la ciencia. Su historia se convirtió en una inspiración para otros, recordándoles que el poder del descubrimiento y la investigación puede llevar a un futuro lleno de innovación y avance científico.

Y así, el cuento de Carlos y su "hoja de ruta" de tesis llega a su fin, pero el comienzo de un nuevo capítulo para aquellos que se atrevan a explorar los misterios y las maravillas de los materiales poliméricos.



En los siguientes capitulos ...
El estudio de estos dispositivos intelgientes a trav\'ez de m\'etodos te\'orico/computacionales no es muy distinto al realizado en un proyecto experimental. El planteo de la composici\'on y estrucutra de nuestros geles, as\'i como tambi\'en el barrido de variables,llamese pH, temperatura, concentraci\'on de sal, es realizado de manera sistem\'atica.
La ventaja de utilizar m\'etodos computacionales es que nos permiten explicar los fenomenos observables a partir del cambio de variables controladas en el laboratorio.
El planteo de teor\'ias que nos ayuden a obtener la informaci\'on termodin\'amica de cada sistema son la base para la presente tesis... 


Al igual que en un proyecto experimental, nuestra idea es poder identifcar las condiciones \'optimas en las cuales los geles puedan ser usados como biomateriales inteligentes. 
La identificaci\'on de estas condiciones es llevado a cabo utilizando teor\'ias que emplean la termodi\'amica estad\'istica. En part\'icular este trabajo hace uso de la Teor\'ia Cl\'asica de la Densidad. Esta \'ultima tendr\'a algunas variaciones dependiendo el tipo de modelado que se use, el cual ser\'a espicificado y explicado en cada uno de los sistemas presentados.

En part\'icular en este cap\'itulo se dar\'a un vistazo general de la Teor\'ia Molecular derivada de la teor\'ia cl\'asica de la densidad. 
Al final del mismo tambi\'en se introducir\'a sobre el tipo de modelado utulizado para poder implementar las teor\'ias....
 


% Chapter 2

\chapter{Films polim\'ericos} % Main chapter title

\label{Chapter2} % For referencing the chapter elsewhere, use \ref{Chapter1}

%\section{papers of films}


Los films polim\'ericos o hidrogeles  consisten en una red de pol\'imeros entrecruzados altamente hidratados, generalmente biocompatibles, dependendiendo de su composici\'on qu\'imca. El ambiente acuoso dentro de los hidrogeles puede proteger a las prote\'inas de la desnaturalización y la agregaci\'on [11e13], mientras permanecen activas y estructuradas cuando se liberan de los hidrogeles [14]. En la administraci\'on oral de f\'armacos, los hidrogeles con respuesta de pH se han investigado en gran medida como veh\'iculos funcionales que pueden encapsular y administrar prote\'inas, evitando su degradaci\'on en el entorno gastrointestinal [15-17].


En este capitulo mostraremos un estudio  de estos sistemas polim\'ericos haciendo uso de de la te\'oria molecular.



\subsection{Te\'oria Molecular en films polim\'ericos}


Este método consiste en minimizar una energía libre generalizada que incluye toda la termodinamica relevante.
Simultáneamente, el método permite una descripción molecular de grano grueso de las diferentes especies químicas que componen el sistema.
Dicha descripción incluye forma, tamaño, distribución de carga y estado de protonación de cada componente molecular.
El sistema en estudio se encuentra en  equilibrio con una solución acuosa que tiene una composición  definida externamente.
Es decir, el pH, la concentración de sal y la concentración de adsorbatos son variables independientes.
La red polimérica que da estructura a los geles, en este casi un film, pueden contener distintos tipos de segmentos: una unidad sensible al pH, ya sea ácida (MAA) o básica (AH), segmentos neutros (VA) o bien segmentos termosencibles (NIPAm), entre otros. \addcite[para-cada-segemento]

Para este capitulo consideraremos un film polim\'erico compuesto por unidades \'acidad de $MAA$.
El mismo se encuentra en equilibrio con una soluci\'on con una temperatura, pH y concentraci\'on de sal y adsorbatos(llamese proteinas, drogas terapeuticas, etc) definidos en el seno de la soluci\'on.


Bajo estas consideraciones es posible definir una energ\'ia libre:

\begin{align}
 	F = -TS_{mez} -TS_{conf,nw} + F_{chem,nw} + F_{chem,pro} + U_{elec} + U_{ste} + U_{VDW} 
\end{align}
 
\noindent En donde $S_{mez}$ es la entropía de traslación (mezcla) de las especies de la solución: moléculas de agua (H$_2$O), iones de hidronio (H$_3$O$^+$), iones de hidróxido (OH$^- $), cationes de sal, aniones de sal y otros adsorbaatos presentes: como drogas o proteinas terapeuticas.
Aquí, consideramos una sal monovalente, NaCl o KCl, y se asueme que está completamente disociada en iones cloruro (Cl$^-$) y sodio ($Na^+$) o potasio ($K^+$) respecticamente. 

$S_{conf,nw}$ representa la entrop\'ia conformacional que resulta de la flexibilidad de la red de polim\'erica, la cual viene dada por todas las conformaciones diferentes que puede asumir la misma.

$F_{chem,nw}$, es la energ\'ia qu\'imica libre que describe el equilibrio entre las especies protonadas y desprotonadas de unidades funcionales (\'acidas/b\'asicas), para nuestro film solo se consideran unidades \'acidas.

De manera similar, $F_{chem,pro}$ describe la protonaci\'on de residuos titulables de los adsorbatos.

$U_{elec}$ y $U_{ste}$ representan, respectivamente, las interacciones electrost\'aticas y las repulsiones est\'ericas.
Las interacciones no electrost\'aticas son representadas en $U_{VDW}$.


Las expresiones explicitas de eq. \ref{eq:semicano} pueden describirse... segun el tipo de modelo elegido y la simetria considerada.. vease modelado...

Como primer termino tenemos la entrop\'ia de mezcla de  las especies mobiles, es decir no se toma en cuenta el sistema polim\'erico.

\begin{equation}
\frac{S_{mez}}{\beta}= \sum_{\gamma}\int_S{dr G(r)\rho_\gamma(r)\left(\ln \left(\rho_\gamma (r)v_w\right) -1 + \beta\mu^0_\gamma\right)}
\end{equation}
\noindent en donde $\beta = \frac{1}{kT}$ y $G(r) = 4\pi r^2$ lo cual corresponde a uns simetria radial. El subindice $\gamma$ tiene en cuenta las moleculas de agua, sus iones y los iones dados por la disociaci\'on de la sal.

Otra especie mobil y que no ha sido considerada en la expresi\'on anterior es los adosrbatos. Para ellos se ha considerado su contribuci\'on energetica como:

\begin{equation}
\frac{1}{\beta}\sum_{\theta}\int_S{dr G(r)\rho_{pro}(\theta,r)\left(\ln \left(\rho_{pro}(\theta,r)v_w\right) -1 + \beta\mu^0_{pro} \right)}
\end{equation}

Subindice $\theta$ puede interpretarse de dos maneras: en primer instancia como un conjunto de configuraciones para un solo adosrbato, tambi\'en puede considerarse a cada configuraci\'on como adsrobatos diferentes, pero con las mismas propiedades qi\'imicas.
Estamos asumiendo que no hay cambios en el potencial qu\'imico  $\mu^0_{pro}$ al cambiar sobre $\theta$ 

Now $S_{conf,nw}$ represents the conformational entropy that results from the flexibility of the polymer network, which can assume many different conformations denoted by the set $\{\alpha\}$. 
\begin{equation}
\frac{S_{conf,nw}}{\beta} = \sum_{\alpha}{P(\alpha)\ln P(\alpha)}
\end{equation}

The next term describes the free energy for the acid-base equilibrium:
For the MAA segments:
\begin{align}
\begin{aligned}
\frac{F_{chem,nw}}{\beta} &= \int_S drG(r) \frac{\left<\phi_{MAA}(r)\right>}{v_{MAA}} \left[f(r)(\ln f(r)+ \beta\mu^0_{MAA^-})\right.\\
&\left.+(1-f(r))(\ln (1-f(r))+\beta\mu^0_{MAAH})\right]    
\end{aligned}
\end{align} 

In the same way the proteins segments equilibrium is writing as:
\begin{align}
\begin{aligned}
\frac{F_{chem,pro}}{\beta} &=\int_S drG(r) \left<\rho_{pro,\tau}(r)\right> \left[g(r)(\ln g(r)+ \beta\mu^0_{\tau p})\right.\\
&\qquad\left.+(1-g(r))(\ln (1-g(r))+\beta\mu^0_{\tau d})\right]   
\end{aligned}
\end{align} 

Titratable units are represented by  $\tau$ subdix \textit{p} and \textit{d} are protonated and deprotonated states respectively. 
\begin{enumerate}
	\item for an unit acid case: $g(r) = f(r)$
	\item basic case: $g(r) = 1-f(r)$
\end{enumerate}

The electrostatic contributions of the free energy $\frac{U_{elec}}{\beta}$ :
\begin{align}
\begin{aligned}
\int_S drG(r)\left[\left(\sum_{\gamma } {\rho_\gamma(r) q_\gamma + \sum_\tau{f_\tau(r) \left<\rho_{pro,\tau}(r)\right> q_\tau} +  f(r)\dfrac{\left<\phi_{MAA}(r)\right>}{v_{MAA}}q_{MAA}}\right)\beta\psi(r) -\frac{1}{2}\beta\epsilon(\nabla\psi(r))^2 \right]\\
\end{aligned}
\end{align} 
\noindent where $\psi(r)$ is the position-dependent electrostatic potential, and $\epsilon$ the medium permittivity. 
%%%%%%%%%%%%%%%%

Finally the energy due the steric repulsion  take account the volume incompressibility
\textcolor{red}{ejemplo para poner la ecuaci\'on en dos reglones}
\begin{align*}
\begin{aligned}
+  &\int_0^\infty drG(r)\left[\left(\sum_{\gamma } {\rho_\gamma(r) q_\gamma + \sum_\tau{f_\tau(r) \left<\rho_{pro,\tau}(r)\right> q_\tau} +  f(r)\dfrac{\left<\phi_{MAA}(r)\right>}{v_{MAA}}q_{MAA}}\right)\beta\psi(r) \right.\\  &\left. \hspace{6em}-\frac{1}{2}\beta\epsilon(\nabla\psi(r))^2 \right]
\end{aligned}
\end{align*}




\subsection{Respuesta al pH}
\textbf{pH} \\
Hidrogeles  compuestos por cadenas de poli\'acidos son sensibles a los cambios de pH. Esta respuesta se debe a la equilibrio qu\'imico de protonaci\'on/desprotonaci\'on de las unidades \'acidas que componen la red. 
Para enteneder el funcionamiento de esta respuesta recordaremos algunos conceptos sobre el comportamiento \'acido/base de moleculas bajo condiciones ideales. 
Estos conceptos nos serviran para entener el equilibrio que ocurre cuando se confinan los monomeros en una red polim\'erica. Los mismos principios ser\'a utilizados para los sistemas  de estudio de los pr\'oximos capitulos.

Si consideramos una soluci\'on diluida de moleculas titulables, estas pueden exhibir dos estados posibles: protonado o desprotonado. En este sentido se define el grado de disoción, $f_d$, el cual  proporciona la fraci\'on de moleculas que se encuentran en estado desprotonado:


\begin{equation}
    f_d = \frac{1}{1+10^{pk_a -pH}}
    \label{eq:diso}
\end{equation}

si consideramos moleculas \'acidas su estado protonado no posee carga, es  neutro, por otro lado su estado desprotonado posee carga. 
Al considerar esto el grado de disociaci\'on, adem\'as nos indica el grado de carga de estas moleculas, $f_c$.
Para molecualas b\'asicas su estado de carga es contrario a las moleculas \'acidas, en consecuencia el grado de carga viene dado por: $f_c =1- f_d$.

En soluciones diluidas el grado de disociaci\'on $f_d$ (y el de carga $f_c$) son completamente determinados por el pH de la soluci\'on y el $pK_a$ intrinseco del par \'acido/base. 
Cuando el pH =$pK_a$ la mitad de los grupos titulables se encuentran en disosiados ($f_d = 0.5$). Para valores de $\pm 1$ coresponden a estados con 90\% y 10\% de disociaci\'on respectivamente.
Es decir, cuando el pH aumenta alrededor del pKa, la transici\'on del 10 al 90 \% de desprotonaci\'on ocurre dentro de dos unidades de pH de la soluci\'on ideal. 

Estas consideraciones de soluci\'on ideal usualmente se utilizan para estimar el grado de carga de las unidades \'acidas dentro de cadenas pol\'imerias. Sin embargo, este comportamiento es diferente para sistemas  en confinamiento. Las unidades  protonables forman parte de una red polim\'erica son un ejemplo de ello, lo cual modifica significativamente su comportamiento de protonaci\'on.


\textbf{Red polim\'erica} \\

A continuaci\'on describiremos el comportamiento de estos sistemas confinados, hidrogeles sensibles al pH.  A diferencia de las soluciones diluidas, las unidades \'acidas en una red de pol\'imeros experimentan repulsiones electrost\'aticas cuando estas se encuentran cargadas. Para reducir la fuerza de las repulsiones dentro de la red, estos grupos se disocian significativamente menos que en condiciones ideales. En la figura \ref{fig:degree-film} se ilustra este comportamiento y muestra el grado medio de carga de los segmentos de una pel\'icula de hidrogel de \'acido poli(metacr\'ilico) (PMAA), que est\'a en contacto con soluciones que tienen diferentes concentraciones de sal.
\begin{figure}
    \centering
    \includegraphics[width=0.9\textwidth]{Figures/graph-film/charge_degree-film.png}
    \caption{Grado de carga del gel como funci\'on del pH. Grado de carga para un monomero aislado en presentado en curva a rayas, se compara para diferentes concentraciones de sal, a mayor concentraci\'on salina m\'as nos acercamos al sistema ideal.}
    \label{fig:degree-film}
\end{figure}



A un pH dado es significativamente menos probable que se cargue una unidad \'acida de la red de lo que se espera seg\'un las consideraciones ideales. La concentraci\'on de sal de la soluci\'on resulta ser una variable cr\'itica que modula este comportamiento de regulación de carga.

A una salinidad relativamente alta, la cantidad de los contraiones dentro del hidrogel crece, lo que da como resultado el apantallamiento de interacciones electrost\'aticas. Las interacciones son ahora de corto alcance. Este apantallamiento de repulsiones dentro de la red permite que el pol\'imero aumente su grado de carga. Se observa que un aumento en la salinidad genera una protonaci\'on que se aproxima al comportamiento ideal. 

En condiciones de baja concentraci\'on de sal solo se encuentran los suficientes contraiones dentro de la red para neutralizar la carga el\'ectrica del pol\'imero. Bajo tales condiciones, el efecto de apantallamiento de los iones de sal se debilita y las interacciones electrost\'aticas se vuelven efectivamente de mayor alcance. Como resultado, la red se carga menos para prevenir o reducir las repulsiones dentro de la red.

Hemos visto que el grado de carga de los film polim\'ericos cambia respecto a monomeros aislados. Esto nos induce a pesnar que las condiciones de ese entorno son diferentes a las que se esperar\'ia para el seno de la soluci]\'on. Exite una una regualci\'on de carga, lo que conlleva a pensar en una regulaci\'on del pH.
Definimos as\'i el pH local que nos proporciona la concentraci\'on de protones  en la posición espacial $r$:
\begin{equation}
    pH(r) = -\log_{10}([H^+](r))
    \label{eq:pH-local}
\end{equation}

Una baja disociaci\'on (un nivel de protonaci\'on alto) de las unidades \'acidas del pol\'imero puede explicarse en terminos del pH local dentro del material. Se define el $pH_{gel}$ como el promedio del pH local dentro del film. Resultados previos han mostrado que esta cantidad esta bien definida \addcite. 
Enfatizaremos la importancia de estos dos terminos $pH_{gel}$ y $pH(r)$ por la informaci\'on que proveen: el estado de carga/protonaci\'on de las unidades titulables en la red polim\'erica. 

Haciendo uso de la eq. \ref{eq:diso} es posible calcular el grado de disoci\'on de la estructura polim\'erica de nuestro hidrogel. El uso del $pH_{gel}$ es indispensable para cuando el pH es distinto al  del seno de la soluci\'on \addcite. El mismo procedimiento se realiza para calcular el estado de protonaci\'on local de las unidades titulables de las especies que se adsorben en el film (ver figura \ref{fig:protein-charge}).

\begin{figure}
    \centering
    \includegraphics[width=0.9\textwidth]{Figures/graph-film/carga-proteinas.png}
    \caption{N\'umero de carga de las proteinas cytocromo c y myoglobina como  funci\'on del pH en el seno de la soluci\'on (bulk). La l\'inea a trazos muestra el cambio en el signo de la carga.}
    \label{fig:protein-charge}
\end{figure}



Sin embargo, aunque esto parece simplificar el problema de establecer la carga neta de cualquier especie dentro del material, incluida la red polim\'erica y las prote\'oinas adsorbidas, determinar los cambios en el pH local tiene la misma complejidad que el problema original (es decir, determinar la carga de la la red). El pH local que se establece dentro del material, as\'i como su valor en la interfaz entre el pol\'imero y la soluci\'on acuosa, es el resultado de la compleja interacci\'on entre la organizaci\'on molecular, los equilibrios qu\'imicos y las interacciones f\'isicas que determinan el equilibrio termodin\'amico a las condiciones impuestas externamente (pH, concentraci\'on de sal). Por ejemplo, la figura \ref{fig:pH-local} muestra el pH dentro de una pel\'icula de hidrogel de PMAA como una funci\'on del pH  y la concentraci\'on de sal, calculado usando teor\'ia molecular.

\begin{figure}
    \centering
    \includegraphics[width=0.9\textwidth]{Figures/graph-film/pH-local.png}
    \caption{pH local del gel como funci\'on del pH en el seno de la soluci\'on (bulk). Cada curva corresponde a una concentraci\'on de sal diferente.}
    \label{fig:pH-local}
\end{figure}

\subsection{Adsorci\'on}
Como se mencion\'o al incio de este capitulo... poder usar estos sistemas de hidrogeles como carries de adsrobatos de utilidad terap\'eutica.
Para ello nos valdremos de la teor\'ia molecular y haciendlo uso de ciertas prote\'inas modelo como lo son el cytocromo c y la myoglobina. estas dos presentan gran estabilidad en un amplio rango de pH y recientemente se ha  investigado la termodinámica  de su adsorci\'on en sistemas polim\'ericos similares. \addcite

Para cuantificar la cantidad de adsrobato adsrobido en el hidrogel utilizamos la expresi\'on:

\begin{align}
\Gamma = \int_V {dr(\rho(r) -\rho_{bulk}}  
\label{adsor}
\end{align}

en donde $\rho(r)$ y $\rho_{bulk}$ son las densidades del  locales y en el seno de la soluci\'on del adsorbato respectivamente, V es el volumen de la soluci\'on. 
Esta adsorci\'on proporciona la masa  en un volumen particular por exceso de la contribuci\'on del bulk. En particular dentro del hidrogel, $\Gamma$ proporciona la cantidad de adsorbato en exceso dentro del material, recibiendo tambi\'en contribuciones de la interfaz de soluci\'on de pol\'imero.

Para estas prote\'inas, la adsorci\'on es una funci\'on no monot\'onica del pH de la soluci\'on (ver Figura \ref{fig:ad-pro}). A pH bajo, estas prote\'inas tienen una carga alta y positiva, pero la red de poli\'acidos solo está d\'ebilmente ionizada (v\'eanse las Figuras \ref{fig:degree-film} y \ref{fig:protein-charge}). A un pH suficientemente alto, por otro lado, el pol\'imero est\'a fuertemente cargado negativamente, pero las prote\'inas tienen una carga d\'ebilmente positiva o incluso negativa. Bajo tales condiciones (muy) \'acidas o alcalinas, las interacciones electrost\'aticas son d\'ebilmente atractivas o repulsivas. No hay fuerza impulsora para la adsorci\'on. A valores de pH intermedios, por el contrario, donde tanto la prote\'ina como la red de poli\'acidos tienen cargas fuertes y opuestas, se produce una adsorci\'on significativa con un m\'aximo necesario en tales condiciones.

La adsorci\'on de prote\'inas depende cr\'iticamente de la concentraci\'on de sal de la soluci\'on. Este comportamiento se ilustra en la figura \ref{fig:ad-pro} que muestra la adsorci\'on de citocromo c  y mioglobina en una pel\'icula de hidrogel de PMAA. La disminuci\'on de las concentraciones de sal mejora la adsorci\'on y ampl\'ia el rango de pH de la adsorci\'on. Por ejemplo, ambos paneles de la \ref{fig:ad-pro} muestran una disminuci\'on de aproximadamente un orden de magnitud en la adsorci\'on cuando se comparan soluciones de NaCl 1 mM y 10 mM. El pH de m\'axima adsorci\'on tambi\'en depende de la salinidad de la soluci\'on. Este comportamiento es a\'un m\'as interesante cuando se considera que una concentraci\'on de sal m\'as baja conduce a una red con carga m\'as d\'ebil, como describimos con anterioridad. En otras palabras, la red de pol\'imero con carga m\'as d\'ebil, a medida que disminuye la concentraci\'on de sal, m\'as prote\'ina es adsorbida. Esta \'ultima afirmaci\'on es cierta en las concentraciones de prote\'ina ($10 \mu M$) y sal de la figura \ref{fig:ad-pro}, donde la adsorci\'on solo modifica ligeramente el grado de carga de la red.

Esta dependencia de la adsorci\'on de la concentraci\'on de sal tiene tres razones principales: en primer lugar, existe el apantallamiento  de las atracciones electrost\'aticas de la red hacia la prote\'inas por parte de los iones de sal. Cuanto menor sea la concentraci\'on de sal, m\'as débil ser\'a el apantallamiento de las interacciones prote\'ina-red, lo que mejora la adsorci\'on. En segundo lugar, a medida que la concentraci\'on de sal disminuye, el pH dentro de las gotas de hidrogel (a un pH general dado). Esto implica que las prote\'inas adsorbidas tienen una carga m\'as positiva tras la adsorci\'on (a medida que disminuye [NaCl]). En tercer lugar, la ganancia entr\'opica de la liberaci\'on de contraiones de la red de pol\'imeros es mayor a medida que disminuye la concentraci\'on de sal, lo que tambi\'en favorece la adsorci\'on de prote\'inas.



\begin{figure}
    \centering
    \includegraphics[width=0.99\textwidth]{Figures/graph-film/ad-proteins.png}
    \caption{Adsorci\'on de proteinas: cytocromo c y myoglobina en panel A y B respectivamente. La concentraci\'on de los adsorbatos es $10 \mu M$}
    \label{fig:ad-pro}
\end{figure}

% Chapter 3

\chapter{Geles polim\'ericos} % Main chapter title

\label{Chapter3} % For referencing the chapter elsewhere, use \ref{Chapter1} 

%----------------------------------------------------------------------------------------

% Define some commands to keep the formatting separated from the content 
\newcommand{\keyword}[1]{\textbf{#1}}
\newcommand{\tabhead}[1]{\textbf{#1}}
\newcommand{\code}[1]{\texttt{#1}}
\newcommand{\file}[1]{\texttt{\bfseries#1}}
\newcommand{\option}[1]{\texttt{\itshape#1}}

%----------------------------------------------------------------------------------------

\section{Modelo sencillo: 2 fases}

Los microgeles son part\'iculas blandas hechas de cadenas de pol\'imeros reticulados que pueden mostrar un comportamiento tanto coloidal como macromolecular \addcite[plamper2017,lyon2012].
Inmersas en soluciones acuosas, estas part\'iculas incorporan y retienen grandes cantidades de agua dentro de su estructura polim\'erica.
Por lo general, sus di\'ametros van desde decenas de nan\'ometros (nanogeles) hasta varias micras.
Sin embargo, la caracter\'istica m\'as llamativa de estas part\'iculas es su capacidad para absorber o liberar solventes y cambiar de tama\~no en respuesta a una variedad de est\'imulos externos.
Este comportamiento de respuesta es reversible y depende de la composici\'on qu\'imica de la red polim\'erica.


Los microgeles compuestos por cadenas polim\'ericas que tienen segmentos \'acidos como el \'acido acr\'ilico o metacr\'ilico (AA y MAA, respectivamente) se hinchan/deshinchan muchas veces en respuesta a cambios en el pH de la soluci\'on \addcite[snowden1996,Zhou1998].
El pH particular que marca el inicio y caracteriza esta transici\'on es el pKa aparente del microgel, que depende de la concentraci\'on de sal de la soluci\'on y frecuentemente difiere del pKa intrínseco del mon\'omero \'acido.
Estos microgeles tambi\'en ajustan su tama\~no en respuesta a cambios en la concentraci\'on de sal de la soluci\'on \addcite[snowden1996].

An\'alogamente, los microgeles de algunos pol\'imeros termosensibles experimentan una transici\'on de fase de volumen (VPT) cuando se calientan por encima de una temperatura caracter\'istica (VPTT o $T_{pt}$)\addcite[Pelton1986,Pelton2000].
Este comportamiento se origina porque tales pol\'imeros son insolubles en agua por encima de cierta temperatura de soluci\'on cr\'itica m\'as baja (LCST) \addcite[Kawaguch2020].
Normalmente, la LCST del pol\'imero y la VPTT de la red reticulada son aproximadamente id\'enticas. 
Este es el caso de las part\'iculas de microgel de poli(N-isopropilacrilamida) (PNIPAm) \addcite[Pelton1986], cuyo volumen colapsa por encima de $32 C$, el LCST del pol\'imero lineal \addcite[Schild1992].

Al tener un VPTT alrededor de la temperatura corporal, los microgeles de PNIPAm han generado un gran inter\'es para aplicaciones biom\'edicas \addcite[Guan2011].
Las estrategias para controlar el VPTT de los microgeles incluyen la s\'intesis de nuevos mon\'omeros termosensibles\addcite[Cai2007, Macchion2019], as\'i como la copolimerizaci\'on con un mon\'omero i\'onico o ionizable \addcite[Hirose1987, Lopez2010].
Este \'ultimo enfoque produce microgeles de respuesta m\'ultiple que son susceptibles a cambios en la temperatura, el pH y la concentraci\'on de sal \addcite[snowden1996, Farooqi2017].
Los microgeles de NIPAm y AA han sido ampliamente estudiados \addcite[Morris1997, Jones2000, Bradley2005, Begum2016];
Tambi\'en se han investigado microgeles de copol\'imeros de NIPAm y MAA \addcite[Dowding2000, Hoare2004, Giussi2015].
El VPTT de estos microgeles de respuesta m\'ultiple depende del pH de la soluci\'on y la concentraci\'on de sal, y la fracci\'on de mon\'omero ionizable en las cadenas de pol\'imero \addcite[Morris1997, Jones2000, Hoare2004, Bradley2005, Lee2008, Wong2009, Hamzavi2016].
Adem\'as, la incorporaci\'on del comon\'omero \'acido proporciona un mecanismo controlado por el pH para la captaci\'on/liberaci\'on de mol\'eculas con carga opuesta, lo que hace que los microgeles de respuesta m\'ultiple sean atractivos para el dise\~no de sistemas funcionales de administraci\'on de f\'armacos \addcite[Liu2017].

Las aplicaciones basadas en microgeles polim\'ericos son cada vez m\'as diversas y cada vez m\'as importantes en diferentes \'areas de la tecnolog\'ia \addcite[plamper2017].
Los nanocompuestos hechos de microgeles de poli(NIPAm-\emph{co}-MAA) que incorporan nanopart\'iculas de oro o plata o redes metal organicas tienen aplicaciones potenciales en el desarrollo de materiales con actividad catal\'itica  para la degradación de contaminantes industriales \addcite[Khan2013, Shi2014, Allegretto2020].
Estos microgeles tambi\'en se pueden aplicar para el desarrollo de emulsiones sensibles al pH, la sal y la temperatura \addcite[Ngai2005, Ngai2006, Brugger2008, Schmidt2011] o como plantillas para el ensamblaje de nanomateriales \addcite[Wong2009].
Los geles polim\'ericos de tama\~no nano/micro de respuesta m\'ultiple son excelentes candidatos para el desarrollo de aplicaciones biom\'edicas m\'inimamente invasivas, incluida la ingeniería de tejidos \addcite[Daly2020].
\addcite[citet:Culver2017A] han utilizado nanogeles de poli(NIPAm-co-MAA) funcionalizados para la uni\'on y detecci\'on de diferentes prote\'inas.
Recientemente se investigaron dispositivos basados en microgeles de poli(NIPAm-co-MAA) para la encapsulación/liberaci\'on del f\'armaco quimioterap\'eutico Doxorrubicina \addcite[Giussi2020, MartinesMoro2020, Pergushov2020].

En este contexto, en este cap\'itulo mostramos el desarrollo de una teor\'ia de equilibrio de dos fases y realizamos una investigaci\'on sistem\'atica del comportamiento termodin\'amico de microgeles compuestos de copol\'imeros aleatorios de NIPAm y un comon\'omero \'acido (MAA).
Este modelo describe la qu\'imica f\'isica detr\'as de todos los siguientes fen\'omenos: el hinchamiento del microgel impulsado por el pH, la dependencia no monot\'onica del tama\~no de part\'icula de la concentraci\'on de sal y el colapso de la red al aumentar la temperatura por encima del VPTT.
Nuestras predicciones brindan una imagen clara de los efectos de la composici\'on de la soluci\'on (pH, concentración de sal) y la qu\'imica del pol\'imero (contenido de MAA, grado de entrecruzamiento) en el VPT.
Tambi\'en investigamos las mejores condiciones para la encapsulaci\'on de doxorrubicina y daunorrubicina dentro de estos microgeles.
Los c\'alculos  ac\'a mostrados son espec\'ificos para el \'acido metacr\'ilico con $pKa = 4,65$, pero el comportamiento fisicoqu\'imico informado puede describir cualitativamente una variedad de microgeles basados en NIPAm que se han modificado con otros monómeros \'acidos que tienen diferente pka y diferente solubilidad a pH bajo.


El comportamiento de los microgeles de poli(NIPAm-co-MAA), incluido su VPT y la interacci\'on con pol\'imeros de carga opuesta, se ha descrito utilizando una variedad de t\'ecnicas experimentales \addcite[Hoare2004,Dowding2000, Kleinen2008, Kleinen2010, Giussi2015, Su2016, Giussi2020].
Tambi\'en se han aplicado teor\'ias  y simulaciones moleculares de grano grueso para investigar el comportamiento de los microgeles polim\'ericos sensibles a est\'imulos \addcite[quesada2011gel, ahualli2016coarse, Landsgesell2019SM].
Estos trabajos se han centrado principalmente en el hinchamiento y otras propiedades de las part\'iculas que tienen una red de pol\'imero permanentemente cargada , y algunos han abordado el efecto de la temperatura y la calidad del solvente \addcite[Jha2011,QuesadaPerez2013,QuesadaPerez2014,moncho-jorda2016a,ahualli2016coarse,AdroherBenitez2017PCCP].
Recientemente, estudios  con simulaciones han considerado la respuesta al pH de microgeles compuestos de pol\'imeros reguladores de carga\addcite[Schroeder2015, Rudov2017, Sean2018, Hofzumahaus2018, Lu2019].
Sin embargo, solo unos pocos trabajos te\'oricos han investigado las propiedades de los microgeles de respuesta m\'ultiple en funci\'on de la temperatura, el pH y la concentración de sal \addcite[CaprilesGonzalez2008, polotsky2013collapse].


\subsection{Fase Microgel}\label{sec:theory}
%%%%%%%%%%%%%%%%%%%%%%%%%%%%%%%%%%%%%%%%%%%%%%%%%%%%%%%%%%%%%%%%%%%%%


Consideremos un modelo de dos fases: un microgel de poli(NIPAm-\emph{co}-MAA) (P(NIPAm-MAA)) (fase $1$, denotada por $MG$) en contacto con una soluci\'on acuosa ( fase $2$, denotada por $s$).
Externamente, podemos controlar la temperatura $T$, el pH y la concentraci\'on de sal de esta soluci\'on, lo que da como resultado que el microgel tenga un radio $R$ y un volumen $V=\frac{4}{3}\pi R^3$.
El potencial termodin\'amico cuyo m\'inimo produce las condiciones de equilibrio dentro de la fase de microgel es el semi-gran potencial can\'onico, $\Omega_{MG}$, que contiene las siguientes contribuciones:
%
\begin{align}
    \begin{aligned}
       \Omega_{MG}=& -TS_{mix} + F_{chem,MAA} +  F_{ela}\\
       & + U_{elec}+  U_{ste} + U_{VDW} -{\sum_{\gamma}
        {\mu_\gamma N_\gamma}}
    \end{aligned}
    \label{eq:free-energy-implicit}
\end{align}
%

\noindent donde $S_{mix}$ es la entropí\'i de traslaci\'on (mezcla) de las especies libres en la fase de microgel: mol\'eculas de agua ($w$), hidronio ($H_3O^+$) e iones de hidr\'oxido ($OH^-$ ), y cationes de sal ($+$) y aniones ($-$).
Aqu\'i consideramos una sal monovalente, $KCl$, y asumimos que est\'a completamente disociada en iones de potasio y cloruro.
$F_{chem,MAA}$ es la energ\'ia química libre que describe la protonación de equilibrio de las unidades MAA.
$F_{ela}$ es la energía libre el\'astica que explica la libertad conformacional de la red polim\'erica.
$U_{elec}$ y $U_{ste}$ representan respectivamente las interacciones electrost\'aticas y las repulsiones est\'ericas.
$U_{VDW}$ es la contribuci\'on que describe las interacciones efectivas pol\'imero-disolvente; incorpora la transici\'on hidrofílica-hidrof\'obica de NIPAm al aumentar la temperatura por encima de su LCST.
Finalmente, la suma de $\gamma$ expresa el equilibrio qu\'imico con la fase de soluci\'on, donde $\mu_\gamma$ y $N_\gamma$ son el potencial qu\'imico y el n\'umero de mol\'eculas de la especie $\gamma$, respectivamente.
Aqu\'i, el subíndice $\gamma$ es  sobre las especies qu\'imicas libres, $\gamma \in \left\{ w, H_3O^+, OH^-, +,- \right\}$.
Hay que tener en cuenta que $\Omega_{MG}$ es un potencial semi grancan\'onico porque la fase de microgel puede intercambiar cada una de estas mol\'eculas con la fase de soluci\'on, mientras que la red de pol\'imero está confinada dentro de la primera.

La forma expl\'icita del potencial termodin\'amico es:




%
\begin{align}
\begin{aligned}
\beta&\frac{\Omega_{MG}(R)}{V}=\\& ~ \sum_{\gamma} \rho_\gamma\left(\ln\left(\rho_\gamma v_w\right) -1 + \beta\mu^0_\gamma\right) \\
& + \frac{\phi_{MAA}}{v_{MAA}} \left[f(\ln f+ \beta\mu^0_{MAA^-})\right.\\
&\qquad\left.+(1-f)(\ln (1-f)+\beta\mu^0_{MAAH})\right] \\
%
& +  \left(\sum_{\gamma } {\rho_\gamma q_\gamma + f\dfrac{\phi_{MAA}}{v_{MAA}}q_{MAA}}\right)\beta\psi_{MG}\\
%
& -\sum_{\gamma }{\rho_\gamma\beta\mu_\gamma}
 -\beta\mu_{H^+}(1-f)\dfrac{\phi_{MAA}}{v_{MAA}}\\
%
& + \chi (T, \phi_{NIPAm})\rho_w \phi_{NIPAm} \\
%
& + \dfrac{3}{2}\dfrac{N_{seg}}{n_{ch} V}\left[\left(\dfrac{R}{R_0}\right)^2 - \ln\dfrac{R}{R_0} -1\right]
%
\end{aligned}
\label{eq:free-energy}
\end{align}

\noindent donde el primer t\'ermino (segunda l\'inea) de $S_{mix}$; $\beta=\frac{1}{k_BT}$ y $k_B$ es la constante de Boltzmann.
La densidad num\'erica de la especie $\gamma$ es $\rho_\gamma$ y $\mu^0_\gamma$ es su potencial qu\'imico est\'andar, que tambi\'en se incluye en esta primera contribución; $v_w$ es el volumen de una mol\'ecula de agua.


La segunda y tercera l\'inea de la ec. \ref{eq:free-energy} (lado derecho) incorporan el equilibrio \'acido-base de las unidades MAA, donde $\phi_{MAA}$ es la fracci\'on de volumen que ocupan estos segmentos, $v_{MAA}$ es la volumen de este segmento, y $f$ es el grado de disociaci\'on o la fracci\'on de estas unidades que se cargan.
La fracci\'on de volumen de los segmentos MAA cargados es $f\phi_{MAA}$, y la de las unidades protonadas o sin carga es $(1-f)\phi_{MAA}$.
Los potenciales qu\'imicos est\'andar son $\mu^0_{MAA^-}$ y $\mu^0_{MAAH}$ para las especies cargadas y protonadas, respectivamente.
Hay qye tener en cuenta que aqu\'i usamos el t\'ermino segmento para identificar las unidades qu\'imicas (monómeros) que componen las cadenas polim\'ericas (MAA y NIPAm).


La siguiente contribuci\'on a la ec. \ref{eq:free-energy} es la energ\'ia electrost\'atica, donde $q_\gamma$ y $q_{MAA}$ son la carga el\'ectrica de las moléculas $\gamma$ y los segmentos MAA, respectivamente.
El potencial electrost\'atico dentro de la fase de microgel es $\psi_{MG}$.


La siguiente contribución (l\'inea 6 de eq.\ref{eq:free-energy}) refuerza el equilibrio qu\'imico entre el microgel y la fase de soluci\'on, donde el segundo t\'ermino representa esos protones unidos a unidades MAA sin carga;
a saber, $\mu_{H^+}\equiv\mu_{H_3O^+}$ se conjuga con el n\'umero total de protones,
$N_{H_3O^+}+N_{MAAH}=V\left(\rho_{H_3O^+}+(1-f)\dfrac{\phi_{MAA}}{v_{MAA}}\right)$.


El siguiente t\'ermino en el potencial termodin\'amico explica la respuesta de PNIPAm a los cambios de temperatura a trav\'es de un par\'ametro de interacci\'on pol\'imero solvente, $\chi$, que depende de la temperatura y la fracci\'on de volumen de NIPAm, $\phi_{NIPAm}$.
Seg\'un \addcite[afroze2000], este par\'ametro de Flory-Huggins se puede expresar como:
%
%


\begin{align}
\begin{aligned}
\chi (T, \phi_{NIPAm}) &=g_0(T) +g_1(T)\phi_{NIPAm} \\
 &~+ g_2(T)\phi_{NIPAm}^2
\end{aligned}
\end{align}

\noindent con
%
%
\begin{align}
\begin{aligned} 
g_k(T)=g_{k0} + \frac{g_{k1}}{T} + g_{k2}T
\end{aligned}
\end{align}


\noindent para  $k=0,1,2$, los coeficinetes son: $g_{00}= -12.947$, $g_{02}=0.044959\,$K$^{-1}$, $g_{10}= 17.920$, $g_{12}= -0.056944$\,K$^{-1}$, $g_{20}= 14.814$, $g_{22}= -0.051419$\,K$^{-1}$  y $g_{k1}\equiv 0$ \addcite[afroze2000]. 


El \'utimo t\'ermino en eq. \ref{eq:free-energy} es la contribuci\'on resultante de la elasticidad de la red de pol\'imero entrecruzado que forma la columna vertebral del microgel.
Esta expresi\'on se describe en \addcite[mocho-jorda2016a] y proviene del modelo de elasticidad del caucho,
donde $N_{seg}$ es el n\'umero total de segmentos en la red de pol\'imero y $n_{ch}$ es el n\'umero de segmentos por cadena de pol\'imero o \emph{longitud de cadena}.
Se tiene en cuenta que la constante de resorte en la energ\'ia libre el\'astica es proporcional al cociente $\dfrac{N_{seg}}{n_{ch}}$, que representa el n\'umero (total) de cadenas de pol\'imero en el microgel.
El radio del microgel seco es $R_0$, lo que satisface con la expresi\'on:

%
%
\begin{align}
\begin{aligned} 
\dfrac{4}{3}\pi R_0^3=V_0&=N_{seg}\Big( x_{MAA} v_{MAA}\\
&\qquad+x_{NIPAm} v_{NIPAm}\Big)
\end{aligned}
\end{align}


\noindent donde $V_0$ es el volumen de la part\'icula seca; $x_{MAA}$ y $x_{NIPAm}$ son la fracci\'on de los segmentos MAA y NIPAm, respectivamente.
Entonces, el n\'umero total de segmentos MAA es $x_{MAA}N_{seg}$ y el de unidades NIPAm es $x_{NIPAm}N_{seg}$, cada uno con un volumen $v_{NIPAm}$.
Los microgeles que consideramos aqu\'i satisfacen $x_{NIPAm}=1-x_{MAA}$.


Deben imponerse dos restricciones f\'isicas al potencial termodin\'amico dado por la ec. \ref{eq:free-energy}.
El volumen del microgel est\'a completamente ocupado por los segmentos de la red y las especies quí\'icas libres.
Esta restricci\'on de incompresibilidad incorpora las repulsiones est\'ericas (volumen excluido) y se puede expresar como:
%

%
%
\begin{align}
\begin{aligned}
\sum_{\gamma } \rho_\gamma v_\gamma  + \phi_{MAA} + \phi_{NIPAm} = 1
\end{aligned}
\label{eq:packing}
\end{align}



\noindent donde $v_\gamma$  es el volumen molecular de la especie $\gamma$, y la fracci\'on de volumen de cada componente de la red son: 
%
%
\begin{align}
\phi_{MAA}&=N_{seg}\dfrac{x_{MAA}v_{MAA}}{\frac{4}{3}\pi R^3}\\
\phi_{NIPAm}&=N_{seg}\dfrac{x_{NIPAm}v_{NIPAm}}{\frac{4}{3}\pi R^3}
\end{align}


La segunda restricci\'on  que se impone es la electroneutralidad del microgel, que puede expresarse como:
%
%
\begin{align}
\begin{aligned}
\sum_{\gamma  } \rho_\gamma q_\gamma + f\frac{\phi_{MAA}}{v_{MAA}}q_{MAA}=0
\end{aligned}
\label{eq:charge-neutrality}
\end{align}


Ahora, el potencial termodin\'amica esta explicitamente en funci\'on de las densidades de todas las especies, el grado de carga del MAA y el radio del microgel, $\Omega_{MG}(R)\equiv\Omega_{MG}(\{\rho_\gamma\},f,R)$.
Para obtener las expresiones de $\{\rho_\gamma\}$ y $f$ que sean consistentes con el equilibrio termodin\'amico, minimizamos $\Omega_{MG}$ con respecto a estas cantidades, sujeto a las restricciones ec. \ref{eq:packing} y ec. \ref{eq:charge-neutrality}; dicho procedimiento conduce a: 
%
%
\begin{align}
\rho_\gamma v_w &= a_\gamma \exp(-\beta\pi_{MG}v_\gamma -\beta\psi_{MG}q_{\gamma})\\
\frac{f}{1-f}&= \frac{K^0_{MAA}}{a_{H^+}}\exp(-\beta\psi_{MG}q_{MAA})\label{eq:fcharge}
\end{align}

\noindent donde $a_\gamma = e^{\beta\mu_\gamma-\beta\mu_\gamma^0}$ es la actividad de la especie $\gamma$, y $\pi_{MG}$ es la presi\'on osm\'otica de la fase de microgel, introducido como un multiplicador de Lagrange para imponer la restricci\'on de incompresibilidad, ec. \ref{eq:packing}.
La constante de equilibrio termodin\'amico que describe la protonaci\'on/desprotonaci\'on MAA es
%
%
\begin{align}
K^0_{MAA}= e^{\beta\mu^0_{MAAH}-\beta\mu^0_{MAA}-\beta\mu^0_{H^+}}
\end{align}

\noindent Esta cantidad se puede calcular directamente a partir del pKa del \'acido.


Para un $R$ dado, las \'unicas inc\'ognitas restantes para determinar $\Omega_{MG}(R)$ son la presi\'on osm\'otica, $\pi_{MG}$ y el potencial electrost\'atico, $\psi_{MG}$.
Estas dos cantidades se pueden calcular resolviendo num\'ericamente la incompresibilidad y la electroneutralidad de la fase de microgel, ec. \ref{eq:packing} y ec. \ref{eq:charge-neutrality}, respectivamente.
Para resolver estas ecuaciones utilizamos un m\'etodo h\'ibrido de Powell sin jacobiano y un c\'odigo FORTRAN desarrollado internamente.


Todas las dem\'as cantidades involucradas en el c\'alculo de $\Omega_{MG}(R)$ son entradas, incluidas las propiedades de las diferentes especies qu\'imicas consideradas, que se resumen en la tabla \ref{table:molecules}.
Usamos $pK_w=14$ para describir el equilibrio de autodisociaci\'on del agua.
Las actividades de todas las especies qu\'imicas libres se pueden calcular a partir de la concentración de estas mol\'eculas en la fase de soluci\'on, como se analiza a continuaci\'on.


\begin{table}
	\centering
%\small
%\begin{tabular}{lcS[table-format=-1]S[table-format=0.3]}
\begin{tabular}{|lccc|}
    \hline
    {Species} & {$pKa$} & {$q$ ($e$)} & {$v$ ($\text{nm}^3$)} \\
      \hline
$H_2O\,(w)$ & ~ & ~ & 0.03\\
$H_3O^+$ & ~ & +1 & 0.03\\
$OH^-$ & ~ & -1 & 0.03\\
$K^+\,(+)$ & ~ & +1 & 0.04\\ 
$Cl^-\,(-)$ & ~ & -1 & 0.047\\
$MAA$ & 4.65 & -1$^\ast$ & 0.09\\
$NIPAm$ & ~ & ~ & 0.12\\
    \hline
  \end{tabular}
 \caption{Propiedades moleculares de las diferentes especies qu\'imicas consideradas.
 	\footnotesize ($^\ast$Para las especies desprotonadas).}
\label{table:molecules} 
\end{table}


%%%%%%%%%%%%%%%%%%%%%%%%%%%%%%%%%%%%%%%%%%%%%%%%%%%%%%%%%%%%%%%%%%%%%
\subsection{Fase soluci\'on}
%%%%%%%%%%%%%%%%%%%%%%%%%%%%%%%%%%%%%%%%%%%%%%%%%%%%%%%%%%%%%%%%%%%%%

En la soluci\'on, el potencial termodin\'amico es:
%
%
\begin{align}
\begin{aligned}
\beta&\frac{\Omega_s}{V}=\\& \sum_{\gamma   } {\rho^s_\gamma\left(\ln(\rho_\gamma^sv_w) -1 + \beta\mu_\gamma^0 - \beta\mu_\gamma\right)}
\end{aligned}
\label{eq:bulk}
\end{align}

\noindent el subindice  $s$  indica las densidades en la soluci\'on.
Al escribir la ec. \ref{eq:bulk}, hemos considerado un volumen de referencia igual al del microgel, $V$, y se ha considerado como cero el potencial electrost\'atico en la fase de soluci\'on.


Una vez que se establece la composición de la solución (pH y concentraci\'on de sal), conocemos las densidades de todas las especies qu\'imicas en esta fase, que deben satisfacer tanto las restricciones de incompresibilidad como de neutralidad de carga:
%
%
\begin{align}
\sum_{\gamma  } \rho_\gamma^s v_\gamma  &=1\label{eq:bulk-packing}\\
\sum_{\gamma  } \rho_\gamma^s q_\gamma  &=0
\end{align}

\noindent con
%
%
\begin{align}
\rho_\gamma^s v_w= a_\gamma \exp(-\beta\pi_s v_\gamma)
\label{eq:bulk-electroneutrality}
\end{align}



\noindent para $\gamma \in \left\{ w, H_3O^+, OH^-, +,- \right\}$, donde $\pi_s$ es la presión osmótica de la fase de solución introducida como un multiplicador de Lagrange por la ecuaci\'on \ref{eq:bulk-packing}.
Despu\'es de conocer $\pi_s$, podemos determinar las actividades de todas las especies qu\'imicas libres.

\subsection{Minimizaci\'on gr\'afica}
%%%%%%%%%%%%%%%%%%%%%%%%%%%%%%%%%%%%%%%%%%%%%%%%%%%%%%%%%%%%%%%%%%%%%

\begin{figure*}[!htb]
\centering
\includegraphics[width=1.\linewidth]{Figures/graph-gel/graph-min.png}
\caption{Potencial termodin\'amico en funci\'on del radio del microgel a diferentes temperaturas, $pH~4.65$ y $cs=10^{-2}M$.
	Cada panel corresponde a un microgel MG100 diferente (longitud de cadena, $n_{ch}=100$) con $10\%$ (A), $35\%$ (B) y $50\%$ (C) MAA.
	Las curvas presentan el potencial termodin\'amico en exceso de la contribuci\'on de la soluci\'on, $\Omega=\Omega_{MG}-\Omega_s$, en algunas unidades convenientes, donde $V_0=\frac{4}{3}\pi R_0^3$ es el volumen de la part\'icula polim\'erica seca.
	En el panel B, los puntos  marcan el radio \'optimo para cada temperatura, que es el m\'inimo local/global de la curva correspondiente (ver tabla \ref{table:optimal-R}).}
\label{fig:graph-min}
\end{figure*}

En este punto, es posible determinar completamente la energí\'i libre de la fase de microgel para cualquier $R$ dado.
Las variables independientes de un c\'alculo son la temperatura, el pH y la concentraci\'on de sal de la soluci\'on en contacto con la fase microgel.
El n\'umero de segmentos en la red de pol\'imero $N_{seg}$, la longitud de la cadena $n_{ch}$ y la fracci\'on de segmentos MAA, $x_{MAA}$, caracterizan completamente el microgel.

We consider microgels with $N_{seg}=10^7$ segments and $n_{ch}=50$, $100$, and $200$, having either $x_{MAA}=0.1$, $0.35$ or $0.5$.
Our goal is to evaluate the effect of increasing or reducing the amount of acidic monomer with respect to poly(NIPAm-\emph{co}-MAA) microgels having $35\%$ MAA, which are typically synthesized in our lab \addcite[Giussi2015,Giussi2020].
These microgels are labeled MG$n_{ch}$-$p_{MAA}$, where $p_{MAA}$ is the percentage of MAA.
For example, MG100-10 is the microgel with  $n_{ch}=100$ and $x_{MAA}=0.1$.


Consideramos microgeles con $N_{seg}=10^7$ segmentos y $n_{ch}=50$, $100$ y $200$, que tienen $x_{MAA}=0,1$, $0,35$ o $0,5$.
El objetivo es evaluar el efecto de aumentar o reducir la cantidad de mon\'omero \'acido con respecto a los microgeles de poli(NIPAm-\emph{co}-MAA) que tienen $35\%$ MAA. % que normalmente se sintetizan en nuestro laboratorio \addcite[Giussi2015 ,Giussi2020].
Estos microgeles est\'an etiquetados como MG$n_{ch}$-$p_{MAA}$, donde $p_{MAA}$ es el porcentaje de MAA.
Por ejemplo, MG100-10 corresponde a un  microgel con $n_{ch}=100$ y $x_{MAA}=0,1$.



To determine the size of the microgel for a given set of conditions, we resource to a graphical minimization procedure.
For each set of conditions (pH, salt, and $T$), we construct $\Omega(R)=\Omega_{MG}(R)-\Omega_{s}(R)$, and find $R_{opt}$, the optimal radius, such that the curve has a local (and global) minimum.
As an example, this procedure is illustrated in figura \ref{fig:graph-min} for MG100 microgels.
The results obtained from the graphical minimization of figura \ref{fig:graph-min} curves are summarized in tabla \ref{table:optimal-R}.

Para determinar el tama\~no del microgel para un conjunto dado de condiciones, recurrimos a una minimizaci\'on gr\'afica del mismo.
Para cada conjunto de condiciones (pH, sal y $T$), construimos $\Omega(R)=\Omega_{MG}(R)-\Omega_{s}(R)$, y encontramos $R_{opt }$, siedo este el radio \'optimo, tal que la curva tenga un m\'inimo local (y global).
Como ejemplo, este procedimiento se ilustra en la figura \ref{fig:graph-min} para microgeles MG100.
Los resultados obtenidos de la minimizaci\'on de las curvas figura \ref{fig:graph-min} se resumen en la tabla \ref{table:optimal-R}.

\begin{table}[!htb]
\centering
\small
  \begin{tabular}{|lccccc|}
   \hline %\multirow{2}{*}{MG100} & 
    %  \multicolumn{4}{c}{Opt. Radius (nm)(MG100)} \\
    	&&   Opt. Radius (nm)(MG100 & && \\
    	\hline
      & {25 $^\circ C$} & {30 $^\circ C$} & {35 $^\circ C$} & {40 $^\circ C$} & {dry, $R_0$} \\
      \hline
    10\% MAA & 215 &  184 &  75  &  74 & 65\\
    35\% MAA &  213 &  193 &  84 & 76 & 64\\
    50\% MAA &  213 & 199 &  172 & 85 & 63\\
    \hline
  \end{tabular}
 \caption{Minimizaci\'on de las curvas de la  figura \ref{fig:graph-min}.
 	Esta tabla resume los radios \'optimos de tres microgeles MG100 a diferentes temperaturas, $pH\,4.65$ y $cs=10^{-2}M$.}
\label{table:optimal-R} 
\end{table}


\subsection{Absroci\'on}
%%%%%%%%%%%%%%%%%%%%%%%%%%%%%%%%%%%%%%%%%%%%%%%%%%%%%%%%%%%%%%%%%%%%%


Para describir la absorci\'on de un analito a la fase de microgel,
el potencial termodin\'amico de la ec. \ref{eq:free-energy} agrega los siguientes t\'erminos:
%
%
%
\begin{align}
\begin{aligned}
\beta&\frac{\Omega_{MG}(R)}{V}= \cdots\\&+ \rho_a\left(\ln\left(\rho_a v_w\right) -1 + \beta\mu^0_a\right) \\
& + \rho_a \sum_\tau n_\tau  \left[g_\tau(\ln g_\tau+ \beta\mu^0_{\tau})\right.\\
&\qquad\left.+(1-g_\tau)(\ln (1-g_\tau)+\beta\mu^0_{\tau H})\right] \\
& +  \left( \rho_a \sum_\tau n_\tau f_\tau q_\tau\right)\beta\psi_{MG}\\
& -\rho_a\beta\mu_a
 -\beta\mu_{H^+} \rho_a \sum_\tau n_\tau g_\tau
\end{aligned}
\label{eq:ads}
\end{align}
%
\noindent La primera l\'inea (lado derecho) representa los grados de libertad de traslaci\'on,
donde $\rho_a$ es la densidad num\'erica del analito y $\mu_a^0$ su potencial qu\'imico est\'andar.
Las siguientes dos l\'ineas describen el equilibrio \'acido-base de las unidades titulables del analito;
el sub\'indice $\tau$ recorre dichas unidades moleculares que tienen un grado de protonaci\'on $g_\tau$ y un volumen $v_\tau$.
El analito tiene $n_\tau$ de estos segmentos;
$\mu^0_{\tau H}$ y $\mu^0_\tau$ son el potencial qu\'imico est\'andar de las especies protonadas y desprotonadas, respectivamente, que se relacionan con la constante de disociaci\'on \'acida:
%
\begin{align}
K^0_{\tau}= e^{\beta\mu^0_{\tau H}-\beta\mu^0_{\tau}-\beta\mu^0_{H^+}}
\end{align}
%

La siguiente l\'inea en la ec. \ref{eq:ads} describe la contribuci\'on del analito a la energ\'ia electrost\'atica, donde $f_\tau$ es el grado de carga de las unidades $\tau$, que es igual a $g_\tau$ si $\tau$ es un grupo b\'asico, o $(1-g_\tau)$ si la unidad es \'acida; $q_\tau$ es la carga de las especies ionizadas.
Los dos \'ultimos t\'erminos dan cuenta del equilibrio qu\'imico entre el microgel y la fase de soluci\'on, donde $\mu_a$ es el potencial qu\'imico del analito.

Adem\'as, la ec. \ref{eq:packing} debe incorporar la fracci\'on total de volumen ocupada por el analito: $\rho_a \sum_\lambda n_\lambda v_\lambda$, donde $\lambda$ recorre todos los tipos de segmentos que forman la mol\'ecula, incluyendo unidades titulables $\{\tau\}\in\{\lambda\}$.
La presencia del analito en la fase de soluci\'on tambi\'en representa contribuciones adicionales al potencial termodin\'amico $\Omega_s$ de ec. \ref{eq:bulk}, que contienen los mismos componentes que ec. \ref{eq:ads}.

De la optimizaci\'on de  $\Omega_{MG}$  se obtiene:
%
\begin{equation}
\frac{f_\tau}{1-f_\tau}=\left(\frac{K^0_\tau}{a_{H^+}}\right)^{\pm 1} e^{-\beta \psi_{MG} q_\tau}
\label{eq:f_ads}
\end{equation}
%
\noindent para el grado de carga de las unidades $\tau$, donde el signo $\pm$ diferencia el caso de un grupo \'acido ($+$) de uno b\'asico ($-$).
Para la densidad del analito obtenemos:
%
\begin{align}
    \begin{aligned}
   \rho_a v_w =&\frac{ \exp{\left(\beta \mu_a - \beta \mu^0_a \right)}}{\prod_\tau \left(1-f_\tau\right)^{n_\tau}}\\
&\quad \cdot\exp{\left(-\beta \pi_{MG} \sum_\lambda n_\lambda v_\lambda \right)} 
    \end{aligned}\label{eq:rho_ads}
\end{align}
%
\noindent where this last equation requires a redefinition of $\mu_a$ and $\mu_a^0$.
Similar expressions to eq. \ref{eq:rho_ads} and eq. \ref{eq:f_ads} are derived for the solution phase.







\begin{figure}[!tb]
\centering
\includegraphics[width=0.35\linewidth]{Figures/graph-gel/dauno-doxo.png}
\caption{Estructura quí\'imica (arriba) y modelo de grano grueso (abajo) aplicado para describir daunorrubicina y doxorrubicina.
	Los segmentos de grano grueso $D1-D4$ se describen en la tabla \ref{table:drugs}.}
\label{fig:dauno-doxo}
\end{figure}



We will consider the absorption of the chemotherapeutic drugs Daunorubicin (Dauno) and Doxorubicin (Doxo) to the P(NIPAm-MAA) microgels under different conditions.
The molecular model applied to describe these analytes is illustrated in figura  \ref{fig:dauno-doxo} and the parametrization is presented in tabla \ref{table:drugs}.\addcite[PerezChavez2020]

Consideraremos la absorci\'on de los f\'armacos quimioterap\'euticos Daunorrubicina (Dauno) y Doxorrubicina (Doxo) a los microgeles P(NIPAm-MAA) en diferentes condiciones.
El modelo molecular aplicado para describir estos analitos se ilustra en la figura \ref{fig:dauno-doxo} y la parametrizaci\'on se presenta en la tabla \ref{table:drugs}.\addcite[PerezChavez2020]

\begin{table}
%\small
%\begin{tabular}{lcS[table-format=-1]S[table-format=0.3]}
\centering
\begin{tabular}{|lccc|}
    \hline
    {CG unit} & {$pKa$} & {$q$ ($e$)} & {$v$ ($\text{nm}^3$)} \\
      \hline
$D1$ & - & 0 & 0.085\\
$D2$ & 7.34 & -1$^\ast$ & 0.085\\
$D3$ & 9.46 & +1$^\ast$ & 0.085\\ 
$D4$ (Doxo) & 8.46 & -1$^\ast$ & 0.035\\
$D4$ (Dauno) & - & 0 & 0.035 \\
    \hline
  \end{tabular}
 \caption{Porpediades moleculares para las distintas unidades de grano grueso usadas para el modelado de las drogas Daunorubicina y Doxorubicina. (ver figura \ref{fig:dauno-doxo}).
\footnotesize ($^\ast$ Para unidades ionizables.)}
\label{table:drugs} 
\end{table}




\section{Resultados y discusi\'on}
%%%%%%%%%%%%%%%%%%%%%%%%%%%%%%%%%%%%%%%%%%%%%%%%%%%%%%%%%%%%%%%%%%%%%



%%%%%%%%%%%%%%%%%%%%%%%%%%%%%%%%%%%%%%%%%%%%%%%%%%%%%%%%%%%%%%%%%%%%%
\subsection{Respuesta al pH y la concentraci\'on de sal}\label{sec:pH_salt}
%%%%%%%%%%%%%%%%%%%%%%%%%%%%%%%%%%%%%%%%%%%%%%%%%%%%%%%%%%%%%%%%%%%%%


\begin{figure}[!ht]
\centering
\includegraphics[width=0.5\linewidth]{Figures/graph-gel/R-pH.png}
\caption{Gr\'afico de tama\~no de microgel (A) y grado de carga (B) en funci\'on del pH para soluciones que tienen diferentes concentraciones de sal y $T=25 ^\circ C$.
	Las cadenas de pol\'imero en el microgel MG100-35 son $n_{ch}=100$-long y tienen $35\% $ MAA.
	La curva de línea punteada en el panel B es la disociaci\'on ideal del \'acido metacr\'ilico ($pKa=4.65$).
	Los c\'irculos de color en las curvas del panel A marcan el pKa aparente del microgel.}
\label{fig:R-pH}
\end{figure}

En esta secci\'on, describiremos el comportamiento de los microgeles en respuesta a cambios en la composici\'on de la soluci\'on.
Nos concentramos en temperaturas por debajo de la LCST del PNIPAm;
el efecto de la temperatura se evaluar\'a en \ref{sec:temperature}.


La Figura \ref{fig:R-pH}A muestra el tama\~no del microgel (radio, $R$) en funci\'on del pH para diferentes concentraciones de sal.
Los microgeles P(NIPAm-MAA) se hinchan al aumentar el pH.
A medida que aumenta el pH, un n\'umero creciente de unidades MAA se desprotona y se carga.
Figura \ref{fig:R-pH}B muestra c\'omo la fracci\'on de MAA cargados ($f$: grado de carga; ver ec. \ref{eq:fcharge}) depende del pH de la soluci\'on.
La hinchaz\'on del microgel observada en el panel A a medida que aumenta el pH es la respuesta a las crecientes repulsiones dentro de la red que resultan del aumento de la carga el\'ectrica en el pol\'imero que se ve en el panel B.


El inicio de la transici\'on de hinchamiento se desplaza a valores de pH m\'as altos cuando se reduce la concentraci\'on de sal (ver fig. \ref{fig:R-pH}A).
Las curvas de disociaci\'on de protones del panel B presentan el mismo desplazamiento a pHs m\'as altos, con respecto al comportamiento ideal de un mon\'omero MAA aislado en soluci\'on diluida.
El pKa aparente de un microgel es el pH al que se desprotonan la mitad de los segmentos MAA;
cuantifica el comportamiento de carga del microgel, fig. \ref{fig:R-pH}B, pero tambi\'en la transici\'on de expansi\'on como vemos en el panel A (ver circulos en $pH=pKa$).
Los pKa aparentes de la fig. \ref{fig:R-pH}B se muestran en la tabla \ref{table:pKa_app}.

\begin{table}[!htb]
\small
  \begin{tabular}{|cc|}
    \hline
      [NaCl] (M)&  pKa app. ($25 ^\circ C$)  \\
      \hline
    $10^{-5}$ & 8.10  \\
    $10^{-4}$ & 7.15 \\
    $10^{-3}$ & 6.15 \\
    $10^{-2}$ & 5.35 \\
    %\azul $10^{-1}$ & \azul 4.80 \\
    ideal (pKa) &  $4.65$  \\
    \hline
  \end{tabular}
 \caption{ pka's aprente de la fig. \ref{fig:R-pH} para un gel MG100-35 a $25 ^\circ C$.}
\label{table:pKa_app} 
\end{table}


Una concentraci\'on relativamente alta de iones de sal dentro del microgel da como resultado la detecci\'on de las repulsiones electrost\'aticas entre los segmentos MAA cargados; estas interacciones repulsivas se vuelven de corto alcance.
Cuando el pH de la soluci\'on aumenta, la disociación de MAA sucede sin un alto costo energ\'etico de repulsiones electrost\'aticas.
En estas condiciones, la desprotonaci\'on de MAA, inducida por la energ\'ia qu\'imica libre (equilibrio \'acido-base), se aproxima al comportamiento ideal o de soluci\'on diluida (compare los casos de alta [sal] con la curva de l\'inea punteada en la fig. \ref{fig:R -pH}B).



Por el contrario, el efecto de apantallamiento se debilita y las repulsiones electrost\'aticas dentro de la red son de mayor alcance para soluciones con baja concentraci\'on de sal.
Incluso si aparecen pocas cargas distantes en la red, interactuar\'an entre s\'i.
Para reducir la contribuci\'on energ\'etica de tales repulsiones electrost\'aticas, es significativamente menos probable que las unidades MAA se carguen en condiciones de baja salinidad;
el pKa aparente aumenta.
El precio a pagar en cambio es aumentar la energ\'ia química libre, cuya contribuci\'on se minimiza cuando el grado de protonaci\'on es ideal.



\begin{figure*}[!htb]
	\centering
	\includegraphics[width=1\linewidth]{Figures/graph-gel/R-cs.png}
	\caption{Gr\'afico del tama\~no del microgel en funci\'on de las concentraciones de sal para diferentes soluciones de pH y $T=25 ^\circ C$.
		Los paneles corresponden a microgeles MG-100 (longitud de cadena, $n_{ch}=100$) que tienen fracciones MAA: $10\%$ (A), $35\%$ (B) y $50\%$ (C).}
	\label{fig:R-cs}
\end{figure*}

La figura \ref{fig:R-cs} ilustra c\'omo el tama\~no de los microgeles de P(NIPAm-MAA) depende de la concentraci\'on de sal para diferentes valores de pH.
A una salinidad relativamente alta, estos microgeles se hinchan con el aumento de la concentraci\'on de sal, lo que es consistente con los resultados de dispersi\'on de luz din\'amica (DLS) de \addcite[citet:Wong2009] para microgeles P(NIPAm-MAA) y concentraciones de NaCl en el rango de $0.1-0.5 M$.

Las curvas de la Figura \ref{fig:R-cs} muestran un comportamiento reentrante, en el que el tama\~no primero aumenta y luego disminuye al aumentar la concentración de sal.
Esta respuesta no monot\'onica es m\'as prominente cuando la carga del pol\'imero aumenta debido a un mayor contenido de pH o MAA (compare diferentes paneles de firgura \ref{fig:R-cs}).



Se han informado transiciones de hinchamiento-deshinchamiento con concentraciones de sal variables para una variedad de sistemas polim\'ericos reguladores de carga.
El grosor de las capas de poli\'acidos d\'ebiles anclados es una funci\'on no monot\'onica de la concentraci\'on de sal de la soluci\'on seg\'un lo predicho por la teor\'ia del campo medio autoconsistente \addcite[Israels1994,Lyatskaya1995,Zhulina1995,Gong2007], que ha sido confirmada por resultados experimentales \addcite [Wu2007].
De manera similar, los resultados te\'oricos predicen que el tama\~no de los polielectrolitos d\'ebiles ramificados en estrella muestra un m\'aximo en funci\'on de la concentraci\'on de sal en la soluci\'on \addcite[Borisov1998,KleinWolterink2002];
Tambi\'en se ha predicho que el espesor de las pel\'iculas de poliácidos d\'ebiles entrecruzados mostrar\'a este comportamiento de hinchamiento reentrante \addcite[Longo2014JCP].

Se ha predicho una transici\'on deshinchaz\'on a hinchaz\'on impulsada por la sal para los nanogeles de polielectrolitos fuertes \addcite[jha2012understanding];
este comportamiento, en el caso de los polielectrolitos quencheados, se atribuy\'o a los efectos de volumen excluidos de los iones absorbidos a altas concentraciones de sal.
M\'as relevante para nuestro estudio, se predijo te\'oricamente una transici\'on de hinchaz\'on a colapso reentrante para microgeles sensibles al pH y al calor \addcite[polotsky2013collapse].
\addcite[citet:polotsky2013collapse] explica que el aumento de la concentraci\'on de sal primero promueve la disociaci\'on de carga de los grupos \'acidos d\'ebiles hasta que se alcanza la saturaci\'on cuando el grado de disociaci\'on alcanza el valor ideal.
M\'as all\'a de este punto, el aumento de la concentraci\'on de sal de la soluci\'on solo mejora la detecci\'on de las repulsiones electrost\'aticas y, por lo tanto, el microgel se deshincha.
Experimentalmente, \addcite[citet:CaprilesGonzalez2008] inform\'o el hinchamiento no mon\'otónico de los microgeles de poli(NIPAm-\emph{co}-AA) (P(NIPAm-AA)) en funci\'on de la concentraci\'on de NaCl usando DLS.

\begin{figure}[!tb]
	\centering
	\includegraphics[width=0.5\linewidth]{Figures/graph-gel/f-cs.png}
	\caption{Curvas del grado de carga MAA (A) y la longitud de Debye (B) dentro de los microgeles MG100-35 en funci\'on de la concentraci\'on de sal para diferentes valores de pH.
		Estos resultados corresponden a las condiciones de figura \ref{fig:R-cs}B.}
	\label{fig:f-cs}
\end{figure}



El aumento de la concentraci\'on de sal de la soluci\'on tiene dos efectos opuestos sobre las propiedades del microgel.
Por un lado, aumenta el apantallamiento de interacciones de carga a medida que se cargan los iones dentro del gel; las repulsiones electrost\'aticas entre los mon\'omeros MAA cargados est\'an cada vez m\'as protegidas.
El alcance efectivo de estas repulsiones se acorta favoreciendo la deshinchamiento.
Por otro lado, este apantallamiento permite una mayor desprotonaci\'on de los mon\'omeros MAA, promovida por el equilibrio \'acido-base.
La disociació\'on de carga favorece el hinchamiento para reducir las repulsiones electrostá\'aticas.


La figura \ref{fig:f-cs} ilustra este doble efecto de aumentar la concentraci\'on de sal en la soluci\'on, lo que conduce al comportamiento de hinchamiento-deshinchamiento.
El panel A muestra que la carga del microgel aumenta monó\'otonamente con concentraci\'on de sal.
En el panel B, usamos la longitud de Debye para cuantificar la extensió\'on de las interacciones electrostá\'aticas % (consulte \cref*{si:eq:debye_length} en \supp).
El alcance efectivo de estas interacciones se acorta dentro del microgel a medida que aumenta la concentració\'on de sal.
Pede observarse en la figura \ref{fig:f-cs} que cuando $pH~3$ la carga dentro de la red de polímero es insignificante, lo que resulta en un hinchamiento apreciable en la figura \ref{fig:R-cs}B.



Esta teoria requiere que el interior del microgel sea de carga neutra.
\addcite[citet:Claudio2009] demostr\'o que esta es una aproximación razonable cuando el microgel es m\'as grande que $R=125\,\text{nm}$ y tiene un 50\% de mon\'omeros cargados.
Los microgeles P(NIPAm-MAA) de este trabajo son m\'as grandes que ese tama\~no en la mayor\'ia de las condiciones, particularmente cuando el pH está por encima del pKa aparente y la mayor\'ia de los grupos MAA est\'an desprotonados.
Adem\'as, los iones de sal se absorben dentro del microgel para reforzar dicha restricci\'on, lo que permite que los segmentos MAA se desprotonen y se carguen el\'ectricamente.
Describimos este efecto como la detecci\'on de las repulsiones electrost\'aticas entre los grupos MAA, que es un concepto funcional que permite una interpretaci\'on clara de muchas caracter\'isticas del comportamiento de estos microgeles. %\cite{Longo2011}




\subsection{Respuesta a la Temperatura]}\label{sec:temperature}
%%%%%%%%%%%%%%%%%%%%%%%%%%%%%%%%%%%%%%%%%%%%%%%%%%%%%%%%%%%%%%%%%%%%%

\begin{figure*}[!htb]
	\centering
	\includegraphics[width=1\linewidth]{Figures/graph-gel/R-T.png}
	\caption{Gr\'afico del tama\~no del microgel en funci\'on de la temperatura para diferentes concentraciones de sal en soluci\'on y $pH~4,65$.
		Los paneles corresponden a microgeles MG-100 (longitud de cadena, $n_{ch}=100$) que tienen diferentes fracciones de MAA: $10\%$ (A), $35\%$ (B) y $50\%$ (C).}
	\label{fig:R-T}
\end{figure*}


En esta secci\'on mostraremos la respuesta de los microgeles de P(NIPAm-MAA) frente a cambios en la temperatura.
En cada panel de la figura \ref{fig:R-T} se muestra el tama\~no de tres geles MG-100 como funci\'on de la temperatura a distintas concentraciones salinas.
A baja temperatura, estos microgeles muestran un estado relativamente hinchado, mientras que a altas temperaturas se produce un estado colapsado (alta densidad de polímero).


Esto ultimo ocurre dado que el NIPAm adquiere un comportamiento hidrofobico por arriba de su LCST, expulsando el solvente de su interior y colapsando su estructura \addcite[sbeih2019structural].
El tama\~no del gel en este estado es robustamente independiente de la concentraci\'on salina o el pH y posee un radio muy cercano al del microgel seco (ver tabla \ref{table:optimal-R})



Por otro lado, el estado hinchado del gel, es dominado por las repulsiones electrost\'aticas entre los semgemtos de MAA cargados y los contraiones absorvidos, como fue descrito en la secci'on \ref{sec:pH_salt}.
El tama\~no y la carga del microgel son funciones mon\'otonamente decrecientes de
la temperatura.
El estado hinchado se caracteriza por un mayor grado de carga %(ver \cref*{si:fig:f-T} en el \supp).
De hecho, el VPT hinchado a colapsado est\'a acompa\~nado por una transici\'on en el grado de carga de los segmentos MAA.


\begin{figure}[!htb]
	\centering
	\includegraphics[width=0.5\linewidth]{Figures/graph-gel/Tpt-pH.png}
	\caption{Gr\'aficos que muestran la temperatura de transici\'on de volumen $T_{PT}$ (A) y la fracci\'on de MAA cargado a esta temperatura (B) en funci\'on del pH para diferentes concentraciones de sal.
		Este microgel P(NIPAm-MAA) tiene una longitud de cadena de $n_{ch}=100$ y un MAA de $35\%$.}
	\label{fig:Tpt-pH}
\end{figure}


En la mayor\'ia de las condiciones, pero no en todas, la transici\'on entre estos dos estados del microgel es brusca y ocurre en un rango estrecho alrededor de una temperatura bien definida ($T_{pt}$).
Comparando los diferentes paneles de figura \ref{fig:R-T}, vemos que aumentar el contenido de MAA de los microgeles conduce a una transici\'on m\'as suave alrededor de $T_{pt}$.



Figura \ref{fig:Tpt-pH}A shows that the $T_{pt}$ increases with pH and salt concentration.
These results are consistent with DLS experiments showing that the VPTT of P(NIPAm-MAA) microgels increases with pH \addcite[Kleinen2008], which has also been observed for P(NIPAm-AA) microgels \addcite[CaprilesGonzalez2008].
We have defined $T_{pt}$ as the inflection point of the $R(T)$ curves of figura \ref{fig:R-T} between the swollen and collapsed states \addcite[Kratz2001].






Panel B of figura \ref{fig:Tpt-pH} shows the degree of charge of MAA segments at the $T_{pt}$.
There is a clear correlation between the dependence of $T_{pt}$ with the pH and salinity and the state of charge of the microgel at the VPT conditions.
The transition temperature increases with pH and salt concentration as does the charge of polymer network.

As opposed to this behavior, The VPTT of permanently charged PNIPAm-based microgels decreases with salt concentration\addcite[Lopez2020].
In this case, the charge of the polymer remains constant while the incorporation of salt ions only weakens the electrostatic repulsions between these charges.







\begin{figure}[!tb]
	\centering
	\includegraphics[width=0.5\linewidth]{Figures/graph-gel/Tpt-pH_MAA.png}
	\caption{(A) Plot of transition temperature $T_{PT}$ as a function of pH for MG-100 microgels (polymer chain length, $n_{ch}=100$ segments) having different MAA contents; $[salt]=0.01 M$;
		(B) Fraction of charged segments $x_{MAA^-}=\frac{N_{MAA^-}}{N_{seg}}$ as a function of pH for the same conditions of panel A (\emph{i.e.}, at the $T_{pt}$); $x_{MAA^-}$ is proportional to the total charge of the polymer; $N_{seg}$ is the same for all microgels.}
	\label{fig:Tpt_MAA}
\end{figure}



The results of figura \ref{fig:Tpt-pH} show that the transition temperature is controlled by the amount of charge inside the microgel.
Indeed, increasing MAA content has the same effect of displacing the VPTT to higher values, as seen in figura  \ref{fig:Tpt_MAA}A.
Once again this behavior results from a more charged polymeric structure.
To compare the state of charge of microgels with different MAA contents, we use the total fraction of charged monomers:
%
\begin{equation}
x_{MAA^-}=\frac{N_{MAA^-}}{N_{seg}}=f x_{MAA}
\end{equation}
%
\noindent where $N_{MAA^-}$ is the number of deprotonated MAA segments; all other quantities have been defined in sec. \ref{sec:theory};
$x_{MAA^-}$ is proportional to the total charge of the microgel network, and
because all microgels have the same total number of segments, the proportionally constant is the same for all MAA contents considered.
figura \ref{fig:Tpt_MAA}B shows that there is a clear correlation between the $T_{pt}$ and the total charge of the microgel (given by $x_{MAA^-}$) when changing pH or the MAA content of the polymer.

%%%%%%%%%%%%%%%%%%%%%%%%%%%%%%%%%%%%%%%%%%%%%%%%%%%%%%%%%%%%%%%%%%%%%
\subsection{Effect of Degree of Crosslinking}
%%%%%%%%%%%%%%%%%%%%%%%%%%%%%%%%%%%%%%%%%%%%%%%%%%%%%%%%%%%%%%%%%%%%%


\begin{figure*}[!tb]
	\centering
	\includegraphics[width=1\linewidth]{Figures/graph-gel/R-all_xlink.png}
	\caption{Plot of microgel size as a function of temperature, salt concentration and pH (panels A, B and C respectively). 
		Different curves correspond to microgels with $50$ (MG050), $100$ (MG100) and $200$ (MG200) segments per polymer chain, all having $35\%$ MAA.}
	\label{fig:R_xlink}
\end{figure*}









Next we analyze how the degree of crosslinking of the polymer network affects the behavior described in the previous sections. 
We have considered microgels with $50$, $100$ and $200$ segments per chain.
Overall, these particles have the same total number of segments.
Figure \ref{fig:R_xlink} shows the response of $35\%$ MAA microgels to changes in temperature (panel A), salt concentration (B), and pH (C).



Microgels with lower degree of crosslinking (higher number of segments per chain) display greater swelling.
This behavior of P(NIPAm-MAA) microgels has been experimentally confirmed\addcite[khan2013preparation].
Qualitatively, the response to salt concentration and pH is similar (panels B and C of figura \ref{fig:R_xlink}, respectively) for all chain lengths considered.
An interesting observation is that decreasing the crosslinking degree leads to a more sharp volume transition when the temperature increases (figura \ref{fig:R_xlink}A);
in addition, $T_{pt}$ increases.
Our results are consistent with the works of \addcite[citet:li1989study] and \addcite[citet:wu1997volume] that reported a change in the volume transition of NIPAm from continuous to discontinuous as the concentration of crosslinker in the synthesis decreases.



In figura \ref{fig:R_xlink} we also see that increasing chain length (decreasing the degree of crosslinking) displaces the $T_{pt}$ to higher temperatures.
This is consistent with the UV spectroscopy results of \addcite[citet:Lee2008] for P(NIPAm-AA) microgels.
This behavior occurs for all the range of conditions explored in this work %(see \cref*{si:fig:Tpt-pH_nch} in the \supp, for example).


The force constant of the elastic contribution to the free energy is inversely proportional to the chain length $n_{ch}$ (see eq. \ref{eq:free-energy}).
By lowering the degree of crosslinking, the more flexible microgel swells, which allows for a higher degree of charge on the polymer network.
In consequence, a higher temperature is required to induce collapse of the polymer network.
We have shown that $T_{pt}$ is strongly correlated to the degree of charge.


The presence of acidic units accentuates the dependence of $T_{pt}$ on chain length because it incorporates protonation equilibrium to the game, but this behavior is intrinsic to the balance between hydrophobic interactions and network elasticity.
Indeed, the transition temperature of pure PNIPAm microgels increases with chain length as well, though the effect is significantly weaker in the absence of MAA segments.



%%%%%%%%%%%%%%%%%%%%%%%%%%%%%%%%%%%%%%%%%%%%%%%%%%%%%%%%%%%%%%%%%%%%%
\subsection{Drug Absorption}
%%%%%%%%%%%%%%%%%%%%%%%%%%%%%%%%%%%%%%%%%%%%%%%%%%%%%%%%%%%%%%%%%%%%%




Multiresponsive microgels are regarded as excellent candidates for the development of functional drug delivery vehicles.
For example, it is known that the extracellular pH of tumor tissue is lower than that of healthy tissue \addcite[Gerweck1996], which makes pH-responsive microgels
ideal for the local administration of anticancer drugs\addcite[Dadsetan2013].
%The tissue-like environment inside microgels can provide stability to encapsulated peptides or proteins and help them retain activity upon release\cite{Malmsten2010}.
Moreover, to achieve intestinal drug delivery via the oral route,
pH-sensitive microgels based on MAA have been extensively studied in by Peppas \emph{et al.} as intelligent carriers that can operate using the different acidity levels along the digestive tract and prevent drug degradation in the stomach\addcite[TorresLugo2002,Carr2010,DuranLobato2014,Sharpe2018].







In this section, we evaluate the ability of P(NIPAm-MAA) microgels to incorporate two chemotherapeutic drugs.
In particular we investigate the best conditions for drug encapsulation under lab conditions.
We consider Doxorubicin (Doxo) and Daunorubicin (Dauno) that are two of the most important anthracyclines used in chemotherapy to treat a wide range of cancers.\addcite[Panis2012,Carvalho2009,aubel1984daunorubicin,come1999dual]
These therapeutics can be followed using fluorescence and adsorbance, which makes them attractive from a research standpoint.\addcite[Serpe2005,ThanHtun2009,PerezChavez2020] 
In addition, these drugs are positively charged under most conditions, which can facilitate their encapsulation within anionic polymer microgels.\addcite[Li2019]]
\addcite[citet:Serpe2005] have studied the thermally activated uptake and release of Doxo from layer-by-layer films of P(NIPAm-AA) microgels and poly(allylamine hydrochloride).
More recently, using time-lapse nuclear magnetic resonance, \addcite[citet:MartinezMoro2020] have described the interaction between Doxo and P(NIPAm-MAA) microgels under different conditions.  



\begin{figure}[!tb]
	\centering
	\includegraphics[width=0.55\linewidth]{Figures/graph-gel/drug_ads.png}
	\caption{Color maps showing the number of microgel-absorbed (A) Daunorubicin and (B) Doxorubicin molecules as a function of the solution pH and salt concentration.
		The solution drug concentration is $1mM$ and $T=25 ^\circ C$.
		the P(NIPAm-MAA) microgel has $n_{ch}=100$ chain length and $35\%$ MAA (MG100-35).}
	\label{fig:drug_ads}
\end{figure}




Figura \ref{fig:drug_ads} shows the number of Dauno (panel A) and Doxo (panel B) molecules inside the microgel as a function of the solution salt concentration and pH.
Clearly, the best conditions for encapsulation of these therapeutic drugs correspond to low salt concentration and pH $6-8$.
Lowering salt concentration favors absorption.
Both Dauno and Doxo have $+1$ net charge at acidic and neutral pH %(see \cref*{si:fig:drugs-Q} in the \supp).
As a result, their absorption has to compete with that of potassium ions to neutralize the negative charge of the polymer network.\addcite[PerezChavez2020].
Figura \ref{fig:f-cs} shows that in the absence of a dissolved drug the microgel charge decreases with lowering the salt concentration, which seems to conflict with the enhanced absorption seen in figura \ref{fig:drug_ads} under these conditions.
However, upon drug absorption the degree of charge of MAA segments increases significantly, particularly under low salt conditions %(see \cref*{si:fig:f-cs_doxo}).
Note also that this behavior is particularly associated with the relatively high drug concentration considered in this work ($1\,$mM).

The fraction of negatively charged MAA segments in the polymer increases with pH, which explains why Dauno/Doxo absorption increases as well (acidic conditions).
Under alkaline conditions, however, the net positive charge of these drugs decreases with increasing pH, which disfavors absorption.
As a result, Dauno/Doxo absorption is a nonmonotonic function of pH.



In our model, the isoelectric points of Dauno and Doxo are $8.4$ and $7.9$, respectively% (see \supp \cref*{si:fig:drugs-Q}).
Figura \ref{fig:drug_ads} shows that the absorption of both molecules can be significant around and above these pH values.
In other words, there is considerable absorption of negatively charged molecules inside the similarly charged polymer network.
Although, indeed these molecules are negatively charged in the solution phase, absorption occurs because the pH drops inside the microgel, allowing the drugs to regulate its electric charge and remain positively charge inside the microgel %(\cref*{si:fig:drugs-pH}, \supp).


The lower isoelectric point of Doxorubicin is due to the deprotonation of its substituent hydroxyl group (see $D4$ in figura \ref{fig:dauno-doxo} and tabla \ref{table:drugs}).
As a result of this additional negative charge under alkaline conditions, the pH range of significant adsorption is slightly wider for Daunorubicin.
\chapter{Impacto de la funcionalizaci\'on de la red polim\'erica en la adsorci\'on de prote\'inas en nanogeles polim\'ericos.}
\label{Chapter-esfericas}
%%%%%%%%%%%%%%%%%%%%%%%%%%%%%%%%%%%%%%%%%%%%%%%%%%
\section{Introducci\'on}
%%%%%%%%%%%%%%%%%%%%%%%%%%%%%%%%%%%%%%%%%%%%%%%%%%


%%%%% Functional vehicles 


El dise\~no de veh\'iculos funcionales para la encapsulaci\'on, transporte y liberaci\'on dirigida de agentes terap\'euticos es uno de los principales desaf\'ios  actuales de la bionanotecnolog\'ia \cite{ye2018review}.
A pesar de estar formulados para tratar espec\'ificamente ciertas enfermedades, muchos medicamentos no logran aprovechar su potencial debido a interacciones no deseadas con el entorno que rodea el objetivo \cite{ibraheem2014administration}.
Las nanopart\'iculas son exploradas como una soluci\'on para evitar estas limitaciones, sirviendo como portadores inteligentes que pueden proteger los medicamentos contra factores de cambios. %pensar
Se han investigado diversos nanotransportadores, incluyendo liposomas, micelas polim\'ericas y part\'iculas inorg\'anicas, por su potencial para transportar y liberar eficazmente cargas moleculares \cite{chamundeeswari2019nanocarriers, lopez2012organic}.
Estos nanotransportadores pueden aumentar el tiempo de circulaci\'on mientras confieren propiedades que ayudan a evadir los sistemas inmunitarios o digestivos \cite{gaucher2010polymeric}.
Adem\'as, al aumentar el tiempo de circulaci\'on de los medicamentos, estos nanotransportadores tambi\'en pueden ayudar a abordar otro problema y reducir la necesidad de dosificaciones frecuentes.


Los nanogeles, en particular, son nanopart\'iculas blandas con di\'ametros menores de $200\,nm$  que pueden absorber y liberar cantidades relativamente grandes de solvente en respuesta a cambios en su entorno.
Dependiendo de la composici\'on qu\'imica del pol\'imero entrecruzado que forma el esqueleto del nanogel, estas part\'iculas pueden expandirse o comprimirse de manera reversible como resultado de cambios en la temperatura \cite{agnihotri2021temperature}, el pH \cite{sharma2022modulating}, la concentraci\'on de sal \cite{saraydin2022calculations} y una variedad de otros est\'imulos externos \cite{jung2020responsive,plamper2017functional, yang2022co}.
Estas propiedades \'unicas de los nanogeles de pol\'imero se pueden aprovechar para dirigir la entrega de medicamentos a microambientes espec\'ificos, como los entornos \'acidos de los tumores \cite{zhang2020construction} o tejidos heridos o inflamados con temperaturas m\'as altas \cite{wu2010core}.
Adem\'as, se pueden incorporar grupos en su superficie polim\'erica, lo que permite que el nanogel se ligue selectivamente a receptores en el objetivo y aumente la especificidad de la entrega \cite{ahadian2020micro, mukherjee2019lipid, torchilin2007micellar, farokhzad2006targeted}.
La capacidad de los nanogeles para liberar medicamentos de manera controlada en entornos espec\'ificos tambi\'en puede ayudar a prevenir la  resistencia a los medicamentos \cite{mukherjee2019lipid}.


Una fracci\'on significativa y creciente de los nuevos medicamentos desarrollados para tratar diferentes enfermedades son prote\'inas \cite{mahmood2023recent}.
La estabilidad de estas prote\'inas es un problema pr\'actico, ya que son f\'acilmente desnaturalizadas por cambios en el pH o la temperatura \cite{frokjaer2005protein}.
Se ha demostrado que los hidrogeles y nanogeles de pol\'imero ayudan a prevenir la desnaturalizaci\'on de las prote\'inas y la p\'erdida de actividad bajo las condiciones mencionadas anteriormente \cite{macdougall2021charged, peppas2004hydrogels}.
La capacidad de mantener la conformaci\'on nativa de las prote\'inas, combinada con el comportamiento sensible a est\'imulos de los nanogeles, los convierte en candidatos adecuados para desarrollar veh\'iculos funcionales para la encapsulaci\'on, transporte y liberaci\'on dirigida de prote\'inas terap\'euticas.
Adem\'as, la red de pol\'imeros de los nanogeles se puede adaptar con diferentes grupos funcionales para controlar simult\'aneamente las interacciones espec\'ificas y no espec\'ificas con las prote\'inas, aumentando as\'i la eficacia de la encapsulaci\'on y entrega.

A pesar de sus muchas ventajas, el desarrollo de sistemas de entrega de medicamentos basados en nanogeles todav\'ia est\'a en sus primeras etapas, y hay muchas cuestiones que deben abordarse antes de que esta tecnolog\'ia pueda desarrollarse extendidamente.
En particular, dado que se requiere que funcionen en fluidos biol\'ogicos, comprender c\'omo se comportan e interact\'uan los nanogeles con las prote\'inas en t\'erminos de la composici\'on de su red de pol\'imeros y la de la soluci\'on de prote\'inas en la cual est\'en presentes es crucial para su aplicaci\'on exitosa.
En este cap\'itulo, utilizamos un enfoque te\'orico para estudiar c\'omo la identidad qu\'imica y la distribuci\'on espacial de los grupos funcionales en la red de pol\'imeros modifican la respuesta del nanogel y su interacci\'on con prote\'inas espec\'ificas, con el foco en la adsorci\'on/desorci\'on de prote\'inas impulsada electrost\'aticamente.
El objetivo es desarrollar una mejor comprensi\'on de los factores que influyen en el rendimiento de estos sistemas e identificar estrategias para mejorar su efectividad en el contexto de los nanoveh\'iculos para la entrega de medicamentos.



Se consideraran  nanogeles compuestos por redes de copol\'imeros que contienen tanto un mon\'omero hidrof\'ilico y  cargado-neutralmente como uno sensible al pH, y sus interacciones con prote\'inas globulares peque\~nas, como el citocromo c, la insulina y la mioglobina, que tienen diferentes puntos isoel\'ectricos.
La interacci\'on de estas prote\'inas con microgeles polim\'ericos i\'onicos o sensibles al pH ha sido estudiada previamente \cite{kabanov2009nanogels,smith2011tunable,sharma2022modulating,klinger2011dual}.
Por ejemplo, Smith y Lyon \cite{smith2011tunable} demostraron que la reducci\'on de la concentraci\'on de sal mejora la uni\'on del citocromo c a microgeles basados en \'acido acr\'ilico.
Estos sistemas tambi\'en se han abordado utilizando teor\'ia y simulaciones moleculares \cite{hagemann2018use,oberle2015competitive}.
Adem\'as, los microgeles polim\'ericos sensibles al pH se eval\'uan en la b\'usqueda de medios m\'as efectivos y menos invasivos para la administraci\'on de insulina \cite{lowman1999oral,wong2018microparticles}, lo que tiene una importancia inmensa en la investigaci\'on biom\'edica actual \cite{chaturvedi2013polymeric} .


Estudiamos nanogeles basados en copol\'imeros de alcohol vin\'ilico (VA) y \'acido metacr\'ilico (MAA; donante de protones) o alilamina (AH; aceptor de protones).
Debido a su biocompatibilidad, estos mon\'omeros se utilizan ampliamente en aplicaciones de administraci\'on de medicamentos \cite{asadi2020common,sarwar2020smart,lowman1999oral}.
Con el objetivo de obtener una comprensi\'on m\'as profunda de los factores que afectan el rendimiento de estos sistemas e identificar estrategias para ajustar sus interacciones con diferentes prote\'inas, desarrollamos y aplicamos una teor\'ia termodin\'amica estad\'istica que permite una descripci\'on a nivel molecular de todas las especies qu\'imicas.
Este m\'etodo incorpora una descripci\'on expl\'icita de las conformaciones de la red cuyo resultado es la capacidad de expandirse/comprimirse el\'asticamente, confinamiento entr\'opico de iones y solventes, qu\'imica de equilibrio \'acido-base, as\'i como interacciones electrost\'aticas y repulsiones est\'ericas.
Espec\'ificamente, investigamos el efecto de la distribuci\'on espacial de unidades sensibles al pH en toda la red de pol\'imeros en el swelling del nanogel y la adsorci\'on de prote\'inas.



%%%%%%%%%%%%%%%%%%%%%%%%%%%%%%%%%%%%%%%%%%%%%%%%%%
\section{Teor\'ia Molecular}
%%%%%%%%%%%%%%%%%%%%%%%%%%%%%%%%%%%%%%%%%%%%%%%%%%
Para el estudio de estos nanogeles se desarroll\'o un formalismo similar al descrito en el capitulo \ref{Chapter-film} para el estudio de films de hidrogeles polim\'ericos con respuesta a est\'imulo.
En este formalismo  buscamos minimizar una energ\'ia libre del sistema. Se incorpora una caracterizaci\'on molecular usando un modelo de grano grueso de las diversas especies qu\'imicas presentes.
El sistema en estudio es un nanogel aislado en equilibrio con una soluci\'on acuosa que tiene una composici\'on de bulk definida externamente.
Es decir, el pH, la concentraci\'on de sal y la concentraci\'on de prote\'ina son nuestras variables independientes.
La red polim\'erica que da estructura al nanogel contiene dos tipos de segmentos: una unidad sensible al pH, ya sea \'acida (MAA) o b\'asica (AH), y un segmento neutro (VA);
los segmentos que describen al entrecruzante en la red se describen como segmentos de carga neutral.

El potencial termodin\'amico que describe el sistema es:

\begin{align}
\begin{aligned}
\Omega_{NG}=& -TS_{mez} -TS_{conf,net} + F_{qca,net} + F_{qca,pro}\\
& + U_{elec} + U_{ste} + U_{VdW} - \sum_{\gamma}{\mu_\gamma N_\gamma} - \mu_{pro} N_{pro}
\end{aligned}
\label{eq:esf:semicano}
\end{align}


\noindent donde $S_{mez}$ es la entrop\'ia de traslaci\'on (y de mezcla) de las especies de la soluci\'on: mol\'eculas de agua (H$_2$O), iones de hidronio (H$_3$O$^+$), iones de hidr\'oxido (OH$^- $), cationes de sal, aniones de sal y prote\'ina.
Consideramos una sal monovalente, NaCl completamente disociada en iones de sodio (Na$^+$) y cloruro (Cl$^-$).
$S_{conf,net}$ representa la entrop\'ia conformacional que resulta de la flexibilidad de la red de pol\'imeros, que puede asumir muchas conformaciones diferentes.
$F_{qca,net}$ es la energ\'ia qu\'imica libre que describe el equilibrio entre las especies protonadas y desprotonadas de unidades funcionales (\'acidas/b\'asicas) en el pol\'imero.
De manera similar, $F_{qca,pro}$ describe la protonaci\'on de residuos titulables de la prote\'ina.
$U_{elec}$ y $U_{ste}$ representan, respectivamente, la energ\'ia asociada a las interacciones electrost\'aticas y las repulsiones est\'ericas.
$U_{VdW}$ contiene las interacciones de Van der Waals entre los distintos segmentos y el solvente.
Finalmente, la suma de $\gamma$ expresa el equilibrio qu\'imico entre el sistema y la soluci\'on bulk que representa un reservorio de pH, concentraciones y temperatura para las part\'iculas libres, donde $\mu_\gamma$ y $N_\gamma$ son el potencial qu\'imico y el n\'umero de mol\'eculas de especie $\gamma$, respectivamente;
el sub\'indice $\gamma$ recorre las especies qu\'imicas libres.
El siguiente t\'ermino tiene en cuenta el equilibrio qu\'imico entre sistema y soluci\'on para la prote\'ina.

Las expresiones expl\'icitas de cada uno de estos componentes, as\'i como la minimizaci\'on del potencial termodin\'amico, es descrita en la siguiente secci\'on.



\subsection{Formalismo te\'orico}\label{sec:esf:tm}

A continuaci\'on describiremos la forma expl\'icita de cada uno de las contribuciones a $\Omega_{NG}$, donde los segmentos protonables del pol\'imero ser\'an considerados como unidades de \'acido metacr\'ilico (MAA). Sin embargo, las expresiones an\'alogas se aplican para el caso de nanogeles que tienen segmentos b\'asicos.

En primera instancia tenemos la entrop\'ia de traslaci\'on y de mezcla de las especies m\'oviles, incluida la prote\'ina:


\begin{align}
	\begin{aligned}
		-\frac{S_{mez}}{k_B}= &\sum_{\gamma}\int_0^\infty{dr G(r)\rho_\gamma(r)\left(\ln \left(\rho_\gamma (r)v_w\right) -1 + \beta\mu^0_\gamma\right)} \\
		&+ \sum_{\theta}\int_0^\infty{dr G(r)\rho_{pro}(\theta,r)\left(\ln \left(\rho_{pro}(\theta,r)\right) -1 + \beta\mu^0_{pro} \right)}
	\end{aligned}
	\label{eq:esf:entropia1}
\end{align}



\noindent donde $\beta = \frac{1}{k_BT}$, $k_B$ es la constante de Boltzmann y $T$ la temperatura absoluta del sistema, $\rho_\gamma(r)$ y $\mu_\gamma$ son la densidad local, a una distancia $r$ del origen de coordenadas, y potencial qu\'imico de la especie $\gamma$ respectivamente.
El sub\'indice $\gamma$ toma en cuenta las mol\'eculas de agua y sus iones (hidronio e hidr\'oxido), y los iones disociados de la sal (Na$^+$, Cl$^-$). $G(r) =4\pi r^2$ es la constante de simetr\'ia de nuestro sistema. Para escribir la ecuaci\'on \ref{eq:esf:entropia1} hemos asumido que nuestro sistema tiene simetr\'ia esf\'erica donde el origen de coordenadas se encuentra en el centro de masa de la red polim\'erica.

En el segundo t\'ermino de la entrop\'ia de mezcla donde se considera el aporte de la prote\'ina,
$\rho_{pro}(\theta,r)$ es la densidad local de la prote\'ina en conformaci\'on  $\theta$.  Una conformaci\'on de la prote\'ina esta definida por la posici\'on relativa de sus unidades o por diferentes rotaciones espaciales de la misma estructura.
La densidad  local total de prote\'ina es: 


\begin{align}
	\left<\rho_{pro}(r)\right> = \sum_\theta{\rho_{pro}(\theta,r)}
\end{align}




$S_{conf,net}$ representa la entrop\'ia conformacional resultante de la flexibilidad de la red polim\'erica que forma al nanogel. Estas conformaciones son  denotadas por el set $\{\alpha\}$. 
\begin{equation}
	\frac{S_{conf,net}}{k_B} = - \sum_{\alpha}{P(\alpha)\ln P(\alpha)}
\end{equation}


\noindent En donde $P(\alpha)$ es la probabilidad que el nanogel se encuentre en la configuraci\'on $\alpha$.
Una conformaci\'on $\alpha$ viene especificada por la posici\'on de todos los segmentos de la red polim\'erica. 
La fracci\'on en volumen de estos segmentos puede expresarse como:

%%%%%% modificacion 1
\begin{align}
	\left< \phi^i(r)\right> = \frac{1}{G(r)} \sum_\alpha{P(\alpha)\phi^i_r(\alpha,r)} 
	\label{eq:esf:ensamble-gel}
 \end{align}

En donde el super\'indice $i$ indica el tipo de segmento ($i = MAA/VA/crosslink$), y la notaci\'on entre brackets, $\langle \rangle$,   hace referencia al promedio de ensamble sobre las conformaciones de la red polim\'erica. 
$\frac{\phi^i_r(\alpha,r)}{G(r)}$  nos proporciona la fracci\'on de volumen que ocupan los segmentos de tipo $i$ entre las esferas conc\'entricas de radio $r$ y $r + dr$, cuando la red est\'a en la configuraci\'on $\alpha$.

%%%%%%%%%%%%%%%%



El siguiente t\'ermino describe la energ\'ia qu\'imica libre originada por el equilibrio \'acido-base de los segmentos de MAA presentes en el nanogel.

\begin{align}
	\begin{aligned}
		\beta F_{qca,net} &= \int_0^\infty drG(r) \frac{\left<\phi^{MAA}(r)\right>}{v_{MAA}} \left[f(r)(\ln f(r)+ \beta\mu^0_{MAA^-})\right.\\
		&\left.+(1-f(r))(\ln (1-f(r))+\beta\mu^0_{MAAH})\right]    
	\end{aligned}
\end{align} 


\noindent donde $f(r)$ es el grado de carga de los segmentos MAA en la capa esf\'erica entre $r$ y $r + dr$.
$\mu^0_{MAA^-}$ y $\mu^0_{MAAH}$ son los potenciales qu\'imicos est\'andar de las especies desprotonadas y protonadas respectivamente. $v_{MAA}$ es el volumen molecular del segmento de MAA.



%%%%%%%%%%%%%%%%%%%
El equilibrio qu\'imico de las unidades proteicas titulables se considera en el siguiente t\'ermino del potencial termodin\'amico:

\begin{align}
	\begin{aligned}
		\beta F_{qca,pro} =\int_0^\infty dr &G(r) \sum_\tau \left<\rho_{pro,\tau}(r)\right> \left[g_\tau(r)(\ln g_\tau(r)+ \beta\mu^0_{\tau p})\right.\\
		&\qquad\left.+(1-g_\tau(r))(\ln (1-g_\tau(r))+\beta\mu^0_{\tau d})\right]
		\label{eq:esf:fca-pro}   
	\end{aligned}
\end{align} 

\noindent en donde $\left<\rho_{pro,\tau}(r)\right>$ representa la densidad local promedio del segmento titulable $\tau$ de la prote\'ina.

El cual es definido como:


\begin{align}
	\left<\rho_{pro,\tau}(r)\right> = \sum_\theta \int_0^\infty dr^\prime \frac{G(r^\prime)}{G(r)} \rho_{pro}(\theta,r^\prime)m_\tau(\theta,r^\prime,r)
	\label{eq:esf:segments-pro-vector}
\end{align}


\noindent en donde $m_\tau(\theta,r^\prime,r) dr$  nos da el n\'umero de segmetos $\tau$  de una prote\'ina en su conformaci\'on $\theta$ con su centro de masa en $r^\prime$, que ocupan el volumen entre las esferas conc\'entricas de radios $r$ y $r + dr$.



N\'otese que el sub\'indice  $\tau$ hace referencia a las unidades/residuos titulables de la prote\'ina, pero estas expresiones son v\'alidas para todos los segmentos de la prote\'ina:

\begin{align}
	\left<\rho_{pro,\lambda}(r)\right> = \sum_\theta \int_0^\infty dr^\prime \frac{G(r^\prime)}{G(r)} \rho_{pro}(\theta,r^\prime)m_\lambda(\theta,r^\prime,r)
	\label{eq:esf:segments-pro}
\end{align}



\noindent donde $\lambda$  describe un segmento arbitrario de la prote\'ina ($\{\tau\}\in\{\lambda\}$).

los sub\'indices $p$ y $d$  de la ecuaci\'on \ref{eq:esf:fca-pro} representan estados protonado y desprotonado respectivamente de un segmento $\tau$. 
De este modo $\mu^0_{\tau p}$ y $\mu^0_{\tau d}$  son los potenciales qu\'imicos est\'andar de estos estados respectivamente.

Utilizando el grado de asosiaci\'on local de protones a los segmentos $\tau$ , $g_\tau (r)$, podemos definir el grado de carga local del segmento $f_\tau (r)$ como:
 
\begin{enumerate}
	\item Para unidades \'acidas: $g_\tau(r) = 1-f_\tau(r)$ (los segmentos $\tau$ se cargan negativamente)
	\item Para unidades b\'asicas: $g_\tau(r) = f_\tau(r)$ (los segmentos $\tau$ se cargan positivamente)
\end{enumerate}

%%%%%%%%%%

La energ\'ia electrost\'atica se define:

\begin{align}
	\begin{aligned}
		\beta U_{elec}= \int_0^\infty drG(r)&\left[\left(\sum_{\gamma } {\rho_\gamma(r) q_\gamma + \sum_\tau{f_\tau(r) \left<\rho_{pro,\tau}(r)\right> q_\tau} +  f(r)\dfrac{\left<\phi_{MAA}(r)\right>}{v_{MAA}}q_{MAA}}\right)\beta\psi(r) \right. \\ &\left.-\frac{1}{2}\beta\epsilon(\nabla\psi(r))^2 \right]
	\end{aligned}
\end{align} 

\noindent donde $\psi(r)$ es el potencial electrost\'atico dependiente de la posici\'on, y $\epsilon$ la permitividad del medio, $q_\gamma$ es la carga de la especie m\'ovil $\gamma$, $q_\tau$ corresponde a la carga del segmento titulable de la prote\'ina y $q_{MAA}$ es la carga de un segmento de MAA.

En este contexto, podemos definir la densidad local de carga: 

\begin{align}
	\left<\rho_q(r)\right> = \sum_{\gamma } {\rho_\gamma(r) q_\gamma + \sum_\tau{f_\tau(r) \left<\rho_{pro,\tau}(r)\right> q_\tau} +  f(r)\dfrac{\left<\phi^{MAA}(r)\right>}{v_{MAA}}q_{MAA}}
	\label{eq:esf:rho-charge}
\end{align}  
%%%%%%%%%%%%%%%%
             
El siguiente t\'ermino en el potencial termodin\'amico se debe a la repulsi\'on est\'erica, el cual se incorpora a trav\'es de la siguiente restricci\'on f\'isica.

\begin{align}
	\begin{aligned}
		1=  {\left[\sum_{\gamma}\rho_\gamma(r) v_\gamma + \sum_\lambda{\left<\rho_{pro,\lambda}(r)\right>v_\lambda} + \sum_i{\left<\phi^i(r)\right>}\right]},~ \forall r
	\end{aligned}
	\label{eq:esf:constraint}
\end{align}


\noindent en donde $v_\lambda$  es el volumen molecular de cada segmento $\lambda$  que compone a la prote\'ina.


%%%%%%%%%%%%%%%

$U_{VdW}$ es la energ\'ia de interacci\'on de Van der Waals ($VdW$). En este sistema se ha asumido que todos los segmentos tienen un car\'acter hidrof\'ilico. Es decir, las interacciones de $VdW$ entre diferentes pares de segmentos y \'estas con mol\'eculas de agua son similares. Como resultado, la energ\'ia de interacci\'on neta $VdW$ representa una constante aditiva a la energ\'ia total del potencial y esta contribuci\'on puede ser ignorada. 
%Esto es posible por los segmentos considerados en la estructura del nanogel, como se mostr\'o en el capitulo anterior, en el modelo de dos fases, se consider\'o la interacci\'on entre los segmentos de NIPAm como un potencial aparte. Por lo que las interacciones de Van der Waals fueron tenidas en cuenta.



Para completar el gran potencial de la  ecuaci\'on  \ref{eq:esf:semicano}, se tiene en cuenta el equilibrio qu\'imico de las especies m\'oviles:
 

\begin{align}
	\begin{aligned}
		\mu_\gamma N_\gamma + \mu_{pro} N_{pro} =\int_0^\infty drG(r)&\left[\sum_{\gamma }{\rho_\gamma(r)\mu_\gamma}
		+ \mu_{pro} \left<\rho_{pro}(r)\right> \right. \\
		& \left. +\mu_{H^+}\sum_{\tau}{g_\tau\left<\rho_{pro,\tau}(r)\right> } +\mu_{H^+}(1-f(r))\dfrac{\left<\phi^{MAA}(r)\right>}{v_{MAA}}\right]
	\end{aligned}
\end{align}


Los primeros dos t\'erminos del lado derecho de la ecuaci\'on explican el equilibrio qu\'imico de las especies m\'oviles $\gamma$ y de las prote\'inas dentro de la soluci\'on.
Los dos \'ultimos t\'erminos consideran los iones de hidr\'ogeno ligados a segmentos protonados de la prote\'ina y a segmentos de  MAA de la red polim\'erica.

Finalmente la forma expl\'icita de nuestro gran potencial es expresado:

%%%%%%%%%%%%
\begin{align}
	\begin{aligned}
		\beta&\Omega_{NG}=\\&  \sum_{\gamma}\int_0^\infty{dr G(r)\rho_\gamma(r)\left(\ln \left(\rho_\gamma (r)v_w\right) -1 + \beta\mu^0_\gamma\right)} \\
		%
		& +\sum_\theta \int_0^\infty{dr G(r)\rho_{pro}(r)\left(\ln (\rho_{pro}(\theta,r)v_w)-1 + \beta\mu^0_{pro} \right)} \\
		%
		& + \sum_{\alpha}{P(\alpha)\ln P(\alpha)} \\
		%
		& +\int_0^\infty drG(r) \frac{\left<\phi^{MAA}(r)\right>}{v_{MAA}} \left[f(r)(\ln f(r)+ \beta\mu^0_{MAA^-})\right.\\
		&\qquad \qquad \qquad\qquad \qquad \quad \left.+(1-f(r))(\ln (1-f(r))+\beta\mu^0_{MAAH})\right] \\
		%
		& +\int_0^\infty drG(r)\sum_\tau \left<\rho_{pro,\tau}(r)\right> \left[g_\tau(r)(\ln g_\tau(r)+ \beta\mu^0_{\tau p})\right.\\
		&\qquad\qquad \qquad\qquad \qquad \qquad\left.+(1-g_\tau(r))(\ln (1-g_\tau(r))+\beta\mu^0_{\tau d})\right] \\
		%
		& +  \int_0^\infty drG(r)\left[\left(\sum_{\gamma } {\rho_\gamma(r) q_\gamma + \sum_\tau{f_\tau(r) \left<\rho_{pro,\tau}(r)\right> q_\tau} +  f(r)\dfrac{\left<\phi^{MAA}(r)\right>}{v_{MAA}}q_{MAA}}\right)\beta\psi(r) \right.\\  &\left. \hspace{6em}-\frac{1}{2}\beta\epsilon(\nabla\psi(r))^2 \right]\\
		%
		&+ \int_0^\infty \beta\pi(r) drG(r){\left(\sum_{\gamma}\rho_\gamma(r) v_\gamma + \sum_{\lambda}{\left<\rho_{pro,\lambda}(r)\right>}{v_\lambda} + \sum_i\left<\phi^i(r)\right> -1\right)}\\
		%
		& -\int_0^\infty drG(r)\left[\sum_{\gamma }{\rho_\gamma(r)\beta\mu_\gamma}
		+ \beta\mu_{pro} \left<\rho_{pro}(r)\right>
		+\beta\mu_{H^+}\sum_{\tau}{g_\tau(r)\left<\rho_{pro,\tau}(r)\right> } \right.\\
		& \left. \hspace{6em} +\beta\mu_{H^+}(1-f(r))\dfrac{\left<\phi^{MAA}(r)\right>}{v_{MAA}}\right]%\\
	\end{aligned}
	\label{eq:esf:potential-energy}
\end{align}


Para esta expresi\'on,  \ref{eq:esf:potential-energy}, se ha introducido la restricci\'on de la incompresibilidad del volumen (ecuaci\'on \ref{eq:esf:constraint} ),y se incorpora un multiplicador de Lagrange $\pi(r)$ para garantizar su cumplimiento , el cual representa la presi\'on osm\'otica del sistema. 

El siguiente paso es la b\'usqueda de las condiciones que minimizan el potencial termodin\'amico. Esto se logra al derivar respecto de las densidades locales $\rho_\gamma(r)$, el potencial electrost\'atico $\psi(r)$, el grado de carga, tanto de los segmentos provenientes de la prote\'ina; $f_\tau (r)$ como del pol\'imero, $f(r)$, adem\'as de la probabilidad de las diferentes conformaciones de la red polim\'erica $P(\alpha)$ y la densidad local de la prote\'ina $\rho_{pro}(\theta,r)$

%En sint\'esis podemos escribir $\Omega = \sum P(\alpha) \int{G(r) dV\omega}$,  siendo $\omega$ el funcional que contempla los funcionales que definen a nuestro gran potencial: 

%\begin{align}
%	\omega=\omega(\rho_\gamma(r), \rho_{pro}(r),\psi(r),f(r),P(\alpha))
%	\label{eq:esf:funcionales-omega}
%\end{align}

En particular la expresi\'on de optimizaci\'on para el grado de carga, $f(r)$ de los segmentos titulables de la red polim\'erica que compone al nanogel:

\begin{align}
	\frac{\partial \beta \Omega_{NG}}{\partial f(r)} = 0
\end{align}

Obteni\'endose:
\begin{align}
	\frac{f(r)}{1-f(r)}= \left(\frac{a_{H^+}}{k^0_{a,MAA}}\right)^{-1} e^{-\beta q_{MAA^-}\psi(r)}
	\label{eq:esf:f-degree}
\end{align}

\noindent En donde  $a_{H^+}=e^{\beta(\mu_{H^+} -\mu^0_{H^+})}$ es la actividad del $H^+$. 

En la expresi\'on anterior, ecuaci\'on \ref{eq:esf:f-degree}, $K^0_{a,MAA}$ es la constante termodin\'amica del equilibrio \'acido-base:

\begin{align}
	\begin{aligned}
	k_{a,MAA}^0=\exp\left(\beta\mu_{MAA}^0 - \beta \mu_{A^-}^0 - \beta \mu^0_{H^+} \right)
	\end{aligned}
	\label{eq:esf:dis-rxn}
\end{align}

Similarmente para los segmentos titulables $\tau$ de la prote\'ina:

\begin{align}
	\frac{f_\tau(r)}{1-f_\tau(r)}= \left(\frac{a_{H^+}}{k^0_{a,\tau}}\right)^{\mp 1} e^{-\beta q_\tau\psi(r)}
	\label{eq:esf:ftau-degree}
\end{align}

 El exponente $\mp \, 1$ hace la diferencia sobre segmentos \'acidos ($-$) o b\'asicos ($+$).

Con la constante termodin\'amica para el equilibrio \'acido/base de los segmentos $\tau$ es:

\begin{align}
	\begin{aligned}
		k_{a,\tau}^0=\exp\left(\beta\mu_{\tau p}^0 - \beta \mu_{\tau d}^0 - \beta \mu^0_{H^+} \right)
	\end{aligned}
	\label{eq:esf:distau-rxn}
\end{align}
%%%%%%%%%%%%%%%
%represent de protonable and deprotonable state of the segment $\iota
Para las especies libres, su densidad localizada se expresa como:


\begin{align}
	\rho_\gamma(r)v_w = a_\gamma \exp{\left(-\beta \psi(r)q_\gamma\right)} \exp{\left(-\beta\pi(r) v_w\right)}
		\label{eq:esf:rho-libres}
\end{align}


En el mismo sentido, para la prote\'ina $\rho(\theta,r)$,

\begin{align}
	\frac{\partial \beta \Omega_{NG}}{\partial \rho(\theta,r)} = 0
\end{align}

 obtenemos:
	

\begin{align}
	\begin{aligned}
		\rho_{pro}(\theta, r)v_w = & \tilde{a}_{pro} \prod_\tau \exp\left[ -\int_0^\infty dr^\prime  m_\tau(\theta,r,r^\prime) \ln f_\tau(r^\prime)\right] \\
		& \times \prod_\lambda \exp\left[ -\int_0^\infty dr^\prime  m_\lambda(\theta,r,r^\prime)\left( \beta\psi(r^\prime) q_\lambda + \beta \pi(r^\prime) v_\lambda \right)\right]
	\end{aligned}
	\label{eq:esf:rho-pro}
\end{align}
	
	\noindent en donde se ha redefinido la actividad de la prote\'ina como:
	
	\begin{align}
		\tilde{a}_{pro} = \exp\left(\beta\mu_{pro} - \beta\tilde{\mu}^0_{pro}\right)
	\end{align}

Con
\begin{align}
	\beta\tilde{\mu}^0_{pro} =  \beta \mu^0_{pro}  + \sum_{\tau,a} C_{n,\tau}\beta\mu^0_{\tau d} 
	+ \sum_{\tau,b} C_{n,\tau}\beta(\mu_{H^+} - \mu^0_{\tau p})
\end{align}


\noindent $\tau,a$ y  $\tau,b$ suman sobre segmentos \'acidos o b\'asicos respectivamente. Adem\'as se ha definido el n\'umero de composici\'on para un segmento $k$, $C_{n,k}$:

	\begin{align}
		\int_0^\infty dr^\prime  m_\lambda(\theta,r,r^\prime) = C_{n,\lambda}\quad \forall \, r
		\label{eq:esf:composition}
	\end{align}

La optimizaci\'on con respecto a la probabilidad de una configuraci\'on $\alpha$ de la red de pol\'imero
\begin{align}
	\frac{\partial \beta\Omega_{NG}}{\partial P(\alpha)} = 0
\end{align}

 resulta en:

\begin{align}
	\begin{aligned}
		P(\alpha)&=\frac{1}{Q}\exp\left[- \sum_i{\int_0^\infty{dr\beta\pi(r)\phi^i_r(\alpha,r)}}\right] \\
		& \times \exp \left[ -\int_0^\infty dr \beta \psi(r)\frac{\phi^{MAA}_r(\alpha,r)}{v_{MAA}} q_{MAA}  \right] \\
		& \times \exp\left[ -\int_0^\infty{ dr\ln(f(r))\frac{\phi^{MAA}_r(\alpha,r)}{v_{MAA}}}\right] \\
	\end{aligned}
	\label{eq:esf:proba-alfa}
\end{align}

\noindent Donde $Q$ es una constante que asegura que $\sum_\alpha P(\alpha) = 1$.


La variaci\'on de $\Omega_{NG}$ con respecto al potencial electrost\'atico da lugar a la ecuaci\'on de Poisson:

\begin{align}
	\epsilon\nabla^2\psi(r) = -\left<\rho_q(r)\right>
	%\label{si:eq:poisson}
\end{align}

Considerando la simetr\'ia de nuestro problema:

\begin{align}
	\epsilon ~ \frac{1}{r^2} \frac{\partial}{\partial r}\left(\frac{\partial \Psi(r)}{\partial r}\right) = -\left<\rho_q(r)\right>
	\label{eq:esf:poisson}
\end{align}

Otra restricci\'on f\'isica a tener en cuenta en este punto es la electro-neutralidad del sistema, que es:

\begin{align}
	\int_0^\infty{drG(r) \left<\rho_q(r)\right>} = 0
\end{align}

Esta restricci\'on se satisface imponiendo las condiciones de contorno adecuadas al resolver la ecuaci\'on \ref{eq:esf:poisson}. Estas condiciones de contorno son:
\begin{align}
	%\begin{aligned}
	&  \lim_{r\to\infty}\psi(r) = 0 \\
	&  \left.\frac{d\psi(r)}{dr}\right|_{r=0} = 0
	\label{eq:esf:contorno}
	% \end{aligned}
	\end{align}
	

Ahora todas las funciones que componen el potencial termodin\'amico $\Omega_{NG}$ se han expresado en t\'erminos del potencial electrost\'atico local $\psi(r)$, la presi\'on osm\'otica dependiente de la posici\'on $\pi(r)$ y algunas cantidades de entrada que incluyen las actividades de las especies libres.
Dada la concentraci\'on de sal, el pH y la concentraci\'on de prote\'inas en la soluci\'on bulk, todas estas actividades se pueden calcular imponiendo la incompresibilidad y la neutralidad de carga a dicha soluci\'on y utilizando la condici\'on de equilibrio de la auto-disoluci\'on del agua.
Entonces, las \'unicas inc\'ognitas restantes son $\psi(r)$ y $\pi(r)$ valor de $r$.
Estas funciones locales se calculan resolviendo num\'ericamente las ecuaciones \ref{eq:esf:constraint} y \ref{eq:esf:poisson}.
 

\subsection{Soluci\'on Bulk}\label{sec:esf:bulk}

La composici\'on qu\'imica de la soluci\'on bulk se encuentra en equilibrio termodin\'amico con el nanogel. Es decir los potenciales qu\'imicos de las especies con movilidad son iguales en cualquier punto del sistema. 
El calculo de estas actividades nos proveen  las condiciones de entrada o contorno para la soluci\'on de nuestro problema.
En esta secci\'on, expresamos esas actividades en t\'erminos de la composici\'on qu\'imica de la soluci\'on bulk.

La soluci\'on bulk  puede considerarse como el l\'imite $r \rightarrow \infty$:
\begin{align}
	\begin{aligned}
		& i)\rho^b_\gamma =\rho_\gamma (r \rightarrow \infty) \\
		& ii) \pi^b = \pi(r \rightarrow \infty) \\
		& iii) f_\tau^b = f_\tau(r \rightarrow \infty)
	\end{aligned}
\end{align}

Adem\'as, las condiciones de contorno expresadas en la ecuaci\'on \ref{eq:esf:contorno} implican que:
\begin{align}
	\psi^b = \psi(r \rightarrow \infty) = 0
\end{align}

En este contexto, para las especies libres (excluyendo la prote\'ina):
\begin{align}
	\rho_\gamma^b v_w = a_\gamma e^{-\beta\pi^bv_w}
	\label{eq:esf:free-bulk}
\end{align}

El grado de carga $f_\tau$ de los segmentos de la  prote\'ina  puede escribirse como:

\begin{align}
	\frac{f^b}{1-f^b} = \left(\frac{a_{H^+}}{K^0_{a,\tau}}\right)^{\mp 1}
\end{align}

la densidad de la prote\'ina $\rho_{pro}^b(\theta)$ es:

\begin{align}
	\begin{aligned}
		\rho^b_{pro}(\theta)v_w = &\tilde{a}_{pro} \prod_\tau\exp\left[-C_{n,\tau} \ln f^b_\tau\right] \\
		&\prod_\lambda \exp \left[-C_{n,\lambda} (\beta\pi^b v_\lambda ) \right]
	\end{aligned}
	\label{eq:esf:bulk-protein}
\end{align}

donde $C_{n,\lambda}$ es el n\'umero de composici\'on para el segmento $\lambda$, definido en la ecuaci\'on  \ref{eq:esf:composition}.

Para la soluci\'on bulk, la restricci\'on de incompresibilidad est\'a dada por:

\begin{align}
	\begin{aligned}
		1= {\sum_{\gamma}\rho^b_\gamma v_\gamma + \sum_\lambda{\left<\rho^b_{pro,\lambda}\right>v_\lambda} }
	\end{aligned}
	\label{eq:esf:bulk-constraint}
\end{align}


y la electro-neutralidad del sistema:

\begin{align}
	\left<\rho^b_q\right> = \sum_{\gamma } {\rho^b_\gamma q_\gamma + \sum_\tau{f^b_\tau \left<\rho_{pro,\tau}\right> q_\tau} =0}
	\label{eq:esf:rhobulk-charge}
\end{align}  

Observamos que las expresiones que definen al bulk de la soluci\'on quedan definidas por la presi\'on osm\'otica del bulk $\pi^b$. Dicho potencial puede obtenerse con la resoluci\'on de las restricciones dadas por las  ecuaciones  \ref{eq:esf:bulk-constraint} y \ref{eq:esf:rhobulk-charge}.

%%%%%%%%%%%%
%%% Aqui estba la resolución numérica
%%%%%%%%%%%%%


%%%%%%%%%%%%%%%%%%%%%%%%%%%%%%%%%%%%%%%%%%%%%%%%%%
\subsection{Modelo Molecular: Prote\'inas}\label{subsec:protein}
%%%%%%%%%%%%%%%%%%%%%%%%%%%%%%%%%%%%%%%%%%%%%%%%%%



Consideramos la interacci\'on del nanogeles de P(MAA-VA) o P(AH-VA) con tres prote\'inas diferentes: citocromo c, insulina y mioglobina.
Para describir estas mol\'eculas, utilizamos un modelo de grano grueso donde cada residuo de amino\'acido se representa mediante una \'unica part\'icula centrada en la posici\'on del carbono $\alpha$.
La secuencia y posici\'on de todos los carbonos $\alpha$ se toman de la estructura cristalogr\'afica obtenida de la base de datos de prote\'inas \cite{berman2000protein}: 2B4Z para el citocromo c \cite{mirkin2008high}, IZNI para la insulina \cite{bentley1976structure} y 3RGK para la mioglobina \cite{hubbard1990x}.

A cada part\'icula de grano grueso en este modelo se le asigna un volumen y un pKa (si la unidad es titulable) seg\'un el amino\'acido que representan; esto se resume en la Tabla \ref{table:Coarse-grain}.
Estos pKa se toman de datos experimentales y representan valores promedio sobre un gran n\'umero de prote\'inas \cite{grimsley2009summary}.
En la mayor\'ia de los casos, el pKa de un residuo no se desv\'ia significativamente del valor promedio.
Sin embargo, en casos espec\'ificos, algunos residuos muestran un pKa diferente.


\begin{table}[htb!]
	\centering
	\small
	\begin{tabular}{|lcc|lcc|}
		\hline
		grupo& $v$ ($\text{nm}^3$)& pka&  grupo& $v$ ($\text{nm}^3$)& pKa\\
		\hline
		Ala & 0.067 & & Met& 0.124& \\
		Arg & 0.148 & 12.5$ (+)$& Phe& 0.135& \\
		Asn & 0.096 & & Pro& 0.09& \\
		Asp & 0.091 & 3.5 $(-)$& Ser& 0.073& \\
		Cys & 0.086 & 6.8 $(-)$& Thr& 0.093& \\
		Cys$^\star$ & 0.086 & & Trp& 0.163& \\
		Gln & 0.114 & & Val& 0.105& \\
		Glu & 0.109 & 4.2$ (-)$& N-t& ${-}$& 7.7$ (+)$\\
		GluA4$^\star$ & 0.109 & 2.62 $(-)$& C-t& ${-}$& 3.3 $(-)$\\
		GluB13$^\star$ & 0.109 & 2.20$ (-)$& Hem1& 0.172& 3.8 $(-)$\\ 
		GluB21$^\star$ & 0.109 & 3.71 $(-)$& Hem2& 0.138& \\
		Gly & 0.048 & & AH& 0.068& 9.5$(+)$\\
		His & 0.118 & 6.6 $(+)$& MAA& 0.085& 4.65$(+)$\\ 
		His18* & 0.118 & 2.4 $(+)$& VA& 0.085& \\
		His26* & 0.118 & 2.9 $(+)$& H$_2$O& 0.033& \\
		His33*& 0.118& 6.35 $(+)$& OH$^-$& 0.033&\\
		Ile& 0.124& & H$_3$O$^+$& 0.033&\\
		Leu& 0.124& & Na$^+$& 0.043&\\
		Lys& 0.135& 10.5$ (+)$& Cl$^-$& 0.047&\\
		\hline
	\end{tabular}
	\caption{Par\'ametros de las part\'iculas de  grano grueso(residuos de amino\'acidos, iones peque\~nos, mol\'eculas de solvente y segmentos polim\'ericos) considerados en el modelo molecular que describe a nuestro sistema.  se diferencias grupos en los cuales experimentalmente, se ha observado que estos residuos tienen un pKa que difiere significativamente del valor promedio \cite{grimsley2009summary}:  $Cys^\star$ representa un residuo de ciste\'ina en un puente di-sulfuro. Las histidinas con  n\'umero y  $^\ast$ se encuentran en la citocromo c, donde el n\'umero indica el orden en la secuencia en la prote\'ina. De manera similar, las glutaminas indicadas de la misma manera y con diferentes pKa ocurren en la secuencia de la insulina \cite{grimsley2009summary}. N-t y C-t denotan los grupos terminales de la secuencia, que a\~naden un pKa adicional a los primeros y \'ultimos segmentos, respectivamente. El complejo hemo, presente tanto en la citocromo c como en la mioglobina \cite{mirkin2008high,hubbard1990x}, se describe utilizando las unidades de grano grueso Hem1 y Hem2. El complejo contiene dos de cada una de estas unidades.}
	\label{table:Coarse-grain} 
\end{table}



Utilizando este modelo molecular, la figura \ref{fig:esf:protein-charge} muestra la carga (n\'umero) de las tres prote\'inas en soluci\'on diluida en funci\'on del pH.
El punto isoel\'ectrico (pI) es el pH en el cual la carga neta de una prote\'ina es cero.
A partir del gr\'afico, obtenemos los valores 9.65 (9.6 \cite{hristova2019isoelectric}, 5.5 (5.3 \cite{guckeisen2019isoelectric}), 7.15 (7.2 \cite{batys2020myoglobin}) para el pI del citocromo c, insulina y mioglobina respectivamente;
los valores entre par\'entesis son los pI de las prote\'inas reportados experimentalmente.


 \begin{figure}[!htb]
     \centering
     \includegraphics[width=0.65\textwidth]{Figures/graphs-gel2/protein-model.pdf}
     \caption{Izquierda: N\'umero de carga de las prote\'inas en una soluci\'on diluida en funci\'on del pH (curvas s\'olidas);
     	los c\'irculos llenos marcan el punto isoel\'ectrico,
     	donde la carga neta de la prote\'ina es cero.
     	La representaci\'on de grano grueso de las prote\'inas se ilustra a la derecha, donde los residuos de amino\'acidos se representan mediante una esfera \'unica (rojo: residuo \'acido; azul: residuo b\'asico; gris: residuos de carga neutral).}
     \label{fig:esf:protein-charge}
 \end{figure}



%%%%%%%%%%%%%%%%%%%%%%%%%%%%%%%%%%%%%%%%%%%%%%%%%%
\subsection{Modelo Molecular: Red polim\'erica}
%%%%%%%%%%%%%%%%%%%%%%%%%%%%%%%%%%%%%%%%%%%%%%%%%%

Adem\'as del modelo de prote\'ina presentado en la secci\'on anterior, necesitamos especificar un modelo molecular para describir la red que compone a nuestro nanogel. Este modelo debe proporcionar un conjunto representativo de configuraciones moleculares de la red polim\'erica. Una conformaci\'on particular de la red se da por la posici\'on espacial de todos sus segmentos.
La red del nanogel est\'a compuesta por cadenas polim\'ericas entrecruzadas de 25 segmentos de longitud. En total, esta red contiene 10054 segmentos. Cada segmento es una representaci\'on simplificada de una unidad neutra (VA), un mon\'omero \'acido/b\'asico (MAA/AH) o un segmento entrecruzante. La tabla \ref{table:Coarse-grain} incluye el volumen y el pKa (si la unidad es titulable) utilizados para describir estas unidades.

La red del nanogel posee una topolog\'ia tipo diamante, donde los segmentos entrecruzantes se colocan en la posici\'on original de los \'atomos de carbono. Los entrecruzantes se conectan a 4 cadenas polim\'ericas. La construcci\'on de esta red se realiz\'o en primera instancia por la traslaci\'on tridimensional de la celda unidad en donde todas las cadenas polim\'ericas se encuentran alargadas, como segundo paso se realiz\'o un corte esf\'erico de radio $R_{cut}$ medido desde el centro de masa de la estructura. El valor de $R_{cut}$ se hace de tal manera de obtener aproximadamente 10000 segmentos en total. 

Originalmente, todas las cadenas polim\'ericas se  conectan a dos entrecruzantes, pero como resultado de este procedimiento, el corte esf\'erico, algunas cadenas quedan lindantes en la superficie de la red, es decir  conectadas a un solo entrecruzante. La mayor\'ia de estas cadenas \emph{colgantes} superficiales son m\'as cortas que 25 segmentos. En conjunto, estas cadenas contienen el 22\% del n\'umero total de segmentos. Para generar las diferentes conformaciones moleculares de la red del pol\'imero, se ha realizado simulaciones de din\'amica molecular usando GROMACS 5.1.2 \cite{lindahl2001gromacs}.

 \begin{figure}[!htb]
     \centering
     \includegraphics[width=0.99\textwidth]{Figures/graphs-gel2/ideal-charge-model.pdf}
     \caption{A: Los nanogeles estan formados por cadenas de copol\'imeros entrecruzadas con un segmento de carga neutral (VA: alcohol vin\'ilico) y una unidad funcional (ya sea MAA: \'acido metacr\'ilico o AH: alilamina).
     	Este esquema ilustra las tres distribuciones de los co-mon\'omeros consideradas; de izquierda a derecha: RF: una distribuci\'on aleatoria de grupos funcionales en toda la red; CF: las unidades funcionales ocupan el centro/n\'ucleo de la red; SF: solo las cadenas colgantes libres en la superficie de la red est\'an funcionalizadas con unidades sensibles al pH.
     	B: Gr\'afico del grado de carga ideal dependiente del pH de la unidad funcional aislada en soluci\'on diluida.}
     \label{fig:esf:gel-topologies}
 \end{figure}


 
 
 Consideramos diferentes nanogeles sensibles al pH que contienen grupos \'acidos (MAA) o b\'asicos (AH), y evaluamos tres topolog\'ias diferentes para la distribuci\'on espacial de estos segmentos funcionales, que se esquematizan en la Figura \ref{fig:esf:gel-topologies}:
 (i) una estructura \emph{aleatoriamente funcionalizada} ( Random Functionalization:  RF) donde los segmentos sensibles al pH se distribuyen aleatoriamente en toda la red,
 (ii) una estructura \emph{funcionalizada en el n\'ucleo} (Core Functionalization: CF), donde las unidades sensibles al pH ocupan el centro del nanogel, y
 (iii) una estructura \emph{funcionalizada en la superficie} (Surface Functionalization: SF) en la cual solo las cadenas colgantes en la superficie de la red son ionizables.
  







%%%%%%%%%%%%%%%%%%%%%%%%%%%%%%%%%%%%%%%%%%%%%%%%%%
\section{Resultados y discusi\'on}
%%%%%%%%%%%%%%%%%%%%%%%%%%%%%%%%%%%%%%%%%%%%%%%%%%






%%%%%%%%%%%%%%%%%%%%%%%%%%%%%%%%%%%%%%%%%%%%%%%%%%
\subsection{Caraterizaci\'on del nanogel}
%%%%%%%%%%%%%%%%%%%%%%%%%%%%%%%%%%%%%%%%%%%%%%%%%%

En esta primera instancia, se examinar\'a el comportamiento (la respuesta) de los nanogeles en funci\'on del pH en ausencia de prote\'inas.

Para cuantificar el tama\~no de un nanogel, utilizaremos el radio medio de la part\'icula, $R$, que se puede calcular utilizando la siguiente f\'ormula:
\begin{align}
	R = \frac{4}{3}\frac{\int_0^\infty{dr\,G(r)\,r \left<\phi(r)\right>}}{\int_0^\infty{dr\,G(r)\left<\phi(r)\right>}}
\end{align}
\noindent donde $r$ es la distancia desde el centro de masa de la red polim\'erica (como se ha mencionado en la secci\'on \ref{sec:esf:tm} se asume simetr\'ia radial);
$\left<\phi(r)\right>$ es la fracci\'on de volumen local de la red polim\'erica;
los corchetes angulares indican el promedio de ensamble sobre las diferentes conformaciones de la red (ver ecuaci\'on \ref{eq:esf:ensamble-gel});
$G(r)=4\pi r^2$ es el \'area de la superficie de una esfera de radio $r$.

\begin{figure}[!htb]
     \centering
     \includegraphics[width=0.65\textwidth]{Figures/graphs-gel2/rr-nano-pH.pdf}
     \caption{Radio promedio, R, en funci\'on del pH para nanogeles de copol\'imero MAA-VA (panel A) y AH-VA (panel B).
     	Se consideran tres estructuras diferentes en cada caso donde las unidades funcionales (MAA/AH) se distribuyen aleatoriamente a lo largo de la red polim\'erica (RF), ocupan el centro de la red (CF), o modifican las cadenas colgantes dentro del pol\'imero, interfaz de soluci\'on (SF).
     	En todos los casos, el $22\%$ de los segmentos de estas redes son sensibles al pH; La concentraci\'on de NaCl es $10^{-3}M$.}
     \label{fig:esf:gel-charge-MAA-AH}
\end{figure}


La Figura \ref{fig:esf:gel-charge-MAA-AH} muestra la relaci\'on entre el radio promedio ($R$) y el pH para las tres estructuras diferentes: RF, CF y SF. En el panel A, se describe un nanogel con segmentos ionizables de MAA, mientras que en el panel B se presenta un nanogel basado en AH. En ambos casos, la concentraci\'on de sal es de 1 mM y la fracci\'on de mon\'omero funcional (MAA o AH) es del $22\%$. Los nanogeles basados en MAA, funcionalizados al azar (RF) y en el n\'ucleo (CF), se hinchan a medida que aumenta el pH (panel A). Esto se debe a que los segmentos MAA se desprotonan y adquieren carga el\'ectrica a medida que el pH aumenta (ver Figura \ref{fig:esf:gel-topologies}B), lo que resulta en repulsiones electrost\'asticas dentro de la red. Para reducir estas interacciones repulsivas, la distancia entre las unidades cargadas de MAA debe aumentar, lo que provoca un aumento en el tama\~no de la red para separar estos segmentos cargados. En resumen, la expansi\'on neta de la red ocurre debido al aumento de la distancia espacial entre las unidades cargadas de MAA, como resultado de la necesidad de disminuir la repulsi\'on entre ellas.



Por otro lado, la red funcionalizada en su superficie con MAA muestra un comportamiento de expansi\'on completamente diferente, como se puede observar en la Figura \ref{fig:esf:gel-charge-MAA-AH}A. Este nanogel se deshincha a medida que las unidades titulables se cargan al aumentar el pH. Para explicar este comportamiento contrario a lo esperado, hemos examinado la distribuci\'on local de segmentos dentro de estas estructuras en diferentes condiciones. Hemos utilizado la distribuci\'on radial de los mon\'omeros funcionales para los nanogeles MAA. Esta cantidad se define como:



%
\begin{align}
    \lambda_{MAA}(r)= 4\pi r^2\left<\phi^{MAA}(r)\right>
\end{align}
%
\noindent en donde $\left<\phi_{MAA}(r)\right>$ da la fracci\'on de volumen local de los segmentos de \'acido metacr\'ilico (ver ecuaci\'on \ref{eq:esf:ensamble-gel})
Hay que tener en cuenta que $\lambda_{MAA}(r) dr$ da el n\'umero de segmentos MAA en la capa esf\'erica entre $r$ y $r+dr$ medido desde el centro del nanogel.
Adem\'as, la integral $\int_0^\infty \lambda_{MAA}(r) dr$ da el n\'umero total de mon\'omeros MAA en la red.


\begin{figure}[!htb]
     \centering
     \includegraphics[width=0.30\textwidth]{Figures/graphs-gel2/dist-MAA.pdf}
     \caption{Distribuci\'on radial de segmentos MAA, $\lambda_{MAA}(r)$, a pH 3 y 7, y $10^{-3}M$ NaCl; cada panel corresponde a un nanogel de  MAA-VA diferente que tienen una funcionalizaci\'on de red particular y 22\% MAA.
     	Estos grupos funcionales est\'an completamente protonados (sin carga) a pH 3 y completamente disociados (cargados) a pH 7.}
     \label{fig:esf:MAA-vs-r-distribution}
 \end{figure}
 %\FloatBarrier

La Figura \ref{fig:esf:MAA-vs-r-distribution} muestra la distribuci\'on radial de los segmentos de MAA para las diferentes redes consideradas. En cada caso, se incluyen resultados para una soluci\'on de pH 3, donde los segmentos MAA tienen carga neutra, y pH 7, donde est\'an completamente cargados (ver Figura \ref{fig:esf:gel-topologies}B). Para una funcionalizaci\'on de tipo aleatoria (Panel A), la distribuci\'on de los segmentos MAA se desplaza hacia la interfaz de soluci\'on del nanogel a medida que la red se carga el\'ectricamente al aumentar el pH. Este desplazamiento ocurre para reducir las repulsiones electrost\'aticas entre los segmentos MAA cargados.

Como resultado, toda la distribuci\'on del pol\'imero tambi\'en se extiende, incluidas las unidades VA de carga neutra (ver Figura \ref{fig:esf:allr-distribution}). El mismo comportamiento tiene lugar para una funcionalizaci\'on central (Panel B), aunque por dise\~no, los segmentos MAA en esta red, ya sea que est\'en cargados o no, es m\'as probable que ocurran a distancias m\'as cortas del centro del nanogel en comparaci\'on con las otras estructuras. El desplazamiento de segmentos a valores m\'as altos de $r$ observado en los paneles A y B de la Figura \ref{fig:esf:MAA-vs-r-distribution} explica el aumento del tama\~no promedio del nanogel con pH observado en la Figura \ref{fig:esf:gel-charge-MAA-AH}A para las estructuras RF y CF.

Por otro lado, la Figura \ref{fig:esf:MAA-vs-r-distribution}C muestra que la distribuci\'on superficial de segmentos de MAA se desplaza hacia el interior cuando la red se carga con el aumento de pH. Para reducir las repulsiones dentro de la red, las cadenas libres (de la superficie) de PMAA, que se asientan en la superficie del nanogel a un pH bajo, tambi\'en intentan ocupar el volumen dentro de la red cuando est\'an cargadas. Este desplazamiento hacia el interior de la distribuci\'on de pol\'imero (Figura \ref{fig:esf:allr-distribution}C) explica el comportamiento de compresi\'on del nanogel tipo SF con el aumento del pH observado en la Figura \ref{fig:esf:gel-charge-MAA-AH}A. N\'otese, sin embargo, que a pesar de este desplazamiento parcial hacia el interior de la red, la posici\'on m\'as probable de los segmentos MAA siempre es la interfaz pol\'imero-soluci\'on para soluciones de pH alto y bajo. Al empujar toda la estructura hacia el interior del nanogel, los segmentos de MAA quedan igualmente expuestos a la soluci\'on



El comportamiento de los nanogeles compuestos por AH es an\'alogo al de las redes basadas en MAA, pero en respuesta al cambio de pH en la direcci\'on opuesta.
Los grupos AH se protonan y se cargan positivamente con la disminuci\'on del pH (ver figura \ref{fig:esf:gel-topologies}B).
Para nanogeles de AH funcionalizados aleatoriamente y en su n\'ucleo, este aumento en la carga el\'ectrica con la disminuci\'on del pH provoca un desplazamiento hacia afuera de la distribuci\'on del segmento. Del mismo modo que se observ\'o para los nanogeles compuestos por MAA% (ver \cref*{fig:AHseg_si}A y B)
, lo que explica la expansi\'on observada en la figura \ref{fig:esf:gel-charge-MAA-AH}B;
mientras tanto, para la estructura tipo SF, el deswelling con la disminuci\'on del pH (figura \ref{fig:esf:gel-charge-MAA-AH}B) es consistente con un desplazamiento hacia adentro del pol\'imero %(ver \cref*{fig:AHseg_si}C) .


La figura \ref{fig:esf:allr-distribution} muestra la distribuci\'on de todos los segmentos que componen la red polim\'erica de los diferentes nanogeles. 
Al igual que en la figura \ref{fig:esf:MAA-vs-r-distribution} presentamos la distribuci\'on para dos diferentes valores de pH. Que corresponden a los estados protonado (pH 3) y desprotonado (pH 7) de los segmentos de MAA.  En los paneles A y B se observa un mayor n\'umero de segmentos a valores m\'as altos de $r$ al cargarse el nanogel. Observ\'andose el swelling reportado en la figura \ref{fig:esf:gel-charge-MAA-AH}A para lo nanogeles tipo RF y CF.
En cambio, el desplazamiento hacia dentro de los segmentos de la red polim\'erica al transicionar de un estado cargado a uno descargado en la distribuci\'on superficial (SF) (figura \ref{fig:esf:allr-distribution}C) explica la disminuci\'on del tama\~no del nanogel observada en la figura \ref{fig:esf:gel-charge-MAA-AH}A.

El comportamiento de los nanogeles compuestos por AH es an\'alogo al de las redes basadas en MAA, pero en respuesta al cambio de pH en la direcci\'on opuesta. Los grupos AH se protonan y se cargan positivamente con la disminuci\'on del pH (ver Figura \ref{fig:esf:gel-topologies}B). Para nanogeles de AH funcionalizados aleatoriamente y en su n\'ucleo, este aumento en la carga el\'ectrica con la disminuci\'on del pH provoca un desplazamiento hacia afuera de la distribuci\'on del segmento, similar a lo observado para los nanogeles compuestos por MAA. Esto explica la explicaci\'on del nanogel  observada en la Figura \ref{fig:esf:gel-charge-MAA-AH}B. Por otro lado, para la estructura tipo SF, la disminuci\'on del tama\~no del nanogel  con la disminuci\'on del pH que se observa en la Figura \ref{fig:esf:gel-charge-MAA-AH}B es consistente con un desplazamiento de los segmentos que componen la red hacia el interior del nanogel.% (ver Figura \ref{fig:AHseg_si}C).

En la figura \ref{fig:esf:allr-distribution} se muestra la distribuci\'on de todos los segmentos que componen la red polim\'erica de los diferentes nanogeles (en este caso basados en MAA). Al igual que en la Figura \ref{fig:esf:MAA-vs-r-distribution}, presentamos la distribuci\'on para dos valores diferentes de pH, correspondientes a los estados protonado (pH 3) y desprotonado (pH 7) de los segmentos de MAA. En los paneles A y B se observa un aumento en la cantidad de segmentos a valores m\'as altos de $r$ a medida que el nanogel se carga el\'ectricamente. Este comportamiento es consistente con el aumento del radio del nanogel observado en la figura \ref{fig:esf:gel-charge-MAA-AH} para las estructuras RF y CF. Por otro lado, el desplazamiento hacia el interior de los segmentos de la red polim\'erica al transicionar de un estado cargado a uno descargado en la distribuci\'on superficial (SF) explica la disminuci\'on del tamañ\~no del nanogel observada en la figura \ref{fig:esf:gel-charge-MAA-AH}B.

\begin{figure}[!htb]
	\centering
	\includegraphics[width=0.30\textwidth]{Figures/graphs-gel2/allseg_SI.pdf}
	\caption{Distribuci\'on radial de todos los segmentos que componen la estructura del nanogel a pH 3 y 7, y $10^{-3}M$ NaCl; cada panel corresponde a una funcionalizaci\'on  diferente de la red polim\'erica.}
	\label{fig:esf:allr-distribution}
\end{figure}




%%%%%%%%%%%%%%%%%%%%%%%%%%%%%%%%%%%%%%%%%%%%%%%%%%
\subsection{Adsorci\'on de prote\'inas en nanogeles basados en MAA}\label{sec:MAA-NGs}
%%%%%%%%%%%%%%%%%%%%%%%%%%%%%%%%%%%%%%%%%%%%%%%%%%



%%%%%%%%%%%%%%%%%%%%%%%%%%%
%%%%% Define Gamma and N(r)
%%%%%%%%%%%%%%%%%%%%%%%%%%%



En la secci\'on anterior, se evalu\'o el impacto de la funcionalizaci\'on de la red y la composici\'on qu\'imica en la respuesta del nanogel a las variaciones de pH en ausencia de prote\'inas. La reorganizaci\'on de los segmentos de la red polim\'erica es resultado de los cambios en el pH, con una dependencia en la elecci\'on de dise\~no, es decir, la distribuci\'on de unidades funcionales dentro de la red.

En esta parte, se mostrar\'a el impacto de esta reorganizaci\'on de la red polimé\'erica en el nivel de adsorci\'on de prote\'inas en diferentes nanogeles, as\'i como la distribuci\'on espacial de las prote\'inas adsorbidas. En particular, se presentar\'a el an\'alisis de la adsorci\'on del citocromo c y mioglobina en las diferentes estructuras de nanogeles basados en MAA. Adem\'as, se realizar\'an estudios de adsorci\'on de insulina, pero en este caso con nanogeles que contienen AH como segmento protonable.% Los resultados de la insulina con MAA se omiten debido a su bajo punto isoel\'ectrico, ya que esta prote\'ina no se adsorbe en nanogeles basados en $MAA$.

Para estos estudios, se considerar\'a un nanogel polim\'erico  con centro de masa centrado en $r=0$ en contacto con una soluci\'on acuosa de prote\'ina con concentraci\'on definida. El n\'umero de prote\'inas adsorbidas dentro de la capa esf\'erica entre $r$ y $r+dr$ se define como la cantidad en exceso. 


\begin{align}
     \langle N(r)\rangle dr = 4\pi r^2 \left(\langle\rho(r)\rangle - \rho_{bulk}\right) dr
\end{align}
%
en donde $\left<\rho(r)\right>$ y $\rho^b_{pro}=\lim\limits_{r\to \infty } \langle\rho(r)\rangle$ son respectivamente la densidad (en n\'umero) local y en el bulk de la prote\'ina.
La integraci\'on de $\langle N(r)\rangle$ produce la \emph{adsorci\'on en exceso} (en adelante, simplemente la adsorci\'on) que cuantifica el n\'umero de prote\'inas incorporadas a la red polim\'erica,


%
\begin{align}
    \Gamma =  \int_0^\infty{  \langle N(r)\rangle dr}
\end{align}
%

%%%%%%%%%%%%%%%%%%%%%%%%%%%
%%%%% Adsorption to MAA NGs
%%%%%%%%%%%%%%%%%%%%%%%%%%%


\begin{figure}[!htb]
	\centering
\includegraphics[width=0.75\textwidth]{Figures/graphs-gel2/ad-maa-pH-proteins.pdf}
\caption{Gr\'aficos de la adsorci\'on en exceso $\Gamma$ de citocromo c (paneles A y B) y mioglobina (paneles C y D) a nanogeles MAA-VA en funci\'on del pH.
	La concentraci\'on de sal es $10^{-3}M$ en los paneles de la izquierda (A y C) y $10^{-2}M$ en los paneles de la derecha
	(B y D);  En los paneles B y C Se muestra en el inset un zoom de la adsorci\'on.
	Estos nanogeles tienen 22\% MAA; la concentraci\'on de prote\'ina es $10^{-6}M$ en cada caso.}
\label{fig:esf:adsorption-vs-pH-cyto-myo}
\end{figure}
 
 La Figura \ref{fig:esf:adsorption-vs-pH-cyto-myo} muestra la adsorci\'on de soluciones, en diluci\'on infinita, de citocromo c (paneles superiores, A y B) y mioglobina (paneles inferiores, C y D) en nanogeles de MAA-VA con diferentes funcionalizaciones en su red. El pH se utiliz\'o como variable independiente en estos c\'alculos, y tambi\'en se evalu\'o el efecto de la concentraci\'on de NaCl al comparar diferentes paneles en la misma l\'inea. Los nanogeles de la Figura \ref{fig:esf:adsorption-vs-pH-cyto-myo} contienen un $22\%$ de MAA, lo que implica que todos los segmentos en las cadenas superficiales de la red son MAA.
 
 Se puede observar que la adsorci\'on de prote\'inas muestra un comportamiento no monot\'onico en funci\'on del pH, alcanzando un m\'aximo en la regi\'on entre pH 5 y 7. Este comportamiento est\'a influenciado por la concentraci\'on de sal y la prote\'ina espec\'ifica considerada. Esta respuesta se puede explicar mediante las interacciones electrost\'aticas y el comportamiento de protonaci\'on de los segmentos de MAA y de las prote\'inas. A medida que el pH aumenta por encima del pKa intr\'inseco de MAA (4.65), las unidades \'acidas del pol\'imero se disocian, cargando negativamente la red. Por encima del pKa del MAA, pero por debajo del punto isoel\'ectrico de cada prote\'ina, esta adquieren carga positiva. En estas condiciones, las interacciones atractivas entre las prote\'inas y la red polim\'erica promueven la adsorci\'on. Sin embargo, en ambos extremos de la escala de pH, estas interacciones son bajas: a pH bajo, el MAA est\'a protonado y tiene carga neutra, mientras que a pH alto, las prote\'inas tienen carga negativa. En ambos casos, esto conduce a una ausencia de adsorci\'on ($\Gamma \approx 0$) o incluso a una desorci\'on ($\Gamma < 0$).
 
 
En general, la adsorci\'on de citocromo c y la de mioglobina son cualitativamente similares. Sin embargo, existen dos diferencias principales: (i) la magnitud de la adsorci\'on ,el citocromo c se adsorbe significativamente m\'as y (ii) el rango de pH en el que ocurre la adsorci\'on (el citocromo c se adsorbe en un rango de pH m\'as amplio debido a su punto isoel\'ectrico m\'as alto, 9.65 en comparaci\'on con 7.15 para la mioglobina). Esto implica que, bajo condiciones similares, el nivel m\'aximo de adsorci\'on de citocromo c se alcanza a un pH ligeramente m\'as alto.

En cuanto a las otras configuraciones, la Figura \ref{fig:esf:adsorption-vs-pH-cyto-myo} muestra que la distribuci\'on central de los segmentos MAA conduce a una adsorci\'on significativamente mayor en la mayor\'ia de las condiciones. Este comportamiento se debe a que dicha distribuci\'on de segmentos MAA permite una incorporaci\'on m\'as efectiva de la prote\'ina adsorbida con carga el\'ectrica opuesta. Por otro lado, el comportamiento de adsorci\'on en las redes funcionalizadas aleatoriamente y en la superficie es sorprendentemente similar en el rango de pH y concentraciones de sal estudiadas, tanto para las diferentes prote\'inas como entre s\'i. Aunque las distribuciones de unidades funcionales entre las estructuras RF y SF difieren significativamente a pH bajo, se vuelven relativamente similares entre s\'i despu\'es de la reorganizaci\'on del nanogel a un pH m\'as alto cuando las unidades MAA se cargan. Esto explica la adsorci\'on comparable de prote\'inas observada en los nanogeles RF y SF (comparece los paneles A y C de la Figura \ref{fig:esf:MAA-vs-r-distribution}).



%%%%%%%%%%%%%%%%%%%%%%%%%%
%%%%% Protein localization
%%%%%%%%%%%%%%%%%%%%%%%%%%

\begin{figure}[!htb]
     \centering
     \includegraphics[width=0.40\textwidth]{Figures/graphs-gel2/cito-adsr-pmf.pdf}
     \caption{Panel A: Gr\'afico de la distribuci\'on radial de las mol\'eculas de citocromo c, $\langle N(r)\rangle$, en funci\'on de su posici\'on, para los nanogeles MAA-VA con diferentes funcionalizaciones.
     	Estas part\'iculas tienen 22\% MAA, el pH es 7, la concentraci\'on de prote\'ina es de $10^{-6}M$ y la concentraci\'on de NaCl es de $10^{-3}M$.
     	El panel B muestra el potencial de la fuerza media, ${PMF}(r)$, que act\'ua sobre el citocromo c para las mismas condiciones que el panel A.}
     \label{fig:esf:adsorption-vs-r-cyto}
 \end{figure}



Para explicar el mejor rendimiento de los nanogeles MAA funcionalizados en su n\'ucleo para la incorporaci\'on de prote\'inas, se muestra en la Figura \ref{fig:esf:adsorption-vs-r-cyto}A la distribuci\'on radial de las mol\'eculas de citocromo c en funci\'on de la distancia $r$ al centro de masa del nanogel. La soluci\'on tiene un pH de 7 y una concentraci\'on de NaCl de $1 $ mM, que corresponden aproximadamente a las condiciones de m\'axima adsorci\'on de esta prote\'ina en la Figura \ref{fig:esf:adsorption-vs-pH-cyto-myo}A. Existe una clara correlaci\'on entre la distribuci\'on de los grupos funcionales a lo largo de la red polim\'erica y la ubicaci\'on del citocromo c adsorbido.

Cuando el centro de la red est\'a funcionalizado, la mayor probabilidad de encontrar las prote\'inas ocurre en el interior del nanogel, entre 20 y 30 nm. En la figura \ref{fig:esf:adsorption-vs-r-cyto}A, el n\'umero m\'aximo de prote\'inas adsorbidas se produce a $r=28$ nm para 1 mM de NaCl. Coherentemente, el perfil de distribuci\'on de los grupos MAA cargados muestra un m\'aximo suave en esta regi\'on espacial (ver Figura \ref{fig:esf:MAA-vs-r-distribution}B, curva roja). Es decir, las prote\'inas adsorbidas se ubican donde pueden estar rodeadas de segmentos de la red con carga el\'ectrica opuesta. Curiosamente, el mismo fen\'omeno ocurre en la adsorci\'on a los nanogeles RF y SF. Las distribuciones de MAA cargados muestran un m\'aximo pronunciado cerca de la superficie del nanogel, entre 45 y 50 nm (ver curvas rojas en la figura \ref{fig:esf:MAA-vs-r-distribution}, paneles A y C). La figura \ref{fig:esf:adsorption-vs-r-cyto}A muestra que el citocromo c se adsorbe preferentemente en estas regiones de alta densidad de MAA, y por ende una alta densidad de carga el\'ectrica.



Al comparar las distribuciones de citocromo c dentro de los nanogeles RF y SF en la Figura \ref{fig:esf:adsorption-vs-r-cyto}A, observamos que los perfiles son relativamente similares entre s\'i. Como era de esperar, si solo se funcionaliza la superficie, el perfil de la prote\'ina se desplaza hacia la interfaz pol\'imero-soluci\'on.

Para cuantificar a\'un m\'as la interacci\'on con los nanogeles, utilizamos el potencial de fuerza media (PMF) que act\'ua sobre una prote\'ina a una distancia $r$ desde el centro de la red polim\'erica. El PMF se define como:

\begin{align}
	\text{PMF}(r) = -k_B T \ln \frac{\langle \rho(r)\rangle}{\rho_{\text{bulk}}}
\end{align}

donde $\lim\limits_{r\to \infty}\text{PMF}(r)=0$, lo que indica que la interacci\'on nanogel-prote\'ina desaparece cuando est\'an suficientemente lejos.

La Figura \ref{fig:esf:adsorption-vs-r-cyto}B muestra el PMF que act\'ua sobre el citocromo c en las mismas condiciones que en el panel A, pero para las tres diferentes funcionalizaciones de nanogeles. En el interior del nanogel, la interacci\'on de la prote\'ina con la estructura CF es la m\'as fuerte, aproximadamente $-8k_B T$ en el rango espacial de $r=0$ a 30 nm. Esta interacci\'on es de relativo corto alcance, ya que disminuye significativamente por encima de $r > 40$ nm.

Por otro lado, las interacciones del citocromo c con los nanogeles RF y SF se extienden m\'as lejos, hasta $55-60$ nm. En el interior del nanogel, estas interacciones son m\'as d\'ebiles que aquellas de la estructura CF. La energ\'ia libre de adsorci\'on es aproximadamente $-6 k_BT$, y se mantiene casi constante dentro del nanogel RF.

Para el nanogel funcionalizado en la superficie, sin embargo, el m\'inimo del PMF tambi\'en es de alrededor de $-6 k_BT$, y ocurre junto a la interfaz pol\'imero-soluci\'on a $r\approx 50$ nm. A diferencia de la estructura RF, esta interacci\'on no es constante dentro del nanogel. 

%En cambio, aumenta de manera monot\'onica a medida que $r$ disminuye, lo que indica que la prote\'ina se aleja de las cadenas superficiales funcionalizadas y se ubica m\'as profundamente en el nanogel.



%%%%%%%%%%
%%%% change in salt

\begin{figure}[!htb]
     \centering
     \includegraphics[width=0.45\textwidth]{Figures/graphs-gel2/gamma-salts-cito.pdf}
     \caption{Gr\'afico de la adsorci\'on en exceso $\Gamma$ del citocromo c en funci\'on de la concentraci\'on de sal a pH 7 para nanogeles MAA-VA con diferentes funcionalizaciones de red que contienen un 22\% de MAA; la concentraci\'on de prote\'ina es de $10^{-6}M$.}
     \label{fig:esf:adsorption-vs-salt-cyto}
 \end{figure}
 

Una caracter\'istica de la adsorci\'on de prote\'inas que nos falta discutir es el efecto de la concentraci\'on de sal.
Tanto para el citocromo c como para la mioglobina, la figura \ref{fig:esf:adsorption-vs-pH-cyto-myo} muestra que la incorporaci\'on de prote\'inas dentro de los diferentes nanogeles se ve significativamente mejorada al disminuir la concentraci\'on de sal en la soluci\'on.
Para caracterizar mejor este comportamiento, la figura \ref{fig:esf:adsorption-vs-salt-cyto} presenta la adsorci\'on de citocromo c en funci\'on de la concentraci\'on de NaCl a pH 7.
Este gr\'afico muestra que todas las funcionalizaciones de la red presentan un comportamiento cualitativamente similar, con una disminuci\'on dr\'astica en la adsorci\'on entre 1 y 10 mM de NaCl.
A 100 mM, todos los nanogeles muestran una adsorci\'on cercana a cero o negativa, dado que es una adsorci\'on por exceso esto \'ultimo significa que hay menos prote\'ina en el interior del nanogel que en la soluci\'on bulk.

Cuando la concentraci\'on de sal en la soluci\'on es alta, tanto los iones Na$^+$ como los iones Cl$^-$ se encuentran en altas concentraciones dentro del nanogel.
Estos iones opacan las atracciones electrost\'aticas entre las cargas positivas de la prote\'ina y las cargas negativas del nanogel, que son la fuerza impulsora para la adsorci\'on de prote\'inas.
En efecto, estas atracciones se vuelven de corto alcance y no son lo suficientemente fuertes como para dar lugar a una adsorci\'on significativa, de ocurrir dicho fen\'omeno.
Por otro lado, si la concentraci\'on de NaCl es menor, estas interacciones electrost\'aticas se ven menos apantalladas y efectivamente tienen un alcance mayor, lo que favorece la adsorci\'on de prote\'inas.
Por lo tanto, la disminuci\'on de la concentraci\'on de sal mejora la adsorci\'on.
Este comportamiento ha sido observado en experimentos; los brushes  de  polielectrol\'iticos muestran un aumento en la adsorci\'on de prote\'inas a baja concentraci\'on de sal \cite{wittemann2006interaction,becker2012proteins, henzler2010adsorption,xu2018interaction}.

Al considerar veh\'iculos para aplicaciones de liberaci\'on de prote\'inas, nuestros resultados sugieren que las mejores condiciones para la encapsulaci\'on corresponden a una baja concentraci\'on de sal.
Los perfiles de adsorci\'on de la figura \ref{fig:esf:adsorption-vs-salt-cyto} son cualitativamente similares para las tres funcionalizaciones, pero el n\'umero de prote\'inas dentro del nanogel siempre es significativamente mayor para la estructura CF.
Esta caracter\'istica puede ser cr\'itica en el dise\~no de veh\'iculos de liberaci\'on para un objetivo que tenga una concentraci\'on de sal intermedia.
El CF incorpora m\'as prote\'inas en las mismas condiciones, pero puede no ser capaz de liberarlas si el objetivo tiene una concentraci\'on de sal intermedia.
Para estas condiciones, la funcionalizaci\'on aleatoria podr\'a liberar toda su carga.



%%%%%%%%%%%%%%%%%%%%%%%%%%%%%%%%%%%%%%%%%%%%%%%%%%
\subsection{Adsorci\'on de insulina en nanogeles basados en  AH} 
%%%%%%%%%%%%%%%%%%%%%%%%%%%%%%%%%%%%%%%%%%%%%%%%%%

La insulina no se adsorbe a los nanogeles de MAA de la secci\'on \ref{sec:MAA-NGs} %(ver \cref*{fig:adsoprtion-vs-pH-insulinMAA_si} en SI).
Esto se debe a que el punto isoel\'ectrico de la insulina y el pKa del MAA est\'an cerca entre s\'i (5.5 y 4.65 respectivamente), lo que significa que para soluciones donde la prote\'ina tiene carga positiva, el nanogel tiene carga neutra, y si el nanogel tiene carga negativa, tambi\'en la tiene la prote\'ina.
En este contexto, se investiga la adsorci\'on de insulina en un nanogel de alilamina, que tiene carga positiva por debajo de su pKa de 9.5, superponi\'endose con el rango donde la insulina tiene carga negativa.
Aparte de los mon\'omeros funcionales, la estructura de estas redes de copol\'imero AH-VA es la misma que la de los nanogeles de MAA-VA descritos anteriormente.


\begin{figure}[!htb]
    \centering
    \includegraphics[width=0.75\textwidth]{Figures/graphs-gel2/insu-PAH.pdf}
    \caption{Gr\'aficos de las mol\'eculas de insulina adsorbidas, $\Gamma$, en funci\'on del pH para nanogeles AH-VA con diferentes funcionalizaciones.
    	El contenido de AH es del 22\% para las redes polim\'ericas del panel A y del 35\% para las del panel B (este \'ultimo grado de funcionalizaci\'on no se alcanza para el nanogel SF).
    	 La concentraci\'on de sal es $10^{-3}$ M de NaCl y [Insulina] = $10^{-6}$ M.}
    \label{fig:esf:adsorption-vs-pH-insulin}
\end{figure}




La figura \ref{fig:esf:adsorption-vs-pH-insulin}A muestra la adsorci\'on de insulina en nanogeles basados en AH con diferentes funcionalizaciones espaciales.
Una vez m\'as, hemos considerado redes con un $22\%$ de mon\'omeros sensibles al pH para poder incluir resultados para el nanogel SF, cuyas cadenas colgantes estan completamente compuestas  de AH. 
Las principales caracter\'isticas de este gr\'afico son cualitativamente similares a las de la adsorci\'on de citocromo c y mioglobina en los nanogeles de MAA (ver figura \ref{fig:esf:adsorption-vs-pH-cyto-myo}).
Es decir, la insulina muestra una adsorci\'on no monot\'onica en funci\'on del pH de la soluci\'on.
Adem\'as, observamos que la distribuci\'on central de los segmentos de AH captura m\'as insulina que las funcionalizaciones aleatorias o superficiales.
Los nanogeles RF y SF muestran perfiles de adsorci\'on dependientes del pH relativamente similares.
Finalmente, una mayor concentraci\'on de sal tiene un efecto cr\'itico en la magnitud de la adsorci\'on de insulina, que disminuye significativamente debido al aumento del apantallamiento de las atracciones electrost\'aticas entre la red y la prote\'ina por los iones m\'oviles
%(las curvas de adsorción para $10^{-2}$,M de NaCl se pueden ver en la figura \ref{fig:esf:adsoprtion-AH-1d-2-insu_si}).

La figura \ref{fig:esf:adsorption-vs-pH-insulin}A muestra que los nanogeles basados en AH son efectivos para encapsular insulina.
Sin embargo, a pesar de las similitudes cualitativas entre los perfiles de adsorci\'on de esta figura y los de citocromo c y mioglobina (\ref{fig:esf:adsorption-vs-pH-cyto-myo}A y C), vemos que la cantidad de mol'eculas de insulina capturadas por los nanogeles de AH es significativamente menor que la de las otras prote\'inas capturadas por los nanogeles basados en MAA.
Por esta raz\'on, tambi\'en hemos evaluado el efecto del grado de funcionalizaci\'on de la red de pol\'imero para mejorar la adsorci\'on de prote\'inas.
La figura \ref{fig:esf:adsorption-vs-pH-insulin}B presenta la adsorci\'on de insulina para nanogeles con un $35\%$ de segmentos de AH.
En este caso, la estructura funcionalizada en la superficie no se incluye porque no hay suficientes segmentos en las cadenas colgantes para llegar a ese porcentaje.
Un mayor contenido de AH promueve una mayor adsorci\'on, como se puede observar al comparar ambos paneles de la figura \ref{fig:esf:adsorption-vs-pH-insulin}.
Una vez m\'as, el nanogel CF adsorbe m\'as insulina que la red RF (m\'as del doble de prote\'inas para las condiciones de estos c\'alculos).




\begin{figure}[!htb]
    \centering
    \includegraphics[width=0.40\textwidth]{Figures/graphs-gel2/insu-ads-pmf.pdf} 
    \caption{A: Gr\'afico de la distribuci\'on local de mol\'eculas de insulina, $\langle N(r) \rangle$, en funci\'on de la posici\'on para nanogeles AH-VA con un 22\% de segmentos sensibles al pH en diferentes configuraciones de red.
    	El pH es 7.5, la concentraci\'on de insulina es de $10^{-6}$ M y la de NaCl es de $10^{-3}$ M.
    	B: Potencial de fuerza media, ${PMF}(r)$ actuando sobre la prote\'ina, en funci\'on de la posici\'on para las mismas condiciones que el panel A.}
    \label{fig:esf:adsorption-vs-r-insulin}
\end{figure}



A continuaci\'on, describimos la distribuci\'on de mol\'eculas de insulina dentro de los nanogeles basados en AH.
La figura \ref{fig:esf:adsorption-vs-r-insulin}A muestra c\'omo se disponen espacialmente las prote\'inas adsorbidas dentro de los diferentes nanogeles basados en AH con un grado de funcionalizaci\'on del $22\%$;
utilizamos ese grado de funcionalizaci\'on para comparar los resultados de los nanogeles CF, RF y SF.
El pH de estos resultados corresponde al m\'aximo de adsorci\'on de la figura \ref{fig:esf:adsorption-vs-pH-insulin}A.
La adsorci\'on de insulina en el nanogel CF no solo es significativamente mayor que la adsorci\'on en los nanogeles RF y SF, sino que tambi\'en ocurre en una posici\'on m\'as profunda dentro de la estructura.
La posici\'on m\'as probable de una mol\'ecula de insulina se encuentra alrededor de $r=25$ nm para el nanogel CF,
mientras que esta posici\'on se desplaza a alrededor de 40-45 y 50 nm para las estructuras RF y SF, respectivamente.

Para los nanogeles basados en MAA, la figura \ref{fig:esf:adsorption-vs-r-cyto}A muestra diferencias relativamente menores entre las distribuciones de citocromo c en los nanogeles RF y SF.
Estas diferencias se acent\'uan ligeramente al observar la adsorci\'on de insulina en los nanogeles AH-VA, como se muestra en la figura \ref{fig:esf:adsorption-vs-r-insulin}A.
La distribuci\'on de insulina se desplaza hacia el interior de la red en el nanogel modificado aleatoriamente en comparaci\'on con la funcionalizaci\'on superficial, donde las prote\'inas tienen m\'as probabilidades de ocupar la vecindad inmediata de la interfaz nanogel-soluci\'on.
Los perfiles de distribuci\'on de insulina siguen siendo relativamente similares para estas dos estructuras.

El panel B de la figura \ref{fig:esf:adsorption-vs-r-insulin} muestra el potencial de fuerza media que act\'ua sobre las mol\'eculas de insulina en las mismas condiciones que el panel A.
La interacci\'on atractiva en la insulina adsorbida var\'ia desde $-8$ hasta $-6 k_B T$ en el interior de la estructura CF y desde $-5$ hasta $-4 k_B T$ en el nanogel RF.
En el nanogel SF, el potencial presenta un m\'inimo de $-4 k_B T$ en la superficie y luego aumenta mon\'otonamente a medida que $r$ disminuye dentro del gel.
En general, los resultados de esta secci\'on muestran, una vez m\'as, que el dise\~no de la red (mediante la s\'intesis del nanogel) proporciona una herramienta para controlar la distribuci\'on y localizaci\'on de prote\'inas dentro del nanogel.






%%%%%%%%%%%%%%%%%%%%%%%%%%%%%%%%%%%%%%%%%%%%%%%%%%
\section{Conclusiones}
%%%%%%%%%%%%%%%%%%%%%%%%%%%%%%%%%%%%%%%%%%%%%%%%%%


En este cap\'itulo investigamos la adsorci\'on de prote\'inas en nanogeles polim\'ericos con diferentes funcionalizaciones espaciales y sensibles al pH. Se desarroll\'o y aplic\'o una teor\'ia termodin\'amica basada en un modelo molecular.
El enfoque se centr\'o en la influencia de las interacciones electrost\'aticas, por lo cual se describieron tres prote\'inas con diferentes puntos isoel\'ectricos (insulina, mioglobina y citocromo c) utilizando un modelo molecular basado en sus estructuras cristalogr\'aficas.
Exploramos las propiedades de los nanogeles sensibles al pH modificados con m\'onomeros de  \'acido metacr\'ilico o de alilamina.
Se consideraron tres configuraciones diferentes, con los segmentos sensibles al pH distribuidos al azar, en el centro o en la superficie de la red del nanogel.

Se examin\'o el comportamiento de estos nanogeles en funci\'on del pH cuando no hay prote\'inas presentes en la soluci\'on.
Los resultados muestran que, para los nanogeles basados en \'acido metacr\'ilico, tanto los estructurados con distribuci\'on distribuidos al azar como los funcionalizados en el n\'ucleo se expanden con el aumento del pH, debido a la desprotonaci\'on y carga de los segmentos de \'acido metacr\'ilico, lo que conduce a repulsiones electrost\'aticas intra-red.
Por otro lado, la red de \'acido metacr\'ilico funcionalizada en la superficie se comprimen a medida que las unidades titulables se cargan con el aumento del pH.
Este comportamiento contra-intuitivo se puede explicar al observar la distribuci\'on local de los segmentos que componen la red dentro de estas estructuras en diferentes condiciones.
La reorganizaci\'on de la red del nanogel en respuesta a cambios en el pH depende de la funcionalizaci\'on espacial espec\'ifica.
Se realiz\'o el mismo an\'alisis para los nanogeles basados en alilamina y se obtuvieron resultados an\'alogos, con la direcci\'on de los est\'imulos invertida;
la diferencia clave radica en el comportamiento de sus unidades sensibles al pH: mientras que el \'acido metacr\'ilico es \'acido, la alilamina est\'a cargada a pH bajo y es neutra a pH alto.


Se evalu\'o la adsorci\'on de citocromo c y mioglobina en las diferentes estructuras de nanogeles basados en MAA, y se encontr\'o que la adsorci\'on de prote\'inas es una funci\'on no monot\'onica del pH.
Los detalles cuantitativos de los perfiles de adsorci\'on dependen de la concentraci\'on de sal y de la prote\'ina espec\'ifica.
La respuesta al pH puede explicarse en t\'erminos de las interacciones electrost\'aticas y el comportamiento de protonaci\'on tanto de los segmentos de MAA como de los residuos de prote\'ina.
La reorganizaci\'on de los segmentos de las cadenas polim\'ericas en los nanogeles como resultado de los cambios de pH depende de la configuraci\'on espacial de las unidades funcionales dentro de la red, lo que tambi\'en regula el nivel y la localizaci\'on de las prote\'inas adsorbidas.
Tambi\'en hemos investigado la adsorci\'on de insulina en nanogeles de alilamina.
Los  resultados muestran que los nanogeles basados en AH son eficaces para encapsular insulina, y un mayor grado de funcionalizaci\'on resulta en una mayor adsorci\'on.

En el contexto del uso de nanogeles sensibles al pH para la liberaci\'on de prote\'inas, los resultados enfatizan la importancia de considerar la distribuci\'on espacial de las unidades funcionales en la red de nanogeles durante la s\'intesis.
Este factor de dise\~no no solo afecta la respuesta mec\'anica macrosc\'opica del nanogel y su nivel de adsorci\'on de prote\'inas, sino que tambi\'en influye en la localizaci\'on de las prote\'inas adsorbidas dentro del nanogel.
La funcionalizaci\'on interna cerca del centro de la red polim\'erica conduce a una mayor encapsulaci\'on de prote\'inas, pero estos adsorbatos son menos accesibles para las interacciones superficiales con un determinado target.
Por otro lado, la distribuci\'on aleatoria de unidades sensibles al pH ofrece un mejor rendimiento si la entrega requiere interacciones prote\'ina-objetivo.
La funcionalizaci\'on superficial del nanogel proporciona una mejor disponibilidad de prote\'inas en la interfaz nanogel-objetivo, aunque potencialmente puede implicar una s\'intesis m\'as compleja.

La dependencia de la adsorci\'on de prote\'inas de la concentraci\'on de sal puede aprovecharse en el dise\~no de portadores funcionales para la entrega de prote\'inas.
Estos hallazgos indican que es mejor encapsular prote\'inas a bajas concentraciones de sal y liberarlas a altas concentraciones salinas, donde interactuan m\'as d\'ebilmente con el nanogel.
Si el entorno objetivo tiene una concentraci\'on de sal intermedia, un nanogel con una distribuci\'on aleatoria de mon\'omeros sensibles al pH es m\'as adecuado. En conclusi\'on, los resultados de este cap\'itulo demuestran el papel cr\'itico que desempe\~na la funcionalizaci\'on de la red y la composici\'on qu\'imica en el control de la respuesta del nanogel a las variaciones de pH y en la adsorci\'on de prote\'inas en estos materiales.
Esta informaci\'on es valiosa para el dise\~no de nanogeles sensibles al pH como veh\'iculos para la encapsulaci\'on, el transporte y la administraci\'on de prote\'inas.


%\include{capitulo2}
%\include{capitulo3}
%\include{conclu}

\appendix
%% Cap'itulos incluidos despues del comando \appendix aparecen como ap'endices
%% de la tesis.
%\include{apendiceA}
%\include{apendiceB}
%\include{apendiceC}

%% Incluir la bibliograf'ia. Mirar el archivo "biblio.bib" para m'as detales
%% y un ejemplo.
\bibliography{gel}

\end{document}
