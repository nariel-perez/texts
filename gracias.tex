\textbf{Agradecimientos}\\

Al llegar a este punto del camino, no puedo evitar mirar atr\'as y reflexionar sobre el viaje que ha sido completar esta tesis. Ha sido un per\'iodo lleno de desaf\'ios, alegr\'ias, tritezas, mucho aprendizaje, todo un crecimiento personal dir\'ia y claro acad\'emico. Este logro no habr\'ia sido posible sin el apoyo, la gu\'ia y la inspiraci\'on de muchas personas que me acompa\~naron en diferentes etapas de este proceso.

En primer lugar, deseo expresar mi m\'as profundo agradecimiento a mis directores de tesis, Dr. Gabriel S. Longo y Dr. Alberto G. Albesa (Gaby y Beto con cari\~no) cuya experiencia, conocimiento y sobre todo mucha paciencia fueron fundamentales para realizar este doctorado. Gracias por confiar en m\'i en todo este camino.

No quiero mencionar m\'as nombres porque son demasiadas personas a las que tendr\'ia que agradecerle y me parecer\'ia injusto que alguien se quede afuera. Compa\~neres de militancia, de casa, de oficinas, de cer\'amica, f\'utbol, familia, etc, etc. 

No puedo dejar de mencionar a las insituciones que hicieron posible todo esto es decir a CONICET, AGENCIA, al INIFTA, la Facultad de Ciencias Exactas, la Universidad Nacional de La Plata. Y en particular la Laboratorio de Materia Blanda lugar que fue mi segundo hogar y en donde conoc\'i y conviv\'i con personas maravillosas. 

Finalmente, mi gratitud se extiende a todos aquellos que, de una manera u otra, contribuyeron a este trabajo, ya sea a trav\'es de charlas, intercambios acad\'emicos o simplemente ofreciendo su amistad y un mate. A todes ustedes, mi m\'as sincero agradecimiento.

...y a las Umas claro.