\chapter{esfericas}

%%%%%%%%%%%%%%%%%%%%%%%%%%%%%%%%%%%%%%%%%%%%%%%%%%
\section{Introduction}
%%%%%%%%%%%%%%%%%%%%%%%%%%%%%%%%%%%%%%%%%%%%%%%%%%


En estas secci\'on... geles polim\'ericos... 
Se aborda una perspectiva diferente con informaci\'on a la que no era posible acceder con el modelo anterior.
El conocimiento  de la estructura interna, o una aproximaci\'on de ella nos abre las puertas al dise\~no de m\'as y mejores materiales...

%%%%%%%%%%%%%%%%%%%%%%%%%%%%%%%%%%%%%%%%%%%%%%%%%%
\section{Method: Molecular Theory}
%%%%%%%%%%%%%%%%%%%%%%%%%%%%%%%%%%%%%%%%%%%%%%%%%%

El estudio de estos nanogeles conlleva un formalismo similar al visto en el capitulo de los films polim\'ericos y al modelo de dos fases.


Es decir buscamos con este m\'etodo  minimizar una energ\'ia libre del sistema.
Incorporaando una caracterizaci\'on molecular de grano grueso de las diversas especies qu\'imicas presentes.

El sistema en estudio es un solo nanogel en equilibrio con una solución acuosa que tiene una composici\'on de bulk definida externamente.
Es decir, el pH, la concentraci\'on de sal y la concentració'o de prote\'ina son las variables independientes.
La red polim\'erica que da estructura al nanogel contiene dos tipos de segmentos: una unidad sensible al pH, ya sea \'acida (MAA) o b\'asica (AH), y un segmento neutro (VA);
los segmentos que componene al entrecruzante en nuestra red se describen como segmentos de carga neutral.

Como vimos en... la energ\'ia libre se convierte en un gran potencial el cual nos permite un mejor manejo y resoluci]'on del sistema.
Este potencial contiene las siguientes contribuciones:

\begin{align}
\begin{aligned}
\Omega_{NG}=& -TS_{mez} -TS_{conf,net} + F_{qca,net} + F_{qca,pro}\\
& + U_{elec} + U_{ste} + U_{vdw} - \sum_{\gamma}{\mu_\gamma N_\gamma} - \mu_{pro} N_{pro}
\end{aligned}
\label{eq:esf:semicano}
\end{align}


\noindent donde $S_{mez}$ es la entrop\'ia de traslaci\'on (y de mezcla) de las especies de la soluci\'on: mol\'eculas de agua (H$_2$O), iones de hidronio (H$_3$O$^+$), iones de hidr\'oxido (OH$^- $), cationes de sal, aniones de sal y prote\'ina.
Consideramos una sal monovalente, NaCl completamente disociada en iones de sodio (Na$^+$) y cloruro (Cl$^-$).
$S_{conf,net}$ representa la entrop\'ia conformacional que resulta de la flexibilidad de la red de pol\'imeros, que puede asumir muchas conformaciones diferentes.
$F_{qca,net}$ es la energ\'ia qu\'imica libre que describe el equilibrio entre las especies protonadas y desprotonadas de unidades funcionales (\'acidas/b\'asicas) en el pol\'imero.
De manera similar, $F_{qca,pro}$ describe la protonaci\'on de residuos titulables de la prote\'ina.
$U_{elec}$ y $U_{ste}$ representan, respectivamente, las interacciones electrost\'aticas y las repulsiones est\'ericas.
$U_{Vdw}$ contiene las interacciones de Van der Waals entre los distintos segmentos y el solvente.
Finalmente, la suma de $\gamma$ expresa el equilibrio qu\'imico entre nuestro sistema y la soluci\'on bulk que representa un ``ba\~no t\'ermico" para las part\'iculas libres, donde $\mu_\gamma$ y $N_\gamma$ son el potencial qu\'imico y el n\'umero de mol\'eculas de especie $\gamma$, respectivamente;
el sub\'indice $\gamma$ recorre las especies qu\'imicas libres.
El siguiente t\'ermino tiene en cuenta el equilibrio descrito anteriormente, pero en esta ocasi\'on para la prote\'ina.

Las expresiones explicitas de cada uno de estos componentes, as\'i como la minimizaci\'on de nuestro gran potencial es descrita en la siguiente secci\'on.



\subsection{Theoretical Framework}\label{sec:esf:tm}

A continuaci\'on describiremos la forma expl\'icita de cada uno de estos t\'erminos, donde los segmentos protonables del nanogel ser\'an considerados como segmentos de \'acido metacrílico (MAA). Sin embargo, las mismas ecuaciones se aplican a los nanogeles que tienen segmentos b\'asicos. Las diferencias se encuentran en el uso del signo del grado de disociaci\'on.

En primera instancia tenemos la entrop\'ia de traslaci\'on y de mezcla de las especies m\'oviles, incluida la prote\'ina:


\begin{align}
	\begin{aligned}
		-\frac{S_{mez}}{k_B}= &\sum_{\gamma}\int_0^\infty{dr G(r)\rho_\gamma(r)\left(\ln \left(\rho_\gamma (r)v_w\right) -1 + \beta\mu^0_\gamma\right)} \\
		&+ \sum_{\theta}\int_0^\infty{dr G(r)\rho_{pro}(\theta,r)\left(\ln \left(\rho_{pro}(\theta,r)\right) -1 + \beta\mu^0_{pro} \right)}
	\end{aligned}
\end{align}



\noindent donde $\beta = \frac{1}{k_BT}$, $k_B$ es la constante de Boltzmann y $T$ la temperatura absoluta del sistema, $\rho_\gamma(r)$ y $\mu_\gamma$ es la densidad local, en la posici\'on $r$, y potencial qu\'imico de la especie $\gamma$ respectivamente.
El sub\'indice $\gamma$ toma en cuenta las mol\'eculas de agua y sus iones (hidronio e hidr\'oxido), y los iones disociados de la sal ($K^+$, $Cl^-$). $G(r)$ es la constante de simetr\'ia de nuestro sistema, en particular para cada para $r$: $G(r) =4\pi r^2$. Esta \'utima se ejemplificara de mejor forma en la secci\'on del modelado molecular.

El segundo t\'ermino de la entrop\'ia de mezcla que considera los aportes entr\'opicos de la prote\'ina.
$\rho_{pro}(\theta,r)$ es la densidad local de la prote\'ina en la conformaci\'on  $\theta$.  La prot\'ina puede tomar cualquier conformaci\'on presenten en su set de conformaciones $\{\theta \}$
La contribuci\'on entr\'opica tambi\'en incluye la rotación espacial.
La densidad media local total de prote\'ina es: 


\begin{align}
	\left<\rho_{pro}(r)\right> = \sum_\theta{\rho_{pro}(\theta,r)}
\end{align}

$S_{conf,net}$ representa la entrop\'ia conformacional resultante de la flexibilidad de la red polim\'erica que forma al nanogel. Conformaciones denotadas por el set $\{\alpha\}$. 
\begin{equation}
	\frac{S_{conf,nw}}{k_B} = - \sum_{\alpha}{P(\alpha)\ln P(\alpha)}
\end{equation}

\noindent En donde $P(\alpha)$ muestra la probabilidad que el nanogles se encuentre un a configuraci\'on $\alpha$.

El siguiente t\'ermino describe la energ\'ia qu\'imica libre originada por el equilibrio \'acido-base de los segmentos de $MAA$ presentes en el nanogel.

\begin{align}
	\begin{aligned}
		\beta F_{qca,net} &= \int_0^\infty drG(r) \frac{\left<\phi_{MAA}(r)\right>}{v_{MAA}} \left[f(r)(\ln f(r)+ \beta\mu^0_{MAA^-})\right.\\
		&\left.+(1-f(r))(\ln (1-f(r))+\beta\mu^0_{MAAH})\right]    
	\end{aligned}
\end{align} 


\noindent donde $f(r)$ es el grado de carga de los segmentos $MAA$ en la capa esf\'erica entre $r$ y $r + dr$.
$\mu^0_{MAA^-}$ y $\mu^0_{MAAH}$ son los potenciales qu\'imicos est\'andar de las especies desprotonadas y protonadas respectivamente. $v_{MAA}$ es el volumen molecular del segmento de $MAA$.
Definimos aden\'as para un tipo de segmento $i$ en la red ($i = MAA/VA/crosslink$):
 

\begin{align}
	\left< \phi_i(r)\right> = \sum_\alpha{P(\alpha)\phi_i(\alpha,r)}
	\label{eq:esf:ensamble-gel}
\end{align}

\noindent en el cual $\left< \phi_i(r)\right> $ es la fracci\'on en volumen promedio, dado por $\phi_i(\alpha,r)$ la cual es una cantidad de entrada en nuestros c\'odigos que muestra la fracci\'on en volumen ocupada localmente en $r$, cuando la red polim\'erica se encuentra en la conformaci\'on $\alpha$.


%%%%%%%%%%%%%%%%%%%
El equilibrio qu\'imico de las unidades proteicas titulables se considera en el siguiente t\'ermino del potencial:

\begin{align}
	\begin{aligned}
		\beta F_{qca,pro} =\int_0^\infty dr &G(r) \sum_\tau \left<\rho_{pro,\tau}(r)\right> \left[g_\tau(r)(\ln g_\tau(r)+ \beta\mu^0_{\tau p})\right.\\
		&\qquad\left.+(1-g_\tau(r))(\ln (1-g_\tau(r))+\beta\mu^0_{\tau d})\right]   
	\end{aligned}
\end{align} 

\noindent en donde $\left<\rho_{pro,\tau}(r)\right>$ representa la densida local promedio del segmento titulable $\tau$ de la prote\'ina.

El cual es definido como:

\begin{align}
	\left<\rho_{pro,\tau}(r)\right> = \sum_\theta \int_o^\infty dr^\prime G(r^\prime) \rho_{pro}(\theta,r)n_\tau(\theta,r^\prime,r)
	\label{eq:esf:segments-pro}
\end{align}

\noindent en donde $n_\tau(\theta,r^\prime,r)$  es un par\'ametro de entrada que nos cuantifica el n\'umero de segmentos $\tau$ entre $r$ y $r + dr$ cuando el centro de masa de la prote\'ina en la configuraci\'on $\theta$ esta en $r^\prime$.

Las unidades titulables son denotadas por el subindice $\tau$  y los subindices $p$ y $d$ representan su estado protonado y desprotonado respectivamente. 
De este modo$\mu^0_{\tau,p}$ y $\mu^0_{\tau,d}$  son los potenciales qu\'imicos est\'andar de estos estados respectivamente.

Hemos definido el grado de asosiaci\'on de protones a los segmentos $\tau$
como: 
 
\begin{enumerate}
	\item Para unidades \'acidas: $g_\tau(r) = 1-f_\tau(r)$ ($\tau$ se carga negativamente)
	\item Para unidades b\'asicas: $g_\tau(r) = f_\tau(r)$ ($\tau$ se carga positivamente)
\end{enumerate}
\noindent en donde $f_\tau(r)$  es el grado de disociaci\'on  para el segmento $\tau$.
%%%%%%%%%%

La energ\'ia electrost\'atica se define:

\begin{align}
	\begin{aligned}
		\beta U_{elecc}= \int_0^\infty drG(r)&\left[\left(\sum_{\gamma } {\rho_\gamma(r) q_\gamma + \sum_\tau{f_\tau(r) \left<\rho_{pro,\tau}(r)\right> q_\tau} +  f(r)\dfrac{\left<\phi_{MAA}(r)\right>}{v_{MAA}}q_{MAA}}\right)\beta\Psi(r) \right. \\ &\left.-\frac{1}{2}\beta\epsilon(\nabla\Psi(r))^2 \right]
	\end{aligned}
\end{align} 

\noindent donde $\Psi(r)$ es el potencial electrost\'atico dependiente de la posici\'on, y $\epsilon$ la permitividad del medio, $q_\gamma$ es la carga de la especie m\'ovil $\gamma$, $q_\tau$ corresponde a la carga del segmento titulable de la prote\'ina . Finalmente $q_{MAA}$ es la de un segmento de $MAA$.

En este contexto, podemos definir la densidad de carga promedio: 

\begin{align}
	\left<\rho_q(r)\right> = \sum_{\gamma } {\rho_\gamma(r) q_\gamma + \sum_\tau{f_\tau(r) \left<\rho_{pro,\tau}(r)\right> q_\tau} +  f(r)\dfrac{\left<\phi_{MAA}(r)\right>}{v_{MAA}}q_{MAA}}
	\label{eq:esf:rho-charge}
\end{align}  
%%%%%%%%%%%%%%%%
             
El siguiente t\'ermino en el potencial termodin\'amico se debe a la repulsi\'on est\'erica, el cual se puede incorporar a trav\'es de la siguiente restricci\'on:

\begin{align}
	\begin{aligned}
		1=  {\left[\sum_{\gamma}\rho_\gamma(r) v_\gamma + \sum_\lambda{\left<\rho_{pro,\lambda}(r)\right>v_\lambda} + \sum_i{\left<\phi_i(r)\right>}\right]},~ \forall r
	\end{aligned}
	\label{eq:esf:constraint}
\end{align}


\noindent en donde $v_\lambda$  es el volumen molecular de cada segmento $\lambda$  que compone a la prote\'ina.
$\left<\rho_{pro,\lambda}(r)\right>$  es definido de la misma forma que se hizo en la ecuaci\'on \ref{eq:esf:segments-pro}.
Es conveniente notar que el subindice $\lambda$ considera a todos los segmentos de la prote\'ia, es decir: $ \tau \in \lambda$.
Finalmente, como se mencion\'o anteriormente el subindice $i$  contempla a todos los tipos de segmentos que componen al nanogel.


%%%%%%%%%%%%%%%

$U_{VdW}$ es la energ\'ia de interacci\'on de Van der Waals ($VdW$). En este sistema se ha asumido que todos los segmentos tienen un car\'acter hidrof\'ilico. Es decir, las interacciones $VdW$ entre diferentes pares de segmentos y \'estas con mol\'eculas de agua son similares. Como resultado, la energ\'ia de interacci\'on neta $VdW$ representa una constante aditiva a la energ\'ia total del potencial.
Por lo tanto, esta contribuci\'on puede ser ignorada. 
Esto es posible por los segmentos considerados en la estructura del nanogel, como se mostr\'o en el capitulo anterior, en el modelo de dos fases, se consider\'o la interacci\'on entre los segmentos de NIPAm como un potencial aparte. Por lo que las interacciones de Van der Waals fueron tenidas en cuenta.



Para completar el gran potencial de ec. \ref{eq:esf:semicano}, se tiene en cuenta el intercambio de especies m\'oviles:

Los primeros dos t\'erminos del lado izquierdo de la ecuaci\'on explican el equilibrio qu\'imico de las especies m\'oviles $\gamma$ y de las prote\'inas dentro de la soluci\'on.
Los dos \'ultimos t\'erminos consideran los iones de hidr\'ogeno de los segmentos protonables de la prote\'ina y los segmentos $MAA$ de la red polim\'erica que forma el nanogel. Notese que se van de la mano con el grado de asociaci\'on $g_\tau$ y $1-f$ para la prote\'ina y la red polim\'erica respectivamente. 


\begin{align}
	\begin{aligned}
		\mu_\gamma N_\gamma + \mu_{pro} N_{pro} =\int_0^\infty drG(r)&\left[\sum_{\gamma }{\rho_\gamma(r)\mu_\gamma}
		+ \mu_{pro} \left<\rho_{pro}(r)\right> \right. \\
		& \left. +\mu_{H^+}\sum_{\tau}{g_\tau\left<\rho_{pro,\tau}(r)\right> } +\mu_{H^+}(1-f(r))\dfrac{\left<\phi_{MAA}(r)\right>}{v_{MAA}}\right]
	\end{aligned}
\end{align}


%%%%%%%%%%
Finalment la forma explicita de nuestro gran potencial es expresado:

%%%%%%%%%%%%
\begin{align}
	\begin{aligned}
		\beta&\Omega_{NG}=\\&  \sum_{\gamma}\int_0^\infty{dr G(r)\rho_\gamma(r)\left(\ln \left(\rho_\gamma (r)v_w\right) -1 + \beta\mu^0_\gamma\right)} \\
		%
		& +\sum_\theta \int_0^\infty{dr G(r)\rho_{pro}(r)\left(\ln (\rho_{pro}(\theta,r)v_w)-1 + \beta\mu^0_{pro} \right)} \\
		%
		& + \sum_{\alpha}{P(\alpha)\ln P(\alpha)} \\
		%
		& +\int_0^\infty drG(r) \frac{\left<\phi_{MAA}(r)\right>}{v_{MAA}} \left[f(r)(\ln f(r)+ \beta\mu^0_{MAA^-})\right.\\
		&\qquad \qquad \qquad\qquad \qquad \quad \left.+(1-f(r))(\ln (1-f(r))+\beta\mu^0_{MAAH})\right] \\
		%
		& +\int_0^\infty drG(r)\sum_\tau \left<\rho_{pro,\tau}(r)\right> \left[g_\tau(r)(\ln g_\tau(r)+ \beta\mu^0_{\tau p})\right.\\
		&\qquad\qquad \qquad\qquad \qquad \qquad\left.+(1-g_\tau(r))(\ln (1-g_\tau(r))+\beta\mu^0_{\tau d})\right] \\
		%
		& +  \int_0^\infty drG(r)\left[\left(\sum_{\gamma } {\rho_\gamma(r) q_\gamma + \sum_\tau{f_\tau(r) \left<\rho_{pro,\tau}(r)\right> q_\tau} +  f(r)\dfrac{\left<\phi_{MAA}(r)\right>}{v_{MAA}}q_{MAA}}\right)\beta\Psi(r) \right.\\  &\left. \hspace{6em}-\frac{1}{2}\beta\epsilon(\nabla\Psi(r))^2 \right]\\
		%
		&+ \int_0^\infty \beta\Pi(r) drG(r){\left(\sum_{\gamma}\rho_\gamma(r) v_\gamma + \sum_{\lambda}{\left<\rho_{pro,\lambda}(r)\right>}{v_\lambda} + \sum_i\left<\phi_i(r)\right> -1\right)}\\
		%
		& -\int_0^\infty drG(r)\left[\sum_{\gamma }{\rho_\gamma(r)\beta\mu_\gamma}
		+ \beta\mu_{pro} \left<\rho_{pro}(r)\right>
		+\beta\mu_{H^+}\sum_{\tau}{g_\tau(r)\left<\rho_{pro,\tau}(r)\right> } \right.\\
		& \left. \hspace{6em} +\beta\mu_{H^+}(1-f(r))\dfrac{\left<\phi_{MAA}(r)\right>}{v_{MAA}}\right]%\\
	\end{aligned}
	\label{eq:esf:potential-energy}
\end{align}


Para esta expresion,  \ref{eq:esf:potential-energy}, se ha introducido nuestra restricci\'on de la incompresibilidad del volumen (ec. \ref{eq:esf:constraint} ), como un  multiplicador  de  Lagrange $\Pi(r)$, este veremos cumple la funci\'on de una presi\'on osm\'otica del sistema. 

Como se menciono al inicio de este capitulo, el siguiente paso es la busqueda de las condiciones que minimizen la energ\'ia total. Esto se logra al derivar respecto de las denidades locales $\rho_\gamma(r)$, el potencial electrost\'atico $\Psi(r)$, el grado de disosiaci\'on, tanto de los segmentos provenientes de la prote\'ina como del nanogel, $f(r)$, adem\'as de la probabilidad de las diferentes conformaciones de la red polim\'erica $P(\alpha)$

En sint\'esis podemos escribir $\Omega = \sum P(\alpha) \int{G(r) dV\omega}$,  con $\omega$ es el funcional que contempla los funcionales que definen a nuestro gran potencial: 

\begin{align}
	\omega=\omega(\rho_\gamma(r), \rho_{pro}(r),\Psi(r),f(r),P(\alpha))
	\label{eq:esf:funcionales-omega}
\end{align}


En particular la expresi\'on obtenida para el grado de disociaci\'on, $f_j(r)$ de los segmentos titulables tanto de la prote\'ina como de la red polim\'erica que compone al nanogel:

\begin{align}
	\frac{f_j(r)}{1-f_j(r)}= \left(\frac{a_{H^+}}{k^0_{a,j}}\right)^{\mp 1} e^{-\beta q_{MAA^-}\Psi(r)}
	\label{eq:esf:f-degree}
\end{align}

\noindent En donde  $a_{H^+}=e^{\beta\Delta\mu_{H^+}}=e^{\beta(\mu_{H^+} -\mu^0_{H^+})}$ es la actividad del $H^+$. El subindice  $j$ se define tal que  $j =\{MAA , \, \tau \}$. El exponente $\mp \, 1$ hace la diferencia sobre segmentos \'acidos o b'asicos respectivamente.

En la expresi\'on anterior, \ref{eq:esf:f-degree}, $K^0_{a,j}$ es la constante termodin\'amica del equilibrio \'acido-base:

\begin{align}
	\begin{aligned}
		& \left[HA\right] \Longleftrightarrow [H^+] +[A^-] \\
		& k_{a,HA}^0=\frac{[H^+][A^-]}{[HA]} \\
		& k_{a,HA}^0=\exp\left(\beta\mu_{HA}^0 - \beta \mu_{A^-}^0 - \beta \mu^0_{H^+} \right)
	\end{aligned}
	\label{eq:esf:dis-rxn}
\end{align}
%%%%%%%%%%%%%%%
%represent de protonable and deprotonable state of the segment $\iota
Para las especiel libres, su densidad local se expresa como:


\begin{align}
	\rho_\gamma(r)v_w = a_\gamma \exp{\left(-\beta \Psi(r)q_\gamma\right)} \exp{\left(-\beta\Pi(r) v_w\right)}
\end{align}


En el mismo sentido, para la prote\'ina $\rho(\theta,r)$:
	
	
	
	\begin{align}
		\begin{aligned}
			\rho_{pro}(\theta, r)v_w = & \tilde{a}_{pro} \prod_\tau \exp\left[ -\int_0^\infty dr^\prime G(r^\prime) n_\tau(\theta,r,r^\prime) \ln f_\tau(r^\prime)\right] \\
			& \times \prod_\lambda \exp\left[ -\int_0^\infty dr^\prime G(r^\prime) n_\lambda(\theta,r,r^\prime)\left(q_\lambda \beta\psi(r^\prime) + v_\lambda \beta \Pi(r^\prime)\right)\right]
		\end{aligned}
	\end{align}
	
	\noindent en donde se ha redefinido la activdiad de la prote\'ina como:
	
	\begin{align}
		\tilde{a}_{pro} = \exp[\beta\mu_{pro} - \beta\tilde{\mu}^0_{pro}]
	\end{align}
	
	En donde:
\begin{align}
	\beta\tilde{\mu}^0_{pro} =  \beta \mu^0_{pro}  + \sum_{\tau,a} C_{n,\tau}\beta\mu^0_{\tau,d} 
	+ \sum_{\tau,b} C_{n,\tau}\beta(\mu_{H^+} - \mu^0_{\tau,p})
\end{align}


\noindent $\tau,a$ y  $\tau,b$ suman obre segmentos \'acidos o b\'asicos respectivamente. Adem\'as hemos definido el n\'umero de composici\'on para un segmento $k$, $C_{n,k}$:
	\begin{align}
		\left[\int_0^\infty dr^\prime G(r^\prime) n_k(\theta,r,r^\prime) = C_{n,k}\right] \quad \forall \, r
		\label{eq:esf:composition}
	\end{align}

	La probabilidad de una conformaci\'on $\alpha$ es expresada como:
	
	\begin{align}
		\begin{aligned}
			P(\alpha)&=\frac{1}{Q}\exp\left[- \sum_i{\int{drG(r)\beta\Pi(r)\phi_i(\alpha,r)}}\right] \\
			& \times \exp\left[\int{\ln(1-f(r))drG(r)\frac{\phi(\alpha,r)}{v_{MAA}}}\right], \\
		\end{aligned}
	\end{align}
	
	\noindent en donde  $Q$ es una constante que satisface: $\sum_\alpha P(\alpha) = 1$
	
	La variaci\'on respecto al potencial electrost\'atico deriva en la ecuaci\'on de Poisson:
	
	\begin{align}
		\epsilon\Delta\psi(r) = -\left<\rho_q(r)\right>
		\label{eq:esf:poisson}
	\end{align}
	
	En la ecuaci\'on anterio se considera la densidad de carga media, la cual definimos en la ecuaci\'on  \ref{eq:esf:rho-charge}.  Para nuestro sistema, adem\'as de la restricci\'on en el volumen, se incluye la electroneutralidad de la soluci\'on, definida por:

	\begin{align}
		\int_0^\infty{drG(r) \left<\rho_q(r)\right>} = 0
	\end{align}
	
	
Para que las condiciones de electroneutralidad y ecuaci\'on se cumplan, es necesario el uso de ciertas condiciones de contorno:	

	\begin{align}
		\begin{aligned}
			&i)  \lim_{r\to\infty}\psi(r) = 0 \\
			&ii) \left.\frac{d\psi(r)}{dr}\right|_{r=0} = 0
			\label{eq:esf:contorno}
		\end{aligned}
	\end{align}
	
Al observar las expresiones de todos los funcionales presentados en ec. \ref{eq:esf:funcionales-omega}
notamos que  todos son definidos por dos potenciales locales: Electrost\'atico $\Psi(r)$ y Presi\'on osm\'otica $\Pi(r)$. Entre otros par\'ametros como las actividades de cada especie y otras cantidades proporcionadas con el modelo molecular a usar.
En la siguiente secci\'on veremos como las actividades de cada componente quedan determinadas con las condiciones iniciales del bulk de la soluci\'on: pH, temperatura, y concentraci\'on de adsorbatos y sal.
Con la determinaci\'on de las activdades es posible resolver el sistema completo haciendo uso de las ecuaciones \ref{eq:esf:constraint} y \ref{eq:esf:poisson} por sustituci\'on de cada una de las expresiones para las densidades locales y el grado de disociaci\'on, as\'i como tambi\'en de la probabilidad de la conformaci\'on de la red polim\'erica. 

\subsection{Soluci\'on Bulk}\label{sec:esf:bulk}

La composici\'on qu\'imica de la soluci\'on a bulk en encuentra en equilibrio termodin\'amico con nuestro nanogel. Qu\'imicamente hablando nos indica que los potenciales qu\'imocos de las especies con m\'obilidad son iguales en cualquier punto del sistema. 
El calculo de estos potenciales, y con ello sus actividades nos proveen  las soluciones iniciales para la resoluci\'on de todo el sistema.
En esta secci\'on, expresamos esas actividades en t\'erminos de la composici\'on qu\'imica de la soluci\'on bulk.

La soluci\'on bulk  puede considerarse como el límite $r \rightarrow \infty$:
\begin{align}
	\begin{aligned}
		& i)\rho^b_\gamma =\rho_\gamma (r \rightarrow \infty) \\
		& ii) \Pi^b = \Pi(r \rightarrow \infty) \\
		& iii) f_\tau^b = f_\tau(r \rightarrow \infty)
	\end{aligned}
\end{align}

Adem\'ass, las condiciones de contorno expresadas en ec. \ref{eq:esf:contorno} implican que:
\begin{align}
	\Psi(r \rightarrow \infty) = 0
\end{align}

En este contexto, para las especies libres (excluyendo la prote\'ina):
\begin{align}
	\rho_\gamma^b v_w = a_\gamma e^{-\beta\Pi^bv_w}
	\label{eq:esf:free-bulk}
\end{align}

El grado de disociaci\'on $f_j$ de la prote\'ina y los segmentos de la red pueden escribirse como:

\begin{align}
	\frac{f_j^b}{1-f_j^b} = \left(\frac{a_{H^+}}{K^0_{a,j}}\right)^{\mp 1}
\end{align}

Finalmente, la densidad de la prote\'ina $\rho_{pro}^b(\theta)$ es:

\begin{align}
	\begin{aligned}
		\rho^b_{pro}(\theta)v_w = &\tilde{a}_{pro} \prod_\tau\exp\left[-C_{n,\tau} \ln f^b_\tau\right] \\
		&\prod_\lambda \exp \left[-C_{n,\lambda} (v_\lambda\beta\Pi^b ) \right]
	\end{aligned}
	\label{eq:esf:bulk-protein}
\end{align}

donde $C_{n,\lambda}$ es el n\'umero de composici\'on para el segmento $\lambda$, definido en ec.  \ref{eq:esf:composition}.

Para la soluci\'on bulk, la restricci\'on de incompresibilidad est\'a dada por:

\begin{align}
	\begin{aligned}
		1= {\sum_{\gamma}\rho^b_\gamma v_\gamma + \sum_\lambda{\left<\rho^b_{pro,\lambda}\right>v_\lambda} }
	\end{aligned}
	\label{eq:esf:bulk-constraint}
\end{align}


En esta ocaci\'on, dado que $\Psi^b = 0$, se observa que los funcionales en el bulk de la soluci\'on se encuentran establecidos por el potencial de presi\'on $\Pi^b$, siendo esta nuestra variable a resolver.

Lo que se puede calcular utilizando las ecuaciones de densidad de las especies libres en soluci\'on, ec.  \ref{eq:esf:free-bulk} y la prote\'ina ec. \ref{eq:esf:bulk-protein} en la nueva restricci\'on, \ref{eq:esf:bulk-constraint}.


\subsection{Resoluci\'on num\'erica}


Para obtener resultados de la minimizaci\'on de la energ\'ia, las ecuaciones integro-diferenciales no lineales descritas anteriormente ( secciones \ref{sec:esf:tm} y \ref{sec:esf:bulk}) deben resolverse num\'ericamente. Para lograr esto, el volumen del sistema se divide en capas de espesor $\delta = 0.5$. En esta divisi\'on se ha considerado una simetr\'ia radial 

 En las ecuaciones presentadas, las sumas sobre capas reemplazan las integrales a lo largo de la coordenada $r$, mientras que las diferencias finitas reemplazan las derivadas.

Reescribiendo, la restricci\'on de incomprensibilidad se expresa como:

\begin{align}
	\begin{aligned}
		1=  {\sum_{\gamma}\rho_\gamma(i_r) v_\gamma + \sum_\lambda{\left<\rho_{pro,\lambda}(i_r)\right>v_\lambda} + \sum_i{\left<\phi_i(i_r)\right>}}
		\label{eq:esf:pi-ir}
	\end{aligned}
\end{align}

Lo que nos da una ecuaci\'on para cada capa $i_r$, en donde cada posici\'on es descrita por la coordenada $r_i = (i_r -0.5)\delta$. 
La variable $i_r$ toma valores de $1$ a $n)r$, en donde $n_r$ es un n\'umero suficientemente grande de capas para que se satisfagan las restricciones impuestas en nuestro sistema. Entre ellas
 $\rho_\gamma(n_r) \approx \rho_\gamma^b$, $\rho_{pro}(\theta,n_r) \approx \rho_{pro}^b(\theta)$ and $\psi(n_r) \approx \psi^b = 0$.

Con estas consideraciones podemos reescribir:


\begin{align}
	\frac{f(j_r)}{1-f(j_r)}= \left(\frac{a_{H^+}}{k^0_{a,j}}\right)^{\mp 1} e^{-\beta q_{MAA^-}\psi(i_r)}
\end{align}


For the free species:

\begin{align}
	\rho_\gamma(i_r)v_w = a_\gamma \exp{[-\beta \psi(i_r)q_\gamma]} \exp{[-\beta\Pi(i_r) v_w]}
\end{align}

Para las especies libres, su densidad local se escribe ahora:

\begin{align}
	\begin{aligned}
		\rho_{pro}(\theta, i_r)v_w = &\tilde{a}_{pro} \prod_\tau\exp\left[- \delta \sum^{n_r}_{j_r = 1} G(j_r) n_\tau(\theta,i_r,j_r) \ln f_\tau(j_r)\right] \\
		& \prod_\lambda \exp \left[- \delta \sum^{n_r}_{j_r = 1} G(j_r) n_\lambda(\theta,i_r, j_r)[v_\lambda\beta\Pi(j_r) + q_\lambda \Psi(j_r)] \right]
	\end{aligned}
\end{align}

La probabilidad de las configuraciones de la red polim\'erica $P(\alpha)T$: 

\begin{align}
	\begin{aligned}
		P(\alpha)&=\frac{1}{Q}\exp\left[- \delta\sum_i{\sum_{i_r =1}^{n_r}{G(i_r)\beta\Pi(i_r)\phi_i(\alpha,i_r)}}\right] \\
		& \exp\left[\delta\sum_{i_r =1}^{n_r}{G(i_r)\ln(1-f(i_r))\frac{\phi(\alpha,i_r)}{v_{MAA}}}\right]
	\end{aligned}
\end{align}

La ecuaci\'on de Poisson se escribe:

\begin{align}
	\epsilon \frac{\Psi(i_r +1) -2 \Psi(i_r) + \Psi(i_r -1)}{\delta ^2} + 2\epsilon \frac{\Psi(i_r +1) -\Psi(i_r)}{(i_r -0.5)\delta ^2}= -\left<\rho_q(i_r)\right>
	\label{eq:esf:poisson-ir}
\end{align}

\noindent en donde la densidad de carga se define:

\begin{align}
	\left<\rho_q(i_r)\right> = \sum_{\gamma } {\rho_\gamma(i_r) q_\gamma + \sum_\tau{f_\tau(i_r) \left<\rho_{pro,\tau}(i_r)\right> q_\tau} +  f(i_r)\dfrac{\left<\phi_{MAA}(i_r)\right>}{v_{MAA}}q_{MAA}}
\end{align}

Nuestras condicones de contorno se reescriben:
\begin{align}
	\frac{\Psi(1) - \Psi(0)}{\delta} = 0
\end{align}

Como se definio en el cap\'itulo donde hablamos sobre los films polim\'ericos,\ref{Chapter2}, definiendo  el pH, concentraci\'on de sal y prote\'ina. temperatura, es posible calcular las variables restantes  $\Pi(i_r)$ y $\Psi(i_r)$ para cada capal $i_r$.
Variables que pueden ser obtenidas al resolver en cada capa las ecuaciones \ref{eq:esf:pi-ir}) y (\ref{eq:esf:poisson-ir}. \ref{eq:film:discreto-poisson}.
De esta forma el n\'umero de ecuaciones totales a resolver es $2n_r$ (dos por cada capa). 
Este sistema de ecuaciones es resuelto usando el m\'etodo de Newton con Jacobiano libre, implementado en c\'odigos FORTRAN desarrollados en el grupo de trabajo.



%%%%%%%%%%%%%%%%%%%%%%%%%%%%%%%%%%%%%%%%%%%%%%%%%%
\subsection{Molecular Model: Proteins}\label{sect:protein}
%%%%%%%%%%%%%%%%%%%%%%%%%%%%%%%%%%%%%%%%%%%%%%%%%%


We consider three different proteins:  cytochrome c, insulin and myoglobin.
To describe these molecules we use a coarse-grained model where each amino acid residue is described by a single particle centered at the position of the $\alpha$-carbon.
The sequence and position of all $\alpha$-carbons are taken from the crystallographic structure in the corresponding protein data bank entry \addcite[berman2000protein]: 2B4Z for cytochrome c \addcite[mirkin2008high], IZNI for insulin \addcite[bentley1976structure], and 3RGK for myoglobin \addcite[hubbard1990x]. 
 
 

The coarse-grained particles in this model are each assigned a volume and a pKa (if the unit is titratable) according to the amino acid that they represent; 
this is summarized in  tabla \ref{table:Coarse-grain}.
These pKa's are taken from experimental data and represent average values over a large number of proteins \addcite[grimsley2009summary].
In most occurrences of a residue, its pKa does not significantly deviate from the average value.
In  specific instances, however, some residues display a different pKa;
these special cases are described in the SI.


\begin{table}
\centering
\small
\begin{tabular}{|lcc|lcc|}
\hline
grupo & v($nm^{-3}$) & pka & grupo & v($nm^{-3}$) & pka \\
\hline
Ala & 0.067 &  & Pro & 0.090 & \\
Arg & 0.148 & $12.5 (+)$& Ser & 0.073 &\\
Asn & 0.096 &  & Thr & 0.093 & \\
Asp & 0.091 & $3.5 (-)$ & Trp & 0.163 &\\
Cys & 0.086 &  & Tyr & 0.141 & $10.3 (-)$\\
Gln & 0.114 & & Val & 0.105 &\\  
Glu & 0.109 & $4.2 (-)$ & H$_2$O & 0.033 & \\ 
Gly & 0.048 &  & OH$^-$ & 0.033 & \\
His & 0.118 & $6.6 (+)$& H$_3$O$^+$ & 0.033 &  \\ 
Ile & 0.124 &  & Na$^+$ & 0.043 & \\ %
Leu & 0.124 &  & Cl$^-$ & 0.047 & \\
Lys & 0.135 & $10.5 (+)$ & AH & 0.068 &  9.5(+)\\
Met & 0.124 & & MAA & 0.085 & $4.65(-)$\\
Phe & 0.135 &   & VA & 0.085 & \\
\hline
\end{tabular}
\caption{Volume and pKa of the coarse-grained particles (amino acid residues, small ions, solvent molecules and polymer segments)  considered in our molecular model.}
\label{table:Coarse-grain} 
\end{table}


Using this molecular model,  figura \ref{fig:protein-charge} shows the charge (number) of the three proteins in dilute solution as a function of  pH.
The isoelectric point (pI) is the pH at which the net charge of a protein is zero.
From the graph, we obtain the values  9.65 (9.6 \addcite[hristova2019isoelectric]), 5.5 (5.3 \addcite[guckeisen2019isoelectric]), 7.15 (7.2 \addcite[batys2020myoglobin]) for the pI of cytochrome c, insulin, and myoglobin respectively;
the values in parentheses are the pI of the proteins reported experimentally. 


 \begin{figure}[!htb]
     \centering
     \includegraphics[width=0.65\textwidth]{Figures/graphs-gel2/protein-model.png}
     \caption{Left: Charge number of the proteins in a dilute solution as a function of  pH (solid-line curves);
     filled circles mark  the isoelectric point,
     where the net charge of the protein is zero.
     The coarse-grained representation of the proteins is illustrated on the right, where amino acid residues are represented by a single sphere (red: acidic; blue: basic; gray: charge-neutral residues).}
     \label{fig:protein-charge}
 \end{figure}






 





%%%%%%%%%%%%%%%%%%%%%%%%%%%%%%%%%%%%%%%%%%%%%%%%%%
\subsection{Molecular model: Nanogel Network}
%%%%%%%%%%%%%%%%%%%%%%%%%%%%%%%%%%%%%%%%%%%%%%%%%%

Besides the protein model presented in sec.  \ref{sect:protein},   we need to specify a molecular model to describe the polymer network that makes the nanogel backbone.
Such model must provide  a set of molecular configurations of the polymer network that is representative of the whole conformational space.
A particular network conformation is given by the spacial position of all its segments.


The nanogel network is composed of 25 segment-long  crosslinked polymer chains. 
In total this network contains 10054 segments.
Each segment is a coarse-grained   representation of either a crosslink, a charge neutral unit (VA) or an acid/basic monomer (MAA/AH). 
tabla \ref{table:Coarse-grain} includes the volume and pKa (if the unit is titratable) used to describe these coarse-grained units.

The polymer network has diamond-like topology, where  crosslinks are placed at the original position of carbon atoms and connected to four polymer chains.
To build this network, we first construct a three-dimensional structure where all the polymer chains are elongated, to then only keep the segments contained within a sphere of radius $R_{cut}$ placed at the center of mass of the structure; $R_{cut}$ is selected such that the network will have 10000 segments approximately.
Originally, all polymer chains connect two crosslinks, but as a result of this procedure some will be left dangling on the network surface,  connected to a single crosslink.
Most of these superficial \emph{dangling} chains are shorter than 25 segments.
Altogether these chains contain 22\% of the total number of segments.
To generate the different molecular conformations of the polymer network, we have performed Molecular Dynamics simulations using GROMACS 5.1.2 \addcite[indahl2001gromacs] (details are given in the SI).

 \begin{figure}[!htb]
     \centering
     \includegraphics[width=0.99\textwidth]{Figures/graphs-gel2/paper2.png}
     \caption{A: The nanogel network consists of crosslinked copolymer chains of a charge-neutral segment (VA: vinyl alcohol) and a functional unit (either MAA: methacrylic acid or AH: allylamine).
    This scheme illustrates the three different comonomer distributions considered;  from left to right: RF: a random distribution of functional groups throughout the network; CF: the functional units occupy the center/core of the network; SF: only the free-end dangling chains on the network surface are functionalized with pH-responsive units.
B: Plot of the ideal pH-dependent degree of charge of the isolated functional unit in dilute solution.}
     \label{fig:gel-topologies}
 \end{figure}














We consider different pH-responsive nanogels, containing either acid (MAA) or basic (AH) groups, and evaluate three different topologies for the spatial distribution of these functional  segments, which are schematized in figura \ref{fig:gel-topologies}: 
(i) a \emph{randomly functionalized} (RF) structure where the pH-responsive segments are spread throughout the network  at random,
(ii) a \emph{core functionalized} (CF) structure, where the  pH-sensitive units occupy the center of the network, and 
(iii) a \emph{surface functionalized} (SF) structure in which only the dangling chains on the network surface  are ionizable. 

 
 
  







%%%%%%%%%%%%%%%%%%%%%%%%%%%%%%%%%%%%%%%%%%%%%%%%%%
\section{Results and discussion}
%%%%%%%%%%%%%%%%%%%%%%%%%%%%%%%%%%%%%%%%%%%%%%%%%%






%%%%%%%%%%%%%%%%%%%%%%%%%%%%%%%%%%%%%%%%%%%%%%%%%%
\subsection{Caraterizaci\'on del nanogel}
%%%%%%%%%%%%%%%%%%%%%%%%%%%%%%%%%%%%%%%%%%%%%%%%%%
En esta primera instancia y como venimos mostrando cap\'itulo a cap\'itulo examinaremos el comportamiento (la respuesta) de los nanogeles en funci\'on del pH cuando en ausencia de  prote\'inas.

Para cuantificar el tama\~no de un nanogel, usaremos el radio  de la partí\'icula, $R$, que se puede calcular usando:
\begin{align}
    R= \frac{4}{3}\frac{\int_0^\infty{dr\,G(r)\,r \left<\phi(r)\right>}}{\int_0^\infty{dr\,G(r)\left<\phi(r)\right>}}
\end{align}
donde $r$ es la distancia desde el centro de masa de la red polim\'erica (en nuestra teor\'ia asume simetr\'ia radial);
$\left<\phi(r)\right>$ es la fracci\'on de volumen local del pol\'imero (incluye todos los tipos de segmentos);
los corchetes angulares indican el promedio de ensamble sobre las diferentes conformaciones de la red (ver ecuaci\'on \ref{eq:esf:ensamble-gel});
$G(r)=4\pi r^2$ es la constante, a un $r$ dado, por la simetr\'ia radial del sistema.

\begin{figure}[!htb]
     \centering
     \includegraphics[width=0.9\textwidth]{Figures/graphs-gel2/rr.png}
     \caption{Conjunto de radio promedio, R, en funci\'on del pH para nanogeles de copo\'imero MAA-VA (panel A) y AH-VA (panel B).
     	Se consideran tres estructuras diferentes en cada caso donde las unidades funcionales (MAA/AH) se distribuyen aleatoriamente a lo largo de la red polim\'erica (RF), ocupan el centro de la red (CF), o modifican las cadenas colgantes dentro del pol\'imero. interfaz de soluci\'on (SF).
     	En todos los casos, el $22\%$ de los segmentos de estas redes son sensibles al pH; La concentraci\'on de NaCl es $10^{-3}M$.}
     \label{fig:gel-charge-MAA-AH}
\end{figure}

La Figura \ref{fig:gel-charge-MAA-AH} muestra el radio promedio $R$ en funci\'on del pH para las tres diferentes estructuras consideradas: RF, CF y SF.
El panel A describe un nanogel cuyo segmeto ionizables es el $MAA$ , mientras que el panel B describe un nanogel basado en $AH$.
En ambos casos, la concentraci\'on de sal es de 1\,mM, y la fracci\'on de mon\'omero funcional ($MAA$ o $AH$) es de $22\%$.
Los nanogeles basados en $MAA$ funcionalizados al azar (RF) y en el n\'ucleo (CF) se hinchan al aumentar el pH (panel A).
Esto se debe a que los segmentos $MAA$ se desprotonan y se cargan el\'ectricamente a medida que aumenta el pH (ver figura \ref{fig:gel-topologies}B), lo que da como resultado repulsiones electrost\'aticas dentro de la red.
La distancia entre las unidades $MAA$ cargadas debe aumentar para reducir estas interacciones repulsivas, con lo cual la red aumenta su tama\~no para colocar estos segmentos m\'as separados.
Resumiendo, para disminuir la repulsi\'on entre unidades $MAA$ cargadas, su distancia espacial debe aumentar, lo que resulta en una expansión neta de la red.




La red $MAA$ funcionalizada en superficie, por otro lado, muestra un comportamiento de expansi\'on completamente diferente en figura \ref{fig:gel-charge-MAA-AH}A.
Este nanogel se deshincha a medida que las unidades titulables se cargan al aumentar el pH.
Para explicar este comportamiento contrario a lo que se espera, observamos la distribuci\'on local de segmentos dentro de nuestras estructuras en diferentes condiciones. Utilizamos la distribuci\'on radial de los mon\'omeros funcionales.
Para los nanogeles $MAA$, dicha cantidad se define como:

%
\begin{align}
    \lambda_{MAA}(r)= 4\pi r^2\left<\phi_{MAA}(r)\right>
\end{align}
%
\noindent en donde $\left<\phi_{MAA}(r)\right>$ da la fracci\'on de volumen local de los segmentos de \'acido metacr\'ilico.
Hay que tener en cuenta que $\lambda_{MAA}(r) dr$ da el n\'umero de segmentos MAA en la capa esf\'erica entre $r$ y $r+dr$ medidos desde el centro del nanogel.
Adem\'as, la integral $\int_0^\infty \lambda_{MAA}(r) dr$ da el n\'umero total de mon\'omeros MAA en la red.


\begin{figure}[!htb]
     \centering
     \includegraphics[width=0.45\textwidth]{Figures/graphs-gel2/dist-MAA.png}
     \caption{Distribuci\'on radial de segmentos MAA, $\lambda_{MAA}(r)$, a pH 3 y 7, y $10^{-3}M$ NaCl; cada panel corresponde a un nanogel de  MAA-VA diferente que tienen una funcionalizaci\'on de red particular y 22\% MAA.
     	Estos grupos funcionales est\'an completamente protonados (sin carga) a pH 3 y completamente disociados (cargados) a pH 7.}
     \label{fig:MAA-vs-r-distribution}
 \end{figure}
 %\FloatBarrier


La figura \ref{fig:MAA-vs-r-distribution} muestra la distribuci\'on radial de los segmentos de $MAA$ para las diferentes redes consideradas.
En cada caso se incluyen  resultados para una solución de pH 3, donde los segmentos $MAA$ tienen carga neutra, y pH 7 donde est\'an completamente cargados (ver figura \ref{fig:gel-topologies}B);
Para una funcionalizaci\'on aleatoria (panel A), la distribuci\'on de los segmentos $MAA$ se desplaza hacia la interfaz de soluci\'on del nanogel a medida que la red se carga el\'ectricamente cuando aumenta el pH.
Este desplazamiento se produce para reducir las repulsiones electrost\'aticas entre los segmentos $MAA$ cargados.

Como resultado, toda la distribuci\'on del pol\'imero tambi\'en se extiende %(ver \cref*{fig:allseg_si}\hl{A y B en el SI})
, incluidas las unidades VA de carga neutra. \textcolor{red}{AGREGAR LOS GRAFICOS DE SI}
El mismo comportamiento tiene lugar para una funcionalizaci\'on central (panel B),
aunque por dise\~no, los segmentos $MAA$ en esta red, ya sea que est\'en cargados o no, es m\'as probable que ocurran a distancias m\'as cortas del centro del nanogel en comparaci\'on con las otras estructuras.
El desplazamiento de segmentos a $r$ m\'as altos observado en los paneles figura \ref{fig:MAA-vs-r-distribution}A y B explica el aumento del tama\~no promedio del nanogel con pH visto en la figura \ref{fig:gel-charge-MAA-AH}A para las estructuras RF y CF.


Por otro lado,
La Figura \ref{fig:MAA-vs-r-distribution}C muestra que la distribuci\'on MAA del nanogel con superficie funcionalizada se desplaza hacia adentro cuando la red se carga con pH.
Para reducir las repulsiones dentro de la red, las cadenas lindantes de PMAA, que se asientan en la superficie del nanogel a un pH bajo, tambi\'en intentan ocupar el volumen dentro de la red cuando están cargadas.
Este cambio hacia adentro de la distribuci\'on de segmentos de pol\'imero %(\hl{ver también} \cref*{fig:allseg_si}C)
explica el comportamiento de deshinchamiento del nanogel SF MAA con el aumento del pH que se observa en la figura \ref{fig:gel-charge-MAA-AH}A.
N\'otese, sin embargo, que a pesar de este desplazamiento parcial hacia el interior de la red, la posici\'on m\'as probable de los segmentos MAA es siempre la interfaz pol\'imero-soluci\'on para soluciones de pH alto y bajo.



El comportamiento de los nanogeles basados en AH es an\'alogo al de las redes basadas en MAA, pero en respuesta al cambio de pH en la direcci\'on opuesta.
Los grupos AH se protonan y se cargan positivamente con la disminuci\'on del pH (ver figura \ref{fig:gel-topologies}B).
Para nanogeles AH funcionalizados aleatoriamente y con n\'ucleo, este aumento en la carga el\'ectrica con la disminución del pH provoca un desplazamiento hacia afuera de la distribuci\'on del segmento % (ver \cref*{fig:AHseg_si}A y B)
, lo que explica la hinchaz\'on de la figura \ref{fig:gel-charge-MAA-AH}B;
mientras tanto, para la estructura SF, el deshinchamiento con la disminuci\'on del pH que se ve en la figura \ref{fig:gel-charge-MAA-AH}B es consistente con un desplazamiento hacia adentro del pol\'imero %(ver \cref*{fig:AHseg_si}C) .





%%%%%%%%%%%%%%%%%%%%%%%%%%%%%%%%%%%%%%%%%%%%%%%%%%
\subsection{Protein adsorption to MAA-based nanogels}\label{sec:MAA-NGs}
%%%%%%%%%%%%%%%%%%%%%%%%%%%%%%%%%%%%%%%%%%%%%%%%%%



%%%%%%%%%%%%%%%%%%%%%%%%%%%
%%%%% Define Gamma and N(r)
%%%%%%%%%%%%%%%%%%%%%%%%%%%

En la secci\'on anterior evalu\'o el impacto de la funcionalizaci\'on de la red y la composici\'on qu\'imica en la respuesta del nanogel a las variaciones de pH en ausencia de prote\'inas.
La reorganizaci\'on de los segmentos de pol\'imero como resultado de los cambios de pH depende de una elecci\'on de dise\~no: la distribuci\'on de unidades funcionales dentro de la red.
Ahora examinaremos el impacto de esta reorganizaci\'on de pol\'imeros en el nivel de adsorci'on de prote\'inas en diferentes nanogeles y la distribuci\'on espacial de las prote\'inas adsorbidas.
Esta secci\'on analiza la adsorció\'on del citocoma c y la mioglobina en diferentes estructuras de nanogel basadas en MAA.
Los resultados de la insulina se omiten en esta secci\'on porque, debido a su bajo punto isoel\'ectrico, esta prote\'na no se adsorbe en nanogeles basados en MAA. % (see fig. \cref*{fig:adsoprtion-vs-pH-insulinMAA_si} in the SI).

Considere un nanogel polim\'erico centrado en $r=0$ en contacto con una soluci\'on acuosa de prote\'ina.
El n\'umero de prote\'inas adsorbidas dentro de la capa esf\'erica entre $r$ y $r+dr$ viene dado por la cantidad en exceso

\begin{align}
     \langle N(r)\rangle dr = 4\pi r^2 \left(\langle\rho(r)\rangle - \rho_{bulk}\right) dr
\end{align}
%
en donde $\left<\rho(r)\right>$ y $\rho_{bulk}=\lim\limits_{r\to \infty } \langle\rho(r)\rangle$ son respectivamente la densidad (en n\'umero) local y en el bulk de la prote\'ina.
La integraci\'on de $\langle N(r)\rangle$ produce la \emph{adsorci\'on en exceso} (en adelante, simplemente la adsorci\'on) que cuantifica el n\'umero de prote\'inas incorporadas a la red de pol\'imeros,


%
\begin{align}
    \Gamma =  \int_0^\infty{  \langle N(r)\rangle dr}
\end{align}
%

%%%%%%%%%%%%%%%%%%%%%%%%%%%
%%%%% Adsorption to MAA NGs
%%%%%%%%%%%%%%%%%%%%%%%%%%%


\begin{figure}[!htb]
\includegraphics[width=0.9\textwidth]{Figures/graphs-gel2/abcd.png}
\caption{Gr\'aficos del exceso de adsorci\'on $\Gamma$ de citocromo c (paneles A y B) y mioglobina (paneles C y D) a nanogeles MAA-VA en funci\'on del pH.
	La concentraci\'on de sal es $10^{-3}M$ en los paneles del lado izquierdo (A y C) y $10^{-2}M$ en los paneles del lado derecho
	(B y D); Se muestra una ampliaci'on de la adsorci\'on.
	Estos nanogeles tienen 22\% MAA; la concentraci\'on de prote\'ina es $10^{-6}M$.}
\label{fig:adsorption-vs-pH-cyto-myo}
\end{figure}



 
 La Figura \ref{fig:adsorption-vs-pH-cyto-myo} muestra la adsorci\'on de soluciones de prote\'ina \'unica de citocromo c (paneles superiores, A y B) y mioglobina (paneles inferiores, C y D) a nanogeles MAA-VA que tienen diferentes funcionalizaciones de red;
 El pH es la variable independiente de estos c\'alculos, pero tambi\'en evaluamos el efecto de la concentraci\'on de NaCl comparando diferentes paneles en la misma l\'inea.
 Los nanogeles de Figura \ref{fig:adsorption-vs-pH-cyto-myo} contienen $22\%$ MAA, lo que significa que todos los segmentos en las cadenas superficiales de la red SF son MAA.
 Primero, discutiremos las caracter\'isticas de la figura \ref{fig:adsorption-vs-pH-cyto-myo} comunes a todos los paneles para luego concentrarnos en el efecto de la funcionalizaci\'on de la red.
 
 
 
 La adsorci\'on de prote\'inas es una funci\'on no monot\'onica del pH con un m\'aximo en la regi\'on entre pH 5-7, que depende de la concentraci\'on de sal y la prote\'ina espec\'ifica.
 Esta respuesta del pH puede explicarse en t\'erminos de las interacciones electrost\'aticas y el comportamiento de protonaci\'on tanto de los segmentos MAA como de las prote\'inas.
 Las unidades \'acidas del pol\'imero se disocian y la red se carga negativamente cuando el pH aumenta por encima de su pKa intr\'inseco (4,65 para MAA).
 Por encima de este pH, pero por debajo de su punto isoel\'ectrico, la prote\'ina tiene carga positiva.
 En estas condiciones, las atractivas interacciones prote\'ina-red impulsan la adsorci\'on.
 Sin embargo, en ambos lados de la escala de pH, estas interacciones son insignificantes (pH bajo) porque el MAA est\'a protonado y tiene carga neutra o repulsivo (pH alto) porque las prote\'inas tienen carga negativa.
 Cualquiera de las dos situaciones conduce a la ausencia de adsorci\'on de prote\'inas ($\Gamma\approx 0$) o desorci\'on ($\Gamma< 0$).
 
 
 
 En general, las adsorciones de citocromo c y mioglobina son cualitativamente similares.
 Hay dos diferencias principales: (i) la magnitud de la adsorci\'on (el citocromo c se adsorbe significativamente m\'as) y (ii) el citocromo c se adsorbe en un rango de pH m\'as amplio, lo que se debe a su punto isoel\'ectrico m\'as alto (9,65 en comparaci\'on con 7,15). para la mioglobina).
 Esto tambi\'en implica que, cuando las dem\'as condiciones son las mismas, el nivel m\'aximo de adsorci\'on de citocromo c tiene lugar a un pH ligeramente superior.
 
 En relaci\'on con las otras configuraciones, la figura \ref{fig:adsorption-vs-pH-cyto-myo} muestra que la distribuci\'on central de los segmentos MAA conduce a una adsorci\'on significativamente mayor en la mayor\'ia de las condiciones.
 Mostramos que este comportamiento ocurre porque tal distribuci\'on de segmentos MAA permite una incorporaci\'on m\'as efectiva de la prote\'ina adsorbida con carga el\'ectrica opuesta.
 Por otro lado, el comportamiento de adsorci\'on de las redes funcionalizadas aleatoriamente y de superficie es sorprendentemente similar dentro del pH y las concentraciones de sal estudiadas, y tambi\'en para las diferentes prote\'inas.
 La distribuci\'on de unidades funcionales entre estructuras RF y SF difiere mucho a pH bajo.
 Sin embargo, tras la reorganizaci\'on del polí\'imero a un pH m\'as alto a medida que se cargan las unidades MAA, estas distribuciones se vuelven relativamente similares entre s\'i (compare los paneles A y C de la Figura \ref{fig:MAA-vs-r-distribution}), lo que explica la adsorci\'on de prote\'inas comparable observado en nanogeles RF y SF.




%%%%%%%%%%%%%%%%%%%%%%%%%%
%%%%% Protein localization
%%%%%%%%%%%%%%%%%%%%%%%%%%

\begin{figure}[!htb]
     \centering
     \includegraphics[width=0.45\textwidth]{Figures/graphs-gel2/cyto-adsr-pmf.png}
     \caption{Panel A: Gr\'afico de la distribuci\'on radial de las mol\'eculas de citocromo c, $\langle N(r)\rangle$, en funci\'on de la posici\'on de los nanogeles MAA-VA con diferentes funcionalizaciones.
     	Estas redes tienen 22\% MAA, el pH es 7, la concentraci\'on de prote\'ina es de $10^{-6}M$ y la concentraci\'on de NaCl es de $10^{-3}M$.
     	El panel B muestra el potencial de la fuerza media, ${PMF}(r)$, que act\'ua sobre el citocromo c para las mismas condiciones que el panel A.}
     \label{fig:adsorption-vs-r-cyto}
 \end{figure}



Para explicar el mejor rendimiento de los nanogeles MAA funcionalizados con n\'ucleo en la incorporaci\'on de prote\'inas,
La figura \ref{fig:adsorption-vs-r-cyto}A muestra la distribuci\'on radial de las mol\'eculas de citocromo c en funci\'on de la distancia $r$ al centro de masa del nanogel.
La soluci\'on tiene pH 7 y NaCl 1\,mM, lo que corresponde aproximadamente a las condiciones de m\'axima adsorci\'on de esta prote\'ina en la Figura \ref{fig:adsorption-vs-pH-cyto-myo}A.
Existe una clara correlaci\'on entre la distribuci\'on de los grupos funcionales a lo largo de la red polim\'erica y la ubicaci\'on del citocromo c adsorbido.



When the center of the network is functionalized, the highest probability of finding the proteins occurs deep inside the nanogel between 20-30\,nm.
In Figura \ref{fig:adsorption-vs-r-cyto}A the maximum number of adsorbed proteins occurs at $r=28$\,nm for $1$\,mM NaCl, and at $r=25$\,nm for 10\,mM NaCl% (\hl{see} \cref*{fig:cyto-vs-r-1d-2_si}).
Congruently, the distribution profile of charged MAA displays a shallow maximum in this spatial region (see figura \ref{fig:MAA-vs-r-distribution}B, red curve).
Namely, adsorbed proteins sit where they can be surrounded by oppositely charged network segments.
Interestingly, the same phenomenon takes place in the adsorption to RF and SF nanogels.
The distributions of charged MAA display a sharp maximum near the nanogel surface, between 45-50\,nm (see red curves in  figura \ref{fig:MAA-vs-r-distribution}, panels A and C).
Fig 6A shows that cytochrome c is most likely to adsorb next to these  regions of high MAA (charge) density.



When comparing the distributions of cytochrome c inside RF and SF nanogels in figura \ref{fig:adsorption-vs-r-cyto}A, we observe that the profiles are relatively similar to each other.
As expected, if only the surface is functionalized, the protein profile shifts towards the polymer-solution interface.
To further quantify the interaction with the nanogels, we use the potential of mean force acting on a protein at a distance $r$ from the center of the polymer network, defined as:
\begin{align}
   {PMF} (r) = -k_B T \ln \frac{\langle \rho(r)\rangle}{\rho_{bulk}}
\end{align}
where $\lim\limits_{r\to \infty}{PMF}(r)=0$, which means that the nanogel-protein interaction vanishes when they are sufficiently far apart.





Figura \ref{fig:adsorption-vs-r-cyto}B shows ${PMF}(r)$ acting on cytochrome c at the same conditions of panel A, and for the three different nanogel functionalizations.
Deep inside the nanogel, protein interaction with the CF structure is the strongest: $-8k_B T$ approximately in the spatial range from $r=0$ to 30\,nm.
This interaction is relatively short-ranged because it decreases significantly above $r approx 40$\,nm.
On the other hand,  cytochrome c interactions with RF and SF nanogels extend longer, up to $55-60$\,nm.
Inside the nanogel, these interactions are weaker than that with the CF structure.
The adsorption free energy is $\sim -6 k_BT$ and it remains roughly constant inside the RF nanogel.
For the surface-functionalized nanogel, on the other hand, the minimum of ${PMF}(r)$ is also $\sim -6 k_BT$, occuring next to the polymer-solution interface ($r\approx 50$\,nm).
As opposed to the RF structure, this interaction is not constant inside the nanogel, but it increases monotonously as $r$ decreases and the protein gets further away from the functionalized superficial dangling chains.












%%%%%%%%%%
%%%% change in salt

\begin{figure}
     \centering
     \includegraphics[width=0.45\textwidth]{Figures/graphs-gel2/gamma-salts-cyto.png}
     \caption{Plot of the excess adsorption $\Gamma$ of cytochrome c as a function of salt concentration at pH 7 for MAA-VA nanogels with different network functionalizations having 22\% MAA; protein concentration is $10^{-6}M$.}
     \label{fig:Adsorption-vs-Salt-cyto}
 \end{figure}
 

A feature of protein adsorption that we have not fully discussed yet is the effect of salt concentration.
For both cytochrome c and myoglobin, figura \ref{fig:adsorption-vs-pH-cyto-myo} shows that the incorporation of proteins inside the different nanogels is significantly enhanced by decreasing the solution salt concentration.
In order further characterize this behavior in  figura \ref{fig:Adsorption-vs-Salt-cyto} presents cytochrome c adsorption as a function of NaCl concentration at pH 7. 
This graph shows that all network functionalizations display a qualitatively similar behavior, with a dramatic decrease in adsorption between 1 and 10\,mM NaCl.
At 100\,mM all nanogels show negligible or negative adsorption.

When the solution salt concentration is high, both Na$^+$ and Cl$^-$ ions are found in high concentrations inside the nanogel.
These ions screen the electrostatic attractions between the positive charges of the protein and the negative charges on the polymer, which are the driving force for protein adsorption.
Effectively, these attractions become short range and are not strong enough to result in significant if any protein adsorption.
If the concentration of NaCl is lower, on the other hand, these electrostatic interactions are less screened and effectively longer ranged, which allows for protein adsorption.
Thus, decreasing the salt concentration enhances adsorption.
ROJO 
{Such behavior has been observed in experiments, which show that there is an increase in protein adsorption at low salt concentration \addcite[becker2012proteins, henzler2010adsorption,xu2018interaction].
Adsorption to planar and spherical polyelectrolyte brushes is significantly favored at lower salt concentrations, as demonstrated using isothermal titration calorimetry. ROJO




In considering carriers for protein delivery applications our results suggest that the best conditions for encapsulation correspond to low salt.
The adsorption profiles of figura \ref{fig:Adsorption-vs-Salt-cyto} are qualitatively similar for the three functionalizations,
but the number of proteins inside the nanogel is always significantly larger for the CF structure.
This feature may be critical in designing delivery vehicles for a target having intermediate salt concentrations.
The CF incorporates more proteins at the same conditions, but it may not be able to release them if the target has intermediate salt concentration.
For these conditions the random functionalization will be able to release all its cargo.















%%%%%%%%%%%%%%%%%%%%%%%%%%%%%%%%%%%%%%%%%%%%%%%%%%
\subsection{Insulin adsorption AH-based nanogels} 
%%%%%%%%%%%%%%%%%%%%%%%%%%%%%%%%%%%%%%%%%%%%%%%%%%




Insulin does not adsorb to the MAA nanogels of sec.  \ref{sec:MAA-NGs} %(see \cref*{fig:adsoprtion-vs-pH-insulinMAA_si} in SI).
This is because the isoelectric point of insulin and the pKa of MAA are close to each other, meaning that for solutions where the protein is positively charged, the nanogel is charge neutral, and if the nanogel is negatively charged so is the protein.
In this context, we decided to investigate insulin adsorption to an allylamine nanogel, which is positively charged below its pKa of 9.5, overlapping with the range where insulin is negatively charged.
Other than the functional monomers, the structure of these AH-VA copolymer networks is the same as that of the MAA-VA nanogels previously described.



\begin{figure}[!htb]
    \centering
    \includegraphics[width=0.9\textwidth]{Figures/graphs-gel2/insu-PAH.png}
    \caption{Plots of adsorbed insulin molecules, $\Gamma$, as a function of pH for  AH-VA nanogels having different functionalizations.
    The AH content is 22\% for the polymer networks of panel A and 35\% for those of panel B (this latter degree of functionalization cannot be achieved for the SF nanogel).
    Other conditions are $10^{-3}$\,M NaCl, and [Insulin] = $10^{-6}$\,M.}
    \label{fig:adsorption-vs-pH-insulin}
\end{figure}






Figura \ref{fig:adsorption-vs-pH-insulin}A shows the adsorption of insulin to AH-based nanogels having different spatial functionalizations.
Again we have considered networks with 22\% of pH-sensitive monomers so that we can include results for the SF nanogel, whose dangling chains are AH homo-polymers.
The main features of this plot are qualitatively similar to those of cytochrome c and myoglobin adsorption to the MAA nanogels (see Figura \ref{fig:adsorption-vs-pH-cyto-myo}).
Namely, insulin displays a  nonmonotonic adsorption as a function of the solution pH.
Moreover, we see that the core distribution of AH segments captures more insuline than either the random or the surface functionalizations.
The RF and SF nanogels display relatively similar pH-dependent adsorption profiles.
Finally, a rising salt concentration has a critical effect on the magnitude of insulin adsorption, which is because of the increasing shielding of network-protein electrostatic attractions by mobile ions (see  figura \ref{fig:adsoprtion-AH-1d-2-insu}).



Despite the qualitative similarities between the adsorptions profiles of
Figura \ref{fig:adsorption-vs-pH-insulin}A and those of cytochrome c and myoglobin (Figura \ref{fig:adsorption-vs-pH-cyto-myo}A and C), we see that the number of insulin molecules captured by the AH-networks is significantly less than that of the other proteins by the MAA-based nanogels.
For this reason, we  will next evaluate the effect of the degree of functionalization of the polymer network to enhance protein adsorption.
Figura \ref{fig:adsorption-vs-pH-insulin}B presents insulin adsorption for nanogels having 35\% of AH segments.
Here, the surface functionalized structure is not included because there are not enough segments in the dangling chains.
A higher HA content drives more adsorption, which results from comparing both panels of Figura \ref{fig:adsorption-vs-pH-insulin}.
Once again, the CF nanogel adsorbs more insulin than the RF network (more than twice as many proteins for the conditions of this calculations).




\begin{figure}[!htb]
    \centering
    \includegraphics[width=0.45\textwidth]{Figures/graphs-gel2/insu-ads-pmf.png}
    \caption{A: Plot of the local distribution of insulin molecules, $\langle N(r)\rangle$, as a function of position for AH-VA nanogels having 22\% pH-sensitive segments in different network configurations.
    pH is 7.5, $10^{-6}$\,M insulin and  $10^{-3}$\,M NaCl.
    B: Potential of mean force,  ${PMF}(r)$, as a function of position for the same conditions as panel A.}
    \label{fig:adsorption-vs-r-insulin}
\end{figure}



Figura \ref{fig:adsorption-vs-r-insulin}A shows how adsorbed proteins are spatially distributed inside the different AH-based nanogels with \%22 degree of functionalization.
The pH of these results corresponds to the adsorption maximum of Figura \ref{fig:adsorption-vs-pH-insulin}A.
Insulin adsorption to the CF nanogel is not only significantly higher than the adsorption to the RF and SF nanogels, but it also occurs deeper inside the structure.
The most likely position of an insulin molecule occurs around $r=25$\,nm for the CF nanogel,
while this position moves to around $40-45$ and 50\,nm for the RF and SF structures respectively.




For MAA-based nanogels, Figura \ref{fig:adsorption-vs-r-cyto}A shows relatively minor differences between the distributions of cytochrome c inside RF and SF networks.
Such are differences slightly accentuated for insulin adsorption to the AH-VA nanogels, as seen in Figura \ref{fig:adsorption-vs-r-insulin}A.
Insulin distribution is displaced to the interior of the network in the randomly modified nanogel as compared to the surface functionalization, where proteins are more likely to occupy the very vicinity of polymer-solution interface.
Overall insulin distribution profiles are still relatively similar for these two structures.
These results show, once again, that network design (polymer synthesis) provides a tool to control the distribution of proteins inside the nanogel.






Panel B of figura \ref{fig:adsorption-vs-r-insulin} shows the potential of mean force acting on insulin molecules at the same conditions as panel A.
The attractive interaction on adsorbed insulin ranges from $-8$ to $-6 k_B T$ inside the CF structure and from $-5$ to $-4 k_B T$ inside RF nanogel.
Inside the SF nanogel the potential presents a minimum of $-4 k_B T$  at the surface and then increases monotonously as $r$ decreases inside the gel.







%%%%%%%%%%%%%%%%%%%%%%%%%%%%%%%%%%%%%%%%%%%%%%%%%%
\section{Conclusions}
%%%%%%%%%%%%%%%%%%%%%%%%%%%%%%%%%%%%%%%%%%%%%%%%%%


We have presented a study of protein adsorption to polymer nanogels having  different pH-responsive functionalizations.
We have derived and applied a thermodynamic theory that can be informed by a coarse-grained molecular model.

The 



