\chapter{Conclusiones generales}

Esta tesis ha abarcado un estudio exhaustivo sobre los hidrogeles polim\'ericos y su capacidad para responder a diversos est\'imulos, entre ellos cambios en el pH, concentraci\'on salina y temperatura, abordando desde la adsorci\'on de poliaminas y la liberaci\'on de f\'armacos hasta el comportamiento de soluciones coloidales de nanogeles bajo diferentes condiciones y concentraciones de las mismas. A trav\'es de una combinaci\'on de enfoques te\'oricos y computacionales, hemos dilucidado la fisicoqu\'imica involucrada en la funcionalidad de estos materiales avanzados.

En el primer cap\'itulo, demostramos c\'omo los hidrogeles basados en PMAA pueden ser dise\~nados para capturar poliaminas, y adem\'as aprovechar este mecanismo para la liberaci\'on de drogas terap\'euticas. La capacidad de estos materiales para liberar f\'armacos en respuesta a concentraciones espec\'ificas de poliaminas subraya su potencial como sistemas inteligentes de entrega de medicamentos.

El estudio de microgeles polim\'ericos con multiple respuesta a est\'imulo revel\'o c\'omo su respuesta a cambios en el pH, la temperatura y la concentraci\'on de sal puede ser manipulada mediante la composici\'on y la estructura del pol\'imero. Este conocimiento es crucial para el dise\~no de sistemas de entrega que requieren precisi\'on en la liberaci\'on de sus cargas \'utiles bajo condiciones fisiol\'ogicas espec\'ificas.

En este mismo sentido, la investigaci\'on sobre la funcionalizaci\'on de la red polim\'erica en la adsorci\'on de prote\'inas en nanogeles resalta la importancia de la distribuci\'on espacial de los segmentos funcionales. Esta funcionalizaci\'on no solo influye en la capacidad de los nanogeles para adsorber y liberar prote\'inas sino tambi\'en en la localizaci\'on de estas prote\'inas dentro de la red, lo que es fundamental para aplicaciones que requieren interacciones espec\'ificas con objetivos celulares.

Por \'ultimo, el an\'alisis de soluciones coloidales compuestas por nanogeles polim\'ericos a diferentes concentraciones proporciona una visi\'on detallada de c\'omo las interacciones entre part\'iculas afectan su comportamiento colectivo en respuesta a est\'imulos. Este conocimiento es vital para el desarrollo de formulaciones coloidales con propiedades ajustadas para aplicaciones espec\'ificas.

En conjunto, los hallazgos presentados en esta tesis ofrecen un avance significativo en nuestro entendimiento de los hidrogeles polim\'ericos y su potencial para revolucionar el campo de los biomateriales. La capacidad de estos materiales para responder de manera precisa a est\'imulos externos abre nuevas posibilidades para su aplicaci\'on en el encapsulado y transporte de medicamentos.