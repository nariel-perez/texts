% Chapter 2

\chapter{Films de hidrogeles polim\'ericos: Encapsaludo de prote\'inas, poliaminas y drogas}
\label{Chapter-film} % For referencing the chapter elsewhere, use \ref{Chapter1}



%%%%%%%%%%%%%%%%%%%%%%%%%%%%%%%%%
%\section{Adsorci\'on de prote\'inas}

%Debido a sus propiedades altamente ajustables y biocompatibles, los hidrogeles de pol\'imero han sido de gran inter\'es como materiales para aplicaciones biom\'edicas. En la administraci\'on oral de medicamentos, estos hidrogeles han sido intensamente investigados como portadores que pueden encapsular y liberar prote\'inas medicinales, protegi\'endolas a trav\'es de la barrera f\'isica y qu\'imica que impone el entorno gastrointestinal. (1) Dentro del microentorno acuoso similar a tejido dentro de la red de pol\'imeros entrecruzados, las prote\'inas son menos susceptibles a la desnaturalizaci\'on y agregaci\'on. (2,3) Adem\'as, las prote\'inas conservan su estructura y actividad cuando se liberan de los hidrogeles de pol\'imero. (4) Adem\'as de la administraci\'on de f\'armacos, los hidrogeles de pol\'imero son de gran inter\'es como componente sensible en muchas otras aplicaciones biom\'edicas (5,6), incluyendo la ingenier\'ia de tejidos, (7) la biosensibilizaci\'on, (8,9) y el dise\~no de materiales biomim\'eticos funcionales. (10)
%El uso de hidrogeles u otros materiales en dispositivos biom\'edicos requiere una comprensi\'on profunda de su interacci\'on con las prote\'inas. Por ejemplo, las lentes de contacto basadas en geles de poli(metacrilato de \'acido) (PMAA) adsorben prote\'inas de la l\'agrima, lo que tiene implicaciones para la comodidad de uso y la aparición de complicaciones inflamatorias. (11,12) Sin embargo, la adsorci\'on selectiva de prote\'inas con propiedades antibacterianas y/o antiinflamatorias, como la lisozima, podr\'ia ser beneficiosa en este y otros contextos. (12) Cuando se aplican en entornos biol\'ogicos, los geles polim\'ericos est\'a expuesto a mezclas multicomponentes, y la adsorci\'on de prote\'inas en el material est\'a gobernada por una interacci\'on compleja entre la red de pol\'imeros y las diferentes especies moleculares presentes en el entorno. 
%La presencia de  biomol\'eculas puede influir en la adsorci\'on de prote\'inas de manera no trivial. Es necesario comprender la qu\'imica f\'isica que rige la adsorci\'on de prote\'inas.
%La adsorci\'on de prote\'inas en materiales polim\'ericos sensibles al pH ha sido objeto de diversos estudios experimentales. Utilizando calorimetr\'ia isot\'ermica, Welsch et al. (17) consideraron la adsorci\'on de lisozima en microgeles n\'ucleo-corteza basados en unidades de poli(acrilato) (PAA). Adem\'as, Zhang et al. (18) estudiaron la influencia del pH en la cin\'etica de adsorci\'on y liberaci\'on de prote\'inas del suero l\'acteo a partir de cuentas de hidrogel basadas en alginato. 
%Adem\'as de estos estudios experimentales, se han realizado estudios te\'oricos y de simulaci\'on sobre la adsorci\'on de prote\'inas. Mediante simulaciones de din\'amica molecular (MD), se ha investigado la adsorci\'on de prote\'inas en diferentes superficies (24,25) y en nanopart\'iculas (26-29). Adem\'as, la interacci\'on entre pol\'imeros cargados en soluci\'on y prote\'inas ha sido objeto de estudios te\'oricos y simulaciones moleculares. (31,32) Sin embargo, la adsorci\'on de proteínas en materiales polim\'ericos o superficies modificadas con pol\'imeros ha sido considerada con menos frecuencia. Yigit et al. (33) desarrollaron diferentes modelos de uni\'on de Langmuir para investigar la adsorci\'on de lisozima en microgeles n\'ucleo-corteza, donde el estado de carga de los microgeles puede ser modificado por el adsorbato. Angioletti-Uberti et al. (34) desarrollaron un enfoque de teor\'ia din\'amica de la densidad para describir la adsorci\'on de lisozima en nanopart\'iculas cargadas recubiertas de pol\'imero. Sun et al. (35) realizaron simulaciones de MD en diferentes valores de pH de la adsorci\'on y complejaci\'on del fragmento de uni\'on al ant\'igeno de trastuzumab en un hidrogel de poli(vinil alcohol). Szleifer y colaboradores desarrollaron una teor\'ia molecular para investigar la adsorci\'on de prote\'inas en capas de pol\'imero injertado. (36,37)
%En este capitulo, presentamos un avance en esa direcci\'on y aplicamos una teor\'ia molecular para investigar las condiciones de equilibrio de la adsorci\'on  de prote\'inas en pelí\'iculas de hidrogel sensibles al pH. Este enfoque permite una descripci\'on a nivel molecular del tama\~no, forma, distribució\'o de carga y grados de libertad conformacionales de todos los componentes del sistema, incluyendo las diferentes prote\'inas y la red de pol\'imeros que forma la estructura de la red. Aqu\'i mostramos que, adem\'as del esperado e importante papel de la carga de la prote\'ina, la protonaci\'on de amino\'acidos es un factor decisivo para determinar la adsorci\'on de las mismas.
%%%%%%%%%%%%%%%%%%%%%%%%%

%\begin{figure}[!htb]
%	\centering
%	\includegraphics[width=0.9\textwidth]{Figures/graph-film/film_model.pdf}
%	\caption{Esquema de simulaci\'on de un film polim\'erico compuesto por MAA. Se muestra adem\'as las prote\'inas usadas y el modelo de grano grueso. En la tabla se ven los segmentos que componen al modelo y la informaci\'on de los pKa asociadas a cada una de ellas.}
%	\label{fig:film:model_film}
%\end{figure}



%%%%%%%%%%%%%%%%%%%%%%%%%%%%%%%%%%%%%%%%%
%%%%%%%%%%%%%%%%%%%%%%%%%%%%%%%%%%%%%%%%%%%%%%

\section{Introducci\'on}

Putrescina, espermidina y espermina son poliaminas que tienen dos, tres y cuatro grupos amino, respectivamente, y que est\'an presentes en todas las c\'elulas vivas.
Las poliaminas son indispensables para el crecimiento celular y son necesarias en muchos procesos intracelulares e intercelulares.
Participan en diversas funciones metab\'olicas, como la replicaci\'on del ADN, la regulaci\'on de canales i\'onicos, la fosforilaci\'on de prote\'inas y la se\~nalizaci\'on extracelular \addcite[Igarashi2010,Soda2011].
Adem\'as, las poliaminas interact\'uan fuertemente con los fosfol\'ipidos de las membranas y, por lo tanto, pueden desempe\~nar un papel importante en la regulaci\'on de las enzimas vinculadas a las membranas \addcite[Moinard2005].
La s\'intesis de estas mol\'eculas esenciales comienza cuando la enzima ornitina decarboxilasa (ODC) cataliza la producción de putrescina \addcite[Soda2011,Casero2009,Pegg2010].

En el entorno de las c\'elulas sanas, las poliaminas se encuentran en concentraciones altas de micromolares a bajas milimolares \addcite[Porter1983, Russell].
Sin embargo, cerca de las c\'elulas tumorales, su concentraci\'on es relativamente mayor.
Numerosos estudios han demostrado que los pacientes con c\'ancer presentan concentraciones elevadas de poliaminas tanto en la sangre como en la orina \addcite[Russell1971], lo que se debe a la mayor actividad de la ornitina decarboxilasa (ODC) \addcite[Soda2011, Agostinelli2010Polyamines, Nowotarski2013].
Concentraciones an\'omalas de poliaminas pueden indicar la presencia de una c\'elula tumoral \addcite[Park2013, Gerner2004], pero tambi\'en un exceso de disponibilidad de poliaminas puede aumentar la velocidad a la que los tumores se propagan y metastatizan \addcite[Soda2011].
Al mismo tiempo, dicho exceso puede inhibir los mecanismos inmunitarios que las c\'elulas tienen para evitar la propagaci\'on del tumor \addcite[Soda2011, Jasnis1994Polyamines].
De hecho, los pacientes con niveles elevados de poliaminas generalmente tienen un pron\'ostico m\'as desfavorable \addcite[Soda2011, Ikeda2011Montmorillonite].
Actualmente, existe un gran inter\'es en terapias contra el c\'ancer que puedan regular/reducir la cantidad de poliaminas en las cercan\'ias de las c\'elulas tumorales para evitar su propagaci\'on \addcite[Soda2011, Aziz1996Potential, Chen2006Combination, Bachrach2004Polyamines].
En este capitulo, proponemos y exploramos tanto te\'oricamente como experimentalmente el concepto de un biomaterial funcional capaz de absorber poliaminas, pero al mismo tiempo utilizar la concentraci\'on excesiva de estos marcadores cerca de las c\'elulas tumorales como un desencadenante para la liberaci\'on de un medicamento terap\'eutico.

Los hidrogeles de cadenas polim\'ericas entrecruzadas actualmente se consideran para diversas aplicaciones en la investigaci\'on biom\'edica \addcite[Wang2019].
Por ejemplo, se han explorado biomateriales basados en hidrogeles sensibles al pH como veh\'iculos de administraci\'on oral de medicamentos que tienen el potencial de encapsular y transportar un agente terap\'eutico a trav\'es del tracto gastrointestinal, protegiendo la carga del medio \'acido del est\'omago y liber\'andola en el ambiente neutral del intestino delgado \addcite[Lowman1999,Zhao2019,Qindeel2019,Li2019].
Estos hidrogeles son sensibles a cambios de pH debido a que contienen un n\'umero significativo de grupos \'acidos d\'ebiles.
En este contexto, consideraremos un filme de hidrogel de poli(metacrilato de \'acido) (MAA).
Los hidrogeles de PMAA son capaces de responder a diversos est\'imulos biol\'ogicos, incluyendo cambios en el pH fisiol\'ogico \addcite[Kanamala2016Mechanisms].
En este sentido, nuestra contribuci\'on actual aborda una pregunta fundamental: ¿Es posible aprovechar las propiedades sensibles al entorno de los hidrogeles de poli\'acido d\'ebil en el desarrollo de un biomaterial que pueda incorporar y atrapar poliaminas al mismo tiempo que libera un medicamento terap\'eutico en respuesta?

La doxorrubicina es una antraciclina que se utiliza com\'unmente en la quimioterapia debido a su eficacia en la lucha contra una amplia gama de c\'anceres, como carcinomas, sarcomas y c\'anceres hematol\'ogicos \addcite[Panis2012].
Es uno de los f\'armacos antineopl\'asicos más potentes y tiene la ventaja experimental de que se puede monitorear mediante fluorescencia y/o absorbancia \addcite[Serpe2005Doxorubicin].
Utilizada sola o en combinaci\'on con otros agentes terap\'euticos, la doxorrubicina es actualmente el compuesto de su clase con el espectro de actividad m\'as amplio \addcite[Carvalho2009].
Adem\'as, debido a que tiene carga positiva, la doxorrubicina puede encapsularse en nanogeles ani\'onicos \addcite[Li2019] o inmovilizarse en la superficie de superficies nanoh\'ibridas cargadas negativamente \addcite[Kazempour2019].

El objetivo de este estudio es caracterizar los hidrogeles de PMAA como materiales capaces de capturar poliaminas y liberar simult\'aneamente doxorrubicina en respuesta.
Para lograr este objetivo, aplicamos una teor\'ia molecular recientemente desarrollada para investigar la absorci\'on de poliaminas y doxorrubicina en pel\'iculas de hidrogel de PMAA desde soluciones.
Esta teorí\'i se formula sobre la base de un potencial termodin\'amico general que tiene en cuenta el costo de energ\'ia libre de protonaci\'on/deprotonaci\'on de unidades titulables, la p\'erdida entr\'opica de confinamiento molecular, los grados de libertad conformacionales y translacionales tanto de la capa polim\'erica como de los absorbentes, y las interacciones electrost\'aticas y est\'ericas.
Para poder aplicar esta teor\'ia es necesario contar con un modelo molecular que incluye una descripci\'on del tama\~no, la forma, la configuraci\'on y el estado de carga local de todas las especies qu\'imicas presentes en el sistema (espermidina, espermina, putrescina, doxorrubicina y la red de PMAA).

Finalmente, se sintetizaron pel\'iculas de PMAA y, eligiendo un conjunto de condiciones representativas de toda la colecci\'on explorada te\'oricamente, llevamos a cabo experimentos de absorci\'on de poliaminas y doxorrubicina seguidos de la t\'ecnica de UV-Vis.
Estos experimentos demuestran la idoneidad del enfoque actual para explorar nuevos biomateriales y su respuesta en el contexto de aplicaciones relacionadas con el c\'ancer.
En resumen, aqu\'i exploramos el comportamiento fisicoqu\'imico de los hidrogeles desde una perspectiva te\'orica, lo que permite un estudio sistem\'atico de su respuesta a cambios en el pH, la concentraci\'on de sal y poliaminas, y tambi\'en desde una perspectiva experimental para comprobar los puntos clave encontrados en nuestros c\'alculos.


\begin{figure}[!htb]
	\centering
	\includegraphics[width=0.7\textwidth]{Figures/graph-film/poliamines_model.png}
	\caption{Esquema de simulaci\'on de un film polim\'erico compuesto por MAA. Se muestra adem\'as las poliaminas usadas  con su respectivo  modelo de grano grueso.}
	\label{fig:film:model_poliamines}
\end{figure}





\section{Metodolog\'ia}


El sistema que estudiamos est\'a esquematizado en la figura \ref{fig:film:model_poliamines}.
Una red de poli(metacrilato de \'acido) entrecruzado con injerto en la superficie est\'a en equilibrio con una soluci\'on que contiene mol\'eculas de agua, iones hidronio e hidr\'oxido, y cloruro de sodio, que est\'a completamente disociado en iones cloruro y sodio.
Adem\'as, esta soluci\'on contiene tanto doxorubicina como una poliamina o ambas especies.
Las poliaminas que hemos considerado son putrescina, espermidina y espermina.

Para estudiar este sistema, aplicamos una teor\'ia molecular que se desarroll\'o recientemente para investigar la absorci\'on en películas de hidrogel sensibles al pH a partir de mezclas de prote\'inas \addcite[Hagemann2018, Longo2019].
Nuestro enfoque se basa en el trabajo de Szleifer y colaboradores para estudiar el comportamiento de capas de polielectrolitos d\'ebiles injertadas \addcite[Nap2006, Gong2007PRL].
El m\'etodo empleado aqu\'i es general y flexible; por ejemplo, se ha aplicado para el estudio de la captura de glifosato en hidrogeles de poli(alilamina) \addcite[PerezChavez2018].

A continuaci\'on, presentaremos una descripci\'on de la teor\'ia con \'enfasis en el modelo molecular introducido para describir la doxorubicina, la putrescina, la espermidina y la espermina.


\section{Te\'oria Molecular } \label{sec:film-teoria}

El m\'etodo propuesto consiste en minimizar una energ\'ia libre generalizada que incluye toda la termodin\'amica relevante que engloba los procesos del sistema polim\'erico con una soluci\'on.
Para tal fin  usamos  una descripci\'on molecular de grano grueso de las diferentes especies qu\'imicas que componen el sistema.
Dicha descripci\'on incluye forma, tama\~no, distribuci\'on de carga (si los hubiese) y estado de protonaci\'on de cada componente molecular en los casos que corresponda.
En esta primera instancia describiremos la fisicoqu\'imica de un film  que  se encuentra en  equilibrio con una soluci\'on acuosa, la cual  tiene una composici\'on  definida externamente (ba\~no de la soluci\'on).
Es decir, el pH, la concentración de sal y la concentraci\'on de adsorbatos son variables independientes.


Nuestro film que posee distintos tipos de segmentos: un segmento que sirve como entrecruzante entre las cadenas polim\'ericas, el cual es considerado neutro, y  una unidad sensible al pH,  para esta ultima, en particular consideraremos un film polim\'erico compuesto por unidades \'acidad de \'acido metacr\'ilico ($MAA$).
Este film  se encuentra en equilibrio con una soluci\'on con una temperatura, pH y concentraci\'on de sal definidas. Adem\'as vamos a considerar que en dicha soluci\'on hay un adsorbato, en este cap\'itulo usaremos una prote\'ina, la cual puede ser cictocromo c o mioglobina. 
Considerando los aspectos anteriores es posible definir una energ\'ia libre:

\begin{align}
 	F = -TS_{mez} -TS_{conf,net} + F_{qca,net} + F_{qca,pro} + U_{elec} + U_{ste} + U_{VDW}
 	\label{eq:film:libre-film}
\end{align}
 
\noindent En donde $S_{mez}$ es la entrop\'ia de traslaci\'on ( y de mezcla) de las especies libres en la soluci\'on: mol\'eculas de agua (H$_2$O), y sus respectivos iones:  hidronio (H$_3$O$^+$), e hidr\'oxido (OH$^- $), cationes y aniones de sal y nuestra prote\'ina modelo.
Aqu\'i, consideramos una sal monovalente, NaCl, la cual est\'a completamente disociada en sus  iones cloruro (Cl$^-$) y sodio ($Na^+$). 

$S_{conf,nw}$ representa la entrop\'ia conformacional que resulta de la flexibilidad de la red polim\'erica, la cual viene dada por todas las conformaciones diferentes que puede asumir la misma.

$F_{qca,net}$, es la energ\'ia qu\'imica libre que describe el equilibrio entre las especies protonadas y desprotonadas de unidades funcionales (\'acidas/b\'asicas), para nuestro film solo se consideran unidades \'acidas.

De manera similar, $F_{qca,pro}$ describe la protonaci\'on de residuos titulables de la prote\'ina.

$U_{elec}$ y $U_{ste}$ representan, respectivamente, las interacciones electrost\'aticas y las repulsiones est\'ericas.
Las interacciones de Van der Waals son representadas en $U_{VdW}$.


Las expresiones explicitas de la ecuaci\'on \ref{eq:film:libre-film} las describimos acontinuaci\'on.

Como primer t\'ermino tenemos la entrop\'ia de mezcla de  las especies m\'oviles, entre ellas consideramos a nuestra prote\'ina modelo:

\begin{align}
	\begin{aligned}
		-\frac{S_{mez}}{k_B}= &A\sum_{\gamma}\int_0^\infty{dz\rho_\gamma(z)\left(\ln \left(\rho_\gamma (z)v_w\right) -1 + \beta\mu^0_\gamma\right)} \\
		&+ A\sum_{\theta}\int_0^\infty{dz\rho_{pro}(\theta,z)\left(\ln \left(\rho_{pro}(\theta,z)\right) -1 + \beta\mu^0_{pro} \right)}
	\end{aligned}
\end{align}

\noindent en donde $\frac{1}{k_B T}$, y $k_B$ es la constante de Boltzmann, $T$ es la temperatura absoluta del sistema. La variable $z$ es la coordenada que mide la distancia a la superficie de soporte de nuestro film, el \'area total de esta superficies es $A$. $\rho_\gamma(z)$ y $\mu_\gamma$ es densidad local, a un $z$ dado, y potencial qu\'imico estadar de la especie $\gamma$ respectivamente.
El sub\'indice $\gamma$ toma en cuenta la mol\'ecula de agua y sus respectivos iones (hidronio e hidr\'oxido), adem'as de los iones provenientes de la sal ($Na^+$ y $Cl^-$). 


El segundo t\'ermino de esta ecuaci\'on corresponde a la entrop\'ia de mezcla de la prote\'ina. $\rho_{pro}(\theta,z)$ es la densidad local de la prote\'ina en la conformaci\'on $\theta$. Es decir $\theta$ recorre sobre las configuraciones de la misma.
Esta conformaciones incluyen rotaciones espaciales de la prote\'ina.
De este modo la densidad local media de la prote\'ia puede expresarse como:


\begin{align}
	\left<\rho_{pro}(z)\right> = \sum_\theta{\rho_{pro}(\theta,z)}
\end{align}


La entrop\'ia conformacional que resulta de la flexibilidad de la red polim\'erica de nuestro film se representa en $S_{conf, nw}$, esta tiene en cuenta todas las configuraciones de un set $\{\alpha\}$.

\begin{equation}
	\frac{S_{conf,net}}{k_B} = - \sum_{\alpha}{P(\alpha)\ln P(\alpha)}
\end{equation}

\noindent en donce $P(\alpha)$ denota la probabilidad de que el film se encuentre en la configuraci\'on $\alpha$.

El siguiente t\'ermino de eq \ref{eq:film:libre-film} describe  la energ\'ia libre dada por  el equilibrio \'acido-base de los segmentos de MAA que componen nuestra red. 

\begin{align}
	\begin{aligned}
		\beta F_{qca,net} &= A\int_0^\infty dz \left<\rho_{MAA}(z)\right> \left[f(z)(\ln f(z)+ \beta\mu^0_{MAA^-})\right.\\
		&\left.+(1-f(z))(\ln (1-f(z))+\beta\mu^0_{MAAH})\right]    
	\end{aligned}
\end{align} 

\noindent en donde $f(z)$ es el grado de carga de los segmentos de MAA entre las capaz $z$ y $z + dz$. 
$\mu^0_{MAA^-}$ y $\mu^0_{MAAH}$ son los potenciales qu\'imico estandar  de las especies protonadas y desprotonadas respectivamente.
Adem\'as se define:

\begin{align}
	\left< \rho_i(z)\right> = \sum_\alpha{P(\alpha)\rho_i(\alpha,z)}
\end{align}
\noindent en el cual $\rho_i(\alpha,z)$  es el ensamble de densidad  promedio local del film. El cual es una variable de entrada que cuantifica la densidad de segmentos del film que  ocupan una capa $z$ cuando la red se encuentra en la conformaci\'on $\alpha$.


%%%%%%%%%%%%%%%%%%%
El equilibrio qu\'imico de las unidades titulables de la prote\'ina es considerada en el siguiente t\'ermino de la energ\'ia libre:

\begin{align}
	\begin{aligned}
		\beta F_{qca,pro} = A\int_0^\infty dz& \sum_\tau \left<\rho_{pro,\tau}(z)\right> \left[g_\tau(z)(\ln g_\tau(z)+ \beta\mu^0_{\tau p})\right.\\
		&\qquad\left.+(1-g_\tau(z))(\ln (1-g_\tau(z))+\beta\mu^0_{\tau d})\right]   
	\end{aligned}
\end{align} 

\noindent en donde $\left<\rho_{pro,\tau}(z)\right>$ representa la densidad local promedio del segmento protonable $\tau$ de la prote\'ina.

Que se define como:
\begin{align}
	\left<\rho_{pro,\tau}(z)\right> = A\sum_\theta \int dz^\prime  \rho_{pro}(\theta,z^\prime)n_\tau(\theta,z^\prime, z)
	\label{eq:film:segments_pro_si}
\end{align}
\noindent en donde $n_\tau(\theta,r^\prime, r)$ es un par\'ametro de entrada que nos da el n\'umero de segmentos $\tau$ entre las capas $z$ and $z+ dz$ cuando el centro de masa de la prote\'ina se encuentra en el a configuraci\'on $\theta$ y en la posici\'on $z^\prime$.

Las unidades titulables pueden estar en estado protonado $\tau, p$ o desprotonado $\tau, d$, los cuales poseen su potenciales qu\'imicos est\'andar $\mu^0_{\tau,p}$ y $\mu^0_{\tau,d}$ respectivamente. 
Adem\'as definimos el grado de asociaci\'on $g_\tau$ para segmento $\tau$ como:


\begin{enumerate}
	\item para unidades \'acidas: $g_\tau(r) = 1-f_\tau(r)$ ( las unidades $\tau$ se cargan negativamente)
	\item para unidades b\'asicas: $g_\tau(r) = f_\tau(r)$ (las  unidades $\tau$ se cargan positivamente  )
\end{enumerate}
en donde  $f_\tau(r)$ es el grado de disociaci\'on de cada segmento $\tau$.
%%%%%%%%%%

La energ\'ia electr\'ostatica se define como:
\begin{align}
	\begin{aligned}
		\beta U_{elec}= A\int_0^\infty dz&\left[\left(\sum_{\gamma } {\rho_\gamma(z) q_\gamma + \sum_\tau{f_\tau(z) \left<\rho_{pro,\tau}(z)\right> q_\tau} +  f(z)\left<\rho_{MAA}(r)\right>q_{MAA}}\right)\beta\psi(z) \right. \\ &\left.-\frac{1}{2}\beta\epsilon(\nabla\psi(z))^2 \right]
	\end{aligned}
\end{align} 

\noindent en donde $\Psi(z)$ es el potencial electrost\'atico dependiente de la posici\'on, $\epsilon$ es la constante de permitividad del medio, $q_\gamma$ es la carga correspondiente a la especie m\'ovil $\gamma$, $q_\tau$ es la carga que adquieren los segmentos titulables de la prote\'ina. Finalmente $q_{MAA}$ es la carga que adquiere el segmento de $MAA$ al desprotonarse.


En este contexto, la densidad de carga media es:

\begin{align}
	\left<\rho_q(z)\right> = \sum_{\gamma } {\rho_\gamma(z) q_\gamma + \sum_\tau{f_\tau(z) \left<\rho_{pro,\tau}(z)\right> q_\tau} +  f(z)\dfrac{\left<\phi_{MAA}(z)\right>}{v_{MAA}}q_{MAA}}
	\label{eq:film:rho_charge}
\end{align}
%%%%%%%%%%%%%%%%

La contribuci\'on siguiente en la energ\'ia libre viene dada por la repulsi\'on esterica entre  todos los segmentos que componen  el sistema. Esta contribuci\'on se incorpora a trav\'es del siguiente restricci\'on:  

\begin{align}
	\begin{aligned}
		1=  {\left[\sum_{\gamma}\rho_\gamma(z) v_\gamma + \sum_\lambda{\left<\rho_{pro,\lambda}(z)\right>v_\lambda} + \left<\rho_{MAA}(z)\right>v_{MAA} \right]},~ \forall z
	\end{aligned}
	\label{eq:film:constraint}
\end{align}
\noindent en donde $v_\gamma$ , $v_\lambda$ y $v_{MAA}$ son los vol\'umenes moleculares de los segmentos $\gamma$ de las especies libres, $\lambda$  en la prote\'ina y los segmentos de $MAA$ del film respectivamente.
$\left<\rho_{pro,\lambda}(z)\right>$ es definido de la misma forma que en la ecu.  \ref{eq:film:segments_pro_si}.
Cabe destacar que el sub\'indice $\lambda$ considera a todos los segmentos de la prote\'ina, es decir $ \tau \in \lambda$.

%%%%%%%%%%%%%%%
La energ\'ia proveniente de las interacciones de Van der Waals se expresa en el t\'ermino $U_{VdW}$. En este trabajo se ha considerado que todos los segmentos del sistema poseen un car\'acter hidrofilico. Es decir la interacci\'on entre cada par de segmentos es similar a su interacci\'on con las mol\'eculas de agua. Como resultado la energ\'ia de interacci\'on de $VdW$ se considera una constante aditiva a la energ\'ia libre, por lo cual puede ser ignorada. En el cap\'itulo \ref{Chapter-geles} mostraremos un ejemplo en donde las interacciones de $VdW$ son tenidas en cuenta muy fuertemente. 

En este punto la energ\'ia libre, ecu. \ref{eq:film:libre-film}, se puede escribir como una funcional de funciones. Esta ultimas est\'an compuesta por la probabilidad de distribuci\'on de segmentos de nuestra red polim\'erica, las densidades locales de cada una de las especies libres, incluidas la densidad de conformaciones de la prote\'ina, los grados de protonaci\'on/disociaci\'on y el potencial electrost\'atico local. Es decir:

\begin{align}
	F = \sum_\alpha \sum_\theta \int_0^\infty dz f(\alpha, \theta,z)
\end{align}

\noindent en donde:

\begin{align}
	 f=  f \left( P(\alpha), \rho_\gamma(z),\rho_{pro}(z), f_\tau(z), f(z), \psi(z)  \right)
	 \label{eq:film:funcionales}
 \end{align}

de forma m\'as explicita:

\begin{align}
	\begin{aligned}
		\beta F=  & A\sum_{\gamma}\int_0^\infty{dz\rho_\gamma(z)\left(\ln \left(\rho_\gamma (z)v_w\right) -1 + \beta\mu^0_\gamma\right)} \\
		%%%%%
		&+ A\sum_{\theta}\int_0^\infty{dz\rho_{pro}(\theta,z)\left(\ln \left(\rho_{pro}(\theta,z)\right) -1 + \beta\mu^0_{pro} \right)} \\
		%%%%%
		&+ \sum_\alpha{P(\alpha)\ln P(\alpha)} \\
		%%%%
		& + A\int_0^\infty dz \left<\rho_{MAA}(z)\right> \left[f(z)(\ln f(z)+ \beta\mu^0_{MAA^-})\right.\\
		& \qquad\qquad\qquad \left.+(1-f(z))(\ln (1-f(z))+\beta\mu^0_{MAAH})\right] \\
		%%%%%
		& + A\int_0^\infty dz \sum_\tau \left<\rho_{pro,\tau}(z)\right> \left[g_\tau(z)(\ln g_\tau(z)+ \beta\mu^0_{\tau p})\right.\\
		&\qquad \qquad \qquad\qquad \qquad\quad \left.+(1-g_\tau(z))(\ln (1-g_\tau(z))+\beta\mu^0_{\tau d})\right]   \\
		%%%%%%%
		 & +A\int_0^\infty dz \left[\left(\sum_{\gamma } {\rho_\gamma(z) q_\gamma + \sum_\tau{f_\tau(z) \left<\rho_{pro,\tau}(z)\right> q_\tau} +  f(z)\left<\rho_{MAA}(r)\right>q_{MAA}}\right)\beta\psi(z) \right. \\ & \qquad \qquad \left.-\frac{1}{2}\beta\epsilon(\nabla\psi(z))^2 \right]
		\end{aligned}
\end{align}


Para completar el sistema de estudio, nuestro film esta el equilibrio con el bulk de la soluci\'on. Este bulk  posee una composici\'on bien definida (pH, Temperatura, concentraci\'on salina y de adsorbatos). Estas cantidades proveen un reservorio con el cual se crea un potencial qu\'imico el cual debe estar en equilibrio con nuestro sistema polim\'erico. En particular estos potenciales corresponden a las especies libres, $\mu_\gamma$, y de la prote\'ina, $\mu_{pro}$.
De esta forma al considerar esta condici\'on de equilibrio nuestra energ\'ia libre se convierte un gran potencial termodin\'amico:

\begin{align}
	\begin{aligned}
		\Omega = &F - \sum_\gamma \mu_\gamma N_\gamma -  \mu_{pro} N_{pro} \\
			= &F -\sum_\gamma A\int_0^\infty dz \mu_\gamma \rho_\gamma(z) -  \mu_{pro} N_{pro}  \\
			& \qquad -A\int_0^\infty \mu_{H^+} \left( \sum_\tau\left< \rho_{pro,\tau}(z) \right>g_\tau(z) + (1-f(z))\left< \rho_{MAA}(z) \right> \right )
			\end{aligned}
		\label{eq:film:equil-qco}
\end{align}

\noindent $N_\gamma$ y $ N_{pro}$ son el n\'umero total de mol\'eculas de las especies libres y la prote\'ina respectivamente. En la \'ultima linea de la expresi\'on ecu. \ref{eq:film:equil-qco} se tiene en cuenta los protones asociados que  provienen de las especies con segmentos titulables: prote\'ina y red polim\'erica respectivamente.


Adicionalmente las condiciones de equilibrio deben satisfacer la condici\'on de incompresibilidad del sistema: ecu. \ref{eq:film:constraint}.  Esta restricci\'on se incorpora como:

\begin{align}
	\Phi = \Omega +A \int_0^\infty dz\pi(z){\left[\sum_{\gamma}\rho_\gamma(z) v_\gamma + \sum_\lambda{\left<\rho_{pro,\lambda}(z)\right>v_\lambda} + \left<\rho_{MAA}(z)\right>v_{MAA} -1 \right]}
\end{align}


\noindent en donde $\Pi(z)$ es un multiplicador local de Lagrange.  Este multiplicar se traduce en un potencial que define la presi\'on osm\'otica local. Finalmente se obtiene un nuevo potencial termodin\'amico para nuestro sistema, el cual se escribe de forma explicita como:
 
\begin{align}
	\begin{aligned}
		\beta \Phi=  & A\sum_{\gamma}\int_0^\infty{dz\rho_\gamma(z)\left(\ln \left(\rho_\gamma (z)v_w\right) -1 + \beta\mu^0_\gamma\right)} \\
		%%%%
		&+ A\sum_{\theta}\int_0^\infty{dz\rho_{pro}(\theta,z)\left(\ln \left(\rho_{pro}(\theta,z)\right) -1 + \beta\mu^0_{pro} \right)} \\
		%%%%%%
		&+ \sum_\alpha{P(\alpha)\ln P(\alpha)} \\
		%%%%%%%
		& + A\int_0^\infty dz \left<\rho_{MAA}(z)\right> \left[f(z)(\ln f(z)+ \beta\mu^0_{MAA^-})\right.\\
		& \qquad\qquad\qquad \left.+(1-f(z))(\ln (1-f(z))+\beta\mu^0_{MAAH})\right] \\
		%%%%%%%
		& + A\int_0^\infty dz \sum_\tau \left<\rho_{pro,\tau}(z)\right> \left[g_\tau(z)(\ln g_\tau(z)+ \beta\mu^0_{\tau p})\right.\\
		&\qquad \qquad \qquad\qquad \qquad\quad \left.+(1-g_\tau(z))(\ln (1-g_\tau(z))+\beta\mu^0_{\tau d})\right]   \\
		%%%%%%%
		& +A\int_0^\infty dz \left[\left(\sum_{\gamma } {\rho_\gamma(z) q_\gamma + \sum_\tau{f_\tau(z) \left<\rho_{pro,\tau}(z)\right> q_\tau} +  f(z) \left<\rho_{MAA}(r)\right>q_{MAA}}\right)\beta\psi(z) \right. \\ & \qquad \qquad \left.-\frac{1}{2}\beta\epsilon(\nabla\psi(z))^2 \right] \\ 
		%%%%%%%%
		& +A \int_0^\infty dz\beta\pi(z){\left[\sum_{\gamma}\rho_\gamma(z) v_\gamma + \sum_\lambda{\left<\rho_{pro,\lambda}(z)\right>v_\lambda} + \left<\rho_{MAA}(z)\right>v_{MAA} -1 \right]} \\
		%%%%%%%%%%%
		&   -\sum_\gamma A\int_0^\infty dz \left(\beta \mu_\gamma \rho_\gamma(z) - \beta \mu_{pro} \left<\rho_{pro}(z)\right> \right)  \\
		&  -A\int_0^\infty \beta\mu_{H^+} \left( \sum_\tau\left< \rho_{pro,\tau}(z) \right>g_\tau(z) + (1-f(z)) \left< \rho_{MAA}(z) \right> \right )
	\end{aligned}
\end{align}
 
Obtenido la expresi\'on que define la energ\'ia del sistema es necesario encontrar las condiciones en las cuales se minimiza la misma. Para ello se deriva este potencial respecto a los funcionales que lo componen.
A continuaci\'on se mostrara la optimizaci\'on de este gran potencial respecto de los funcionales presentados en  ecu. \ref{eq:film:funcionales}.

En part\'icular la optimizaci\'on respecto a la densidad de las especies libres, $\rho_\gamma$ resulta en:

\begin{align}
	\rho_\gamma(z)v_w = a_\gamma \exp\left[-\beta q_\gamma\psi(z)\right] \exp\left[-\beta v_\gamma\pi(z)\right]
	\label{eq:film:free-species}
\end{align}

\noindent en donde la actividad de la especie $\gamma$ se define como:
\begin{align}
	a_\gamma = \exp[\beta\mu_\gamma - \beta\mu^0_\gamma]
\end{align}

En esta expresi\'on se ve la influencia de los potenciales qu\'imicos de las especies libres,  $\mu_\gamma$, los cuales  deben estar en equilibrio con el bulk de la soluci\'on. Las actividades qu\'imicas est\'an completamente determinadas por la composici\'on (pH, T, concentraci\'on salina) del seno de la soluci\'on.
 
El grado de disociaci\'on de los segmentos de $MAA$ viene dado por:

\begin{align}
	\frac{f(z)}{1-f(z)} = \frac{K^0_a}{a_{H^+}} \exp[-\beta q_{MAA}\psi(z)]
	\label{eq:film:degree-film}
\end{align}

\noindent en donde se define la constante termodin\'amica del equilibrio \'acido-base para los segmentos de $MAA$ como:

\begin{align}
	K_{a,MAA}^0 = \exp[-\beta\mu^0_{MAAH} -\beta\mu^0_{MAA^-} -\beta\mu^0_{H^+}]
	\label{eq:film:ka-acido-base}
\end{align}

Del mismo modo el grado de disociaci\'on de los segmentos titulables $\tau$:

\begin{align}
	\frac{f_\tau(z)}{1-f_\tau(z)} = \left(\frac{a_{H^+}}{K^0_{a,\tau}}\right)^{\mp 1} \exp[-\beta q_\tau \psi(z)]
	\label{eq:film:degree-protein}
\end{align}


\noindent la constante termodin\'amica para el equilibrio de los segmentos $\tau$ se define de igual forma que en ecu. \ref{eq:film:ka-acido-base}, el exponente,$\mp 1$, cambia si se trata de segmentos \'acidos o b\'asicos respectivamente.

Optimizando respecto a la probabilidad de las conformaciones de la red polim\'erica se obtiene:

\begin{align}
	\begin{aligned}
	P(\alpha) = &\frac{1}{Q}\exp\left[ -A \int^\infty_0 \rho_{MAA}(\alpha, z) \ln f(z)\right] \\
	%%%%
	&\exp\left[ -A \int^\infty_0 \rho_{MAA}(\alpha, z) \beta q_{MAA} \psi(z)\right] \\
	& \exp\left[ -A \int^\infty_0 \rho_{MAA}(\alpha, z) \beta v_{MAA} \pi(z)\right] 
	\end{aligned}
	\label{eq:film:probabilidad}
\end{align}

\noindent en donde:
\begin{align}
	\begin{aligned}
		Q = &\sum_\alpha \left\{ \exp\left[ -A \int^\infty_0 \rho_{MAA}(\alpha, z) \ln f(z)\right]\right\} \\
		%%%
		& + \sum_\alpha\left\{ \exp\left[ -A \int^\infty_0 \rho_{MAA}(\alpha, z) \beta q_{MAA} \psi(z)\right]  \right\} \\
		%%%
		& + \sum_\alpha\left\{ \exp\left[ -A \int^\infty_0 \rho_{MAA}(\alpha, z) \beta v_{MAA} \pi(z)\right]  \right\} 
	\end{aligned}
\end{align}

Constante con la cual se tiene en cuenta que la sumatoria de las probabilidades de cada conformaci\'on de la red polim\'erica sea 1:

\begin{align}
	\sum_\alpha P(\alpha) = 1                 
\end{align}

La densidad local de nuestra prote\'ina en una conformaci\'on $\theta$ se deriva de la expresi\'on:

\begin{align}
	\begin{aligned}
	\rho_{pro}(\theta, z)v_w = &\tilde{a}_{pro} \prod_\tau\exp\left[-A\int_0^\infty dz^\prime n_\tau(\theta,z,z^\prime) \ln f_\tau(z^\prime)\right] \\
	& \prod_\lambda \exp \left[-A\int^\infty_0 dz^\prime n_\lambda(\theta,z, z^\prime)[v_\lambda\beta\pi(z^\prime) + q_\lambda \psi(z^\prime)] \right]
	\end{aligned}
\label{eq:film:proteina}
\end{align}

En esta expresi\'on se ha redefinido el potencial qu\'imico est\'andar de la prote\'ina y si los segmentos son de naturaleza \'acida $\tau, a$ o b\'asica $\tau,b$:

\begin{align}
	\beta\tilde{\mu}^0_{pro} =  \beta \mu^0_{pro}  + \sum_{\tau,a} C_{n,\tau}\beta\mu^0_{\tau,d} 
	+ \sum_{\tau,b} C_{n,\tau}\beta(\mu_{H^+} - \mu^0_{\tau,p})
\end{align}

\noindent en donde se define el n\'umero de composici\'on, $C_{n,j}$, para un segmento $j$:
\begin{align}
	C_{n,j} = A\int_0^\infty dz \, n_j(\theta, z^\prime, z), \forall z^\prime
	\label{eq:film:n-coord}
\end{align}

 resultando:
\begin{align}
	\tilde{a}^0_{pro} = \exp[\beta\mu_{pro} - \beta\tilde{\mu}^0_{pro}]
	\label{eq:film:actividad-pro}
\end{align}

La variaci\'on de nuestra energ\'ia respecto del potencial electrost\'atico que da origen a la ecuaci\'on de Poisson:

\begin{align}
	\epsilon \nabla^2 \Psi(z) = - \left< \rho_q (z)\right>
\end{align}

%%%%
%% qué sentido tiene esto ??
En esta expresi\'on podemos observar  el acoplamiento local entre las interacciones f\'isicas, la organizaci\'on molecular, los grados de libertad, conformaciones y equilibrios qu\'imcos. Para ello hay que tener en cuenta la densidad de carga definida en ecu. \ref{eq:film:rho_charge} 
%%%%

Para la resoluci\'on de nuestro sistema, es decir que el mismo  se encuentre en equilibrio, se han impuesto ciertas restricciones, como la incompresibilidad o equilibrio de potenciales qu\'imicos ecuaciones \ref{eq:film:constraint} y \ref{eq:film:equil-qco} respectivamente. Otra restricci\'on que se impone es la electro neutralidad de la soluci\'on: 

\begin{align}
	\int_0^\infty dz \left< \rho_q (z)\right> = 0
\end{align}

Esta restricci\'on se satisface en la ecuaci\'on de Poisson al considerar las condiciones de contorno adecuadas, las cuales definimos:

\begin{align}
	\begin{aligned}
		&i)  \lim_{z\to\infty}\psi(z) = 0 \\
		&ii) \left.\frac{d\psi(z)}{dr}\right|_{z=0} = 0
		\label{eq:film:contorno}
	\end{aligned}
\end{align}

Estas condiciones significan que el potencial electrost\'atico se desvanece a medida que nos alejamos de nuestro film polim\'erico,  y que el medio diel\'ectrico en el cual se encuentra el film  se extiende desde su superficie de soporte, respectivamente. 

En este punto hemos mostrados las expresiones que optimizan a nuestro gran potencial, y c\'omo cada uno de estos funcionales: $P(\alpha), \rho_\gamma(z),\rho_{pro}(z), f_\tau(z), f(z), \psi(z) $ a su vez  terminan siendo definidos por dos potenciales locales: Electrost\'atico $\psi(z)$ y Presi\'on osm\'otica $\pi(z)$. 
Como primera conclusi\'on de esta teor\'ia es que es posible calcular y describir la termodin\'amica del sistema dadas las condiciones de laboratorio: pH, concentraci\'on de sal y adsorbatos, temperatura. Utilizando y resolviendo las ecuaciones de incompresibilidad y  electro-neutralidad, adem\'as de otros par\'ametros impuestos en la entrada: vol\'umenes moleculares, cargas y constantes de disociaci\'on. as\'i como la distintas conformaci\'on que pueden adquirir cada elemento del sistema, es decir la distribuci\'on espacial de cada segmentos en sus conformaciones
Estas par\'ametros de entrada son provisto al c\'odigo mediante el uso de un modelo molecular. 

La forma de obtener estas variables desconocidas, $\psi(z)$ y $\pi(z)$  es realizando una soluci\'on num\'erica (puede verse en el anexo \ref{sec:film:reso-numerica}) por sustituci\'on en la diferentes ecuaciones en las que interact\'uan: la densidad de las especies libres ecu. \ref{eq:film:free-species}, los grados de disociaci\'on de los segmentos del film y de los  titulables de la prote\'ina ecuaciones \ref{eq:film:degree-film} y \ref{eq:film:degree-protein} respectivamente, la probabilidad de las conformaciones de la red polim\'erica ecu. \ref{eq:film:probabilidad} y la densidad local de la prote\'ina ecu. \ref{eq:film:proteina}.

Una vez obtenidos los potenciales $\pi(z)$ y $\psi(z)$ es posible derivar  cualquier cantidad termodin\'amica de inter\'es  a partir de la energ\'ia libre o haciendo uso de alguna expresi\'on explicita. 

Por ejemplificar la fracci\'on de volumen local ocupada por la prote\'ina puede ser calculada como:

\begin{align}
	\left< \phi_{pro}(z) \right> = A\int_0^\infty dz^\prime \sum_\theta \rho(\theta, z^\prime)\sum_\lambda n_\lambda(\theta, z^\prime, z)v_\lambda
\end{align}

Con esta cantidad es posible cuantificar la adsorci\'on de la prote\'ina en el film. 

\section{Soluci\'on Bulk}\label{sec:film:bulk-solution}

Como mencionamos en la secci\'on anterior, la energ\'ia del sistema es expresada como un potencial gran can\'onico el cual proviene de un ensamble gran can\'onico, en el cual el sistema de estudio  esta en equilibrio con un ``ba\~no de  la soluci\'on". Traducido a nuestro trabajo significa que el film y sus alrededores se encuentran en constante equilibrio con el bulk de la soluci\'on en la que est\'an bien definidas las variables como pH, temperatura y concentraci\'on de sal y/o prote\'ina. 
Dadas estas condiciones la resoluci\'on del bulk de la soluci\'on es indispensable para llevar a cabo nuestras simulaciones. Las soluciones nos proporcionan la informaci\'on de las actividades de las especies m\'oviles, y con ella sus potenciales qu\'imicos. Informaci\'on que como vimos anteriormente es necesaria para la resolución de las ecuaciones de incompresibilidad y de Poisson.

El procedimiento te\'orico es en esencia el mismo: escritura de la energ\'ia libre (y su respectivo potencial) y minimizaci\'on respecto de las funcionales  que lo componen.

De esta forma el potencial dado por el bulk de la soluci\'on se escribe:

\begin{align}
	\begin{aligned}
		\beta\frac{ \Phi^b}{V}=  & \sum_{\gamma}{\rho^b_\gamma\left(\ln \left(\rho^b_\gamma v_w\right) -1 + \beta\mu^0_\gamma\right)} \\
		%%%%
		&+ \sum_{\theta}{\rho^b_{pro}(\theta)\left(\ln \left(\rho^b_{pro}(\theta)\right) -1 + \beta\mu^0_{pro} \right)} \\
		%%%%%%
		& + \sum_\tau \left<\rho^b_{pro,\tau}\right> \left[g^b_\tau(\ln g^b_\tau+ \beta\mu^0_{\tau p}) +(1-g^b_\tau)(\ln (1-g^b_\tau)+\beta\mu^0_{\tau d})\right]   \\
		%%%%%%%
		& +\left[\left(\sum_{\gamma } {\rho^b_\gamma q_\gamma + \sum_\tau{f^b_\tau \left<\rho^b_{pro,\tau}\right> q_\tau} }\right)\beta\psi^b -\frac{1}{2}\beta\epsilon(\nabla\psi^b)^2 \right] \\ 
		%%%%%%%%
		& +\beta\pi^b{\left[\sum_{\gamma}\rho^b_\gamma v_\gamma + \sum_\lambda{\left<\rho^b_{pro,\lambda}\right>v_\lambda}  -1 \right]} \\
		%%%%%%%%%%%
		&   -\sum_\gamma \left(\beta \mu_\gamma \rho^b_\gamma - \beta \mu_{pro} \left<\rho^b_{pro}\right> \right)   -\beta\mu_{H^+} \left( \sum_\tau\left< \rho^b_{pro,\tau} \right>g_\tau  \right )
	\end{aligned}
	\label{eq:film:pot-bulk}
\end{align}

\noindent en donde el super\'indice $b$ denota el Bulk de la soluci\'on. 
 Los ensambles, expresiones entre brackets ($\left<\rho\right>$) se definen de la misma forma que en la secci\'on anterior. Teniendo las salvedades que no hay dependencia sobre la coordenada $z$:
 \begin{align}
 	\left<\rho^b_{pro}\right> = \sum_{\theta}\rho^b_{pro}(\theta)
 \end{align}
 Adem\'as se ha considerado las condiciones a $z \to \infty$:

\begin{align}
	\begin{aligned}
		& i)\rho^b_i =\rho_i (z \rightarrow \infty)   \\
		& ii)  \pi^b = \pi(z \rightarrow \infty) \\
		& iii) g_\tau^b = g_\tau(z \rightarrow \infty)  
	\end{aligned}
\end{align}
\noindent en donde $i$ hace referencia a las especies libres y la prote\'ina. 

en consecuencia para las especies libres $\gamma$ resulta:
\begin{align}
	\rho^b_\gamma v_w = a_\gamma \exp\left[ -\beta q_\gamma \psi^b -\beta \pi^b v_\gamma \right]
	\label{eq:film:rhofree-bulk}
\end{align}

El grado de disosiaci\'on de los segmentos $\tau$ titulables de la prote\'ina:

\begin{align}
		\frac{f^b_\tau}{1-f^b_\tau} = \left(\frac{a_{H^+}}{K^0_{a,\tau}}\right)^{\mp 1} \exp[-\beta q_\tau \psi^b]
\end{align}


finalmente para la densidad de prote\'ina se obtiene:

\begin{align}
	\begin{aligned}
		\rho_{pro}(\theta)v_w = &\tilde{a}_{pro} \prod_\tau \exp\left[-cn_\tau \ln f^b_\tau\right] \\
		& \prod_\lambda \exp \left[-cn_\lambda (v_\lambda\beta\pi^b + q_\lambda \psi^b) \right]
	\end{aligned}
	\label{eq:film:rhopro-bulk}
\end{align}

\noindent en donde $\tilde{a}_{pro}$ y $cn_\tau$ (y $cn_\lambda$) son definidos en las ecuaciones  \ref{eq:film:actividad-pro} y \ref{eq:film:n-coord} respectivamente. 

Podemos observar nuevamente que nuestros funcionales quedan en funci\'on, valga la redundancia, por la presi\'on osm\'otica $\pi^b$ y el potencial electrost\'atico $\psi^b$
Sin embargo si consideramos las condiciones de contorno dada en la ecu. \ref{eq:film:contorno} para la ecuaci\'on de Poisson, vemos que en la soluci\'on bulk se debe cumplir: $\psi^b = 0$. 

Esto  muestra que para nuestro ba\~no de soluci\'on la principal inc\'ognicta es la presi\'on osm\'otica: $\pi^b$.
La cual es posible obtenerla por resoluci\'on num\'erica al sustituir las ecuaciones  \ref{eq:film:rhofree-bulk} ,  \ref{eq:film:rhopro-bulk}  y sus respectivas actividades (ecu. \ref{eq:film:act-bulk-free} y \ref{eq:film:act-bulk-pro} ) en la nueva condici\'on de incompresibilidad dada por:

\begin{align}
	\sum_\gamma \rho^b_\gamma v_\gamma + \sum_\lambda\left< \rho^b_{pro,\lambda}\right> v_\lambda = 1
	\label{eq:incom-bulk}
\end{align}

Como se mencion\'o al inicio de esta secci\'on la resoluci\'on del bulk de la soluci\'on, en concreto el c\'alculo de $\pi^b$, nos provee la informaci\'on para las actividades de las especies m\'oviles:

\begin{align}
	a_\gamma =\frac{\rho^b_\gamma v_w}{\exp\left[-\beta \pi^b v_\gamma - \beta q_\gamma\psi^b\right]}
	\label{eq:film:act-bulk-free}
\end{align}

 y para la prote\'ina:

\begin{align}
	\begin{aligned}
		\tilde{a}_{pro} = & \rho_{pro}(\theta)v_w\exp\left[cn_\tau \ln f^b_\tau\right] \\
		& \exp \left[cn_\lambda (v_\lambda\beta\pi^b + q_\lambda \psi^b) \right]
	\end{aligned}
\label{eq:film:act-bulk-pro}
\end{align}


Hay que tener en cuenta que las densidades en el bulk de la soluci\'on son par\'ametros de entrada en cada c\'alculo. Una vez que se establecen el pH, la concentraci\'on de sal y de nuestra prote\'ina, estas densidades se determinan completamente (usando la electro-neutralidad de la soluci\'on del bulk y la aut- disociaci\'on de equilibrio del agua).

\section{Modelo Molecular}

Para poder desarrollar la teor\'ia molecular es necesario contar con un modelo en el cual puedan ser introducidas todas las caracter\'isticas molecular del sistema, y por lo cual la teor\'ia recibe su nombre.
El tipo de modelado a usar es crucial dado que puede definir diferentes variables, tiempos de c\'alculos, tipos de interacciones, entre otros. 
En esta tesis hacemos uso de modelos de grano grueso en los cuales una \textbf{unidad de grano grueso} esta compuesta por un grupo de \'atomos con los cuales  se busca hacer una representaci\'on simplificada de un sistema complejo que tiene como objetivo simular su comportamiento.
La elecci\'on de cuantos \'atomos conforman una unidad de grano grueso depende de las caracter\'isticas a modelar. 
Para nuestros modelos utilizamos una definici\'on en donde no es la cantidad sino de la importancia qu\'imica y relevante para responder las preguntar que se plantean en el sistema. Nos enfocamos en segmentos de mol\'eculas que posean carga o puedan ser titulables, as\'i como aquellos que posean alg\'un tipo de interacci\'on que es considerada en las ecuaciones de potencial qu\'imico usado.
%%%%%%%%%%%%%%%%%%%%%%%%%%%%%%%%%%%%%%

\subsubsection{Molecular Model}


\begin{figure}[ht]
	\centering
	\includegraphics[width=0.5\textwidth]{Figures/graph-film/chargeAds.png}
	\caption{Average net charge number of doxorubicin and the polyamines as a function of pHin dilute solutions.
	The right-hand side axis gives the degree of charge of an isolated methacrylic acid unit. The vertical dash-dotted line indicates physiological pH. (Theoretical predictions.)}
	\label{fig:film:model}
\end{figure}


Este formalismo te\'orico requiere una representaci\'on molecular de todas las especies qu\'imicas en el sistema. La figura \ref{fig:film:model_poliamines} incluye el esquema de grano grueso utilizado para el MAA y las especies en soluci\'on. Las poliaminas (putrescina, espermidina y espermina) se representan utilizando sus diferentes grupos amino (N1 a N5). El modelo de doxorubicina incluye los anillos (D1, D2 y D4) as\'i como el grupo carbox'ilico (D3). El volumen, pKa y carga de estos grupos, al igual que los de las mol\'eculas de agua e iones, se presentan en la tabla \ref{table:CG}. Las geometr\'ias moleculares, as\'i como los valores de pKa asignados a las diferentes unidades de grano grueso, se han obtenido de la literatura \addcite[agostinelli2010polyamines, casero2009recent, puchem].

La red polim\'erica est\'a compuesta por cadenas entrecruzadas de 50 segmentos de longitud; cada segmento de cadena es una representación de grano grueso de una unidad de MAA (ver la figura \ref{fig:film:model_poliamines}). Esta red tiene una topolog\'ia similar a la de un diamante, con las unidades de entrecruzamiento con una coordinaci\'on de cuatro \addcite[Mann2005, QuesadaPerez2012M, Kosovan2015, Hofzumahaus2018]. Las conformaciones de la red se generan mediante simulaciones de din\'amica molecular. El set de configuraciones fueron obtenidas de un trabajo anterior, realizado en el mismo grupo de investigaci\'on. Puede verse la descripci\'on en \addcite[Hagemann2018].



Con el esquema de pKa descrito anteriormente, la figura \ref{fig:film:model_poliamines} muestra la carga el\'ectrica promedio de cada poliamina y la de la doxorubicina. El gr\'afico tambi\'en muestra el grado de carga de un mon\'omero de MAA aislado en una soluci\'on diluida. La fuerza impulsora para la absorci\'on son las atracciones electrost\'aticas entre el absorbato y el MAA. La carga positiva de todos los absorbatos disminuye con el aumento del pH. El pol\'imero se vuelve m\'as negativamente cargado a medida que aumenta el pH. Por lo tanto, esperamos que la absorci\'on sea una funci\'on no mon\'otona del pH de la soluci\'on. Adem\'as, el punto isoel\'ectrico de la doxorubicina se encuentra alrededor del pH neutro, lo que implica que el f\'armaco es neutro en cuanto a carga en condiciones fisiol\'ogicas y tiene carga negativa a valores de pH m\'as altos.



\begin{table}[!ht]
	\begin{centering}
		\centering
		%\footnotesize
		\setlength{\tabcolsep}{2.2pt}
		\begin{tabular}{|cccc|}
			\hline 
			\hspace{0pt}CG unit   & $pKa$ & $q$  & $v$ ($\text{nm}^3$)  \\ \hline
			$N_1$& 9.9 & (+1) &0.051\\
			$N_2$& 10.9& (+1) & 0.051\\ 
		     $N_3$& 8.4& (+1)& 0.063\\
		     $N_4$&7.9& (+1) & 0.063\\
		     $N_5$& 10.1& (+1)& 0.051\\
		      $D_1$&  - & 0 &0.085\\
		     $D_2$& 7.34 & (-1) & 0.085\\ 
			 $D_3$&  8.46& (-1)& 0.035\\
			$D_4$&  9.46 & (+1) &0.085\\ 
			$MAA$&  4.65 & (-1) & 0.065\\
			$H_2O$ & - & 0 & 0.03\\
			$H_3O^+$ & - & +1 & 0.03\\
			$OH^-$ & - & -1 & 0.03\\
			$Na^+$ & - & +1 & 0.033\\ 
			$Cl^-$ & - & -1 & 0.033\\ \hline
		\end{tabular}		
		\caption{Coarse grain model. The pKa values assigned to the different coarse grain units have all been obtained from the literature \addcite[agostinelli2010polyamines, casero2009recent, puchem], and the molecular volumes correspond to Van der Waals values. \footnotesize{() indicate the charge of the ionized chemical species.}}
		\label{table:CG}
	\end{centering}
\end{table}

%%%%%%%%%%%%%%%%%%%%%%%%%%%%%%%%%%%%%%%%%%


\begin{figure*}[!htb]
	\centering
	\includegraphics[width=0.7\textwidth]{Figures/graph-film/exp_synt_scheme.png}
	\caption{Steps of the thin PMAA film synthesis, represented as: A) ATRP initiator on substrates; B) t-BMA film growth; C) hydrolysis of t-BMA.
		Contact angle measurements can be found in the SI, which confirm successful substrate modification after each step.}
	\label{fig:film:synthesis_scheme}
\end{figure*}




\subsection{Experimentos}


Con el fin de respaldar la teor\'ia y el modelado molecular, se prepararon films de hidrogel polim\'ericos basadas en poli(metacrilato de \'acido) y se llevaron a cabo experimentos de carga y liberaci\'on de doxorubicina en presencia de espermina y espermidina. El \'angulo de contacto del film se midi\'o en cada paso para seguir el procedimiento de s\'intesis.



\subsubsection{Crecimiento controlado de pel\'iculas delgadas de PMAA.}

Los films  de PMAA se obtuvieron mediante la t\'ecnica de Polimerizaci\'on Radical por Transferencia de \'Atomo (ATRP por sus siglas en ingl\'es ), que permite la polimerizaci\'on iniciada en la superficie. La s\'intesis completa de la pel\'icula de PMAA consta de tres pasos principales, como se representa en la figura \ref{fig:film:synthesis_scheme}. Estos pasos son los siguientes:
1) Inmovilizaci\'on del iniciador de ATRP en los sustratos (portaobjetos de vidrio y obleas de silicio de un solo pulido).
2)Procedimiento de ATRP utilizando ter-butil metacrilato (t-BMA) y N,N'-Metilenbisacrilamida (BIS) como agente de entrecruzante
3)Hidr\'olisis de los films resultantes de ter-butil metacrilato entrecruzado con \'acido trifluoroac\'etico para obtener las pel\'iculas de PMAA.


\emph{1) Inmovilizaci\'on del iniciador de ATRP en los sustratos.}
 Despu\'es de limpiar con agua jabonosa, etanol y acetona en un ba\~no ultras\'onico, los portaobjetos de vidrio y las obleas de silicio se modificaron mediante inmersi\'on en una solución al 2\% v/v de 2-bromo-2-metil-N-(3-(tri-etoxisilil)propil)propenamida (preparada seg\'in lo informado previamente \addcite[Yameen2008]) en etanol seco durante 1 hora a $30^\circ C$. Luego, los sustratos se lavaron con etanol y se curaron durante 2 horas en un horno $60^\circ C$ bajo vac\'io. El \'angulo de contacto medido (goni\'ometro de\'angulo de contacto Ramè-Hart modelo 290) utilizando agua fue de alrededor de $63.6^\circ \pm 0.1^\circ$ sin cambios con el tiempo, ligeramente menor que el del portaobjetos de vidrio ($64.7^\circ \pm 0.2^\circ$).% El conjunto completo de mediciones se muestra en el \SuppInfo (\SI).




\emph{2) Crecimiento de la película de ter-butil metacrilato.}
 Las polimerizaciones de ATRP se llevaron a cabo de acuerdo con \addcite[Brown2009] Se prepar\'o una soluci\'on de t-BMA (15 ml, 92 mmol, Aldrich 98\%), BIS (422 mg, 2.76 mmol, Aldrich 99\%), CuBr$_2$ (4.1 mg, 0.018 mmol, Aldrich 99.999\%), y N,N,N,N,N-Pentametildietilentriamina (PMDETA, 0.12 ml, 0.55 mmol, Aldrich, 99\%) disuelta en DMSO (15 ml) y se desgasific\'o mediante burbujeo de N$_2$ durante una hora a temperatura ambiente. Luego, se agregó CuBr (26.5 mg, 0.18 mmol, Aldrich 99.999\%) y la mezcla se dej\'o bajo N$_2$ durante 15 minutos. Simult\'aneamente, los sustratos con iniciador se sellaron en tubos Schlenk y se desgasificaron mediante ciclos de vac\'io/N$_2$. Luego, la mezcla de reacci\'on se inyect\'o en estos tubos Schlenk para cubrir completamente las muestras. La mezcla se dej\'o reposar durante 24 horas bajo N$_2$, y luego se retiraron los sustratos y se lavaron con DMSO, acetona y se secaron con N$_2$. El \'angulo de contacto medido fue de $89.3^\circ \pm 0.1^\circ$ sin cambios con el tiempo.

\emph{3) Hidrólisis del ter-butil metacrilato entrecruzado obtenido.}
 Las pel\'iculas entrecruzadas de PtBMA se sumergieron en una soluci\'on de \'acido trifluoroac\'etico (Aldrich 99\%) en CH$_2$Cl$_2$ (50\% v/v) durante una hora a temperatura ambiente. Luego, los sustratos se lavaron varias veces con agua y se dejaron sumergidos en agua durante 10 minutos antes de secarlos con N$_2$. El \'angulo de contacto medido utilizando agua disminuy\'o con el tiempo, desde un valor inicial de alrededor de 69$^\circ$ hasta 63$^\circ$; este comportamiento indica una exitosa hidr\'olisis de la película de PtBMA en una pel\'icula de PMAA m\'as hidrofílica, exponiendo grupos carbox\'ilicos, que se hinchan/hidratan con el tiempo.


\begin{figure*}[htb]
	\centering
	\includegraphics[width=0.7\textwidth]{Figures/graph-film/exp_doxo_load_scheme.png}
	\caption{Scheme showing doxorubicin loading to the PMAA hydrogel film and release in presence of spermidine$^{\ast\ast}$.}
	\label{fig:film:exp_doxo_scheme}
\end{figure*}



\subsubsection{Captura y liberación de doxorrubicina en presencia de poliaminas}


La carga de doxorrubicina (Doxo, Aldrich 98.0-102.0\% en clorhidrato de doxorrubicina) se realiz\'o sumergiendo las pel\'iculas de PMAA en una soluci\'on de la droga al $10^-2$ M en agua ultrapura \textit{Milli-Q} ($18.2 M \Omega cm$) durante 24 horas en el refrigerador (ver figura \ref{fig:film:exp_doxo_scheme}). Se utilizaron las bandas de absorci\'on de Doxo en el rango de 450-550 nm para caracterizar su carga en el film utilizando un espectrofot\'ometro Lambda 35 de Perkin Elmer. Para ello, despu\'es de la carga, los sustratos se sumergieron durante unos segundos en agua ultrapura para eliminar las mol\'eculas adheridas a la superficie y se secaron con un flujo de N2. Finalmente, los sustratos se colocaron verticalmente en el camino \'optico del espectrofot\'ometro y se realizaron los an\'alisis, abarcando el rango de 200-800 nm, con una velocidad de escaneo de 480 nm/min y un ancho de ranura de 1 nm.

La liberaci\'on se realiz\'o sumergiendo los sustratos cargados en agua o en soluciones de espermidina (Aldrich$ >$ 99\%) o espermina (Aldrich $>$ 97\%), cada una al 2.8\% en peso/volumen (ver figura \ref{fig:film:exp_doxo_scheme}) en diferentes momentos. La liberaci\'on se sigui\'o de la misma manera que la carga. Estas soluciones de poliaminas se neutralizaron con hidr\'oxido de sodio al 0.1 M, lo que aument\'o la fuerza i\'onica desde el agua pura y llev\'o a un pH final de 7. Todos los experimentos se realizaron utilizando agua ultrapura \textit{Milli-Q}. La concentraci\'on de poliamina se eligi\'o para explorar un r\'egimen de saturaci\'on, equivalente al extremo de la regi\'on explorada en las predicciones te\'oricas (ver más abajo). De esta manera, se pueden comparar cualitativamente las cantidades de carga de amina equivalentes.

%%%%%%%%%%%%%%%%%%





%%%%
\section{Resultados} \label{sec:film:resultados}

\subsection{Theoretical Results}

La primera pregunta que abordamos es c\'omo responden los hidrogeles de PMAA a las soluciones de poliaminas. El enfoque termodin\'amico permite considerar la absorci\'on dentro de la red polim\'erica a partir de soluciones que contienen estas aminas. Definimos la absorci\'on como:

\begin{align}
	\begin{aligned}
		\Gamma_i= \int_0^\infty{(\rho_i(z) -\rho_i^{bulk})dz}
	\end{aligned}
	\label{eq:film:adsorption}
\end{align}


\noindent Donde la coordenada $z$ mide la distancia desde la superficie que sostiene el film, y el sub\'indice $i$ se refiere al absorbato de inter\'es. 
La funci\'on local $\rho_i(z)$ es la densidad del absorbato, y $\rho_i^{bulk}=\lim_{z\to\infty} \rho_i(z)$ es la densidad en la soluci\'on a granel, lejos del film  de hidrogel. 

Esta definici\'on de $\Gamma$ cuantifica la masa de mol\'eculas absorbidas por unidad de \'area en exceso de la contribuci\'on impuesta por la soluci\'on a granel. 

La expresi\'on dada por la ecu. \ref{eq:film:adsorption} describe la absorci\'on de cada una de las aminas, pero tambi\'en es v\'alida para la doxorrubicina.


\begin{figure*}[!htb]
	\centering
	\includegraphics[width=0.7\textwidth]{Figures/graph-film/amines_ads.png}
	\caption{Plot of the absorption, $\Gamma$, of spermine (A), spermidine (B) and putrescine (C) as a function of its (bulk) solution concentration.
		The various curves in each panel  correspond to different pH values (around $7$) and physiological salt conditions, $[NaCl]=100 \,mM$. Shaded regions indicate the range of healthy polyamine concentrations\addcite[Soda2011] 
		There is no doxorubicin in these solutions. (Theoretical predictions.)}
	\label{fig:film:amines-ads}
\end{figure*}




La figura  \ref{fig:film:amines-ads} muestra las isotermas de absorci\'on para cada una de las poliaminas en funci\'on de su concentraci\'on en la soluci\'on en el bulk. Debido a que existen informes que indican que los alrededores de las c\'elulas tumorales son \'acidos \addcite[vaupel1989blood, Tannock1989, Raghunand1999, rofstad2006acidic, schmaljohann2006thermo, Koltai2016], se incluyen isotermas de absorci\'on para diferentes valores de pH. Enfatizamos que el primer objetivo es investigar la capacidad de los hidrogeles de PMAA para secuestrar poliaminas en condiciones fisiol\'ogicas; luego, los resultados de la figura \ref{fig:film:amines-ads} corresponden a soluciones en ausencia de doxorrubicina y con $100 \, mM$ de $[NaCl]$. Como era de esperar, la absorci\'on aumenta con la concentraci\'on creciente de poliaminas. Esto indica que se alcanza un r\'egimen no saturado con la concentraci\'on explorada, como se puede ver f\'acilmente en las isotermas de absorci\'on.

M\'as cerca de las células sanas, la concentraci\'on de poliaminas se encuentra en el rango de $10^{-4}$ a $10^{-3}\, M$, lo que puede aumentar en un orden de magnitud o m\'as alrededor de los tumores \addcite[Soda2011]. Los resultados de la figura \ref{fig:film:amines-ads} sugieren que los hidrogeles de PMAA pueden responder a estas condiciones capturando cantidades crecientes de aminas. En particular, la putrescina muestra un comportamiento aparentemente de encendido/apagado en torno al inicio de las concentraciones saludables a patol\'ogicas. Adem\'as, excepto para las soluciones m\'as ácidas (pH 5), este comportamiento de absorci\'on se mantiene para diferentes valores de pH en torno a las condiciones fisiol\'ogicas, lo que apunta a la capacidad del hidrogel para adaptarse a peque\~nos cambios de pH causados por c\'elulas da\~nadas.

Las poliaminas absorbidas se distribuyen de manera m\'as o menos homog\'enea dentro de la película de hidrogel . Sin embargo, esta distribuci\'on est\'a correlacionada con la densidad del pol\'imero: se encuentran concentraciones m\'as altas de poliamina en las regiones de fracci\'on de volumen localmente alta de PMAA. Por lo tanto, las poliaminas tienen una mayor probabilidad de encontrarse cerca de las intersecciones de la red, donde ocurre la mayor densidad de pol\'imero. Un comportamiento similar ha sido predicho por \addcite[Sai2020] mediante simulaciones de din\'amica molecular, quienes sugirieron que los anfífilos de crom\'oforos se autoensamblan en los nodos de hidrogeles polielectrol\'iticos cruzados qu\'imicamente.


En condiciones similares, el hidrogel absorbe m\'as espermina que espermidina y putrescina (compare los paneles A, B y C de la fig.  \ref{fig:film:amines-ads}, respectivamente), lo que se puede explicar sobre la base de la carga positiva neta de las poliaminas: cuanto m\'as cargada se encuentre la amina, m\'as se absorber\'a en lel film porque una carga positiva m\'as alta reduce el costo entr\'opico de la confinaci\'on del contrai\'on al tiempo que permite las atracciones electrost\'aticas con el pol\'imero MAA.

\addcite[citet: Schimka2017] han estudiado la interacci\'on entre microgeles de pol\'imero y surfactantes fotosensibles con poliaminas sint\'eticas como grupos principales. Estos surfactantes tienen diferentes n\'umeros de grupos amino con cargas entre $+1$ y $+3$. Los microgeles ani\'onicos se deshinchan cuando se exponen a una concentraci\'on suficientemente alta de surfactante (dependiendo del estado de isomerizaci\'on del surfactante). Su estudio muestra tanto experimental como te\'oricamente que se necesita una mayor concentraci\'on de surfactante para desencadenar el deshinchamiento de los microgeles cuando se reduce la carga del grupo principal de poliamina. Este comportamiento indica que aumentar el n\'umero de grupos amino facilita la captaci\'on del surfactante por parte del microgel, en acuerdo con los resultados aquí presentados.




\begin{figure}[!htb]
	\centering
	\includegraphics[width=0.35\textwidth]{Figures/graph-film/doxo_load.png}
	\caption{Color map showing the absorption of doxorubicin (see \ref{eq:film:adsorption}) as function of the composition of the solution that embeds the hydrogel, \emph{i.e.}, pH and $[NaCl]$;
		the concentration of the drug in the solution is $[Doxo]=1\, mM$. (Theoretical predictions.)}
	\label{fig:film:doxo-load}
\end{figure}


A continuaci\'on, consideramos la capacidad de las pel\'iculas de hidrogel de PMAA para incorporar doxorrubicina (Doxo). Los hidrogeles de cadenas de poli\'acido entrecruzadas son sensibles a cambios en la fuerza i\'onica de la soluci\'on o la concentraci\'on de sal \addcite[Zhang2000]. Este comportamiento puede no ser relevante en entornos biol\'ogicos con una concentraci\'on de iones altamente regulada, pero controlar la concentraci\'on de sal es fundamental en el laboratorio para aumentar la carga del agente terap\'eutico dentro del material. Por lo tanto, hemos considerado la absorci\'on de doxorrubicina en condiciones de laboratorio t\'ipicas, abarcando un amplio rango de pH y $[NaCl]$, en contraposici\'on a las condiciones fisiol\'ogicas.

La figura  \ref{fig:film:doxo-load} muestra la absorci\'on en exceso de Doxo, calculada utilizando la ecu. \ref{eq:film:adsorption}. Esta figura resalta el efecto de la disminuci\'on de la salinidad para aumentar la absorci\'on del f\'armaco. Debido a que la carga de la doxorrubicina es $+1$ (a bajo pH), su absorci\'on compite con la de los iones de sodio para neutralizar la carga del pol\'imero. Por lo tanto, las mejores condiciones para su incorporaci\'on en la pel\'icula corresponden a la reducci\'on de la disponibilidad de contrai\'on de sal.


\begin{figure}[!htb]
	\centering
	\includegraphics[width=0.35\textwidth]{Figures/graph-film/doxo-charge.png}
	\caption{Plot showing doxorubicin charge number inside the PMAA network as a function of pH, for different salt concentrations (solid lines) as well as the average charge of the molecule in the bulk solution (dashed line);
		$[Doxo]=1\, mM$. (Theoretical predictions.)}
	\label{fig:film:doxo-charge}
\end{figure}

En este modelo, la doxorrubicina tiene su punto isoeléctrico (pI) a pH $7.8$. Se puede observar que en  la figura  \ref{fig:film:doxo-load} se muestra que las condiciones \'optimas para la absorci\'on impulsada electrost\'aticamente ocurren en valores de pH cercanos al pI. Este resultado se puede explicar considerando que el pH disminuye dentro del film (ver SI), un efecto que se ha predicho para el interior de una variedad de sistemas polim\'ericos cargados, desde cadenas individuales, capas de pol\'imero injertado, polielectrolitos con estructura de estrella, hasta geles con diferentes topologías \addcite[Nap2006, Borisov2011, Longo2011, Polotsky2013, Murmiliuk2018]. Este efecto resulta en una regulaci\'on de la carga, especialmente por parte del grupo difen\'olico de la doxorrubicina (ver D2 en la fig. \ref{fig:film:model_poliamines} con pKa~$7.3$), que se protona al absorberse. Dentro del film , las mol\'eculas de Doxo absorbidase stán, en promedio, m\'as cargadas positivamente que en la soluci\'on; este comportamiento se describe en la figura \ref{fig:film:doxo-charge}. A un pH fijo, el efecto de reducir las concentraciones de sal de la soluci\'on es aumentar la carga promedio de las mol\'eculas absorbidas. En particular, el pI (aparente) de la Doxo absorbida puede aumentar varias unidades de esta manera.

\begin{figure*}[!htb]
	\centering
	\includegraphics[width=0.7\textwidth]{Figures/graph-film/doxo_release-amines.png}
	\caption{Plot of the absorption of doxorubicin (see ecu.\ref{eq:film:adsorption})as a function of pH for solutions containing spermine (A), spermidine (B) and putrescine (C).
		In each panel, the various solid-line curves correspond to solutions having different concentration of the polyamine, while the dashed-line curve corresponds to a solution without polyamines. All results correspond to $1\, mM$ Doxo and $100\, mM$ $NaCl$ solutions. (Theoretical predictions.)}
	\label{fig:film:doxo-release}
\end{figure*}



\begin{figure}[!htb]
	\centering
	\includegraphics[width=0.35\textwidth]{Figures/graph-film/DG67.png}
	\caption{Plot showing Doxo retention, $\Gamma_{ret}$ (see  ecu. \ref{eq:film:doxo-DG}),  as a function of the concentration of amines for solutions having $100\, mM$~$NaCl$ and $ pH 6$ (top panel) $ pH 7$ (bottom panel). (Theoretical predictions.)}
	\label{fig:film:doxo-DG}
\end{figure}

A continuaci\'on, evaluamos si la doxorrubicina puede liberarse del hidrogel cuando el material es\'a en contacto con soluciones de poliaminas. La figura \ref{fig:film:doxo-release} muestra la absorci\'on de Doxo en el film a partir de soluciones que contienen diferentes concentraciones de poliaminas. Como referencia, los gr\'aficos tambi\'en incluyen la absorci\'on de doxorrubicina a partir de soluciones que no contienen poliaminas. Nuevamente, consideramos una $[NaCl]$ fisiol\'ogica de $100\, mM$, pero ampliamos el rango de valores de pH para describir las condiciones que podr\'ian ocurrir alrededor de tejidos enfermos \'acidos. La absorci\'on de doxorrubicina es una funci\'on no mon\'otona del pH de la soluci\'on, con un m\'aximo entre $6$ y $7$. Este comportamiento no mon\'otono era de esperar, ya que la carga neta negativa del pol\'imero es una funci\'on creciente del pH de la soluci\'on (ver figura \ref{fig:film:model}), mientras que la carga positiva promedio de la Doxo disminuye hasta volverse negativa eventualmente a valores de pH lo suficientemente altos (ver figura  \ref{fig:film:doxo-charge}).



\begin{figure}[!htb]
	\centering
	\includegraphics[width=0.45\textwidth]{Figures/graph-film/Release_water_2.png}
	\caption{Doxorubicin loading to and release from PMAA films in water, represented as the change in absorbance (at $\lambda=490\:nm$) relative to the initial loading at different release times.
		Inset: Plot of the UV-Vis spectra of PMAA films after 24 h in contact with Doxo (red; $t=0$).
		The various curves illustrate drug release after different release times to water.
		The black curve is included as a reference and corresponds to the signal of the bare glass support. $A_G$: Glass substrate absorbance; $A_0$: Fully loaded hydrogel absorbance.}
	\label{fig:film:exp_doxo-water}
\end{figure}



La absorci\'on de doxorrubicina disminuye cuando las poliaminas est\'an presentes en la soluci\'on (compare las diferentes curvas de l\'inea s\'olida en la figura \ref{fig:film:doxo-release} con la curva de l\'inea discontinua en cada panel), lo cual se debe a la captura de poliaminas como se describe en la figura \ref{fig:film:amines-ads}. Esta disminuci\'on en la absorci\'on de Doxo en comparaci\'on con las soluciones sin el f\'armaco se puede interpretar como una liberaci\'on del material. Dado que la espermina es la m\'as cargada, es la m\'as eficiente en prevenir la absorci\'on de Doxo (ver panel A de la figura \ref{fig:film:doxo-release}).



\begin{figure*}[!ht]
	\centering
	\includegraphics[width=0.6\textwidth]{Figures/graph-film/fig11.png}
	\caption{Schematic representation of Doxo release by polyamine uptake (A). Doxorubicin loading to and release from PMAA films in spermine (B) and spermidine (C) solutions represented as the change in absorbance (at $\lambda=490\:$nm) relative to the initial loading at different release times. $A_G$: Glass substrate absorbance; $A_0$: Fully loaded hydrogel absorbance.}
	\label{fig:film:exp_doxo-amines}
\end{figure*}

Para cuantificar a\'un m\'as c\'omo el f\'armaco se desorbe de la pel\'icula de hidrogel cuando se incorporan aminas en la soluci\'on, definimos la reducci\'on relativa en la absorci\'on de doxorrubicina:


\begin{align}
	\begin{aligned}
		\Gamma_{ret}= \dfrac{\Gamma_\text{Doxo}([Amine])}{\Gamma_\text{Doxo}^0}
	\end{aligned}
	\label{eq:film:doxo-DG}
\end{align}


\noindent donde $\Gamma_{ret}$ es la retenci\'on de Doxo, $\Gamma_\text{Doxo}([Amina])$ es la absorci\'on de doxorrubicina para soluciones de amina (ver curvas de l\'inea s\'olida en la figura \ref{fig:film:doxo-release}), y $\Gamma_\text{Doxo}^0=\Gamma_\text{Doxo}([Amina]=0)$ es la absorci\'on en soluciones sin poliaminas (ver la figura \ref{fig:film:doxo-load} y las curvas de línea discontinua en la figura \ref{fig:film:doxo-release}).

La figura \ref{fig:film:doxo-DG} muestra la retenci\'on de Doxo en presencia de cada una de las poliaminas en condiciones fisiol\'ogicas y a un pH de aproximadamente 6. Si todo el f\'armaco se mantiene dentro de la pel\'icula para una soluci\'on de poliamina, entonces $\Gamma_{ret}\approx 1$; mientras que $\Gamma_{ret}\approx 0$ indica que todo el f\'armaco ha sido liberado. Los valores negativos de $\Gamma_{ret}$ pueden ocurrir porque $\Gamma_\text{Doxo}([Amina])$, que es una cantidad en exceso en el bulk, puede tomar valores negativos, lo que indica que en esas condiciones las aminas evitan la absorci\'on de Doxo en la pel\'icula.

La retenci\'on de doxorrubicina en el hidrogel disminuye a medida que aumenta la concentraci\'on de poliamina. Esto es cierto para todas las aminas y para pH 7 y 6 (ver figura \ref{fig:film:doxo-DG}). Como se mencion\'o anteriormente, la espermina es la m\'as eficiente en promover la liberaci\'on de Doxo del hidrogel. Sin embargo, a altas concentraciones de amina, todas las poliaminas pueden impulsar la liberaci\'on de cantidades significativas de doxorrubicina.


\subsection{Espectro UV-Visible}

La carga de doxorrubicina en el hidrogel se caracteriz\'o mediante experimentos de UV-Vis, analizando la absorbancia a $\lambda=490\:$nm. Para abordar la liberaci\'on de Doxo despu\'es de la exposici\'on al agua, los datos de absorbancia se normalizaron en consecuencia, restando la absorbancia del sustrato de vidrio ($A_G$) y luego dividiendo por la absorbancia del hidrogel completamente cargado ($A_0$), tambi\'en con la resta de $A_G$. La figura \ref{fig:film:exp_doxo-water} muestra el cambio relativo en la absorbancia despu\'es de un tiempo de liberaci\'on dado para un hidrogel previamente cargado durante 24 h. El inset muestra los espectros UV-Vis despu\'es de diferentes tiempos de liberaci\'on en agua. Los resultados experimentales en $t=0$ (los espectros rojos mostrados en el inset de la figura \ref{fig:film:exp_doxo-water}) evidencian que las pel\'iculas de PMAA incorporan doxorrubicina dentro de su red polim\'erica. La absorbancia a $\lambda=490\, nm$ para ese espectro corresponde a la barra roja en la figura principal. Despu\'es de un tiempo de exposici\'on suficientemente largo al agua, la liberaci\'on de las pel\'iculas es solo parcial, lo que demuestra la afinidad del f\'armaco por la pel\'icula. Sin embargo, existe una clara y significativa liberaci\'on inicial del f\'armaco en los primeros 10 minutos, un comportamiento que no se observa en las etapas posteriores de liberaci\'on, donde las medidas se reducen a los mismos espectros (ver el inset de la figura \ref{fig:film:exp_doxo-water}).


La liberaci\'on de Doxo puede ser desencadenada al exponer la pel\'icula de hidrogel a poliaminas, como se representa en la figura \ref{fig:film:exp_doxo-amines}A. Para evaluar este comportamiento, se realizaron experimentos de liberaci\'on utilizando spermidina y espermina como poliaminas modelo. La  figura \ref{fig:film:exp_doxo-amines} muestra los cambios en la absorbancia relativa para las pel\'iculas de PMAA cargadas con Doxo y su liberaci\'on en contacto con soluciones de spermidina (panel B) y spermina (panel C) en diferentes tiempos de exposici\'on. La putrescina no se consider\'o en los experimentos debido a su baja absorci\'on en comparaci\'on con la spermina y la spermidina (ver figura \ref{fig:film:amines-ads}). Los resultados muestran que la liberaci\'on del f\'armaco en presencia de estas poliaminas es m\'as r\'apida que en agua, lo que respalda las predicciones de nuestra teor\'ia y modelado molecular: los hidrogeles de PMAA act\'uan como plataformas de absorci\'on para las poliaminas, y tal incorporaci\'on puede desencadenar la liberaci\'on de un f\'armaco previamente cargado.




La figura \ref{fig:film:exp_doxo-amines} B y C muestran que la liberaci\'on en presencia de espermina es más r\'apida que con la incorporaci\'on de spermidina: despu\'es de una exposici\'on de 15 segundos a la espermina, ya no hay se\~nal UV de Doxo, mientras que se requiere un contacto tres veces m\'as largo con spermidina para alcanzar la misma etapa de liberaci\'on. Adem\'as, la magnitud de la liberaci\'on de Doxo utilizando spermidina es menos significativa en comparaci\'on con la liberaci\'on casi completa inducida por la spermina. Estos resultados indican que el n\'umero de grupos amino determina su retenci\'on en la pel\'icula de hidrogel, lo que concuerda con las predicciones de nuestra teor\'ia molecular (ver figur \ref{fig:film:doxo-DG}) y el trabajo de \addcite[citet: Schimka2017].


\begin{figure}[!htb]
	\centering
	\includegraphics[width=0.45\textwidth]{Figures/graph-film/fig12.png}
	\caption{(A) Schematic representation of Doxo re-loading/release after polyamine absorption. (B) The change in absorbance (at $\lambda=490\:$nm) relative to the initial Doxo loading in the hydrogel film (dashed line). $A_G$: Glass substrate absorbance; $A_h$: Absorbance for the initial Doxo loading in the hydrogel film (before polyamine absorption).}
	\label{fig:film:exp_doxo-reload}
\end{figure}


Despu\'es de la carga de poliaminas y la liberaci\'on concomitante de Doxo, las pel\'iculas de PMAA se expusieron a una nueva soluci\'on de Doxo para evaluar si la absorci\'on de poliaminas impediría o no la reabsorci\'on de doxorrubicina, como se muestra en la figur \ref{fig:film:exp_doxo-reload}A. En otras palabras, estos experimentos tienen como objetivo responder si una vez liberada por la absorci\'on de poliaminas, la doxorrubicina se volver\'ia a absorber en el hidrogel, disminuyendo su concentraci\'on en el medio. En l\'inea con el comportamiento esperado, la retenci\'on de espermina por la pel\'icula de hidrogel dificulta significativamente la reabsorci\'on de Doxo dentro de la red polim\'erica \textcolor{red}{(ver Fig. S6 en el SI)}. Por otro lado, los films de hidrogel cargadas con spermidina en contacto con Doxo muestran la posibilidad de recargar el f\'armaco, lo que sugiere una interacci\'on diferente, dependiendo de la poliamina. Como se muestra en el panel B de la figura \ref{fig:film:exp_doxo-reload}, la spermidina permite una recarga inicial de Doxo en el film, pero esta carga se libera f\'acilmente al medio, en comparaci\'on con los resultados de la figura \ref{fig:film:exp_doxo-water}, donde incluso despu\'es de 1 hora de exposici\'on al agua, el Doxo no se libera. Estos resultados tambi\'en resaltan la interacci\'on m\'as fuerte de la espermina (en comparaci\'on con la spermidina) con la red polim\'erica.

En resumen, nuestros resultados experimentales y te\'oricos muestran el potencial de los films de hidrogel basadas en PMAA para cargar f\'armacos terap\'euticos, que se explor\'o utilizando la Doxorrubicina como f\'armaco modelo. Si bien el hidrogel funciona como una plataforma de liberaci\'on controlada, la presencia de altas concentraciones de poliaminas en el medio circundante puede desencadenar la liberaci\'on del f\'armaco al interactuar con la red polim\'erica.



\section{Conclusiones}

Las poliaminas son esenciales para el metabolismo celular, ya que est\'an involucradas en diferentes procesos celulares y sirven como nutrientes para el crecimiento celular. En c\'elulas da\~nadas que muestran un crecimiento descontrolado, como los tumores, hay un aumento en la necesidad de nutrientes, lo que provoca un exceso medible en las concentraciones de poliaminas en los compartimentos extracelulares. Este aumento en la concentraci\'on de poliaminas juega un papel clave en la aceleraci\'on de la diseminaci\'on de tumores y puede servir como indicador de c\'elulas cancerosas. Concentraciones an\'omalas de poliaminas pueden ser la base para terapias que se centren en la eliminaci\'on de nutrientes y la prevenci\'on de met\'astasis, así\'i como para evaluar el avance de la quimioterapia.

En este contexto, nuestro objetivo en este trabajo fue demostrar el siguiente concepto: ¿se pueden emplear hidrogeles basados en PMAA para desarrollar biomateriales capaces de capturar poliaminas y liberar un f\'armaco terap\'eutico en respuesta? Hemos desarrollado un modelo molecular de grano grueso para describir la doxorrubicina, la putrescina, la espermidina y la espermina. Utilizando este modelo molecular y una teor\'ia termodin\'amica, nuestros resultados proporcionan información sobre la absorci\'on de poliaminas y doxorrubicina en pel\'iculas de hidrogel de PMAA.

Estos resultados predicen la capacidad de los hidrogeles de PMAA para capturar cantidades crecientes de poliaminas a medida que aumenta la concentraci\'on del medio en condiciones de sal fisiol\'ogica. Esta absorci\'on aumenta con el n\'umero de grupos amino; la espermina muestra la preferencia m\'as fuerte por el hidrogel. Este comportamiento resulta de la interacci\'on entre la carga positiva de los grupos amino y los segmentos de MAA desprotonados (cargados negativamente).



Hemos considerado la carga de un f\'armaco contra el c\'ancer como la doxorrubicina en el hidrogel de PMAA. Las condiciones \'optimas de encapsulaci\'on en el laboratorio corresponden a una baja concentraci\'on de sal y un pH entre 6 y 7. Este resultado es contrario a la intuici\'on, ya que el punto isoelectroel\'ectrico de la doxorrubicina se encuentra alrededor de un pH neutro, lo que puede explicarse por la protonaci\'on del f\'armaco al absorberse en el medio de pH m\'as bajo dentro del hidrogel.

Cuando tanto la doxorrubicina como una poliamina est\'an presentes en la soluci\'on en contacto con el hidrogel, las poliaminas dificultan significativamente la absorci\'on del f\'armaco (en comparaci\'on con las soluciones que contienen solo doxorrubicina). Este efecto se intensifica a medida que aumenta la concentraci\'on de poliaminas en la soluci\'on. La eficiencia para excluir el f\'armaco de la pel\'icula aumenta con la carga de la poliamina (es decir, el n\'umero de grupos amino); la espermina es la m\'as eficiente de las tres aminas consideradas para lograr esta respuesta. Como hemos demostrado, este comportamiento se puede interpretar como la liberaci\'on de doxorrubicina al capturar poliaminas.

Los films  de PMAA se sintetizaron mediante polimerizaci\'on radical de transferencia de \'atomos y se caracterizaron en condiciones espec\'ificas para respaldar nuestras predicciones te\'oricas. La absorci\'on y liberaci\'on de doxorrubicina en agua, soluciones de espermidina y espermina se estudiaron mediante espectroscopía UV-Vis. Estos resultados experimentales respaldan claramente el concepto propuesto de encapsulaci\'on de doxorrubicina dentro de estos films y la liberaci\'on de f\'armacos dependiente de poliaminas. % Utilizando las predicciones te\'oricas de este trabajo como gu\'ia, estamos trabajando actualmente en una caracterización sistemática del sistema experimental.

En resumen, nuestros estudios combinados te\'oricos y experimentales indican que los hidrogeles basados en \'acido polimetacr\'ilico tienen un gran potencial para servir como componente funcional en biomateriales que pueden capturar poliaminas y liberar un f\'armaco terap\'eutico en respuesta.