\chapter{Soluciones coloidales} \label{chapter:mc:soluciones}
	

	
	\section{Introducci\'on}
	
	Las soluciones coloidales de nanogeles polim\'ericos han ganado atenci\'on significativa debido a sus propiedades \'unicas y aplicaciones potenciales en varios campos. Los nanogeles son part\'iculas polim\'ericas reticuladas con dimensiones en el rango coloidal \cite{10.1002/pola.27653}. Estas part\'iculas combinan las propiedades de los pol\'imeros y de los coloides, como su capacidad de dispersar la luz, que les confiere propiedades \'opticas \'unicas o su capacidad de adsorber solvente en el que se encuentran, permiti\'endoles estabilidad en diferentes medios; en particular en medios biol\'ogicos, les permite ser utilizados en aplicaciones biom\'edicas \cite{lyon2012polymer}. Son part\'iculas blandas, deformables y permeables con una estructura formada por una red polim\'erica. \cite{lyon2012polymer}. En soluci\'on, pueden absorber el solvente y expandir su volumen significativamente hacia un estado de baja densidad de pol\'imero \cite{karg2019nanogels, perez2021thermodynamic}. La concentraci\'on de estas soluciones de nanogeles, adem\'as posee influencia en su estabilidad, como su solubilidad en agua y tama\~no \cite{10.3390/polym13234071}. 
	
	Estas soluciones son capaces de responder a diversos est\'imulos externos como la temperatura, la presi\'on, el pH, la fuerza i\'onica y la presencia de diferentes biomoleculas.
	La complejidad de esta respuesta a est\'imulo de las soluciones, a su vez, genera una mayor complejidad en cualquier dispersi\'on compuesta por ellas \cite{lyon2012polymer}.
	En este, sentido adem\'as de conocer como responden los nanogeles a diversos est\'imulos, es necesario estudiar c\'omo se ve modificada la respuesta de estas part\'iculas cuando forman soluciones coloidales. 
	En particular la  capacidad de responder a diferentes est\'imulos externos viene mediada por la composici\'on de la red polim\'erica. Los nanogeles compuestos por cadenas de pol\'imero que contienen segmentos \'acidos como \'acido acr\'ilico o metacr\'ilico (AA y MAA, respectivamente) se expanden o comprimen en respuesta a cambios en el pH de la soluci\'on \cite{snowden1996colloidal, Zhou1998}. De la misma forma, nanogeles compuestos pol\'imeros termosensibles, por ejemplo, PNIPAm, experimentan una transici\'on de fase volum\'etrica cuando se calientan por encima de una temperatura caracter\'istica \cite{Pelton1986, Pelton2000}. Este comportamiento se origina porque estos pol\'imeros son insolubles en agua por encima de cierta temperatura cr\'itica m\'inima de soluci\'on (LCST, por sus siglas en ingl\'es) \cite{Kawaguchi2020}.
	Este tipo de  transiciones, tanto en respuesta al pH como a la temperatura, son reversibles. Es decir, se aplica el est\'imulo externo en una direcci\'on provocando la expansi\'on de la part\'icula (swelling), la aplicaci\'on del est\'imulo en direcci\'on opuesta resulta en la compresi\'on del nanogel (deswelling).
	La incorporaci\'on de co-mon\'omeros termosensibles y con respuesta al pH a la red polim\'erica permite la obtenci\'on de nanogeles multi-responsivos, los cuales son atractivos para el dise\~no de sistemas inteligentes que funcionen en diferentes \'areas de la tecnolog\'ia \cite{plamper2017functional}. Se han estudiado microgeles de copol\'imeros de NIPAm y \'acido metacr\'ilico (MAA) \cite{Dowding2000, Hoare2004, Giussi2015}, encontr\'andose que su temperatura de cambio de fase depende del pH de la soluci\'on, la concentraci\'on de sal y la fracci\'on de mon\'omero ionizable en las cadenas de pol\'imero \cite{Morris1997, Jones2000, Hoare2004, Bradley2005, Lee2008, Wong2009, Hamzavi2016}. La incorporaci\'on del co-mon\'omero de MAA proporciona un mecanismo controlado por el pH para la adsorci\'on/liberaci\'on de mol\'eculas de carga de signo opuesto, lo que hace que los microgeles multiresponsivos sean atractivos para el dise\~no de diversos sistemas, entre ellos la administraci\'on de medicamentos \cite{Liu2017}.
	En otros trabajos \cite{scotti2022softness, urich2016swelling} se han enfoncado en el estudio de la  arquitectura y composici\'on de los hidrogeles, lo que permite adaptar estas part\'iculas para que posean propiedades espec\'ificas.
	Del mismo modo se han estudiado soluciones coloidales, algunos estudios muestran a los nanogeles como part\'iculas esf\'ericas que interact\'uan \'unicamente con un potencial de n\'ucleo duro, es decir, esferas duras \cite{karg2019nanogels}. \citet{alziyadi2023osmotic} utilizaron simulaciones de din\'amica molecular con una teor\'ia de Poisson-Boltzmann para estudiar microgeles cargados superficialmente. En otros estudios, \cite{scotti2022softness, scheffold2020pathways}, se ha hecho \'enfasis en soluciones con concentraciones altas de microgeles, en las cuales las part\'iculas se empaquetan densamente y forman una fase cristalina o amorfa. Esta capacidad de comprimirse y formar diferentes fases de los nano/microgeles se convierte en un aspecto clave que influye en las interacciones a nivel macrosc\'opico.
	Todas estas propiedades \'unicas de los nanogeles polim\'ericos, incluida su estabilidad coloidal, los convierten en candidatos prometedores para una amplia gama de aplicaciones tales como biosensores \cite{zhang2012ultrathin, islam2014responsive}, ingenier\'ia de tejidos \cite{matricardi2013interpenetrating, van2011biopolymer}, regeneraci\'on \'osea \cite{bai2018bioactive}, materiales biomim\'eticos \cite{green2016mimicking, wu2010multifunctional}, entre muchas otras aplicaciones biom\'edicas \cite{Daly2020}.
	
	El cap\'itulo \ref{Chapter-geles} se dedic\'o al desarrollo de una teor\'ia termodin\'amica para el entendimiento de microgeles con respuesta a la temperatura, pH y concentraci\'on de sal. Los resultados presentados en dicho cap\'itulo describen la fisicoqu\'imica detr\'as de los fen\'omenos del swelling de estos microgeles impulsada por el pH, la dependencia no mon\'otona del tama\~no de part\'icula con la concentraci\'on de sal. El aumento de la salinidad de la soluci\'on puede provocar tanto expansi\'on como compresi\'on de la red polim\'erica, dependiendo del rango de concentraci\'on de sal.
	En este nuevo cap\'itulo damos un paso m\'as adelante mostrando un estudio sistem\'atico de soluciones coloidales compuestos por nanogeles de P(NIPAm-co-MAA). Para este prop\'osito se har\'a uso de una metodolog\'ia basada en un algoritmo Monte Carlo en el cual se combinan potenciales de interacci\'on part\'icula-part\'icula y de termodin\'amica desarrollada en \cite{perez2021thermodynamic}. 
	
	%%%% aqui motivación del capitulo
	%%% lo que sigue es lo que se hace, no la motivación.
	
	En este nuevo cap\'itulo se incorporan potenciales entre part\'iculas de la soluci\'on de tal forma de estudiar los efectos de la concentraci\'on de nanogeles en la soluci\'on de los mismos.
	Estos potenciales buscan incorporar la naturaleza de los nanogeles de trabajo: su permeabilidad, capacidad de absorber solvente y adquirir carga el\'ectrica.
	Por un lado, se agrega el potencial de Hertz, el cual tiene en cuenta la elasticidad de los nanogeles, lo que permite describir su capacidad de deformarse.
	El potencial de Yukawa es un potencial de interacci\'on electrost\'atica entre dos part\'iculas. Se puede utilizar para describir la interacci\'on electrost\'atica entre dos nanogeles polim\'ericos, que pueden tener carga el\'ectrica. El potencial de Yukawa tiene en cuenta la carga el\'ectrica de los nanogeles y las condiciones del medio en el que se encuentran, en particular la fuerza i\'onica del medio.
	El uso de estos potenciales permite describir los efectos que tiene la concentraci\'on de nanogeles sobre la respuesta a est\'imulo, como lo son el pH, la temperatura y la concentraci\'on salina del medio.
	
	
	
	
	
	\section{Metodolog\'ia}
	
	Se realiz\'o un estudio sistem\'atico de la comportamiento de soluciones de nanogeles, variando la concentraci\'on salina, el pH y la concentraci\'on de part\'iculas. Adem\'as, se estudi\'o  el efecto de la temperatura sobre los nanogeles. En las siguientes secciones se describen los potenciales de interacci\'on entre part\'iculas y el potencial termodin\'amico que da origen a la energ\'ia intramolecular. 
	
	
	
	\subsection{Potencial termodin\'amico intramolecular}
	
	La energ\'ia libre interna de cada nanogel se calcula en base al potencial termodin\'amico desarrollado por \citet{perez2021thermodynamic} que tiene en cuenta  la elasticidad de la red polim\'erica, la energ\'ia libre qu\'imica dada por los segmentos titulables (segmentos de MAA), la energ\'ia libre asociada a la entrop\'ia de mezcla de los iones que componen la soluci\'on, la energ\'ia por interacciones electrost\'aticas internas o con el solvente, as\'i como la energ\'ia generada por efectos de repulsi\'on est\'erica. Es decir, la fisicoqu\'imica relevante del nanogel
	
	Para obtener este potencial termodin\'amico, desarrollamos un modelo de dos fases. La primera fase est\'a ocupada por nuestro nanogel, cuya red polim\'erica esta compuesta por poli(NIPAm-co-MAA) (P(NIPAm-MAA)). Esta fase se denota por $NG$. La fase nanogel se encuentra en contacto con una soluci\'on acuosa (fase 2, denotada por $s$), en la cual estan presentes las mol\'eculas de agua, hidronio e hidr\'oxido, as\'i como tambi\'en los iones de prevenientes de la sal del medio ($K^+$ y $Cl^-$).
	Con este modelo, podemos controlar variables externas como la temperatura $T$ y la composici\'on de la soluci\'on, es decir, el pH y la concentraci\'on de sal. Bajo estas condiciones, un nanogel aislado asume un radio $R$ y, con ello, un volumen $V=\frac{4}{3}\pi R^3$.
	El potencial termodin\'amico cuyo m\'inimo produce las condiciones de equilibrio dentro de la fase de nanogel es un semi gran potencial, $\Omega_{NG}$.
	
	
	
	%
	\begin{align}
		\begin{aligned}
			\Omega_{NG}=& -TS_{mez} + F_{qca,MAA} +  F_{ela}\\
			& + U_{elec}+  U_{ste} + U_{VdW} -{\sum_{\gamma}
				{\mu_\gamma N_\gamma}}
		\end{aligned}
		\label{eq:mc:free-energy-implicit}
	\end{align}
	%
	
	
	\noindent En donde $S_{mez}$ es la entrop\'ia de traslaci\'on (y mezcla) de las especies libres en la fase del nanogel: mol\'eculas de agua ($w$), hidronio ($H_3O^+$) e iones de hidr\'oxido ($OH^-$), y cationes de sal ($+$) y aniones ($-$).
	Hemos considerado una sal monovalente, $KCl$,  completamente disociada en iones de potasio y cloruro.
	
	\begin{align}
		-\frac{S_{mez}}{k_B}	= \sum_{\gamma} \rho_\gamma\left(\ln\left(\rho_\gamma v_w\right) -1 + \beta\mu^0_\gamma\right) 
	\end{align}
	
	\noindent en donde  $\beta=\frac{1}{k_BT}$ , $T$ es la temperatura del sistema  y  $k_B$ es la constante de Boltzmann. La densidad num\'erica de la especie $\gamma$ es $\rho_\gamma$ y $\mu^0_\gamma$ es su potencial qu\'imico est\'andar,  $v_w$ es el volumen de una mol\'ecula de agua. Adem\'as $\gamma \in \left\{ w, H_3O^+, OH^-, +,- \right\}$.
	
	$F_{qca,MAA}$ es la energ\'ia libre qu\'imica que describe la protonaci\'on de equilibrio de las unidades de MAA.
	
	
	\begin{align}
		\beta F_{qca, MAA} =  \frac{\phi_{MAA}}{v_{MAA}} \left[f(\ln f+ \beta\mu^0_{MAA^-}) +(1-f)(\ln (1-f)+\beta\mu^0_{MAAH})\right]
	\end{align}
	
	
	\noindent donde $\phi_{MAA}$ es la fracci\'on de volumen que ocupan estos segmentos (dentro del volumen del nanogel aislado), siendo $v_{MAA}$ su volumen, y $f$ es el grado de disociaci\'on del mismo. 
	La fracci\'on de volumen de los segmentos MAA cargados es $f\phi_{MAA}$, y la de las unidades protonadas (o sin carga el\'ectrica) es $(1-f)\phi_{MAA}$.
	Los potenciales qu\'imicos est\'andar son $\mu^0_{MAA^-}$ y $\mu^0_{MAAH}$ para las especies desprotonadas (cargadas) y protonadas, respectivamente.
	Se define como segmento a las unidades qu\'imicas que componen las cadenas de la red polim\'erica (MAA y NIPAm).
	
	
	La energ\'ia libre el\'astica que describe la libertad conformacional de la red polim\'erica es $F_{ela}$: 
	
	\begin{align}
		\beta F_{ela} = \dfrac{3}{2}\dfrac{N_{seg}}{n_{ch} }\left[\left(\dfrac{R}{R_0}\right)^2 - \ln\dfrac{R}{R_0} -1\right]
	\end{align}
	
	En donde $N_{seg}$ es el n\'umero total de segmentos en la red de pol\'imero y $n_{ch}$ es el n\'umero de segmentos por cadena de pol\'imero o \emph{longitud de cadena}.
	La constante de elasticidad en esta energ\'ia es proporcional al cociente $\dfrac{N_{seg}}{n_{ch}}$, que representa el n\'umero total de cadenas de pol\'imero en el nanogel.
	El radio del nanogel seco es $R_0$, que satisface:
	
	%
	%
	\begin{align}
		\begin{aligned} 
			\dfrac{4}{3}\pi R_0^3=V_0&=N_{seg}\Big( x_{MAA} v_{MAA} +x_{NIPAm} v_{NIPAm}\Big)
		\end{aligned}
	\end{align}
	
	
	\noindent donde $V_0$ es el volumen de la part\'icula seca; $x_{MAA}$ y $x_{NIPAm}$ son la fracci\'on de los segmentos MAA y NIPAm en el nanogel, respectivamente. %En las soluciones coloidales que vamos a considerar, todas las part\'iculas tienen el mismo volumen seco.
	El n\'umero total de segmentos MAA es $x_{MAA}N_{seg}$ y el de unidades NIPAm es $x_{NIPAm}N_{seg}$, estos \'ultimos con un volumen $v_{NIPAm}$.
	Los nanogeles que consideramos aqu\'i satisfacen $x_{NIPAm}=1-x_{MAA}$.
	
	
	$U_{elec}$ y $U_{ste}$ representan, respectivamente, las energ\'ias que resultan de las interacciones electrost\'aticas y las repulsiones est\'ericas.
	
	\begin{align}
		\beta U_{elec} =\left(\sum_{\gamma } {\rho_\gamma q_\gamma + f\dfrac{\phi_{MAA}}{v_{MAA}}q_{MAA}}\right)\beta\psi_{MG}
	\end{align}
	
	\noindent donde $q_\gamma$ y $q_{MAA}$ son la carga el\'ectrica de las moléculas $\gamma$ y de los segmentos desprotonados de MAA, respectivamente.
	El potencial electrost\'atico dentro de la fase de nanogel es $\psi_{NG}$. En la fase soluci\'on el potencial electrost\'atico es nulo $\psi_s = 0$
	
	Se impone una restricci\'on de electro-neutralidad del microgel, que puede expresarse como:
	%
	%
	\begin{align}
		\begin{aligned}
			\sum_{\gamma  } \rho_\gamma q_\gamma + f\frac{\phi_{MAA}}{v_{MAA}}q_{MAA}=0
		\end{aligned}
		\label{eq:mc:charge-neutrality}
	\end{align}
	
	Las interacciones est\'ericas se incorporan como una segunda restricci\'on al sistema, la cual consiste en que el volumen de la fase nanogel est\'a completamente ocupado por los segmentos de la red y las especies qu\'imicas libres.
	
	%
	\begin{align}
		\begin{aligned}
			\sum_{\gamma } \rho_\gamma v_\gamma  + \phi_{MAA} + \phi_{NIPAm} = 1
		\end{aligned}
		\label{eq:mc:packing}
	\end{align}
	
	
	\noindent donde $v_\gamma$  es el volumen molecular de la especie $\gamma$, y la fracci\'on de volumen de cada componente de la red es: 
	%
	%
	\begin{align}
		\phi_{MAA}&=N_{seg}\dfrac{x_{MAA}v_{MAA}}{\frac{4}{3}\pi R^3}\\
		\phi_{NIPAm}&=N_{seg}\dfrac{x_{NIPAm}v_{NIPAm}}{\frac{4}{3}\pi R^3}
	\end{align}
	
	
	%%%%% 
	
	$U_{VdW}$ es la contribuci\'on que describe las interacciones efectivas pol\'imero-solvente; para este trabajo  se ha realizado la siguiente aproximaci\'on: 
	
	\begin{align}
		U_{VdW} = U_{NIPAm-w} + U_{MAA-w}
	\end{align}
	\noindent en donde $U_{NIPAm-w}$ incorpora la transici\'on hidrof\'ilica-hidrof\'obica de PNIPAm al aumentar la temperatura por encima de su LCST. 
	Del mismo modo $U_{MAA-w}$ hace cuenta de la interacci\'on entre los segmentos de MAA y agua.
	Para el presente potencial intramolecular se ha considerado a los segmentos de MAA completamente hidrof\'ilicos y por tanto $U_{MAA-w} = 0$
	
	\begin{align}
		\beta U_{VdW} = U_{NIPAm-w} = \chi (T, \phi_{NIPAm})\rho_w \phi_{NIPAm}
	\end{align}
	
	
	Este  t\'ermino modela la respuesta de PNIPAm a los cambios de temperatura a trav\'es de un par\'ametro de interacci\'on pol\'imero solvente, $\chi$, que depende de la temperatura y la fracci\'on de volumen de NIPAm, $\phi_{NIPAm}$.
	Seg\'un  \citet{afroze2000}, este par\'ametro de Flory-Huggins se puede expresar como:
	%
	%
	
	
	\begin{align}
		\begin{aligned}
			\chi (T, \phi_{NIPAm}) &=g_0(T) +g_1(T)\phi_{NIPAm} \\
			&~+ g_2(T)\phi_{NIPAm}^2
		\end{aligned}
	\end{align}
	
	\noindent con
	%
	%
	\begin{align}
		\begin{aligned} 
			g_k(T)=g_{k0} + \frac{g_{k1}}{T} + g_{k2}T
		\end{aligned}
	\end{align}
	
	
	\noindent para  $k=0,1,2$, los coeficientes son: $g_{00}= -12.947$, $g_{02}=0.044959\,$K$^{-1}$, $g_{10}= 17.920$, $g_{12}= -0.056944$\,K$^{-1}$, $g_{20}= 14.814$, $g_{22}= -0.051419$\,K$^{-1}$  y $g_{k1}\equiv 0$. \cite{afroze2000}
	
	
	
	
	Finalmente, la sumatoria sobre $\gamma$, en ecuaci\'on \ref{eq:mc:free-energy-implicit} expresa el equilibrio qu\'imico con la fase de soluci\'on, donde $\mu_\gamma$ y $N_\gamma$ son el potencial qu\'imico y el n\'umero de mol\'eculas de la especie $\gamma$, respectivamente.
	%Aqu\'i, el subíndice $\gamma$ identifica  las especies qu\'imicas libres, $\gamma \in \left\{ w, H_3O^+, OH^-, +,- \right\}$.
	Hay que tener en cuenta que $\Omega_{NG}$ es un semi-gran potencial porque la fase del nanogel puede intercambiar cada una de estas mol\'eculas con la fase de soluci\'on, mientras que la red de pol\'imero est\'a confinada dentro de la primera.
	
	
	\begin{align}
		\sum_\gamma N_\gamma \mu_\gamma = \sum_{\gamma }V{\rho_\gamma\beta\mu_\gamma}
		+ \beta\mu_{H^+}(1-f)V\dfrac{\phi_{MAA}}{v_{MAA}}
		\label{eq:mc:equilibrio_qco}
	\end{align}
	
	El segundo t\'ermino en la derecha de la ecuaci\'on \ref{eq:mc:equilibrio_qco} representa a los protones asociados a unidades de MAA;
	a saber, $\mu_{H^+}\equiv\mu_{H_3O^+}$ se conjuga con el n\'umero total de protones,
	
	\begin{align}
		N_{H_3O^+}+N_{MAAH}=V\left(\rho_{H_3O^+}+(1-f)\dfrac{\phi_{MAA}}{v_{MAA}}\right)
		\label{eq:mc:equilibrio}
	\end{align}
	
	
	
	Con todo esto la forma expl\'icita del potencial termodin\'amico es:
	
	
	
	
	%
	\begin{align}
		\begin{aligned}
			\beta&\frac{\Omega_{NG}(R)}{V}=\\& ~ \sum_{\gamma} \rho_\gamma\left(\ln\left(\rho_\gamma v_w\right) -1 + \beta\mu^0_\gamma\right) \\
			& + \frac{\phi_{MAA}}{v_{MAA}} \left[f(\ln f+ \beta\mu^0_{MAA^-})\right.\\
			&\qquad\left.+(1-f)(\ln (1-f)+\beta\mu^0_{MAAH})\right] \\
			%
			& + \dfrac{3}{2}\dfrac{N_{seg}}{n_{ch} V}\left[\left(\dfrac{R}{R_0}\right)^2 - \ln\dfrac{R}{R_0} -1\right] \\
			%
			& +  \left(\sum_{\gamma } {\rho_\gamma q_\gamma + f\dfrac{\phi_{MAA}}{v_{MAA}}q_{MAA}}\right)\beta\psi_{NG}\\
			%
			& +\beta\pi_{NG} \left[ \sum_{\gamma } \rho_\gamma v_\gamma  + \phi_{MAA} + \phi_{NIPAm} -1 \right] \\
			%
			& + \chi (T, \phi_{NIPAm})\rho_w \phi_{NIPAm} \\
			%
			& -\sum_{\gamma }{\rho_\gamma\beta\mu_\gamma}
			-\beta\mu_{H^+}(1-f)\dfrac{\phi_{MAA}}{v_{MAA}}\\
			%
			%
		\end{aligned}
		\label{eq:mc:free-energy}
	\end{align}
	
	
	
	
	\noindent En donde $\pi_{NG}$ es la presi\'on osm\'otica de la fase nanogel, introducida como un multiplicador de Lagrange para imponer la restricci\'on de incompresibilidad, ecuaci\'on \ref{eq:mc:packing}.
	
	
	Finalmente el potencial termodin\'amico esta escrito expl\'icitamente en funci\'on de las densidades de todas las especies, el grado de carga del MAA y el radio del nanogel, $\Omega_{NG}(R)\equiv\Omega_{NG}(\{\rho_\gamma\},f,R)$.
	Para obtener las expresiones de $\{\rho_\gamma\}$ y $f$ de tal forma que sean consistentes con el equilibrio termodin\'amico \'unico del nanogel aislado, minimizamos $\Omega_{NG}$ respecto a estas cantidades, y  sujeto a las restricciones ecuaciones  \ref{eq:mc:packing} y  \ref{eq:mc:charge-neutrality}; dicho procedimiento conduce a: 
	%
	%
	\begin{align}
		\rho_\gamma v_w &= a_\gamma \exp(-\beta\pi_{NG}v_\gamma -\beta\psi_{NG}q_{\gamma})\\
		\frac{f}{1-f}&= \frac{K^0_{MAA}}{a_{H^+}}\exp(-\beta\psi_{NG}q_{MAA})\label{eq:mc:fcharge}
	\end{align}
	
	\noindent donde $a_\gamma = e^{\beta\mu_\gamma-\beta\mu_\gamma^0}$ es la actividad de la especie $\gamma$. 
	La constante de equilibrio termodin\'amico que describe la protonaci\'on/desprotonaci\'on de los segmentos de MAA es:
	%
	%
	\begin{align}
		K^0_{MAA}= e^{\beta\mu^0_{MAAH}-\beta\mu^0_{MAA}-\beta\mu^0_{H^+}}
	\end{align}
	
	\noindent Esta cantidad es posible calcularla directamente a partir del pKa del \'acido.
	
	
	Si se considera  un valor de  $R$, las \'unicas inc\'ognitas restantes para determinar $\Omega_{NG}(R)$ son la presi\'on osm\'otica, $\pi_{NG}$ y el potencial electrost\'atico, $\psi_{NG}$.
	Estas dos cantidades se pueden calcular resolviendo num\'ericamente la incompresibilidad y la electro-neutralidad de la fase nanogel, ecuaci\'on \ref{eq:mc:packing} y ecuaci\'on \ref{eq:mc:charge-neutrality}, respectivamente.
	Para resolver estas ecuaciones utilizamos un m\'etodo h\'ibrido de Powell sin jacobiano y un c\'odigo FORTRAN desarrollado en nuestro grupo de investigaci\'on.
	En resumen, es posible calcular la variaci\'on de  $\Omega_{NG}(R)$ en funci\'on del radio del nanogel $R$, y encontrando su m\'inimo calcular el valor del radio \'optimo del nanogel para unas condiciones dadas (pH, temperatura, concentraci\'on de sal).
	
	Una descripci\'on m\'as detallada del calculo del potencial termodin\'amico intramolecular puede encontrarse en la referencia  \cite{perez2021thermodynamic} y en el cap\'itulo \ref{Chapter-geles}
	
	\subsection{Energ\'ia intermolecular: Interacci\'on entre nanogeles.}\label{sec:mc:energia_intra}
	
	Para el estudio de las soluciones de nanogeles hemos considerado la energ\'ia atribuida a las interacciones entre los mismos. Este aporte aporte energ\'etico originado por la interacci\'on entre las part\'iculas que componen la soluci\'on viene dado por dos tipos de potenciales: el primero de estos toma en cuenta la capacidad de deformaci\'on de los nanogeles, potencial de Hertz, y el segundo que describe la interacción electrost\'atica mediada por la composici\'on de la soluci\'on, potencial de Yukawa.
	La suma de las energ\'ias dada por estos potenciales se define como $U_{inter}$.
	\begin{align}
		U_H + U_Y = U_{inter}
		\label{eq:mc:u_inter}
	\end{align}
	\noindent en donde $U_H$ y $U_Y$ representan al potencial de Hertz y el el potencial de Yukawa respectivamente.
	En las siguientes líneas describiremos a los dos potenciales considerados que componen a $U_{inter}$.\\
	
	\textbf{Potencial de Hertz} \\
	
	El primer potencial, $U_H$, cuantifica las interacciones el\'asticas efectivas entre cada par de part\'iculas. Es decir la energ\'ia origninada por la deformaci\'on de los nanogeles, cualidad intr\'inseca de nuestras part\'iculas.  El potencial de Hertz permite que los nanogeles puedan deformarse al momento de interaccionar entre ellos, en nuestro modelo implica un peque\~no solapamiento de las part\'iculas, lo cual hace al sistema m\'as flexible respecto a un modelo de esferas duras. 
	
	\begin{align}
		\begin{aligned}
			\beta u^H_{ij} (r) = \left(\frac{1-r}{a_i + a_j}\right)^{5/2}\times b_{ij}
		\end{aligned}
		\label{eq:mc:hertz_ij}
	\end{align}
	
	\noindent En donde $a_i$ y $a_j$ corresponde a los radios de las part\'iculas $i$ y $j$ respectivamente y siendo $r$ la distancia entre sus centros de masas. $b_{ij}$ es la constante de interacci\'on entre las part\'iculas $i$ y $j$. Esta constante se define para cada par de part\'iculas:
	
	\begin{align}
		b_{ij} = \frac{8}{15}\left[\frac{1-v_i^2}{Y_i} + \frac{1-v_j^2}{Y_j}  \right]^{-1} \times(a_i +a_j)^2 \sqrt{a_ia_j}
		\label{eq:mc:bij_param}
	\end{align}
	
	\noindent en donde $v_i$ y $v_j$ es el radio de Poisson para cada nanogel.
	El radio de Poisson es una medida de c\'omo se contrae un material cuando se estira. Se define como la raz\'on entre la deformación transversal y la deformación longitudinal, este generalmente oscila entre 0.3 y 0.5. Un valor de $v$ de 0.5 indica que el pol\'imero no experimenta cambios de volumen al ser estirado \cite{bertoldi2010negative}.
	Por otro lado el m\'odulo de  Young $Y$  es un par\'ametro  que mide la rigidez de un material y se define como la raz\'on entre el esfuerzo y la deformaci\'on longitudinal cuando el material est\'a sometido a una carga \cite{ku2011review}.
	La teor\'ia de escalado para geles polim\'ericos en buenos solventes \cite{de1979scaling,hu2012polymer} predice que este m\'odulo es proporcional a la temperatura, el tamaño y composici\'on del gel (n\'umero de segmetos, densidad de entrecruzante,etc).
	\begin{align}
		Y \sim \frac{TN^*}{a^3}
	\end{align}
	\noindent en donde $T$ es la temperatura del sistema $a$ el radio del nanogel y $N^*$ hace referencia al n\'umero de segmentos/densidad de entrecruzamiento. \\
	
	
	\textbf{Potencial de Yukawa} \\
	
	El potencial de Yukawa describe las interacciones electrost\'aticas considerando la composici\'on del medio, en particular de la fuerza i\'onica.
	El potencial de Yukawa fue previamente utilizados para el estudio de las propiedades estructurales de suspensiones de microgeles i\'onicos \citet{weyer2018concentration}, polim\'eros estrellas.\cite{denton2003counterion}. Para nuestro sistema el potencial de yukawa cuantifica las repulsiones electrost\'aticas originadas por la carga adquirida de los nanogeles debido a la desprotonaci\'on  de los monomeros de MAA que compone a cada part\'icula.

	
	De forma expl\'icita se define el potencial de Yukawa en dos part\'iculas como: 
	
	\begin{align}
		\beta u^Y_{ij}(r) = l_Bq_i q_j \frac{e^{\kappa(a_i + a_j -r)}}{(1 +\kappa a_i)(1 + \kappa a_j)} \frac{e^{-\kappa r}}{r} 
		\label{eq:mc:yukawa}
	\end{align}
	
	\noindent En donde $q_i$ y $q_j$ son las cargas netas que poseen los nanogeles $i$ y $j$ respectivamente. $a_i$, $a_j$ y $r$ se definen de la misma forma que en la ecuaci\'on \ref{eq:mc:hertz_ij}. El valor $\kappa$ corresponde a la constante de apantallamiento de Debye, y $l_B$ a la constante de Bjrrum, siendo estas \'ultimas dos variables las que consideran las propiedades del medio en la que se encuentran inmersos los nanogeles. 
	En particular, sobre la constante de apantallamiento de Debye nos dice que una concentraci\'on alta de sal, y con ello de sus iones, se traduce en un mayor apantallamiento de las cargas el\'ectricas interactuantes de los nanogeles.  Obteniendose interacciones de corto alcanze.
	De forma contraria, a menores concentraciones de iones, las interacciones entre nanogeles se vuelven de largo alcance, debido al menor apantallamiento.\\
	
	\textbf{Energ\'ia intermolecular}
	
	Al considerar la naturaleza de cada potencial, Hertz y Yukawa, podemos considerar su efectividad en funci\'on de la distacia $r$ que separa a un par de part\'iculas $a_i$ y $a_j$.
	
	\begin{align}
		U_{inter}(r) = \begin{cases} U_H  & \text{if } r \leq a_i + a_j \\ U_Y & \text{if } r > a_i + a_j \end{cases} 
		\label{eq:mc:HY-potential}
	\end{align}
	
	\noindent en d\'onde el potencial de Hertz solo actua en casos de superposici\'on de part\'iculas $r \leq a_i + a_j$ y Yukawa para interacciones por fuera del solapamiento.
	Finalmente se definen explicitamente $U_H$ y $U_Y$ como:
	\begin{align}
		& U_H =\frac{1}{2} \sum^{Np}_{i,i \neq j} \beta u^H_{ij} \\
		& U_Y = \frac{1}{2} \sum^{Np}_{i,i \neq j} \beta u^Y_{ij}
	\end{align}
	
	\noindent siendo estos la suma sobre todos los pares de particulas $i$ $j$ que interactuan en la soluci\'on.
	
	
	\subsection{Simulaciones Monte Carlo} \label{sec:mc:mc}
	
	El modelado de las soluciones se realiz\'o utilizando el m\'etodo de Monte Carlo. Este se basa en la evaluaci\'on de diferentes estados o configuraciones del sistema de trabajo a trav\'es de la generaci\'on de n\'umeros aleatorios.
	
	En este cap\'itulo, se trabaja con soluciones de nanogeles en la cual un cambio de configuraci\'on consiste en el movimiento y cambio de tama\~no que puede ser un aumento o disminuci\'on  del radio de una part\'icula seleccionado al azar. Este esquema de trabajo puede visualizarse en la figura \ref{fig:mc:pasos_mc} en la cual se muestra un paso del algoritmo propuesto. Una part\'icula, en el estado \textbf{A}, de todas las presentes en la soluci\'on es cambiada de posici\'on, y en este caso, se disminuye su tama\~no, obteni\'endose el estado \textbf{B}.
	%% Faltaria mencionar el grafico
	
	\begin{figure}[!htb]
		\centering
		\includegraphics[width=0.75\textwidth]{Figures/modelos/mc_model.pdf}
		\caption{Esquema de un paso en el algoritmo de Metropolis-MonteCarlo. El paso de un estado \textbf{A} a uno \textbf{B} esta mediado por el cambio de energ\'ia entre ambas configuraciones. }
		\label{fig:mc:pasos_mc}
	\end{figure}
	
	La probabilidad de aceptar o no dichos cambios en el sistema viene dado por:
	
	\begin{align}
		P = min \{e^{-(\Delta U_{inter} + \Delta \Omega_{intra})},1\}
		\label{eq:mc:probabilidad}
	\end{align}
	
	\noindent en donde $\Delta U_{inter}$ y $\Delta\Omega_{intra}$ son los cambios en energ\'ia de los estados inicial y final. 
	
		De la expresi\'on, ecuaci\'on \ref{eq:mc:probabilidad}, puede notarse que, si el cambio de energ\'ia es negativo, es decir, se pasa de un estado menos estable a uno energ\'eticamente m\'as estable, este paso es aceptado autom\'aticamente. Se obtiene una probabilidad de 1. 
	
	
	Si el cambio de energ\'ia es desfavorable, es decir, $\Delta U_{inter} + \Delta \Omega_{intra} > 0$, la probabilidad de aceptarlo es menor a uno. En este caso, se sortea un n\'umero aleatorio y, si este es menor o igual a la probabilidad de aceptaci\'on, el paso se acepta. De lo contrario, se rechaza. 
	
	Se define para obtener la probabilidad de aceptar o no un paso de la simulaci\'on:
	
	\begin{align}
		\begin{aligned}
			& \Delta \Omega_{intra} = \Omega(R_B) - \Omega(R_A) \\
			& \Delta U_{inter} = U_{inter,B} - U_{inter,A}
		\end{aligned}
	\end{align}
	\noindent en donde $\Omega(R_B)$ y $\Omega(R_A)$ es la energ\'ia intramolecular para un nanogel con radio $R_B$ y radio $R_A$ provenientes de los estados \textbf{B} y \textbf{A} respectivamente. Del mismo modo se define $\Delta U_{inter}$, como la diferencia de energ\'ia intermolecular de los estados \textbf{A} y \textbf{B}.
	

	
	\section{Nanogel de P(NIPAm-MAA) en soluci\'on diluida: fluctuaci\'on de volumen} \label{sec:fluctuacion-volumen}
	
	El estudio de las soluciones de nanogeles necesita un previa caracterizaci\'on y comprensi\'on de los sistemas aislados, en particular nos interesa caracterizar la fluctuaci\'on en volumen que pueden tener estos sistemas al varias las condiciones de su entorno.  E sposible sustraer esta informaci\'on a partir del potencial termodin\'amico de un nanogel aislado en soluci\'on.
	
	%El c\'alculo de su potencial termodin\'amico y c\'omo puede usarse para estudiar las fluctuaciones en volumen de estos sistemas aislados.
	La respuesta a est\'imulo se estudi\'o en detalle en la referencia \cite{perez2021thermodynamic}, donde adem\'as de observar los cambios en el pH, temperatura y concentraci\'on de sal, se estudia c\'omo son modificadas sus propiedades con la estructura del microgel.
	
	El nanogel considerado en este estudio posee $N_{seg} = 10^4$ segmentos con $n_{ch} = 100$ segmentos por cadena, las cuales tienen un 35\% de MAA en su estructura polim\'erica, $x_{MAA} = 0.35$. El 65\% restante est\'a compuesto por mon\'omeros de NIPAm.
	
	En la figura \ref{fig:mc:energy-intra}, se observa el potencial termodin\'amico asociado a un nanogel aislado en funci\'on de su radio. Las curvas presentadas corresponden a tres concentraciones salinas. El pH del sistema es 4.65, el cual coincide con el pKa intr\'inseco del mon\'omero de MAA aislado. La temperatura para este sistema es de $25 ^\circ C$. De cada una de las curvas se obtiene el m\'inimo local/total de la energ\'ia intramolecular de un nanogel aislado. Este m\'inimo energ\'etico corresponde al radio \'optimo del nanogel aislado. Se puede notar que hay un aumento en el radio \'optimo del nanogel a medida que disminuye la concentraci\'on salina.
	
	Bajo condiciones de pH 4.65, hay una fracci\'on de segmentos de MAA cargados el\'ectricamente dentro de la red polim\'erica; en consecuencia, esta se expande para disminuir las repulsiones electrost\'aticas. Dicho efecto se magnifica con la disminuci\'on de la concentraci\'on de sal, la cual proporciona contraiones que apantallan las cargas.
	
	Este resultado fue reportado en \cite{perez2021thermodynamic} y discutido en el cap\'itulo \ref{Chapter-geles} sobre el rol que cumple la concentraci\'on salina como est\'imulo de los hidrogeles polim\'ericos.
	
	
	\begin{figure}[!htb]
		\centering
		\includegraphics[width=0.45\textwidth]{Figures/graph-mc/interna.png}
		\caption{Energ\'ia libre de un nanogel aislado en funci\'on del tama\~no del mismo. Las curvas corresponden a diferentes condiciones de pH. La concentraci\'on salina es 1 mM y la temperatura $25 ^\circ C$.}
		\label{fig:mc:energy-intra}
	\end{figure}
	
	
	\subsection{Fluctuaci\'on en Volumen}\label{sec:mc:fluctuacion}
	
	Fluctuaci\'on de volumen es una variaci\'on del volumen de un sistema, que se produce debido a la peque\~nos cambios en la temperatura del sistema. Son una medida de la inestabilidad del tama\~no, en nuestro caso, de un nanogel. A mayor fluctuaci\'on, mayor es la inestabilidad.
	Se puede calcular la fluctuaci\'on en volumen mediante \cite{callen1991thermodynamics}:
	
	\begin{align}
		\frac{\Delta V}{V} = \sqrt{\frac{k_BT}{V}\kappa_T}
	\end{align}
	
	\noindent en donde $\Delta V$ es la varianza del volumen del nanogel que se define como:
	\begin{align}
		\Delta V = \sqrt{\langle V^2\rangle - \langle V \rangle^2}
	\end{align}
	Adem\'as, $k_B$ es la constante de Boltzmann, $T$ es la temperatura del sistema y $\kappa_T$ es la incompresibilidad isot\'ermica. Es una medida de la resistencia a la compresi\'on del sistema. A mayor $\kappa_T$, mayor rigidez.
	Se define como la relación entre la variaci\'on de presi\'on y  volumen de un fluido a temperatura constante, considerando nuestro potencial termodin\'amico, $\Omega_{NG}$, se puede obtener la siguiente forma:
	
	
	\begin{align}
		\begin{aligned}
			\kappa_T & = -\frac{1}{V} \left( \frac{\partial V}{\partial P}\right)_T \\
			& =\frac{1}{V} \left( \frac{\partial^2 \Omega_{NG}}{\partial V^2}\right)^{-1}_T \\
			& \kappa_T  = 12 \pi R_{eq} \left( \frac{\partial^2 \Omega_{NG}}{\partial R^2}\right)^{-1}_{T,R=R_{eq}}
		\end{aligned}
	\end{align}
	\noindent en la cual $R_{eq}$ hace referencia al radio de equilibrio, es decir al m\'inimo de las curvas de la energ\'ia $ \Omega_{NG}$, presentadas en \ref{fig:mc:energy-intra}. Todas las dem\'as variables fueron previamente definidas en la secc\'ion \ref{sec:mc:energia_intra}
	
	\begin{figure}
		\centering
		\includegraphics[width=0.45\linewidth]{Figures/graph-mc/fluct-pH.png}
		\caption{Gr\'afico de la fluctuaci\'on en volumen de un nanogel aislado. El nanogel esta compuesto por $2\times 10^5$ segmentos (mon\'omeros) repartidos en 200 cadenas de 1000 cadenas cada una}
		\label{fig:mc:flut-pH}
	\end{figure}
	
	En base a estas consideraciones, se confeccion\'o la figura \ref{fig:mc:flut-pH}. En esta figura, se observa la fluctuaci\'on de un nanogel aislado en funci\'on del pH de la soluci\'on. Se presentan curvas correspondientes a tres condiciones salinas: 1, 10 y 100 mM en KCl. La temperatura es de 25$^\circ$C, condiciones en las cuales el PNIPAm posee un comportamiento hidrof\'ilico. En estas condiciones, las fluctuaciones mostradas en esta figura no se ven afectadas por la presencia de este co-mon\'omero.
	En t\'erminos generales, se observa que la fluctuaci\'on del sistema a las diferentes condiciones presentadas es de muy baja magnitud, del orden de $10^{-3}$. Sin embargo, dentro de estos valores, se puede notar un aumento de la fluctuaci\'on al aumentar el pH hasta estabilizarse en valores de pH mayores a 6.
	A bajos valores de pH, el nanogel no posee carga el\'ectrica, ya que los mon\'omeros de MAA que componen la red polim\'erica se encuentran protonados (estamos por debajo del pKa intr\'inseco del MAA). En estas condiciones, un cambio en el tama\~no del nanogel requiere cambios en la energ\'ia dada por la adsorci\'on de solvente y de los iones presentes en el sistema, adem\'as de la contribuci\'on el\'astica que conlleva el cambio de radio.
	Con el aumento en el grado de carga de la red polim\'erica se esperar\'ia una entrada de contraiones. A valores por debajo del pka intr\'insico del MAA la red polim\'erica que compone al nanogel no posee carga el\'ectrica, en consecuencia las fluctuacionesson bajas. A pH alto se obtiene el mayor grado de carga posible, sin cambios en la carga las fluctuaciones de estabilizan.
	
	En particular, altas concentraciones salinas promueven una desprotonaci\'on de los segmentos de MAA para peque\~nos aumentos de pH. Esto se debe a que los iones presentes en la soluci\'on pueden apantallar la carga el\'ectrica adquirida por los grupos carbox\'ilicos de los mon\'omeros de MAA. Como resultado, la entrada de iones favorece la fluctuaci\'on en volumen del nanogel.
	A medida que aumenta el pH, m\'as unidades de MAA adquieren carga el\'ectrica, por lo cual un cambio en su volumen se ve favorecido. Para minimizar las repulsiones el\'ectricas de la red polim\'erica, el nanogel puede expandirse o adsorber m\'as cantidad de contraiones presentes en la soluci\'on. Como resultado, se observa un aumento en la fluctuaci\'on del sistema, el cual se intensifica con el aumento de la concentraci\'on salina.
	
	%%%%%%%%%%%%%%%%%%
	A condiciones de pH altos, un valor de 7, el nanogel se encuentra completamente desprotonado, cargado el\'ectricamente. La fluctuaci\'on del sistema es mayor y se estabiliza como consecuencia de que no es posible un cambio en el grado de carga el\'ectrica del nanogel. 
	En una situaci\'on de pH intermedios, la desprotonaci\'on es incompleta y el nanogel puede expandirse o contraerse, por lo cual la fluctuaci\'on en el sistema es la mayor de todas.
	
	%%%%%%%%%%%%%%%%%%%%%%%%%%%
	
	En la figura \ref{fig:mc:fluct-T} se muestra el efecto de la temperatura en las fluctuaciones en el volumen del nanogel. Se presentan tres curvas correspondientes a tres valores caracter\'isticos de pH: por debajo y por encima del pH intr\'inseco del MAA, pH 3 y 7, respectivamente, y pH 4.65, igual al pKa del MAA aislado.
	
	A pH 3 el nanogel no posee carga el\'ectrica, y como mencioanamos antes, las fluctuaciones a este valor de pH son bajas. El aumento de \'estas es por efecto del cambio de la temperatura, en particular por el cambio en la hidroficidad de PNIPAm. Sin embargo, no se observan cambios apreciables a medida que se aumenta la temperatura del sistema, incluso cerca de la temperatura de transici\'on del PNIPAm, alrededor de los 32$^\circ$C. Por arriba de esta temperatura, el nanogel adopta un estado colapsado y no hay ninguna fuerza impulsora que promueva un cambio en su volumen.
	%%%%%%%%%%%%%%
	%%% 
	
	En la misma figura, \ref{fig:mc:fluct-T}, a valores de pH 4.65 y 7, los mon\'omeros de MAA se encuentran mayormente desprotonados, la red polim\'erica adquiere carga el\'ectrica. 
	A temperatura baja las fluctuaciones son mayores que a pH 3 por efecto de la desprotonaci\'on de la red polim\'erica. En particular son mayores a pH 4.65 dado que es posible una regulaci\'on de carga el\'ectrica, a pH 7 todos los segmentos de MAA se encuentran desprotonados.
	Con el aumento de temperatura hay un aumento de las fluctuaciones del sistema.
	Esto ocurre hasta llegar a la temperatura caracter\'istica de transici\'on volum\'etrica del PNIPAm, al seguir aumentando la temperatura se orginia un colapso en la estructura del nanogel, siendo las interacciones entre los segmentos de NIPAm las predominantes y estabilizando un tama\~no espec\'ifico del nanogel.
	Las fluctuaciones en este cambio de fase volum\'etrico son las mayores. Pasada la transici\'on las fluctuaciones decaen a los valores m\'as bajos posibles.
	
	
	%%%%%%%%%%%%%%%%%%%%%%%%%%	
	
	\begin{figure}
		\centering
		\includegraphics[width=0.45\linewidth]{Figures/graph-mc/fluct-T.png}
		\caption{Gr\'afico de la fluctuaci\'on en volumen de un nanogel aislado. El nanogel esta compuesto por $2\times 10^5$ segmentos (mon\'omeros) repartidos en 200 cadenas de 1000 cadenas cada una}
		\label{fig:mc:fluct-T}
	\end{figure}
	
	
	%%%%%%%%
	El estudio de las fluctuaciones en volumen de estos sistemas aislados nos muestran la baja magnitud en las mismas. No hay una fuerza impulsora significativa para que los nanogeles cambien su volumen. Esto puede traducirse en que en una soluci\'on coloidal, diluida, la poblaci\'on de tama\~nos de nanogeles sea \'unica o con muy poca disperci\'on. 
	
	En la siguiente secci\'on se mostraran los resultados de soluciones de nanogeles, desde concentraciones diluidas a una muy concentrada.
	Se ver\'a cual es el rol de las interacciones entre las part\'iculas y si se mantiene o no la tendencia de las fluctuaciones en volumen encontradas en esta secci\'on.
	
	\section{Resultados: Soluciones}
	
	En esta secci\'on mostraremos los resultados obtenidos en el estudio sistem\'atico de tres soluciones de nanogeles compuestos por segmentos de N-isopropilamina y \'acido metacr\'ilico. La red polim\'erica, P(NIPAm-MAA), posee un 35\% de MAA. Para cada una de estas soluciones, se estudiaron su respuesta a cambios en el pH, la temperatura y la concentraci\'on salina. En cada caso, se comparar\'a con un nanogel a diluci\'on infinita, es decir, un sistema aislado. Las soluciones consideras corresponden a 0.10 mg/ml, 1.0 mg/ml y 5.0 mg/ml las cuales fueron seleccionadas en base a la fracción en volumen que ocupar\'ian al alcanzar el tama\~no esperado en un sistema aislado.
	
	Como mencionamos en la secci\'on \ref{sec:mc:mc}, la metodolog\'ia utiliza simulaciones Monte Carlo. Estas se componen de $5\times 10^6$ pasos; cada paso de simulaci\'on se compone de una compresi\'on/expansi\'on y cambio de posici\'on espacial de una part\'icula que compone la soluci\'on. La caja de simulaci\'on est\'a compuesta por 100 nanogeles. Los par\'ametros de construcci\'on de cada nanogel han sido especificados en la secci\'on \ref{sec:fluctuacion-volumen}: Nanogeles con $N_{seg} = 10^4$ segmentos y $n_{ch} = 100$ segmentos por cadena. Estas cantidades fueron elegidas en base a un estudio sistem\'atico de las fluctuaciones en volumen de los nanogeles. Estos resultados mostraron que, cuanto m\'as r\'igidos (es decir, cadenas m\'as cortas) son los nanogeles, menores son las fluctuaciones. %En este sentido, para observar efectos en la concentraci\'on de estas part\'iculas, es necesario poseer fluctuaciones lo m\'as grandes posibles.
	
	
	
	\subsection{Efecto de la concentraci\'on de nanogeles}
	En la figura \ref{fig:mc:densidad-probabilidad} se muestra la densidad de probabilidad de distribuci\'on de tama\~nos de tres soluciones de nanogeles. Las condiciones del bulk de cada una de las soluciones corresponden a una temperatura de 25$^\circ$C y una concentraci\'on de sal de 1 mM en $[KCl]$, en cada panel se muestra un pH de estudio distinto, 3, 4.65, y 5 y 7 para los paneles A, B, C y D respectivamente. Estos valores son considerados por estar por debajo (pH 3) y por arriba (pH 5) del pKa intr\'inseco de los monomeros de MAA (pH 4.65) y condiciones de neutralidad (pH 7), que son condiciones cercanas al pH fisiol\'ogico. Las diferentes curvas dentro de cada panel representan una concentraci\'on diferente, 0.10, 1.0 y 5.0 mg/ml.
	
	\begin{figure}
		\centering
		\includegraphics[width=0.99\linewidth]{Figures/graph-mc/sizes-phs.pdf}
		\caption{Densidad de probabilidad de tama\~no para 3 diferentes soluciones, 0.1, 1.0 y 5.0 mg/ml. Cada panel corresponde a un pH de trabajo diferente: 3, 4.65, 5.50 y 7 para los paneles A, B, C, y D respectivamente. Otras condiciones de trabajo son: temperatura 25$^\circ$C y 1 mM en [KCl].}
		\label{fig:mc:densidad-probabilidad}
	\end{figure}
	
	La figura \ref{fig:mc:densidad-probabilidad} nos muestra, de forma general, que los perfiles de densidad de probabilidad en tama\~no de los nanogeles var\'ia con la concentraci\'on y el pH del medio. Se puede observar que el aumento del pH desplaza las curvas hacia valores mayores de radio. Lo que se puede explicar por el aumento en el n\'umero de segmentos de MAA que se desprotonan a medida que se aumenta el pH. En la secci\'on \ref{sec:mc:phs_salt_temp} daremos una explicaci\'on m\'as detallada de este fen\'omeno.
	
	Observando la figura \ref{fig:mc:densidad-probabilidad}A notamos que todas las concentraciones de nanogeles exhiben picos definidos, lo que sugiere una distribuci\'on de tama\~no m\'as uniforme. La variaci\'on entre de los radios es como m\'aximo 2 nm. Bajo estas condiciones no se aprecia ninguna fuerza impulsora que modifique significativamente el tama\~no de los nanogeles, de hecho no se puede apreciar una diferencia importante entre cada una de las soluciones de las part\'iculas.
	
	A medida que el pH aumenta a 4.65, figura \ref{fig:mc:densidad-probabilidad}B, la distribuci\'on correspondiente a una concentraci\'on de 5 mg/ml se diferencia de las otras dos. El desplazamiento es hacia menores valores de radio. Esta separaci\'on es aun mayor a medida que se aumenta el pH, panel C. Finalmente en la figura \ref{fig:mc:densidad-probabilidad}D a pH 7, las diferencias entre las concentraciones se incrementan hasta observar una separaci\'on entre las concentraciones consideradas. Esta disminuci\'on del tama\~no puede explicarse al considerar los potenciales de interacci\'on entre part\'iculas. Al estar las part\'iculas muy pr\'oximas entre s\'i, interact\'uan bajo la influencia de los potenciales de Hertz y Yukawa. Para disminuir la energ\'ia dada por estas interacciones, los nanogeles buscan disminuir su tama\~no, si bien esto conlleva un aumento en la energ\'ia libre interna del nanogel, ver figura \ref{fig:mc:energy-intra}, el aumento de la energ\'ia libre interna es menor que el producido por las interacci\'on intermoleculares. A medida que disminuimos la concentraci\'on de nanogeles, estos pueden aumentar su tama\~no para acercarse m\'as a su radio ideal. Es por ello que en ausencia de interacciones de a pares, una diluci\'on infinita, las part\'iculas tienden a aumentar su radio hasta minimizar la energ\'ia interna.
	
	La capacidad de fluctuaci\'on de tama\~no de los nanogeles es fuertemente afectada por la capacidad de los nanogeles de interactuar con otro vecino, lo cual est\'a interrelacionado con la concentraci\'on de la soluci\'on.
	En ese sentido, la figura \ref{fig:mc:rdf} muestra la distribuci\'on radial para las tres diferentes soluciones de nanogeles. Consideramos las mismas condiciones del bulk de la figura \ref{fig:mc:densidad-probabilidad}, es decir, pH 4.65, temperatura de 25$^\circ$C y concentraci\'on salina de 1 mM.
	
	Observamos que la soluci\'on con concentraci\'on m\'as alta de nanogeles ($\rho$ = 5.0 mg/ml) exhibe un perfil de distribuci\'on radial caracter\'istico de sistemas l\'iquidos, donde se pueden distinguir picos marcados en la distribuci\'on que indican la presencia de una estructura ordenada a corta distancia, lo que sugiere una cierta estructuraci\'on en la disposici\'on de las primeras part\'iculas vecinas dentro de la soluci\'on.
	
	Por otro lado, las soluciones m\'as diluidas ($\rho$ = 0.10 y 1.0 mg/ml) muestran un perfil que asciende gradualmente, alcanzando un m\'aximo menos definido y manteni\'endose as\'i con peque\~nas variaciones. Este comportamiento es an\'alogo al de un sistema gaseoso, donde la distribuci\'on de part\'iculas es m\'as aleatoria y menos estructurada.
	
	Estos resultados pueden indicar que las interacciones entre los nanogeles son m\'as frecuentes en soluciones de mayor concentraci\'on, lo que conduce a una mayor correlaci\'on espacial y, por ende, a un mayor ordenamiento, donde se observ\'o un comportamiento de un l\'iquido. En contraste, a baja concentraci\'on, presenta menos interacciones, las part\'iculas est\'an m\'as espaciadas, reflejando un sistema con menor correlaci\'on entre part\'iculas.
	
	\begin{figure}
		\centering
		\includegraphics[width=0.45\linewidth]{Figures/graph-mc/rdf.png}
		\caption{Densidad de distribuci\'on radial para las tres soluciones de nanogeles. Las curvas corresponden a las tres concentraciones consideradas: 0.10, 1.0, 5 mg/ml. Las condiciones del bulk de la soluci\'on son pH 4.65, [KCl] = 1 mM, la temperatura del sistema es 25$^\circ$C.}
		\label{fig:mc:rdf}
	\end{figure}
	
	
	
	\subsection{Respuesta a est\'imulos: pH, concentraci\'on salina y temperatura.}\label{sec:mc:phs_salt_temp}
	
	\textbf{pH}
	
	En esta secci\'on estudiamos la influencia en la respuesta a est\'imulos, en particular los resultados que mostramos m\'as abajo surgen de los cambios en el pH, la concentraci\'on salina y la temperatura. En primera instancia, veremos c\'omo se comportan las soluciones con la variaci\'on del pH. %Teniendo en cuenta que las soluciones de nuestros nanogeles est\'an influenciadas por su concentraci\'on, veremos c\'omo estas soluciones 
	
	La Figura \ref{fig:mc:xvspH} ilustra c\'omo var\'ian ciertas propiedades medias de las soluciones de nanogeles en funci\'on del pH. Se destacan el tama\~no promedio $\left<R\right>$, el grado de disociaci\'on $\left<f\right>$ y la fracci\'on en volumen $\left<\phi\right>$, correspondientes a los paneles A, B y C, respectivamente. A una temperatura de 25$^\circ$C y con una concentraci\'on de $[KCl]$ de 1 mM, se observa principalmente la respuesta de los segmentos de MAA, mostrando que el pol\'imero NIPAm es hidrof\'ilico bajo estas condiciones. Se han analizado tres concentraciones distintas de nanogeles, $\rho$ = 0.10, 1.0 y 5.0 mg/ml, para estudiar el efecto de la concentraci\'on. En los paneles A y B, las curvas punteadas representan el comportamiento en un sistema de diluci\'on infinita.
	
	\begin{figure}[!htb]
		\centering
		\includegraphics[width=0.30\linewidth]{Figures/graph-mc/xvspH.pdf}
		\caption{Variaci\'on de las propiedades medias de la soluci\'on de nanogeles en funci\'on del pH de trabajo. El panel A muestra el tama\~no $\left<R\right>$, el B el grado de carga $\left<f\right>$ y el C la fracci\'on en volumen $\left<\phi\right>$. La temperatura es de 25$^\circ$C y la concentraci\'on salina de 1 mM en $[KCl]$. Cada curva corresponde a un grado de empaquetamiento. Los nanogeles aislados se muestran en los paneles A, B y C.}
		\label{fig:mc:xvspH}
	\end{figure}
	
	El radio medio de los nanogeles crece con el aumento del pH, independientemente de la concentraci\'on contemplada en nuestro estudio. Esto se atribuye a la influencia del pH en la carga de los segmentos de MAA: a un pH m\'as elevado, m\'as segmentos de MAA se desprotonan, incrementando la carga el\'ectrica negativa del nanogel. En el panel B (\ref{fig:mc:xvspH}B), se observa que el aumento del pH tambi\'en eleva el grado de disociaci\'on de los segmentos de MAA. Este crecimiento en tama\~no es resultado de las repulsiones electrost\'aticas en la red polim\'erica de los nanogeles, como se report\'o en \cite{perez2021thermodynamic}, donde se indica que estas repulsiones disminuyen al distanciar los centros de carga de la estructura del nanogel.
	
	El efecto de la concentraci\'on de los nanogeles se evidencia al examinar las distintas curvas en cada panel, correspondientes a diferentes concentraciones. El comportamiento de la soluci\'on m\'as diluida, 0.10 mg/ml, es similar al del sistema aislado en todo el rango de pH. La concentraci\'on intermedia, 1.0 mg/ml, presenta una din\'amica parecida, aunque muestra diferencias alrededor del pH 6. En cambio, la soluci\'on m\'as concentrada, 5.0 mg/ml, solo se asemeja al sistema aislado a pH inferiores a 5.
	
	En t\'erminos generales, al aumentar la concentraci\'on de nanogeles, se observa una reducci\'on en la respuesta esperada de un sistema aislado, con una disminuci\'on de m\'as del 40\% para la soluci\'on m\'as concentrada. A diferencia de un sistema aislado, las soluciones de nanogeles presentan interacciones energ\'eticas intermoleculares, influenciadas por los potenciales de Hertz y Yukawa. Estas diferencias en tama\~no o grado de carga se explican no solo por la energ\'ia libre interna de cada nanogel, sino tambi\'en por las interacciones entre ellos. Esto se hace evidente al observar la figura \ref{fig:mc:xvspH}C: en la cual se muestra c\'omo cambia la fracci\'on en volumen con respecto al cambio de pH, aumentando r\'apidamente con este y a\'un m\'as para concentraciones m\'as altas. Una mayor fracci\'on en volumen implica una mayor interacci\'on entre las part\'iculas. Con m\'as part\'iculas en un mismo volumen, la proximidad entre ellas aumenta, lo que intensifica su interacci\'on.
	
	Los potenciales de Hertz y Yukawa, detallados en la secci\'on \ref{sec:mc:energia_intra}, incrementan la energ\'ia del sistema a medida que los nanogeles se aproximan. El aumento en el volumen de las part\'iculas disminuye la distancia entre ellas, resultando en un aumento de la energ\'ia total del sistema. Para reducir esta energ\'ia de interacci\'on, las part\'iculas adquieren un tama\~no menor al esperado en un sistema aislado. Esta diferencia con respecto al sistema aislado conlleva un incremento en la energ\'ia libre interna de los nanogeles, pero se compensa con una disminuci\'on en la energ\'ia de interacci\'on. Este efecto se amplifica con el aumento en la concentraci\'on, como se observa en las diferentes curvas de la figura \ref{fig:mc:xvspH}A. Adem\'as, este cambio en el tama\~no de los nanogeles afecta su grado de carga, generando un efecto entr\'opico que impulsa a los iones a salir de los entornos confinados de los nanogeles de menor tama\~no hacia la soluci\'on, lo que a su vez reduce el apantallamiento en los segmentos de MAA desprotonados. La energ\'ia libre qu\'imica necesaria para desprotonar los segmentos de MAA es mayor, lo que se traduce en un desplazamiento hacia valores m\'as altos de pH para alcanzar el mismo grado de carga que un sistema ideal.
	
	\textbf{Concentraci\'on salina}
	
	La figura \ref{fig:mc:reentrante} exhibe el comportamiento del tama\~no medio $\left<R\right>$ de los nanogeles en funci\'on de la concentraci\'on salina a un pH de 4.65 y 7, paneles A y B respectivamente, la temperatura del sistema es de 25$^\circ$C. Las curvas s\'olidas representan la respuesta a est\'imulo para tres concentraciones de nanogeles $\rho =$ 0.1, 1.0 y 5.0 mg/ml, mientras que la l\'inea punteada muestra la respuesta de un nanogel aislado. En ambos paneles se observa una transici\'on reentrante caracter\'istica, donde inicialmente el tama\~no de los nanogeles aumenta con la concentraci\'on salina hasta un m\'aximo antes de disminuir, indicando un colapso en el tama\~no de las part\'iculas. Este fen\'omeno se atribuye a la interacci\'on entre los mon\'omeros cargados y la adsorci\'on de contraiones provenientes del KCl. Bajo estas condiciones de pH (ya sea 4.65 o 7) hay una proporci\'on de los segmentos de MAA que se encuentran desprotonados, vease la figura \ref{fig:mc:xvspH}B, con mayor magnitud a pH 7. A bajas concentraciones salinas, el apantallamiento sobre estos segmentos cargados el\'ectricamente es insuficiente, lo que conlleva a un aumento del tama\~no de los nanogeles para minimizar las repulsiones electrost\'aticas. Si se aumenta la concentraci\'on de contraiones, en primera instancia se favorece la protonaci\'on de los segmentos de MAA \cite{perez2021thermodynamic}, provocando un aumento del \textit{swelling} de los nanogeles presentes en la soluci\'on. Con una mayor concentraci\'on de sal, el incremento en la adsorci\'on de contraiones aumenta lo que conduce a un mejor apantallamiento de los contraiones, reduciendo las repulsiones entre los segmentos de MAA desprotonados. En este punto se esperar\'ia que este aumento en el apantallamiento de las cargas permita que m\'as segmentos se desprotonen y se aumente a\'un m\'as el radio de las part\'iculas, sin embargo, el efecto observado es el contrario: colapso de la estructura de los nanogeles. El exceso de contraiones en el interior de los nanogeles tiene como consecuencia que las interacciones electrost\'aticas se vuelvan de muy corto alcance, permitiendo que los centros de carga puedan acercarse mucho m\'as, lo que facilita una relajaci\'on y disminuci\'on en la energ\'ia el\'astica de los nanogeles. Esta transici\'on reentrante es m\'as evidente a pH 7 donde los segmentos de MAA desprotonados (con carga el\'ectrica) son mayor\'ia.
	
	\begin{figure}
		\centering
		\includegraphics[width=0.75\linewidth]{Figures/graph-mc/r-salts-pHs.pdf}
		\caption{Variaci\'on del radio medio ($\left<R\right>$) con la concentraci\'on de sal para diferentes concentraciones de nanogeles. El pH es 4.65 y 7 para el panel A y B respectivamente. La temperatura del sistema es de 25$^\circ$C. Las curvas s\'olidas corresponden a $\rho=$ 0.10, 1.0 y 5.0 mg/ml, y a trazos se incorpora la respuesta de un nanogel aislado.}
		\label{fig:mc:reentrante}
	\end{figure}
	
	Respecto al efecto de la concentraci\'on de las soluciones, se puede observar que a medida que aumenta la concentraci\'on de nanogeles, menor es la respuesta de estas part\'iculas a cambios en la concentraci\'on salina. Esto \'ultimo se evidencia mejor en la figura \ref{fig:mc:reentrante}B a un pH de 7. Esto se debe a que en estas condiciones, el tama\~no y grado de carga de estas part\'iculas est\'an bien diferenciados, figura \ref{fig:mc:xvspH}A y B. La diferencia se puede atribuir al costo energ\'etico de las interacciones entre nanogeles, el cual es compensado con un menor tama\~no de los mismos. Una mayor vecindad de nanogeles, ver figura \ref{fig:mc:rdf}, hace que interact\'uen a\'un m\'as entre ellos, aumentando la energ\'ia intermolecular del sistema. Al igual que en la figura \ref{fig:mc:xvspH}, es la energ\'ia libre interna de los nanogeles la que aumenta como consecuencia de poder reducir la energ\'ia atribuida a los potenciales de Hertz y Yukawa. A medida que se diluye lo suficiente la soluci\'on, nos acercamos al comportamiento ideal, es decir, un nanogel aislado donde las part\'iculas pueden aumentar su tama\~no para reducir su energ\'ia libre interna sin mayor costo en las interacciones intermoleculares.
	
	
	\textbf{Temperaratura}
	
	A continuaci\'on haremos un an\'alisis sobre la respuesta de nuestras soluciones de nanogeles cuando el est\'imulo que se les aplica son cambios en la temperatura.
	En este sentido en la figura \ref{fig:mc:dfs_energy} se muestra la energ\'ia libre interna de un nanogel aislado en funci\'on de su tama\~no y como es sus respuesta con el cambio de la temperatura. En el panel A, con un pH del bulk de 4.65, podemos observar como al aumentar la temperatura se da la aparici\'on de un nuevo m\'inimo en la energ\'ia libre a valores m\'as chicos de radio. Hay paso de un m\'inimo alrededor de los 60 nm hac\'ia valores de 12 nm. Esto se debe al efecto de hidroficidad del PNIMPam que componen la red de las part\'iculas. Este m\'inimo global se va profundizando al aumentar la temperatura. Por lo cual se esperar\'ia un estado colapsado bien definido al aumentar la temperatura de las soluciones de trabajo. Por otro lado en el panel B se considera un pH del bulk de 7, a diferencia del panel A, vemos que la coexistencia de dos m\'inimos en la energ\'ia libre interna, uno correspondiente a bajos valores de radio, un estado colapsado del nanogel, y uno a valores mayores valores de R. A temperaturas bajas el m\'inimo energ\'etico correspondiente al estado colapsado es un m\'inimo local, pasando a ser uno global al aumentar mucho m\'as la temperatura.
	Esta coexistencia de estados se debe a que a pH 7 existe una buena proporci\'on de segmentos de MAA desprotonados, con lo cual las repulsiones electrost\'aticas entre estos segmentos favorecen, a bajas temperaturas, un estados swelling, mientras que las interacciones de PNIPAm prefieren una estructura colapsada. En la siguiente figura se presentar\'a los resultados al considerar prioritario los dos m\'inimos de a energ\'ia libre de los nanogeles.
	
	\begin{figure}[!htb]
		\centering
		\includegraphics[width=0.65\linewidth]{Figures/graph-mc/aa.png}
		\caption{Energ\'ia libre interna de un nanogel aislado como funci\'on de sus radio. Las diferentes curvas muestran el cambio de la energ\'ia al aumentar la temperatura, la concentraci\'on salina es 1 mM en [KCl]. Los paneles A y B corresponden a pH 4.65 y 7 respectivamente.}
		\label{fig:mc:dfs_energy}
	\end{figure}
	
	La figura \ref{fig:mc:temperatura-r} muestra la respuesta a la temperatura de las soluciones estudiadas. El panel A corresponde a un pH de 4.65, el panel B y C muestran la respuesta a est\'imulo de temperatura a un pH de 7,  y la concentraci\'on es 1 mM en $[KCl]$. Las diferentes curvas corresponden a las diferentes concentraciones: 0.10, 1.0 y 5.0 mg/ml, y a trazos se presenta la respuesta de un nanogel aislado.
	
	En todos los paneless se puede apreciar c\'omo, superando cierta temperatura, los nanogeles colapsan. Esto es causado por el efecto de la temperatura sobre el PNIPAm que conforma la estructura de las part\'iculas estudiadas. Esta caracter\'istica es propia de los pol\'imeros de PNIPAm \cite{perez2021thermodynamic}, los cuales poseen una temperatura cr\'itica m\'inima de soluci\'on, que siendo superada aumenta la hidrofobicidad del PNIPAm observ\'andose un colapso en la estructura de la que forman parte. Se puede observar que una vez superada esta temperatura cr\'itica todas las concentraciones contempladas tienden al mismo estado de deswelling y este mismo es inherente al pH de la soluci\'on bulk. El radio, de aproximadamente 12 nm, que adquieren las part\'iculas es tan peque\~no que el efecto de las interacciones entre pares desaparece, predominando fuertemente solo la energ\'ia libre intramolecular de los nanogeles.
	
	En la figura \ref{fig:mc:temperatura-r}A vemos que a condiciones de pH 4.65 no hay diferencias apreciables entre las distintas concentraciones de nanogeles y una part\'icula aislada. La diferencia en la temperatura de transici\'on de estos sistemas es menor a 0.2$^\circ$C. Bajo estas condiciones, pH 4.65, no hay diferencias apreciables para las tres soluciones, v\'ease figura \ref{fig:mc:xvspH}A y B, consideradas. El estado de carga el\'ectrica de los nanogeles en estas condiciones es el mismo. En este sentido, la fuerza impulsora, es decir, la hidrofobicidad del PNIPAm, no tiene ning\'un impedimento para llegar al deswelling del sistema y al disminuir el tama\~no de cada part\'icula por efecto de la hidrofobicidad de los segmentos del PNIPAm, tambi\'en se reducen las interacciones entre nanogeles. Estos resultados son consistentes con la existencia de un solo m\'inimo en la energ\'ia libre de los nanogeles, ver figura \ref{fig:mc:dfs_energy}A. Es decir hay una predominancia en el estado de estructura colapsada de las part\'iculas. 
	
	
	\begin{figure}[!htb]
		\centering
		\includegraphics[width=0.99\linewidth]{Figures/graph-mc/rvsTs.pdf}
		\caption{Tama\~no medio de la soluci\'on de nanogeles como funci\'on de la temperatura de la soluci\'on. pH 4.65, $[KCl]$ 1m M. Curvas s\'olidas corresponde a concentraciones $\rho =$ 0.10, 1.0 y 5.0 mg/ml de nanogel, a trazos se presenta un sistema con una part\'icula polim\'erica aislada.}
		\label{fig:mc:temperatura-r}
	\end{figure}

	Continuando con nuestra descripci\'on del efecto de la temperatura vemos en la figura \ref{fig:mc:temperatura-r}B y C el comportamiento de las soluciones cuando el pH de la soluci\'on tiene un valor de 7. Observamos una respuesta cualitativamente similar al panel A, el paso de un estado swelling a uno colapsado al superar una temperatura cr\'itica. Sin embargo a diferencial de la condici\'on de pH 4.65 presentamos dos respuesta diferentes para un mismo pH (7). Esto se debe a la elecci\'on de las condiciones iniciales con las que realizan las simulaciones. Esta elecci\'on se atribuye a la energ\'ia libre interna de los nanogeles, figura \ref{fig:mc:energy-intra}B.
	
	Para ambos paneles podemos destacar que el estado inicial, hinchado, de las part\'iculas disminuye con el aumento de la concentraci\'on de los nanogeles. Este comportamiento se atribuye  a los potenciales de Hertz y Yukawa para los cuales un aumento de la concentraci\'on significa una mayor energ\'ia de interacci\'on de a pares a medida que aumenta el tama\~no por efecto del pH, v\'ease figura \ref{fig:mc:xvspH}A, con lo cual para minimizar la energ\'ia total del sistema las part\'iculas adquieren un menor tama\~no respecto al esperado en un sistema aislado.
	 Por otro lado, vemos que nuevamente el estado colapsado es igual para las tres soluciones siendo este el mismo que para un sistema aislado. 
	
	En el panel \ref{fig:mc:temperatura-r}B se ha considerado radios de nanogeles en estado seco, deshidratado (R = 10 nm), mientras que en el panel \ref{fig:mc:temperatura-r}C se ha considera un estados swelling de las part\'iculas, R = 50 nm. 
	La elecci\'on de estos dos estados se ha realizado en base a la energ\'ia libre interna presentada en la figura \ref{fig:mc:dfs_energy}B, se han considerado condiciones iniciales de tama\~no de los nanogeles  valores de radios a la izquierda de los m\'inimos energ\'eticos.
	 En el panel B se observa que la temperatura de transici\'on es menor que para un nanogel aislado. Se observa que la disminuci\'on de tama\~no de las part\'iculas, para reducir la cantidad de interacci\'on entre nanogeles, va de la mano con la disminuci\'on de la energ\'ia libre de las part\'iculas por efecto hidrofobico del PNIPAm.
	 Estos resultados son consistentes con la respuesta al pH y concentraci\'on salina de las soluciones. Si embargo, al consideramos el panel C, observamos un aumento de casi 90 $^\circ$C, respecto al sistema aislado, hasta alcanzar una temperatura de transici\'on de alrededor de 136 $^\circ$C. Las condiciones iniciales de las part\'iculas, R= 50 nm, le otorga una estructura lo suficientemente expandida para permitir la desprotonaci\'on de m\'as segmentos de MAA, respecto a su estado seco, siendo esto el principal obst\'aculo para los nanogeles de alcanzar su estado colapsado por el efecto del PNIPAm. En la figura  \ref{fig:mc:temperatura-r}B puede notarse que hay una barrera energ\'etica alta entre los dos m\'inimos la cual disminuye al aumentar la temperatura, por ello solo a temperaturas lo suficientemente altas, m\'as de 100 $^\circ$C es posible pasar esta barrera. 
	Este desplazamiento a una temperatura m\'as alta, se atribuye a las repulsiones electrost\'aticas internas de los nanogeles. A pH 7 y un radio inicial mayor hay una fracci\'on alta de mon\'omero de MAA desprotonados, por lo cual un colapso en los nanogeles significa un acercamiento de los centros de carga, lo que aumentar\'ia la energ\'ia libre intramolecular. El colapso de los nanogeles implica un desplazamiento del  equilibrio \'acido/base de los mon\'omeros de MAA  hacia mayores pHs, para as\'i evitar su desprotonaci\'on y con ello las repulsiones el\'ectricas. Este desplazamiento es compensado energ\'eticamente con el aumento de la temperatura. En consecuencia, se necesita una temperatura m\'as alta, respecto al panel A, para hacer colapsar la estructura de las part\'iculas.
	
	 Finalmente podemos destacar el rol de la concentraci\'on de part\'iculas, el cual no parec\'ia tener efecto en la transici\'on swelling-colapso de la red polim\'erica.  En el panel \ref{fig:mc:temperatura-r}C vemos como la transici\'on pasa de ser tan brusca a una transic\'on m\'as suave, curva correspondiente a la concentraci\'on de 5 mg/ml. La mayor energ\'ia puesta en juego por los potenciales de interacci\'on y una barrera energ\'etica entre un estados expandido/colapsado que disminuye con la temperatura se acoplan para permitir una transici\'on m\'as gradual.
	

	\section{Conclusiones}
	
	%%%%%%%%%%%
		En este cap\'itulo se investig\'o el comportamiento de soluciones compuestas por nanogeles polim\'ericos a diferentes grados de empaquetamiento, es decir, a diferentes concentraciones. Se aplic\'o metodolog\'ia en base a simulaciones Monte Carlo. El enfoque se centr\'o en la influencia de la concentraci\'on de part\'iculas sobre la respuesta a est\'imulos de nanogeles compuestos por P(NIPAM-MAA). Se consideraron tres est\'imulos diferentes: cambios en el pH, la temperatura y la concentraci\'on de sal. Adem\'as del efecto de la concentraci\'on de nanogeles en la soluci\'on en estos est\'imulos.
	
		Los perfiles encontrados para las distintas concentraciones de trabajo nos indican el comportamiento de un sistema gaseoso (concentraciones de 0.10 y 1 mg/ml), mientras que a una concentraci\'on de 5 mg/ml se observa un comportamiento m\'as cercano a un l\'iquido. Para las concentraciones m\'as bajas vemos  que sus propiedades se desv\'ian m\'inimamente del sistema a diluci\'on infinita. Estos comportamiento indican la mayor regularidad de interacci\'on entre part\'iculas, en consecuencia se observa una disminuci\'on del tama\~no para minimizar la energ\'ia obtenida por los potenciales de Hertz y Yukawa, 
		 A su vez estos cambios son responsables de la diferente respuesta a los est\'imulos de pH, temperatura y concentraci\'on de sal respecto a un nanogel aislado. En general, el comportamiento reportado es cualitativamente similar a un sistema con diluci\'on infinita, siendo una disminuci\'on en la magnitud del comportamiento lo que predomina al aumentar la concentraci\'on de nanogeles. En particular, la respuesta a cambios de pH tambi\'en conlleva un aumento en el pKa aparente de la soluci\'on en concentraciones altas.	
		
		La respuesta a la temperatura sigue con los mismo lineamentos reportados en la respuesta al pH y concentraci\'on salina, es decir un comportamiento cualitativo al sistema en diluci\'on inifinita. En particular a pH 4.65 no se encontr\'o diferencia alguna sobre un nanogel aislado, pero a pH 7 si se reporta un comportamiento interesante el cual depende de las condiciones iniciales en las que se encuentra la soluci\'on de nanogeles. 
		Si partimos desde una un agregado de part\'iculas en esta seco vemos que la concentraci\'on de la soluci\'on disminuye la temperatura de transici\'on de la misma, por otro lado si el punto de partida son soluciones con nanogeles en estado swelling, y pH alto, la temperatura necesaria para colapsar la estructura de estos nanogeles aumenta dr\'asticamente. 


	%%%%
	%%  TEMPERATURA
	%Otra particularidad que se mostr\'o es el efecto de la temperatura.A diferencia de observar un cambio en la temperatura de transici\'on, del mismo modo que del pKa aparente de las soluciones, se encontr\'o que las tres soluciones estudiadas conflu\'ian a un mismo radio medio y a una misma temperatura de transici\'on. Los nanogeles en estado colapsado no logran interactuar con otras part\'iculas vecinas, en consecuencia no hay efecto del grado de empaquetamiento en estas condiciones.
	
	
	
	
	
	
	
	