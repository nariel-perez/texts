\chapter{Soluciones coloidales}

%%%%%%%%%%%%%%%%%%%%%%%%%%%%%%%%%%%%%%%%%%%%%%%%%%
\section{Introducci\'on}

en estos ultimos capitulos hemos hecho incapie en la importancia y potencial uso que tienes los geles polimericos. Haciendo enfasis en sus uso para el secuestro de proteinas y/o farmacos de interes terapeutico.
En ese sentido en los dos ultimos capitulos hemos presentados dos tipos de modelos con los cuales es posible hacer un estudio de la fisicoquimica que involucra a un nano/microgel aislado en solucion.
En un primer capitulo, \ref{Chapter-geles} hicimos referencia a un modelos robusto con el cual pudimos explicar la respuesta de microgeles con respuesta multiestimulo... 
En el siguiente capitulo, \ref{Chapter-esfericas}, complejisamos nuestro modelo para investigar como afecta la estructura interna a la respuesta de estos geles, demostrando que el dise\~no  y sintesis,  es un factor importante para optimizar los procesos de encapsulacion y liberacion....
En este ultimo capitulo presentamos el comportamiento de la solucion de estos nano/microgeles en solucion. 


Estudio sobre las soluciones coloidales de nanogeles polimericos....

\section{M\'etodo: Simulaci\'on Monte Carlo}


La simulaci\'on de sistemas de nanogeles en soluci\'on es una herramienta invaluable para comprender su comportamiento y sus propiedades termodin\'amicas. En este estudio, se emple\'o el m\'etodo Monte Carlo, en particular el algoritmo de Metropolis-Monte Carlo, para modelar la soluci\'on de nanogeles.
El m\'etodo de Metropolis-Monte Carlo se fundamenta en la generaci\'on aleatoria de n\'umeros y la evaluaci\'on de distintos estados o configuraciones del sistema en estudio. En cada paso de la simulaci\'on, se selecciona de manera aleatoria una configuraci\'on y se calcula su energ\'ia o probabilidad seg\'un un modelo predefinido que describe las interacciones entre las part\'iculas de nanogeles y el solvente.
La simulaci\'on Monte Carlo, mediante el algoritmo de Metropolis-Monte Carlo, ofrece una perspectiva \'unica para explorar las propiedades de los nanogeles en soluci\'on y su comportamiento en diferentes condiciones. 
En el caso espec\'ifico de nuestra soluci\'on de nanogeles, una configuraci\'on consiste en el movimiento y aumento de tama\~no de una part\'icula (nanogel) seleccionada al azar. 
En este sentido la probabilidad de acetar o no dichos cambios en el sistema biene dado por:

\begin{align}
	P(s) = min \{e^{-(\Delta U^s_{inter} + \Delta U^s_{intra})},1\}
\end{align}

En donde $P(s)$ indica la probabilidad de aceptaci\'on del estado $s$, como mencionamos este consta de un cambio en el tama\~no y posici\'on de una nanogel. $\Delta U^s_{intra}$ corresponde al cambio en la energ\'ia interna de cada nanogel, debido a un cambio en su tama\~no, la energ\'ia de estos estados es obtenida usando una teor\'ia termodin\'amica detallada en el cap\'itulo  \ref{Chapter-geles}. Como moestramos en ese cap\'itulo el modelo de dos fases nos permite obtener la informaci\'on termodin\'amica necesarioa para explicar el comportamiento de una nanogel aislado.
El costo energetico de adsorber solvente, con sus respectivo iones, energ\'ia el\'astica... 
Por otro lado la energ\'ia intermolecular, $\Delta U_{inter}$, viene dada por la interacci\'ion de a pares de todos los nanogeles en soluci\'on. 
Para esta interacci\'on de a pares se ha utilizado un potencial combinado: Hertz-Yukawa. Este potencial fue previamente utilzado en \addcite[weyer2018concentration], en los cuales \addcite[weyer2018concentration] 
estudiaron la hinchaz\'on y las propiedades estructurales de suspensiones de microgeles i\'onicos su teor\'ia inlcuye las interacciones el\'asticas efectivas de Hertz y una teor\'ia de interacciones electrost\'aticas efectivas dependientes de la densidad (Yukawa). 
(un poco m\'as de los potenciales)
Porque los podemso usar para nuestro sistema..
materiales blandos que son deformables, por tando es posible modelarlos usando un potencial de Hertz, en el cual se considera la superposici\'on de las part\'iculas involucradas.
Yukawa por su parte es un potencial del tipo columbico mediado por el medio dilectrico en el que se encuentra. Se ven invlucradas las cargas que envuelven a cada nanogel considerando la longitud de Debye. 
Podemos resumir nuestro potencial total de trabajo como:

\begin{align}
	U_{inter}(r) = \begin{cases} U_H + U_Y & \text{if } r \leq a_i + a_j \\ U_Y & \text{if } r > a_i + a_j \end{cases} 
	\label{eq:HY-potential}
\end{align}

en donde $r$ es la distancia entre dos nanogeles, definida como la distancia de sus centros. 
$a_i$ y $a_j$ representan los radios de el nanogel $i$ y $j$ respectivamente. Por ello la condici\'on $r \leq a_i +a_j$ representa la superposici\'on de estas particulas, deformandose y entrando en juego el potencial de Hertz.
Para mayores distancia a la suma de los radios de cada nanogel se pone en juego el potencial de Yukawa. Para el cual se define:
\begin{align}
	\beta u_Y(r) = q_i q_j \frac{e^{\kappa(a_i + a_j)}}{(1 +\kappa a_i)(1 + \kappa a_j)} \frac{e^{-\kappa r}}{r} 
	\label{eq:yukawa}
\end{align}

En donde $q_i$ y $q_j$ son las cargas netas que poseen los nanogeles $i$ y $j$ respectivamente. El valor $\kappa$ corresponde a la constante de apantallamiento de Debye. 
El pontecial de Hertz, $U_H$, se define como:
\begin{align}
	\begin{aligned}
		& \beta u_H (r) = \left(\frac{1-r}{a_i + a_j}\right)^{5/2}\times b_{ij} \\
		& b_{ij} = \frac{6}{5\pi}\frac{N_{chains}}{\frac{4}{3}\pi R_0^3}(a_i + a_j)^2(a_ia_j)^{1/2}(a_i^3 + a_j^3) 
	\end{aligned}
\end{align}

En donde $b_{ij}$ es la constante de interacci\'on entre la part\'icula $i$ y $j$ la cual tiene en cuanta el n\'umero de cadenas que compone al nanogel: $N_{chains}$, los radios de las part\'iculas interactuantes $a_i$ y $a_j$ respectivamente y el valor de $r$ que determina la distancia al centro de masa del par interactuante. 