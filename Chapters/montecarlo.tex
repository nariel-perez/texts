\chapter{Soluciones coloidales}

%%%%%%%%%%%%%%%%%%%%%%%%%%%%%%%%%%%%%%%%%%%%%%%%%%
\section{Introducci\'on}

en estos ultimos capitulos hemos hecho incapie en la importancia y potencial uso que tienes los geles polimericos. Haciendo enfasis en sus uso para el secuestro de proteinas y/o farmacos de interes terapeutico.
En ese sentido en los dos ultimos capitulos hemos presentados dos tipos de modelos con los cuales es posible hacer un estudio de la fisicoquimica que involucra a un nano/microgel aislado en solucion.
En un primer capitulo, \ref{Chapter-geles} hicimos referencia a un modelos robusto con el cual pudimos explicar la respuesta de microgeles con respuesta multiestimulo... 
En el siguiente capitulo, \ref{Chapter-esfericas}, complejisamos nuestro modelo para investigar como afecta la estructura interna a la respuesta de estos geles, demostrando que el dise\~no  y sintesis,  es un factor importante para optimizar los procesos de encapsulacion y liberacion....
En este ultimo capitulo presentamos el comportamiento de la solucion de estos nano/microgeles en solucion. 


Estudio sobre las soluciones coloidales de nanogeles polimericos....

\section{M\'etodo: Simulaci\'on Monte Carlo}

Para modelar la soluci\'on de nanogeles se utiliz\'o el m\'etodo Monte Carlo, espec\'ificamente el algoritmo de Metropolis-Monte Carlo. Este m\'etodo se basa en la generaci\'on aleatoria de n\'umeros y la evaluaci\'on de diferentes estados o configuraciones del sistema de trabajo. En cada paso, se selecciona aleatoriamente una configuraci\'on y se calcula su energ\'ia o probabilidad de acuerdo a un modelo predefinido. Se aplican reglas para la aceptaci\'on o rechazo de dicha configuraci\'on. Este proceso se repite muchas veces para obtener resultados estad\'isticamente significativos y se utilizan para estimar propiedades macrosc\'opicas del sistema.

En el caso espec\'ifico de la soluci\'on de nanogeles, una configuraci\'on consiste en el movimiento y aumento de tama\~no de una part\'icula (nanogel) seleccionada al azar. El m\'etodo Metropolis Monte Carlo se utiliza para simular estas configuraciones y obtener resultados estad\'isticos significativos.

La probabilidad de acetar o no dichos cambios en el sistema biene dado por:

\begin{align}
	P = min \{e^{-(\Delta U_{inter} + \Delta U_{intra})},1\}
\end{align}

En donde $\Delta U_{intra}$ corresponde a la energ\'ia interna de cada nanogel obtenida usando una teor\'ia termodin\'amica detallada en el cap\'itulo  \ref{Chapter-geles}. De esta ultima teor\'ia se obtiene la termodinamica necesarioa para explicar el comportamiento de una nanogel aislado.
El costo energetico de adsorber solvente, con sus respectivo iones, energ\'ia el\'astica... 
Por otro lado la energ\'ia intermolecular, $\Delta U_{inter}$, viene dada por la interacci\'ion de a pares de todos los nanogeles en soluci\'on. 
Para esta interacci\'on de a pares se ha utilizado un potencial combinado: Hertz-Yukawa. Este potencial fue previamente utilzado en \addcite[weyer2018concentration], en los cuales \addcite[weyer2018concentration] 
estudiaron la hinchaz\'on y las propiedades estructurales de suspensiones de microgeles i\'onicos su teor\'ia inlcuye las interacciones el\'asticas efectivas de Hertz y una teor\'ia de interacciones electrost\'aticas efectivas dependientes de la densidad (Yukawa). 

Podemos resumir nuestro potencial de trabajo como:

\begin{align}
	U_{inter}(r) = \begin{cases} U_H + U_Y & \text{if } r \leq a_i + a_j \\ U_Y & \text{if } r > a_i + a_j \end{cases} 
	\label{eq:HY-potential}
\end{align}

para lo que se define:
\begin{align}
	\beta u_Y(r) = q_i q_j \frac{e^{\kappa(a_i + a_j)}}{(1 +\kappa a_i)(1 + \kappa a_j)} \frac{e^{-\kappa r}}{r} 
	\label{eq:yukawa}
\end{align}

En donde $q_i$ y $q_j$ son las cargas netas que poseen los nanogeles $i$ y $j$ respectivamente. De mismo modo se define el radio de cada part\'icula como $a_i$. El valor $\kappa$ corresponde a la constante de apantallamiento de Debye y $r$ es definido como la distancia entre los centros de masa de cada nanogel.

El pontecial de Hertz, $U_H$, se define como:
\begin{align}
	\begin{aligned}
		& \beta u_H (r) = \left(\frac{1-r}{a_i + a_j}\right)^{5/2}\times b_{ij} \\
		& b_{ij} = \frac{6}{5\pi}\frac{N_{chains}}{\frac{4}{3}\pi R_0^3}(a_i + a_j)^2(a_ia_j)^{1/2}(a_i^3 + a_j^3) 
	\end{aligned}
\end{align}

En donde $b_{ij}$ es la constante de interacci\'on entre la part\'icula $i$ y $j$ la cual tiene en cuanta el n\'umero de cadenas que compone al nanogel: $N_{chains}$, los radios de las part\'iculas interactuantes $a_i$ y $a_j$ respectivamente y el valor de $r$ que determina la distancia al centro de masa del par interactuante. 