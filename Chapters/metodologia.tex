\chapter{Hoja de ruta}
\label{ruta}

Este cap\'itulo busca guiar al lector en la distribución de la presente tesis. 


En su primer cap\'itulo, nos adentramos en el  mundo de los materiales blandos, la esencia misma de estos materiales y su relevancia en la industria y la ciencia moderna. Se har\'a enfas\'is en los hidrogeles polim\'ericos dentro de los cuales encotraremos a los films y a los micro y nano geles. Las motivaciones y objetivos son presentados. 
 Adem\'as se mencionar\'a el tipo enfoque y las herramientas que se utilizaron para llevar a cabo los estudios presentados en la tesis.

El primer sistema polim\'erico, los films, se retoma en los antecedentes, \ref{Chapter-film}, en la cual se da primera aproximaci\'on de parte de quien les escribe a la Teor\'ia Molecular. Las ventajas del uso de la termodin\'amica estad\'istica y como su combinaci\'on con las propiedades moleculares de los sistemas nos permite explicar toda la termodin'amica necesaria para la descipci\'on y predicci\'on de su comportamiento en condiciones determinadas.  
Se  revisan conceptos generales del comportamiento de estas cadenas polim\'ericas entrecruzadas en soluci\'on.
En part\'icular se mostrar\'a la respuesta  de los films poliméricos a cambios en el pH y la concentraci\'on de sal.  En este capitulo también encontraremos resultados sobre la aplicabilidad en el secuestro de dos prote\'inas modelos (citocromo y mioglobina), la elección de estos resultados, respuesta a pH, concentraci\'on salida, adsoric\'on de prote\'inas,  es debido a que los mismos son retomados en los cap\'itulos subsecuentes.

Es importante destacar que en  el transcurso de este aprendizaje logró la presentación de trabajos en donde se hace uso de los films polim\'ericos en el secuestro de moleculas, con aplicabilidad en la agroqu\'imcia  \addcite[film - glifosato], y principalmente en su uso de secuestro de f\'armacos \addcite[polimainas] trabajo con el cual nos acercamos  a los objetivos del presente trabajo.


Emocionado por su primer éxito, Carlos se aventuró en el siguiente capítulo: los micro y nano geles poliméricos. Descubrió que estos geles eran como pequeñas partículas con habilidades especiales, capaces de responder a estímulos externos y adaptarse a diferentes entornos. A medida que profundizaba en su investigación, Carlos comprendió la importancia de diferenciar entre micro y nano geles, explorando los métodos utilizados para sintetizarlos y caracterizarlos en el laboratorio. Fascinado por las aplicaciones emergentes de estos geles, desde la entrega de medicamentos hasta la ingeniería de tejidos, Carlos comenzó a trazar conexiones entre los films poliméricos y los geles poliméricos, revelando un mundo de posibilidades interconectadas.

Guiado por su deseo de exploración y conocimiento, Carlos pasó al siguiente capítulo: la metodología experimental. Aquí, descubrió los secretos detrás de los experimentos y análisis rigurosos. Se adentró en los materiales utilizados en su investigación y aprendió los procedimientos precisos para sintetizar tanto los films poliméricos como los micro y nano geles poliméricos. Cada paso, cada reacción química, se convirtió en un viaje hacia la comprensión y la perfección.

Después de meses de trabajo duro y dedicación, llegó el momento más esperado: los resultados y la discusión. Carlos presentó sus hallazgos al mundo, revelando datos reveladores y sorprendentes. A medida que profundizaba en la interpretación de los resultados, Carlos se sintió como un mago de la ciencia, conectando los puntos y desvelando nuevos conocimientos. Compartió sus conclusiones con la comunidad científica, analizando la relevancia de su trabajo y las posibles áreas de mejora.

Finalmente, Carlos escribió las conclusiones de su tesis. En este capítulo final, resumió sus descubrimientos y evaluó el éxito de sus objetivos. Reconociendo el valor de su investigación, Carlos compartió las contribuciones que su trabajo podría hacer al campo de estudio. Con entusiasmo, dejó abierta la puerta para futuras investigaciones, instando a otros a continuar su legado y desafiar los límites del conocimiento científico.

Y así, el viaje de Carlos llegó a su fin. Con su tesis completada, se despidió del mundo de los films poliméricos y los micro y nano geles poliméricos, pero no sin llevar consigo un tesoro invaluable: el conocimiento y la pasión por la ciencia. Su historia se convirtió en una inspiración para otros, recordándoles que el poder del descubrimiento y la investigación puede llevar a un futuro lleno de innovación y avance científico.

Y así, el cuento de Carlos y su "hoja de ruta" de tesis llega a su fin, pero el comienzo de un nuevo capítulo para aquellos que se atrevan a explorar los misterios y las maravillas de los materiales poliméricos.



En los siguientes capitulos ...
El estudio de estos dispositivos intelgientes a trav\'ez de m\'etodos te\'orico/computacionales no es muy distinto al realizado en un proyecto experimental. El planteo de la composici\'on y estrucutra de nuestros geles, as\'i como tambi\'en el barrido de variables,llamese pH, temperatura, concentraci\'on de sal, es realizado de manera sistem\'atica.
La ventaja de utilizar m\'etodos computacionales es que nos permiten explicar los fenomenos observables a partir del cambio de variables controladas en el laboratorio.
El planteo de teor\'ias que nos ayuden a obtener la informaci\'on termodin\'amica de cada sistema son la base para la presente tesis... 


Al igual que en un proyecto experimental, nuestra idea es poder identifcar las condiciones \'optimas en las cuales los geles puedan ser usados como biomateriales inteligentes. 
La identificaci\'on de estas condiciones es llevado a cabo utilizando teor\'ias que emplean la termodi\'amica estad\'istica. En part\'icular este trabajo hace uso de la Teor\'ia Cl\'asica de la Densidad. Esta \'ultima tendr\'a algunas variaciones dependiendo el tipo de modelado que se use, el cual ser\'a espicificado y explicado en cada uno de los sistemas presentados.

En part\'icular en este cap\'itulo se dar\'a un vistazo general de la Teor\'ia Molecular derivada de la teor\'ia cl\'asica de la densidad. 
Al final del mismo tambi\'en se introducir\'a sobre el tipo de modelado utulizado para poder implementar las teor\'ias....
 

