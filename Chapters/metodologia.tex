\chapter{Metodolog\'ia}
\label{Metodologia}

El estudio de estos dispositivos intelgientes a trav\'ez de m\'etodos te\'orico/computacionales no es muy distinto al realizado en un proyecto experimental. El planteo de la composici\'on y estrucutra de nuestros geles, as\'i como tambi\'en el barrido de variables,llamese pH, temperatura, concentraci\'on de sal, es realizado de manera sistem\'atica.
La ventaja de utilizar m\'etodos computacionales es que nos permiten explicar los fenomenos observables a partir del cambio de variables controladas en el laboratorio.
El planteo de teor\'ias que nos ayuden a obtener la informaci\'on termodin\'amica de cada sistema son la base para la presente tesis... 


Al igual que en un proyecto experimental, nuestra idea es poder identifcar las condiciones \'optimas en las cuales los geles puedan ser usados como biomateriales inteligentes. 
La identificaci\'on de estas condiciones es llevado a cabo utilizando teor\'ias que emplean la termodi\'amica estad\'istica. En part\'icular este trabajo hace uso de la Teor\'ia Cl\'asica de la Densidad. Esta \'ultima tendr\'a algunas variaciones dependiendo el tipo de modelado que se use, el cual ser\'a espicificado y explicado en cada uno de los sistemas presentados.

En part\'icular en este cap\'itulo se dar\'a un vistazo general de la Teor\'ia Molecular derivada de la teor\'ia cl\'asica de la densidad. 
Al final del mismo tambi\'en se introducir\'a sobre el tipo de modelado utulizado para poder implementar las teor\'ias....
 

\section{Teor\'ia Molecular}


Esta teor\'ia aborda el efecto de la distribución espacial de los grupos sensibles al pH en la termodinámica de la adsorción de proteínas en los sistemas poliméricos


This theory was developed in other works see \addcite[nap2006weak,longo2011molecular] and has been useful for the study of systems such as polymeric films with response to pH \addcite[longo2011molecular,longo2019protonation], study if ligand-receptor binding \addcite[longo2005ligand], and adsorption of different proteins \addcite[hagemann2018use,cathcarth2022competitive] or other molecules: glyphosate and polyamides \addcite[perez2020triggering, perez2018using]



Este m\'etodo consiste en minimizar una energ\'ia libre generalizada que incluye toda la termodin\'amica relevante de cada sistema.
Simult\'aneamente, el m\'etodo permite una descripci\'on molecular de , utilizando modelos de grano grueso, de las diferentes especies qu\'imicas que componen el sistema.
Dicha descripci\'on incluye forma, tama\~no, distribuci\'on de carga y/o estado de protonaci\'on de cada componente molecular.

En esta teor\'ia se trabaja en condiciones de equilibro por lo cual nuestro  sistema en estudio se encuentra en  equilibrio con una soluci\'on acuosa (bulk o ba\~no de la soluci\'on) que tiene una composici\'on  definida externamente.
Es decir, el pH, la concentraci\'on de sal y la concentraci\'on de adsorbatos son variables independientes.

Otro aspecto que tiene en cuenta  es la estructura de la red polim\'erica de los geles. Estos pueden contener distintos tipos de segmentos, los cuales son los que le dan la respuesta a est\'imulo y por ello considerados como materiales inteligentes. Dichas unidades qu\'imcas pueden ser sencibles al pH. Para este trabajo se estudian monomero de \'acido metacr\'ilico (MAA) y alilamida (AH) los cuales tiene comportamientos \'acido y b\'asico respectivamente. Como veremos en los resultado la respuesta en el pH es tambi\'en influenciada por la concentrac\'on salina del medio. 
Otro tipo de segmentos son los  neutros, como el alcholo vinilico (VA),  o bien segmentos termosencibles (NIPAm)
Las unidades de que representan los entrecruzamientos de cadena se describen como segmentos de carga neutral.

El semi-gran potencial de este sistema contiene las siguientes contribuciones:
\begin{align}
\begin{aligned}
\Omega_{NG}=& -TS_{mez} -TS_{conf,nw} + F_{chem,nw} + F_{chem,pro}\\
& + U_{elec} + U_{ste} + U_{VDW} - {\sum_{\gamma}{\mu_\gamma N_\gamma}}
\end{aligned}
\label{eq:semicano}
\end{align}
\noindent donde $S_{mez}$ es la entropía de traslación (mezcla) de las especies de la solución: moléculas de agua (H$_2$O), iones de hidronio (H$_3$O$^+$), iones de hidróxido (OH$^- $), cationes de sal, aniones de sal y otros adsorbaatos presentes: como drogas o proteinas terapeuticas.
Aquí, consideramos una sal monovalente, NaCl o KCl, y se asueme que está completamente disociada en iones cloruro (Cl$^-$) y sodio ($Na^+$) o potasio ($K^+$) respecticamente. 

$S_{conf,nw}$ representa la entropía conformacional que resulta de la flexibilidad de la red de polímeros, que puede asumir muchas conformaciones diferentes.
$F_{chem,nw}$, es la energía química libre que describe el equilibrio entre las especies protonadas y desprotonadas de unidades funcionales (ácidas/básicas) en el polímero.
De manera similar, $F_{chem,pro}$ describe la protonación de residuos titulables de la proteína.
$U_{elec}$ y $U_{ste}$ representan, respectivamente, las interacciones electrostáticas y las repulsiones estéricas.
Las interacciones no electrostaticas son representadas en $U_{VDW}$ en las cuales se puede mencionar la interaccino NIPam-agua.
Finalmente, la suma de $\gamma$ expresa el equilibrio químico entre nuestro sistema y la solución  que representa el bulk para las partículas libres, donde $\mu_\gamma$ y $N_\gamma$ son el potencial químico y el número de moléculas de especie $\gamma$, respectivamente;
el subíndice $\gamma$ recorre las especies químicas libres, incluidas las proteínas.
En este punto, tenga en cuenta que $\Omega_{NG}$ es un semi-gran potencial porque el sitema polimerico puede intercambiar cada una de estas moléculas libres con la solución , mientras que la red de polímeros está confinada dentro de nuestro sistema.


\section{Modelado}

Se utiliza un modelo de grano grueso en el cual las unidades consideradas son representativas de las moleculas y contienen la informaci\'on molecular.

\subsection{Generaci\'on de configuraciones}

\subsubsection{Modelo RIS}
Las distintas configuraciones de las cadenas del polímero son generadas utilizando un modelo de isomeros rotacionas (RIS por sus siglas en inglés) donde cada segmento puede tomar cualquiera de las tres posibles orientaciones isoenergéticas que le corresponden \addcite[Flory]
La longitud del segmento entre monomeros es $0.5 nm$ y su volumen es $v_{AH}= 0.065 nm^3$. Se generaron cadenas con 25, 50 y 100 segmentos. Para cada uno de estos largos de cadenas se generaron aleatoriamente un total de $10^6$ configuraciones distintas. Este número de configuraciones es suficientemente grande como para describir las propiedades del polímero adecuadamente, pero al mismo tiempo permite el cálculo de resultados en un tiempo razonable.

\subsubsection{gromacs}
Para la generaci\'on del modelo, en concreto de las configuracines, de nuestro migroles se partir\'a de una celda base, la cual se expander\'a en el espacio de tal forma de obtener las dimensiones que necesitemos.
La estrucutura obtenida  ser\'a truncada/cortada para obtener una esfera que dará origen a la topología inicial del gel.

Las simulaciones de, din\'amica molecular, y sus respectivos an\'alisis se harán haciendo uso del Software GROMACS. \addcite[11] 
Finalmente todas las configuraciones serán  ponderadas dada la energía libre atribuida a cada una. \addcite[12]

La estrucutura base que se utilizar\'a es la celda unitaria del diamanate ...\textcolor{red}{ por qu\'e ??}...
Dada la arquitectura de los geles, cadenas entrelazadas, es necesario definir los nodos o crosslinks del gel,  en nuestro modelo los crosslinks corresponden a la posici\'on de lo \'atomos de carbono en la estructura base (celda de diamante).
expansion y el corte esféricos, el cual es el punto de partida para la topolog\'ia de nuestro gel.

Para nuestro esquema, trabajaremos con geles con un radio de corte de 109,\textcolor{red}{en unidades arbitrarias... sigma}, además se utilizarán cadenas de 25 segmentos cada una\footnote{Dado el corte esférico realizado, las cadenas externas poseen menor cantidad}. Nuestro gel finalmente estará compuesto por aproximadamente 10 mil segmentos (10054).
