\prefacesection{Gu\'ia del Navegante...}
\label{ruta}

 
En esta primera instancia se busca guiar al lector en la distribuci\'on de la presente tesis....se presenta un breve recorrido de cada uno de los cap\'itulos escritos.
Cada uno de ellos contiende la informaci\'on necesaria para autosustentarse, pero hay una interconexi\'on entre ellos en donde se continua y responden preguntas que no se pod\'ian delucidar con los modelos anteriores, monstrando as\'i un estudio sitematico de los sistemas polim\'ericos.
Se espera que la lectura sea agradable... \textcolor{red}{muy prologo de libro}....

En el primer cap\'itulo, nos adentramos en el mundo de los materiales blandos, la esencia misma de estos materiales y su relevancia en la industria y la ciencia moderna. Introducción de tema y problema a abordar.  Se har\'a \'enfasis en los hidrogeles polim\'ericos, dentro de los cuales encontraremos a los films y a los micro y nanogeles. Se presentar\'an las motivaciones y objetivos. Adem\'as, se mencionar\'a el enfoque utilizado y las herramientas empleadas para llevar a cabo los estudios presentados en la tesis.

El primer sistema polim\'erico, los films, se retoma como partes de los antecedentes (Cap\'itulo \ref{Chapter-film}), en los cuales se da una primera aproximaci\'on de parte de quien escribe a la Teor\'ia Molecular. Se destacan las ventajas del uso de la termodin\'amica estad\'istica y c\'omo su combinaci\'on con las propiedades moleculares de los sistemas nos permite explicar toda la termodin\'amica necesaria para la descripci\'on y predicci\'on de comportamientos en condiciones determinadas. Se revisan conceptos generales de la respeusta a estimulo de estas cadenas polim\'ericas entrecruzadas en soluci\'on. En particular, se mostrar\'a la respuesta de los films polim\'ericos a cambios en el pH y la concentraci\'on de sal. En este cap\'itulo tambi\'en se encontrar\'an resultados sobre la aplicabilidad en el secuestro de dos prote\'inas modelos (citocromo y mioglobina). La elecci\'on de estos resultados, respuesta a pH, concentraci\'on salina y adsorci\'on de prote\'inas, se debe a que son retomados en los cap\'itulos subsecuentes. En particular las prote\'inas se han seleccionado debido a su uso en resultados experimentales en sistemas polim\'ericos (\addcite). Adem\'as, en los siguientes sistemas de estudio se incorporar\'an nuevos adsorbatos: insulina y las drogas doxorubicina (y su derivado daunorubicina). La insulina se ha seleccionado debido a su importancia en tratamientos contra la diab\'etes y la b\'usqueda de mejores transportadores para mejorar su dosificaci\'on. El mismo criterio se ha utilizado en la elecci\'on de las drogas doxorubicina y daunorubicina, antraciclinas muy utilizadas en tratamientos anticancer\'igenos.
Es importante destacar que durante este acercamiento a la teor\'ia molecular se logr\'o la publicaci\'on de trabajos en los que se hace uso de los films polim\'ericos en el secuestro de mol\'eculas, con aplicabilidad en la agroqu\'imica \cite{perez2018using, perez2019molecular}, y principalmente en su uso de secuestro de f\'armacos \cite{perez2020triggering}, trabajo con el cual nos acercamos a uno de los objetivos de esta tesis. 

En el cap\'itulo \ref{Chapter-geles}, nos encontraremos con una primera aproximaci\'on hacia los geles polim\'ericos. En particular, se presentará un estudio de la termodin\'amica de estos sistemas. En este cap\'itulo nos valemos de un sistema de dos fases tipo Donnan. Una descripci\'on detallada del mismo se presentar\'a en las primeras hojas del cap\'itulo. Los resultados mostrados forman parte de un trabajo publicado \cite{perez2021thermodynamic}, en el cual se plantea el uso de un modelo robusto con el cual se explican los fen\'omenos observados experimentalmente, desde la fisicoqu\'imica. Se muestra la capacidad de estos microgeles como encapsuladores de drogas terap\'euticas y se predicen las condiciones \'optimas de laboratorio para dicha tarea.

En el cuarto cap\'itulo, extendemos los conocimientos adquiridos con nuestro modelo de dos fases para incorporar estructura definida a los geles de trabajo. Adem\'as de la respectiva descripci\'on que conlleva el estudio de estos geles, se escribe un apartado sobre el trabajo que tom\'o la obtenci\'on de las configuraciones que dan origen a la estructura de los nanogeles.

Finalmente, en el quinto cap\'itulo, se estudia el comportamiento de los nanogeles en soluci\'on.


