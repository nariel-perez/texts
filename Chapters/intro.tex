%% Los cap'itulos inician con \chapter{T'itulo}, estos aparecen numerados y
%% se incluyen en el 'indice general.
%%
%% Recuerda que aqu'i ya puedes escribir acentos como: 'a, 'e, 'i, etc.
%% La letra n con tilde es: 'n.

\chapter{Introducci'on}
\label{Chapter1} % Change X to a consecutive number; for referencing this chapter elsewhere, use \ref{ChapterX}

%----------------------------------------------------------------------------------------
%	SECTION 1
%----------------------------------------------------------------------------------------

\section{Materia Blanda}

Durante la decada pasada el uso de materiales blandos ha sido de mucho interes debido a la cantidad de materiales e innovaciones que se optienen con ellos.

La versatilidad de estos los han convertido en materiales  importantes en una amplia variedad de aplicaciones tecnológicas.
Se han utilizado como  espumas y adhesivos, son excelente detergentes, est\'an presentes en la industria de los  cosméticos y pinturas, adem\'as de ser ampliamente usados en aditivos alimentarios . El campo de la medicina, la industria farmaceutica, ha encontrado en estos materiales una oportunidad para la creaci\'on de de trasnportadores de drogas m\'as eficientes biocompatibles y biodegradables.

En estas ultimas aplicaciones los  films y geles  polim\'ericos han sido pioneros en su uso y han tenido un creciente inter\'es.
La capacidad de adsorber solventes y la estructura de cadenas entrelazadas proporciona condiciones adecuadas para la adsorcion y proteccion de adsorbatos de interés. \addcite 
Estas características permiten que las drogas/proteinas adsorbidas interactuen con el solvente en el cual son solubles. La proteccion es debido a la restricción de entrada de otros agentes, además de evitar el movieminto libre de la droga. Aspecto muy importante si se consideran proteinas y se quiere evitar su desplegamiento.
Sin embargo la caracteris más notoria de estos materiales es su capacidad de responder a diversos estimulos. 
Muchos estudios han reportado la respuesta a la temperatura de estos materiales, entre los m\'as conocidos se encuentra PNIPAm, Pluronics, elastin-like polypeptides (ELP) and poly(N- vinylcaprolactam) (PNVCL).

Interest in thermoresponsive polymers has steadily grown over many decades, and a great deal of work has been dedicated to developing temperature sensitive macromolecules that can be crafted into new smart materials. 

However, the overwhelming majority of previously reported temperature-responsive polymers are based on poly(N-isopropylacrylamide) (PNIPAM), despite the fact that a wide range of other thermoresponsive polymers have demonstrated similar promise for the preparation of adaptive materials. Herein, we aim to highlight recent results that involve thermoresponsive systems that have not yet been as fully considered. Many of these (co)polymers represent clear opportunities for advancements in emerging biomedical and materials fields due to their increased biocompatibility and tuneable response. By highlighting recent examples of newly developed thermoresponsive polymer systems, we hope to promote the development of new generations of smart materials.


Este comportamiento responsivo hace de estos materiales excelentes candidados en el secuestro de particulas, intercambios ionicos lo cual permite el dise\~no de transportadores. 
\textcolor{green}{here}

El diseño de dispositivos inteligentes optimizados para tareas específicas basados en estos sitemas blandos demanda un estricto control sobre sus propiedades fisicoquímicas y la respuesta de estos geles.
Por lo tanto, su diseño  requiere el desarrollo de modelos que permitan entender y predecir cómo el comportamiento observado resulta de la interrelación entre la estructura química del polímero que lo componen, la organización molecular bajo condiciones de confinamiento,  las interacciones entre especies y las propiedades del medio en el que se encuentren.


Por ejemplo, en la actualidad desarrollar materiales que respondan a múltiples estímulos (pH y temperatura) de manera predecible y controlable representa un desafío enorme. \addcite
\begin{enumerate}
    \item matariales con respuesta a Temperaruta y como cambia su respuesta al modifiar la arquitectura... film y gel.
    \item para materiales con solo respuesta a pH, como el cambio en la concentraci\'on salina del medio afecta la respuesta de los geles.
    \item trabajos experimentales de copolimeros, como se afectan entre ellos la respuesta de cada uno.
\end{enumerate}

...fin de la tesis, estudio te\'orico haciendo uso de simulaciones computaciones para el desarrollo de nuevos materiales capaces de responder a estimulo, una caracterizaci\'on exhaustiva de sus propiedades fisicoqu\'imcas.

Descripci\'on generica de los obsetivos... uso de la TM y familiarizaci\'on de la misma con hidrogeles de homopolimeros, como responden y pueden servir para la captura de moleculas. Uso de estos materiales para fines ambientales...
Del mismo modo estudiar microgeles en un modelo sencillo como primera aproximaci\'on. Fisicoquimca de geles homogeneos y con respuesta a multi estimulo. Adsorci\'on de drogas terapeuticas, condiciones optimcas de encapsulamiento.
Complejizaci\'on de la teor\'ia para modificar parametros de estructura, es decir como afectan los cambios de topologia de estos geles.
Adsorci\'on de biomoleculas. Nueva inormaci\'on sobre la arquitectura de los geles y su relaci\'on con la adsorci\'on de proteinas. 
Finalmente, m\'etodo estocastico para el estudio de soluciones de geles polimericos. En un modelo sencillo.
C\'omo la concentraci\'on afecta la distribuci\'on de tama\~no, adsorci\'on, etc.
%-----------------------------------
%	SUBSECTION 1
%-----------------------------------
\subsection{Objetivos}

Los {\bf objetivos específicos} del presente plan de trabajo son los siguientes:
%
\begin{enumerate}
\item Desarrollar un modelo mecano-estadístico utilizando TM para describir la respuesta a cambios de pH, temperatura y concentración de sal en microgeles formados por homopolímeros.%\label{objetivo_1}
\item Extender dicho modelo para investigar el comportamiento de microgeles de copolímeros con respuesta a múltiples estímulos.%\label{objetivo_2}
\item Estudiar los mecanismos de adsorción de diferentes biomoléculas en los microgeles en función de las condiciones del medio y la estructura/composición química de las cadenas poliméricas.%\label{objetivo_3}
\item Desarrollar un modelo combinando simulaciones de TM y Dinámica Molecular (DM) para estudiar el comportamiento de estos microgeles en soluciones relativamente concentradas.%\label{objetivo_4}
\end{enumerate}
%


\subsection{Films poliméricos}

Los hidrogeles de cadenas de polímeros reticulados se consideran actualmente para diversas aplicaciones en la investigación biomédica.%\cite{Wang2019}
 Estos materiales pueden parecerse al tejido biol\'ogico y dise\~narse para responder a cambios ambientales como variaciones de temperatura, pH, fuerza i\'onica y en la concentraci\'on de algunas biomol\'eculas. Como resultado, los hidrogeles polim\'ericos actualmente son candidatos prometedores para el desarrollo de una variedad de biomateriales con aplicaciones para biodetección [1,2], ingenier\'ia de tejidos [3,4], regeneraci\'on \'osea [5], materiales biomim\'eticos [6,7], administraci\'on de f\'armacos [8,9] y muchas otras aplicaciones biom\'edicas [10].
%\cite{Lowman1999,Zhao2019,Qindeel2019,Li2019}
Estos hidrogeles responden a los cambios de pH porque contienen una cantidad significativa de grupos ácidos débiles.
%-----------------------------------
%	SUBSECTION 2
%-----------------------------------

\subsection{Subsection 2}
Morbi rutrum odio eget arcu adipiscing sodales. Aenean et purus a est pulvinar pellentesque. Cras in elit neque, quis varius elit. Phasellus fringilla, nibh eu tempus venenatis, dolor elit posuere quam, quis adipiscing urna leo nec orci. Sed nec nulla auctor odio aliquet consequat. Ut nec nulla in ante ullamcorper aliquam at sed dolor. Phasellus fermentum magna in augue gravida cursus. Cras sed pretium lorem. Pellentesque eget ornare odio. Proin accumsan, massa viverra cursus pharetra, ipsum nisi lobortis velit, a malesuada dolor lorem eu neque.

%----------------------------------------------------------------------------------------
%	SECTION 2
%----------------------------------------------------------------------------------------

\section{Main Section 2}

Sed ullamcorper quam eu nisl interdum at interdum enim egestas. Aliquam placerat justo sed lectus lobortis ut porta nisl porttitor. Vestibulum mi dolor, lacinia molestie gravida at, tempus vitae ligula. Donec eget quam sapien, in viverra eros. Donec pellentesque justo a massa fringilla non vestibulum metus vestibulum. Vestibulum in orci quis felis tempor lacinia. Vivamus ornare ultrices facilisis. Ut hendrerit volutpat vulputate. Morbi condimentum venenatis augue, id porta ipsum vulputate in. Curabitur luctus tempus justo. Vestibulum risus lectus, adipiscing nec condimentum quis, condimentum nec nisl. Aliquam dictum sagittis velit sed iaculis. Morbi tristique augue sit amet nulla pulvinar id facilisis ligula mollis. Nam elit libero, tincidunt ut aliquam at, molestie in quam. Aenean rhoncus vehicula hendrerit.




Existen dos tipos de citas bibliograf'icas: usa \verb|\citep{..}| para
citas en \emph{par'entesis} y \verb|\citet{..}| para citas
en el \emph{texto}. Por ejemplo, estudios reciente han mostrado nuevos e
interesantes modelos que se pueden aplicar para reformular teor'ias
f'isicas~\citep{NewCam97}. Mientras que, el trabajo de \citet{Rofl06} fue
considerado muy divertido por una significativa fracci'on de la comunidad
de investigadores. Tambi'en es posible citar a varios trabajos en una sola
referencia \citep{Lamport86,Knuth84}.

Estos comandos para producir citas bibliograficas son provistos por
el paquete \textsf{natbib}. Para obtener m'as informaci'on, consulta la
documentaci'on de ese paquete~\citep{doc:natbib}. Por su parte, en
la documentaci\'on de \textsf{geometry} puedes encontrar detalles
adicionales sobre el sistema para ajustar los m'argenes del
documento~\citep{doc:geometry}. Lo que sigue
es un mont'on de texto sin sentido en lat'in que utilizaremos para llenar
algunas p'aginas.

Lorem ipsum dolor sit amet, consectetuer adipiscing elit. Integer arcu nisl,
consectetuer ut, vehicula nec, blandit id, nulla. Vestibulum in odio a odio
volutpat sollicitudin. Donec congue porta tellus. Ut quis est sed velit
blandit fringilla. Nunc lobortis dui vitae sapien. In tincidunt magna eget
purus. Nam lorem quam, vehicula in, dictum et, congue eget, odio. Curabitur
gravida mi id dui. Aliquam erat volutpat. Fusce velit turpis, accumsan vel,
tincidunt at, aliquet at, sem. Cras viverra eros ac orci. Aenean vestibulum,
lorem sed luctus congue, arcu pede ultricies libero, at posuere felis nulla
et leo.

Fusce rhoncus lobortis orci. Quisque suscipit dolor. In tincidunt dictum
elit. Cras metus. Donec nibh mi, ornare a, ullamcorper in, gravida non,
augue. Aliquam erat volutpat. Aliquam commodo tellus sed dolor. Sed urna.
Phasellus blandit orci sit amet nulla. Fusce vel eros. Aenean ultrices
sodales mi. Aliquam erat volutpat. Fusce orci sem, sollicitudin convallis,
auctor a, sollicitudin vitae, dui. Sed massa. Duis luctus lectus ut lacus.

Morbi felis tellus, placerat quis, congue pretium, consectetuer at, tortor.
Nunc condimentum mattis urna. Donec dolor erat, fringilla ut, auctor ac,
vestibulum ut, velit. Aliquam convallis magna ac neque. Praesent varius
congue augue. Nulla adipiscing urna faucibus diam. Mauris porta sapien ut
justo. Donec suscipit tortor gravida ligula. Aliquam ac purus et massa
scelerisque vehicula. Maecenas a libero. Class aptent taciti sociosqu ad
litora torquent per conubia nostra, per inceptos himenaeos. Pellentesque
sit amet est eget metus tincidunt semper. Phasellus nec purus. Proin
venenatis lectus vel elit. Pellentesque augue quam, tincidunt sed, pretium
ut, feugiat id, odio. Aenean eu nibh et quam dignissim facilisis.

Suspendisse adipiscing. Maecenas tincidunt placerat justo. Ut mattis nunc ac
orci. Vestibulum quis velit sed massa vulputate posuere. Duis rhoncus lacus.
Quisque non lacus et nibh molestie tincidunt. Nulla tortor pede, auctor id,
eleifend sit amet, ultrices id, risus. Duis et lectus. Suspendisse interdum,
magna ut porta porta, quam tellus suscipit ligula, cursus consectetuer purus
erat et dolor. Phasellus venenatis, risus malesuada lacinia placerat, lectus
tellus lobortis ligula, eu porttitor tellus nibh eu enim. Morbi vel erat in
sem pharetra molestie. Duis tellus. In ipsum. Vivamus ac augue sed dui
hendrerit pulvinar. In dui erat, molestie ut, lacinia at, sagittis sed, nisi.
Maecenas libero. Nam volutpat dictum erat.

Fusce laoreet sapien ut lorem. Mauris sed leo a mi luctus sollicitudin.
Donec ornare nisi id dolor. Ut eros metus, tristique quis, ultrices ac,
accumsan cursus, est. Pellentesque mollis posuere sapien. Morbi nec augue.
Cum sociis natoque penatibus et magnis dis parturient montes, nascetur
ridiculus mus. Duis tristique, ipsum in tincidunt gravida, nunc nulla
vehicula felis, elementum eleifend nunc elit id magna. Cum sociis natoque
penatibus et magnis dis parturient montes, nascetur ridiculus mus. Curabitur
rhoncus dui ut sapien.

Sed orci. Nunc nisi lorem, convallis nec, porttitor at, porttitor et, erat.
Lorem ipsum dolor sit amet, consectetuer adipiscing elit. Donec luctus, velit
quis lacinia pulvinar, risus urna malesuada nisl, vel hendrerit erat enim ac
enim. Aliquam sapien dolor, fringilla quis, consequat auctor, sodales id,
est. In imperdiet est et dui. Cras libero lacus, feugiat a, auctor ut,
vulputate sollicitudin, orci. Ut tellus velit, rutrum tristique, eleifend sit
amet, auctor consectetuer, sapien. Fusce eget justo. Nam auctor lorem at
purus. Vestibulum ante ipsum primis in faucibus orci luctus et ultrices
posuere cubilia Curae; Pellentesque pretium enim sed tortor. Sed luctus velit
at ligula. Nunc id elit. Curabitur lacus. Mauris placerat nibh sit amet
turpis. Fusce varius, justo et ultrices dictum, urna risus rhoncus ipsum, sed
ultricies nunc arcu eu risus. Nam vitae purus.

Proin augue. Duis vehicula mauris sollicitudin sapien. Nam tristique lacus
nec nisl. Praesent quis enim. Vestibulum vel velit in purus luctus mattis.
Mauris ullamcorper tempor lorem. Quisque rutrum. Praesent enim nibh,
pellentesque non, lacinia accumsan, euismod a, lorem. Etiam fringilla
iaculis mauris. Aenean adipiscing purus in lacus. Maecenas quis nibh. Ut
non augue at mauris elementum luctus. Duis varius tincidunt mi. Aliquam
justo massa, auctor nec, malesuada interdum, mollis ac, mauris. Maecenas
ultricies gravida dui. Aliquam arcu elit, pretium eu, gravida ac, molestie
eu, enim. Etiam facilisis orci eget est. Integer eu orci non felis tincidunt
consectetuer. Sed imperdiet ultrices nibh.
