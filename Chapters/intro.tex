%% Los cap'itulos inician con \chapter{T'itulo}, estos aparecen numerados y
%% se incluyen en el 'indice general.
%%
%% Recuerda que aqu'i ya puedes escribir acentos como: 'a, 'e, 'i, etc.
%% La letra n con tilde es: 'n.

\chapter{Introducci\'on}
\label{Chapter1} % Change X to a consecutive number; for referencing this chapter elsewhere, use \ref{ChapterX}

%----------------------------------------------------------------------------------------
%	SECTION 1
%----------------------------------------------------------------------------------------

\section{Hidrogeles polim\'ericos}

%%%%%%%%%%%%%%%
%%%%%%%%%%%%%%%recorte

\begin{color}{blue}
Durante la decada pasada el uso de materiales blandos ha sido de mucho interes debido a la cantidad de materiales e innovaciones que se optienen con ellos.
La versatilidad de estos los han convertido en materiales  importantes en una amplia variedad de aplicaciones tecnológicas.
Se han utilizado como  espumas y adhesivos, son excelente detergentes, est\'an presentes en la industria de los  cosméticos y pinturas, adem\'as de ser ampliamente usados en aditivos alimentarios . El campo de la medicina, la industria farmaceutica, ha encontrado en estos materiales una oportunidad para la creaci\'on de de trasnportadores de drogas m\'as eficientes biocompatibles y biodegradables.
En estas ultimas aplicaciones los  films y geles  polim\'ericos han sido pioneros en su uso y han tenido un creciente inter\'es.
La capacidad de adsorber solventes y la estructura de cadenas entrelazadas proporciona condiciones adecuadas para la adsorcion y protecci\'on de adsorbatos de inter\'es. \addcite 
\end{color}

En particular los hidrogeles consisten en una red de pol\'imeros reticulares (entrecruzados) altamente hidratados y generalmente biocompatibles. Estos materiales pueden parecerse a los tejidos biol\'ogicos y pueden ser dise\~nados para responder a cambios ambientales como variaciones en la temperatura, el pH, la fuerza i\'onica y la concentraci\'on de algunas biomol\'eculas. Como resultado, los hidrogeles polim\'ericos son actualmente candidatos prometedores para el desarrollo de una variedad de biomateriales con aplicaciones en biosensores, ingenier\'ia de tejidos, regeneraci\'on \'osea, materiales biomim\'eticos, administraci\'on de medicamentos y muchas otras aplicaciones biom\'edicas. \cite{Daly2020}
	
	
Inmersas en soluciones acuosas, estos hidrogeles pueden incorporar y retener grandes cantidades de agua, y de manera m\'as general el solvente en el cual se encuentren inmersos, dentro de su estructura polim\'erica.
Sin embargo, la caracter\'istica m\'as llamativa de estas sistemas es su capacidad para adsorber o liberar estos solventes y cambiar de tama\~no en respuesta a una variedad de est\'imulos externos.
El entorno acuoso dentro de los hidrogeles puede proteger a las prote\'inas de la desnaturalizaci\'on y agregaci\'on, mientras que estas se mantienen activas y estructuradas cuando se liberan de los hidrogeles. \addcite
En la administraci\'on oral de medicamentos, los hidrogeles con respuesta al pH han sido ampliamente investigados como veh\'iculos funcionales que pueden encapsular y liberar prote\'inas, evitando su degradaci\'on en el entorno gastrointestinal. \addcite
Este comportamiento de respuesta es  generalmente reversible y depende de la composici\'on qu\'imica de la red polim\'erica.


\section{Respuesta a estimulo pH, sal y Temperatura}

	
Los microgeles compuestos por cadenas polim\'ericas que tienen segmentos \'acidos como el \'acido acr\'ilico o metacr\'ilico (AA y MAA, respectivamente) se hinchan-deshinchan muchas veces en respuesta a cambios en el pH de la soluci\'on que los contienen \cite{snowden1996colloidal}.
El pH en el cual se marca el inicio y caracteriza esta transici\'on es el pKa aparente del microgel, que depende de la concentraci\'on de sal de la soluci\'on y frecuentemente difiere del pKa intrí\'iseco del mon\'omero \'acido.
Estos microgeles tambi\'en ajustan su tama\~no en respuesta a cambios en la concentraci\'on de sal de la soluci\'on \cite{snowden1996colloidal}.
	
An\'alogamente, los microgeles de algunos pol\'imeros termosensibles experimentan una transici\'on de fase de volumen (VPT por sus siglas en ingl\'es) cuando se calientan por encima de una temperatura caracter\'istica (VPTT o $T_{pt}$) \cite{Pelton1986,Pelton2000}.
Este comportamiento se origina porque tales pol\'imeros son insolubles en agua por encima de cierta temperatura de soluci\'on cr\'itica m\'as baja (LCST) \cite{Kawaguchi2020}.
Normalmente, la LCST del pol\'imero y la VPTT de la red  son aproximadamente id\'enticas. 
Este es el caso de las part\'iculas de microgel de poli(N-isopropilacrilamida) (PNIPAm) \cite{Pelton1986}, cuyo volumen colapsa por encima de $32   ^\circ C$, siendo le mismo para el  pol\'imero lineal \cite{Schild1992}.
	
Al tener un VPTT alrededor de la temperatura corporal, los microgeles de PNIPAm han generado un gran inter\'es para aplicaciones biom\'edicas \cite{Guan2011}.
Las estrategias para controlar el VPTT de los microgeles incluyen la s\'intesis de nuevos mon\'omeros termosensibles  \cite{Cai2007,Macchione2019}, as\'i como la copolimerizaci\'on con un mon\'omero i\'onico o ionizable  \cite{Hirose1987,Lopez2020}.
Este \'ultimo enfoque produce microgeles de respuesta m\'ultiple que son susceptibles a cambios en la temperatura, el pH y la concentraci\'on de sal  \cite{snowden1996colloidal, Farooqi2017}.
Los microgeles de NIPAm y AA han sido ampliamente estudiados \cite{Morris1997, Jones2000,Bradley2005,Begum2016};
Tambi\'en se han investigado microgeles de copol\'imeros de NIPAm y MAA  \cite{Dowding2000,Hoare2004,Giussi2015}.
El VPTT de estos microgeles de respuesta m\'ultiple depende del pH de la soluci\'on y la concentraci\'on de sal, y la fracci\'on de mon\'omero ionizable en las cadenas de pol\'imero  \cite{Morris1997,Jones2000, Hoare2004, Bradley2005, Lee2008,Wong2009,Hamzavi2016}.



\section{Encapsulado y Liberaci\'on de medicamentos}

El inter\'es en hidrogeles en especial aquellos con di\'ametros menores a 200 nm ha crecido rotudamente debido a su tiempo m\'as prolongado de circulaci\'on en el sistema circulatorio. La naturaleza incipientemente hidrof\'ilica de estas estructuras, los nanogeles son generalmente biocompatibles y poseen una gran capacidad para incorporar mol\'eculas hu\'esped o analitos, tanto org\'anicos como inorg\'anicos, y prevenir su degradaci\'on por el medio externo. El ambiente acuoso dentro de los hidrogeles puede proteger a las prote\'inas de la desnaturalización y la agregaci\'on [11y13], mientras permanecen activas y estructuradas cuando se liberan de los hidrogeles [14]. En la administraci\'on oral de f\'armacos, los hidrogeles con respuesta de pH se han investigado en gran medida como veh\'iculos funcionales que pueden encapsular y administrar prote\'inas, evitando su degradaci\'on en el entorno gastrointestinal [15-17].
Adem\'as, su escaso tama\~no les permite responder r\'apidamente luego de recibir el est\'imulo \cite{tanaka1979kinetics} . Por todas estas razones, los nanogeles polim\'ericos son hoy en d\'ia una de las primeras opciones consideradas al dise\~nar biomateriales con funciones espec\'ificas \cite{soni2016nanogels, sabir2019polymeric}. El est\'imulo que dispara la respuesta de los nanogeles puede ser suministrado por un gradiente en la composici\'on fisiol\'ogica, ya sea natural o inducido por un estado patol\'ogico. La versatilidad de estos materiales dif\'icilmente puede ser alcanzada con otro tipo de nanopart\'iculas, incapaces de responder a cambios en las condiciones del medio que pueden ser relativamente moderados. El desaf\'io en la actualidad es aprender a controlar esta respuesta para canalizarla en diferentes aplicaciones.


Por citar, \citet{Brugger2008} encontraron que el pH durante la s\'intesis tiene un impacto significativo en la composici\'on y, por lo tanto, en las propiedades del microgel y su capacidad para ser utilizado como un estabilizador sensible a est\'imulos.
Resultados similares fueron estudiados por otros autores en donde se destaca el uso de emulsiones sensibles al pH, la sal y la temperatura  \cite{Ngai2005,Ngai2006, Schmidt2011} o como plantillas para el ensamblaje de nanomateriales \cite{Wong2009}.
Haciendo de estos sistema no solo valiosos en el encapsulado y liberaci\'on de medicamentos, sino también como secuestradores de diferentes adsorbatos.

Del mismo modo \citet{Culver2017A}han utilizado nanogeles de poli(NIPAm-co-MAA) funcionalizados para la uni\'on y detecci\'on de diferentes prote\'inas. 
Recientemente se investigaron dispositivos basados en microgeles de poli(NIPAm-co-MAA) para la encapsulación/liberaci\'on del f\'armaco quimioterap\'eutico Doxorrubicina \cite{Giussi2020, MartinezMoro2020, Pergushov2020}. Estos autores mostraron que el uso de microgeles para la liberaci\'on contralada de sustancias bioactivas con carga opuesta. 
La incorporaci\'on del comon\'omero \'acido proporciona un mecanismo controlado por el pH para la captaci\'on/liberaci\'on de mol\'eculas con carga opuesta, lo que hace que los microgeles de respuesta m\'ultiple sean atractivos para el dise\~no de sistemas funcionales de administraci\'on de f\'armacos \cite{Liu2017}.

En sitentesis el  rango de las potenciales aplicaciones biom\'edicas de los nanogeles es extenso e incluye desarrollos,mas especificamente en medicina, contra los trastornos neurol\'ogicos, las enfermedades cardiovasculares, oftalmol\'ogicas, inflamatorias y autoinmunes, as\'i como tambi\'en avances en el diagn\'ostico por im\'agenes, la ingenier\'ia de tejidos, la reconstrucci\'on \'osea y el manejo de la diabetes y el dolor. Por ejemplo, los nanogeles de \'acido hialur\'onico est\'an siendo evaluados para inhibir la acumulaci\'on de la prote\'ina beta-amiloide en el manejo del Alzheimer. En el tratamiento de la diabetes, se investigan nanogeles sensibles a la glucosa y nuevos m\'etodos de administraci\'on de insulina basados en nanogeles.

Los nanogeles de pol\'imeros termosensibles pueden ser utilizados para la administraci\'on localizada de anest\'esicos. Como veh\'iculos para el suministro de drogas, los nanogeles polim\'ericos pueden administrar f\'armacos de peso molecular bajo, oligonucle\'otidos, prote\'inas terap\'euticas e incluso combinaciones de drogas, lo cual es esencial en terapias contra el c\'ancer y las enfermedades infecciosas.

En muchos casos, la v\'ia oral es preferible para la administraci\'on de f\'armacos, ya que es menos invasiva y presenta otras ventajas que mejoran la calidad de vida de los pacientes. En este \'ambito, los nanogeles con respuesta al pH son de particular inter\'es debido a los cambios de pH que ocurren a lo largo del tracto digestivo, desde un medio \'acido en el est\'omago (pH 1.2-2) hasta uno neutro o moderadamente alcalino en el intestino delgado (pH 7-8). Adem\'s, algunos compartimentos celulares involucrados en la captaci\'on de f\'armacos, como los endosomas tempranos, tienen un pH levemente \'acido. La diferencia de pH que existe entre la superficie de la piel  y el torrente sangu\'ineo puede ser aprovechada para la administraci\'on transd\'ermica de f\'armacos utilizando nanogeles con respuesta al pH [48]. Por otro lado, el microambiente alrededor del tejido canceroso puede presentar un pH m\'as ácido en comparaci\'on con las condiciones fisiol\'ogicas habituales [49-52], por lo que los nanogeles con respuesta al pH est\'an siendo evaluados para la administraci\'on de medicamentos en el tratamiento del c\'ancer [33,53]. Por ejemplo, se han utilizado nanogeles de quitina para la administraci\'on de doxorubicina en diferentes tipos de c\'ancer, incluyendo pulm\'on, mama, h\'igado y pr\'ostata [54]. Del mismo modo, los nanogeles de pol\'imeros termosensibles tienen un gran potencial para la liberaci\'on dirigida tanto a c\'elulas cancerosas como a tejidos inflamados o lesionados, los cuales presentan una temperatura ligeramente superior a la corporal.

	


	
\begin{color}{red}
En este contexto, nuestro esta tesis tiene como objetivo avanzar en el entendimiento b\'asico de la fisicoqu\'imica que subyace en la actuaci\'on de los geles polim\'ericos como biomateriales y su interacci\'on con prote\'inas y otras biomol\'eculas. Adem\'as, esta investigaci\'on busca explorar nuevas estrategias de funcionalizaci\'on de estos hidrogeles para controlar su respuesta e interacci\'on. Para lograrlo, utilizaremos simulaciones moleculares por computadora, las cuales nos brindan acceso a informaci\'on esencial que a menudo no est\'a disponible en el laboratorio. Al mismo tiempo, el conocimiento adquirido en nuestros estudios tiene como objetivo guiar el dise\~no en el laboratorio de geles polim\'ericos con propiedades \'optimas para aplicaciones espec\'ificas en el campo de los biomateriales.

\end{color}

\section{Enfoque te\'orico}


El control de la funci\'on y el comportamiento de un biomaterial requiere comprender su interacci\'on con las prote\'inas. Por ejemplo, las lentes de contacto basadas en \'acido poli(metacr\'ilico) (PMAA) con respuesta al pH est\'an expuestas al fluido lagrimal, que contiene cientos de prote\'inas. La adsorci\'on de algunas de estas prote\'inas debe evitarse, ya que afecta la comodidad de uso y puede provocar inflamaci\'on; sin embargo, la adsorci\'on selectiva de prote\'inas con propiedades antibacterianas y antiinflamatorias, como la lisozima, podr\'ia ser beneficiosa. 
La interacci\'on entre prote\'inas y superficies polim\'ericas est\'a gobernada por una compleja interacci\'on entre diferentes grados de libertad. La capacidad tanto del adsorbato como del material adsorbente para protonar/deprotonar, regular su carga el\'ectrica y modificar el entorno cercano, contribuye a esta complejidad.

\textcolor{red}{Descripcion de diferentes teorias para explicar comportamiento de estos nanogles, diferentes escalas de modelado. Resultados obtenidos...}.

El comportamiento de los microgeles de poli(NIPAm-co-MAA), incluido su VPT y la interacci\'on con pol\'imeros de carga opuesta, se ha descrito aplicando teor\'ias  y simulaciones moleculares en varios niveles de resoluci\'on para investigar el comportamiento de los microgeles polim\'ericos sensibles a est\'imulos \cite{ahualli2016coarse,Landsgesell2019SM}.
\citet{quesada2011gel} ha simulado el comportamiento de geles compuestos  polielectrolitos y termosensibles utilizando  simulaciones de Monte Carlo, logrando explicar el comportamiento de hinchamiento de estas part\'iculas.
\citet{ahualli2016coarse}
enplearon simulaciones de grano grueso empleadas para geles polielectrol\'iticos. Este enfoque computacional,se basaron en interacciones part\'icula-partícula entre unidades de pol\'imero. 

En estos trabajos se han centrado principalmente en el hinchamiento y otras propiedades de las part\'iculas que tienen una red de pol\'imero permanentemente cargada , y algunos han abordado el efecto de la temperatura y la calidad del solvente \cite{Jha2011, QuesadaPerez2013, moncho-jorda2016a, ahualli2016coarse, AdroherBenitez2017PCCP}.
Recientemente, estudios  con simulaciones han considerado la respuesta al pH de microgeles compuestos de pol\'imeros reguladores de carga \cite{Schroeder2015,Rud2017,Sean2018, Hofzumahaus2018,Lu2019}.
Sin embargo, solo unos pocos trabajos te\'oricos han investigado las propiedades de los microgeles de respuesta m\'ultiple en funci\'on de la temperatura, el pH y la concentración de sal  \cite{CaprilesGonzalez2008,polotsky2013collapse}.

\citet{polotsky2013collapse} basa su teor\'ia en equilibrios osm\'oticos y teniendo en cuenta expl\'icitamente el equilibrio de ionizaci\'on dentro de sus microgeles. Llegando a predecir patrones complejos en la dependencia de las dimensiones de las part\'iculas de microgel. Es decir sus par\'ametros de control.
	


\textcolor{red}{Enganche a los objetivos....}
%%%%%%%%%%%%%%%%%recorte
%%%%%%%%%%%%%%%%%%%%


%-----------------------------------
%	SUBSECTION 1
%-----------------------------------
\section{Objetivos}

Los {\bf objetivos específicos} del presente plan de trabajo son los siguientes:
%
\begin{enumerate}
\item Desarrollar un modelo mecano-estadístico utilizando TM para describir la respuesta a cambios de pH, temperatura y concentración de sal en microgeles formados por homopolímeros.%\label{objetivo_1}
\item Extender dicho modelo para investigar el comportamiento de microgeles de copolímeros con respuesta a múltiples estímulos.%\label{objetivo_2}
\item Estudiar los mecanismos de adsorción de diferentes biomoléculas en los microgeles en función de las condiciones del medio y la estructura/composición química de las cadenas poliméricas.%\label{objetivo_3}
\item Desarrollar un modelo combinando simulaciones de TM y Dinámica Molecular (DM) para estudiar el comportamiento de estos microgeles en soluciones relativamente concentradas.%\label{objetivo_4}
\end{enumerate}
%

Finalmente el \textbf{objetivo general} de esta tesis consiste en \emph{desarrollar una descripción comprensiva del comportamiento y respuesta a estímulo de los microgeles poliméricos mediante el uso de modelos teóricos y computacionales basados en las interacciones moleculares.}
