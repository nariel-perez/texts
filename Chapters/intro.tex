%% Los cap'itulos inician con \chapter{T'itulo}, estos aparecen numerados y
%% se incluyen en el 'indice general.
%%
%% Recuerda que aqu'i ya puedes escribir acentos como: 'a, 'e, 'i, etc.
%% La letra n con tilde es: 'n.

\chapter{Introducci'on}
\label{Chapter1} % Change X to a consecutive number; for referencing this chapter elsewhere, use \ref{ChapterX}

%----------------------------------------------------------------------------------------
%	SECTION 1
%----------------------------------------------------------------------------------------

\section{Hidrogeles polim\'ericos}

%%%%%%%%%%%%%%%
%%%%%%%%%%%%%%%recorte

\begin{color}{blue}
Durante la decada pasada el uso de materiales blandos ha sido de mucho interes debido a la cantidad de materiales e innovaciones que se optienen con ellos.

La versatilidad de estos los han convertido en materiales  importantes en una amplia variedad de aplicaciones tecnológicas.
Se han utilizado como  espumas y adhesivos, son excelente detergentes, est\'an presentes en la industria de los  cosméticos y pinturas, adem\'as de ser ampliamente usados en aditivos alimentarios . El campo de la medicina, la industria farmaceutica, ha encontrado en estos materiales una oportunidad para la creaci\'on de de trasnportadores de drogas m\'as eficientes biocompatibles y biodegradables.

En estas ultimas aplicaciones los  films y geles  polim\'ericos han sido pioneros en su uso y han tenido un creciente inter\'es.
La capacidad de adsorber solventes y la estructura de cadenas entrelazadas proporciona condiciones adecuadas para la adsorcion y protecci\'on de adsorbatos de inter\'es. \addcite 
\end{color}

	Los hidrogeles consisten en una red de pol\'imeros reticulares (entrecruzados) altamente hidratados y generalmente biocompatibles. Estos materiales pueden parecerse a los tejidos biol\'ogicos y pueden ser dise\~nados para responder a cambios ambientales como variaciones en la temperatura, el pH, la fuerza i\'onica y la concentraci\'on de algunas biomol\'eculas. Como resultado, los hidrogeles polim\'ericos son actualmente candidatos prometedores para el desarrollo de una variedad de biomateriales con aplicaciones en biosensores, ingenier\'ia de tejidos, regeneraci\'on \'osea, materiales biomim\'eticos, administraci\'on de medicamentos y muchas otras aplicaciones biom\'edicas. \cite{Daly2020}
	
	
	Inmersas en soluciones acuosas, estos hidrogeles pueden incorporan y retienen grandes cantidades de agua dentro de su estructura polim\'erica.
	Sin embargo, la caracter\'istica m\'as llamativa de estas sistemas es su capacidad para adsorber o liberar solventes y cambiar de tama\~no en respuesta a una variedad de est\'imulos externos.
	Este comportamiento de respuesta es  generlmente reversible y depende de la composici\'on qu\'imica de la red polim\'erica.
	
	El entorno acuoso dentro de los hidrogeles puede proteger a las prote\'inas de la desnaturalizaci\'on y agregaci\'on, mientras que estas se mantienen activas y estructuradas cuando se liberan de los hidrogeles. \addcite
	
	En la administraci\'on oral de medicamentos, los hidrogeles con respuesta al pH han sido ampliamente investigados como veh\'iculos funcionales que pueden encapsular y liberar prote\'inas, evitando su degradaci\'on en el entorno gastrointestinal. \addcite



\section{Respuesta a estimulo pH, sal y Temperatura}

	
	Los microgeles compuestos por cadenas polim\'ericas que tienen segmentos \'acidos como el \'acido acr\'ilico o metacr\'ilico (AA y MAA, respectivamente) se hinchan/deshinchan muchas veces en respuesta a cambios en el pH de la soluci\'on que los contienen \cite{snowden1996colloidal}.
	El pH en el cual se marca el inicio y caracteriza esta transici\'on es el pKa aparente del microgel, que depende de la concentraci\'on de sal de la soluci\'on y frecuentemente difiere del pKa intrí\'iseco del mon\'omero \'acido.
	Estos microgeles tambi\'en ajustan su tama\~no en respuesta a cambios en la concentraci\'on de sal de la soluci\'on \cite{snowden1996colloidal}.
	
	An\'alogamente, los microgeles de algunos pol\'imeros termosensibles experimentan una transici\'on de fase de volumen (VPT por sus siglas en ingl\'es) cuando se calientan por encima de una temperatura caracter\'istica (VPTT o $T_{pt}$) \cite{Pelton1986,Pelton2000}.
	Este comportamiento se origina porque tales pol\'imeros son insolubles en agua por encima de cierta temperatura de soluci\'on cr\'itica m\'as baja (LCST) \cite{Kawaguchi2020}.
	Normalmente, la LCST del pol\'imero y la VPTT de la red  son aproximadamente id\'enticas. 
	Este es el caso de las part\'iculas de microgel de poli(N-isopropilacrilamida) (PNIPAm) \cite{Pelton1986}, cuyo volumen colapsa por encima de $32   ^\circ C$, siendo le mismo para el  pol\'imero lineal \cite{Schild1992}.
	
	Al tener un VPTT alrededor de la temperatura corporal, los microgeles de PNIPAm han generado un gran inter\'es para aplicaciones biom\'edicas \cite{Guan2011}.
	Las estrategias para controlar el VPTT de los microgeles incluyen la s\'intesis de nuevos mon\'omeros termosensibles  \cite{Cai2007,Macchione2019}, as\'i como la copolimerizaci\'on con un mon\'omero i\'onico o ionizable  \cite{Hirose1987,Lopez2020}.
	Este \'ultimo enfoque produce microgeles de respuesta m\'ultiple que son susceptibles a cambios en la temperatura, el pH y la concentraci\'on de sal  \cite{snowden1996colloidal, Farooqi2017}.
	Los microgeles de NIPAm y AA han sido ampliamente estudiados \cite{Morris1997, Jones2000,Bradley2005,Begum2016};
	Tambi\'en se han investigado microgeles de copol\'imeros de NIPAm y MAA  \cite{Dowding2000,Hoare2004,Giussi2015}.
	El VPTT de estos microgeles de respuesta m\'ultiple depende del pH de la soluci\'on y la concentraci\'on de sal, y la fracci\'on de mon\'omero ionizable en las cadenas de pol\'imero  \cite{Morris1997,Jones2000, Hoare2004, Bradley2005, Lee2008,Wong2009,Hamzavi2016}.



\section{Encapsulado y Liberaci\'on de medicamentos}


	Uno de los principales desaf\'ios en el campo de la administraci\'on de f\'armacos es lograr una entrega controlada y sostenida de los medicamentos al sitio de acci\'on deseado en el organismo. 
	Los sistemas de delivery de drogas son dispositivos que permiten una administraci\'on controlada y  localizada en el organismo, mejorando la eficacia de su componente activio y reduciendo as\'i los efectos secundarios. 
	Los hidrogeles polim\'ericos ofrecen una soluci\'on prometedora a este desaf\'io debido a sus propiedades f\'isicas y qu\'imicas ajustables, que les permiten encapsular y liberar f\'armacos.
	
	
	\citet{Brugger2008} enocntraron que el pH durante la s\'intesis tiene un impacto significativo en la composici\'on y, por lo tanto, en las propiedades del microgel y su capacidad para ser utilizado como un estabilizador sensible a est\'imulos.
	Resultados similares fueron estudiados por otros autores en donde se destaca el uso de emulsiones sensibles al pH, la sal y la temperatura  \cite{Ngai2005,Ngai2006, Schmidt2011} o como plantillas para el ensamblaje de nanomateriales \cite{Wong2009}.
	Haciendo de estos sistema no solo valiosos en el encapsulado y liberaci\'on de medicamentos, sino también como secuestradores de diferentes adsorbatos.
	
	\citet{Culver2017A}han utilizado nanogeles de poli(NIPAm-co-MAA) funcionalizados para la uni\'on y detecci\'on de diferentes prote\'inas. 
	Recientemente se investigaron dispositivos basados en microgeles de poli(NIPAm-co-MAA) para la encapsulación/liberaci\'on del f\'armaco quimioterap\'eutico Doxorrubicina \cite{Giussi2020, MartinezMoro2020, Pergushov2020}. Estos autores mostraron que el uso de microgeles para la liberaci\'on contralada de sustancias bioactivas con carga opuesta. 
	
	
	La incorporaci\'on del comon\'omero \'acido proporciona un mecanismo controlado por el pH para la captaci\'on/liberaci\'on de mol\'eculas con carga opuesta, lo que hace que los microgeles de respuesta m\'ultiple sean atractivos para el dise\~no de sistemas funcionales de administraci\'on de f\'armacos \cite{Liu2017}.
	
	
	
	
	En los últimos años el interés científico en los nanogeles poliméricos se ha centrado en aquellos con 
	diámetros menores a 200 nm, ya que incorporados al torrente sanguíneo pueden circular por períodos 
	más prolongados. Debido a la naturaleza incipientemente hidrofílica de estas estructuras, los NGs son 
	generalmente biocompatibles y poseen una gran capacidad para incorporar moléculas huésped o 
	analitos, tanto orgánicos como inorgánicos, y prevenir su degradación por el medio externo. Además, su 
	escaso tamaño les permite responder rápidamente luego de recibir el estímulo.18 Por todo esto, los NGs 
	poliméricos son hoy en día una de las primeras opciones consideradas al diseñar biomateriales con 
	funciones específicas.19,20 El estímulo que dispara la respuesta del NG puede ser suministrado por un 
	gradiente en la composición fisiológica, natural o inducido por un estado patológico. La versatilidad de 
	estos materiales difícilmente puede ser alcanzada con otro tipo de nanopartículas, incapaces de 
	responder a cambios en las condiciones del medio que pueden ser relativamente moderados. El desafío 
	en la actualidad es aprender a controlar esta respuesta para canalizarla en diferentes aplicaciones.
	4
	El rango de las potenciales aplicaciones biomédicas de los NGs es extenso e incluye desarrollos 
	contra los trastornos neurológicos21 y las enfermedades cardiovasculares,22 oftalmológicas,23
	inflamatorias24 y autoinmunes,25 así como también avances en el diagnóstico por imágenes,26 la 
	ingeniería de tejidos,27 la reconstrucción ósea28 y el manejo de la diabetes29,30 y el dolor.31 Por ejemplo, 
	los NGs de ácido hialurónico son evaluados para inhibir la acumulación de la proteína beta-amiloide en 
	el manejo del Alzheimer.21 En el tratamiento de la diabetes, se investigan NGs sensibles a la glucosa y 
	

	nuevos métodos de administración de insulina basados en NGs.
	
	Nanogeles de polímeros termosensibles pueden ser utilizados para la administración localizada de anestésicos

 Como vehículos para 
	el suministro de drogas los NGs poliméricos pueden administrar fármacos de peso molecular bajo,32,33
	oligonucleótidos,34,35 proteínas terapéuticas36,37 e incluso combinaciones de drogas,38,39 lo cual en la 
	actualidad es esencial en terapias contra el cáncer y las enfermedades infecciosas.40
	En muchos casos, la vía oral es preferible para la administración de fármacos, ya que es menos 
	invasiva y presenta otras ventajas que mejoran la calidad de vida de los pacientes.41 En este ámbito, los 
	NGs con respuesta al pH son de particular interés debido a los cambios de pH que ocurren a lo largo del 
	tracto digestivo, desde un medio ácido en el estómago (pH 1.2-2) hasta uno neutro o moderadamente 
	alcalino en el intestino delgado (pH 7-8).42–44 Además, algunos compartimentos celulares involucrados en 
	la captación de fármacos, como los endosomas tempranos, tienen un pH levemente ácido.45–47 La 
	diferencia de pH que existe entre la superficie de la piel y el torrente sanguíneo puede ser aprovechada 
	para la administración transdérmica de fármacos utilizando nanogeles con respuesta al pH.48 Por otro 
	lado, el microambiente en torno al tejido canceroso puede presentar un pH más ácido respecto de las 
	típicas condiciones fisiológicas;49–52 los NGs con respuesta al pH son evaluados para la administración 
	de medicamentos para el tratamiento del cáncer.33,53 Por ejemplo, NGs de quitina han sido utilizados 
	para la administración de doxorubicina a distintos tipos de cáncer, incluyendo pulmón, mama, hígado y 
	próstata.54 En el mismo sentido, los nanogeles de polímeros termo-sensibles tienen un gran potencial 
	para la liberación tanto a células cancerosas como a tejido inflamado y/o herido, los cuales presentan 
	una temperatura levemente superior a la corporal.

\section{enfoque te\'orico}


El control de la funci\'on y el comportamiento de un biomaterial requiere comprender su interacci\'on con las prote\'inas. Por ejemplo, las lentes de contacto basadas en \'acido poli(metacr\'ilico) (PMAA) con respuesta al pH est\'an expuestas al fluido lagrimal, que contiene cientos de prote\'inas. La adsorci\'on de algunas de estas prote\'inas debe evitarse, ya que afecta la comodidad de uso y puede provocar inflamaci\'on; sin embargo, la adsorci\'on selectiva de prote\'inas con propiedades antibacterianas y antiinflamatorias, como la lisozima, podr\'ia ser beneficiosa. 
Sin embargo, la interacci\'on entre prote\'inas y superficies polim\'ericas est\'a gobernada por una compleja interacci\'on entre diferentes grados de libertad. La capacidad tanto del adsorbato como del material adsorbente para protonar/deprotonar, regular su carga el\'ectrica y modificar el entorno cercano, contribuye a esta complejidad.

Descripcion de diferentes teorias para explicar comportamiento de estos nanogles, diferentes escalas de modelado.
Resultados obtenidos....


Los films polim\'ericos o hidrogeles  consisten en una red de pol\'imeros entrecruzados altamente hidratados, generalmente biocompatibles, dependendiendo de su composici\'on qu\'imca. El ambiente acuoso dentro de los hidrogeles puede proteger a las prote\'inas de la desnaturalización y la agregaci\'on [11y13], mientras permanecen activas y estructuradas cuando se liberan de los hidrogeles [14]. En la administraci\'on oral de f\'armacos, los hidrogeles con respuesta de pH se han investigado en gran medida como veh\'iculos funcionales que pueden encapsular y administrar prote\'inas, evitando su degradaci\'on en el entorno gastrointestinal [15-17].


En este capitulo mostraremos un estudio  de estos sistemas polim\'ericos haciendo uso de de la te\'oria molecular.


El comportamiento de los microgeles de poli(NIPAm-co-MAA), incluido su VPT y la interacci\'on con pol\'imeros de carga opuesta, se ha descrito aplicando teor\'ias  y simulaciones moleculares en varios niveles de resoluci\'on para investigar el comportamiento de los microgeles polim\'ericos sensibles a est\'imulos \cite{ahualli2016coarse,Landsgesell2019SM}.
\citet{quesada2011gel} ha simulado el comportamiento de geles compuestos  polielectrolitos y termosensibles utilizando  simulaciones de Monte Carlo, logrando explicar el comportamiento de hinchamiento de estas part\'iculas.
\citet{ahualli2016coarse}
enplearon simulaciones de grano grueso empleadas para geles polielectrol\'iticos. Este enfoque computacional,se basaron en interacciones part\'icula-partícula entre unidades de pol\'imero. 

En estos trabajos se han centrado principalmente en el hinchamiento y otras propiedades de las part\'iculas que tienen una red de pol\'imero permanentemente cargada , y algunos han abordado el efecto de la temperatura y la calidad del solvente \cite{Jha2011, QuesadaPerez2013, moncho-jorda2016a, ahualli2016coarse, AdroherBenitez2017PCCP}.
Recientemente, estudios  con simulaciones han considerado la respuesta al pH de microgeles compuestos de pol\'imeros reguladores de carga \cite{Schroeder2015,Rud2017,Sean2018, Hofzumahaus2018,Lu2019}.
Sin embargo, solo unos pocos trabajos te\'oricos han investigado las propiedades de los microgeles de respuesta m\'ultiple en funci\'on de la temperatura, el pH y la concentración de sal  \cite{CaprilesGonzalez2008,polotsky2013collapse}.

\citet{polotsky2013collapse} basa su teor\'ia en equilibrios osm\'oticos y teniendo en cuenta expl\'icitamente el equilibrio de ionizaci\'on dentro de sus microgeles. Llegando a predecir patrones complejos en la dependencia de las dimensiones de las part\'iculas de microgel. Es decir sus par\'ametros de control.
	



%%%%%%%%%%%%%%%%%recorte
%%%%%%%%%%%%%%%%%%%%


%-----------------------------------
%	SUBSECTION 1
%-----------------------------------
\section{Objetivos}

Los {\bf objetivos específicos} del presente plan de trabajo son los siguientes:
%
\begin{enumerate}
\item Desarrollar un modelo mecano-estadístico utilizando TM para describir la respuesta a cambios de pH, temperatura y concentración de sal en microgeles formados por homopolímeros.%\label{objetivo_1}
\item Extender dicho modelo para investigar el comportamiento de microgeles de copolímeros con respuesta a múltiples estímulos.%\label{objetivo_2}
\item Estudiar los mecanismos de adsorción de diferentes biomoléculas en los microgeles en función de las condiciones del medio y la estructura/composición química de las cadenas poliméricas.%\label{objetivo_3}
\item Desarrollar un modelo combinando simulaciones de TM y Dinámica Molecular (DM) para estudiar el comportamiento de estos microgeles en soluciones relativamente concentradas.%\label{objetivo_4}
\end{enumerate}
%

Nuestro objetivo general.... 
