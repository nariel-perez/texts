\prefacesection{Gu\'ia del Navegante}
\label{ruta}

 
Esta tesis presenta una investigaci\'on exhaustiva sobre los hidrogeles polim\'ericos, abordando su estudio desde una perspectiva te\'orica/computacional. La intenci\'on de esta gu\'ia es ofrecer una visi\'on general de la investigaci\'on realizada, proporcionando al lector un primer vistazo de lo desarrollado e investigado a lo largo de este doctorado.

En el primer cap\'itulo, nos adentramos en el mundo de los hidrogeles polim\'ericos, haciendo \'enfasis en sus aplicaciones biom\'edicas. Se presentar\'an las motivaciones y objetivos, adem\'as de mencionar el enfoque utilizado y las herramientas empleadas para llevar a cabo los estudios presentados.

La investigaci\'on se inicia con una exploraci\'on detallada de films de hidrogeles polim\'ericos, subrayando su versatilidad y su papel en la soluci\'on de problemas complejos en diversas \'areas de aplicaci\'on. Mediante el uso de la teor\'ia molecular, se examina c\'omo la respuesta de estos materiales a diversos est\'imulos pueden ser manipuladas para dise\~nar sistemas m\'as eficientes y efectivos para aplicaciones espec\'ificas. Este cap\'itulo, inspirado en trabajos publicados \cite{perez2018using, perez2019molecular} y en especial \cite{perez2020triggering}, utiliz\'o films para la encapsulaci\'on de f\'armacos.

Continuando en el mundo de los hidrogeles polim\'ericos y acerc\'andonos m\'as a nuestros objetivos, se confeccion\'o el cap\'itulo \ref{Chapter-geles}, en el cual se presenta un estudio de la termodin\'amica de microgeles polim\'ericos con m\'ultiples est\'imulos. Los resultados mostrados forman parte de un trabajo publicado \cite{perez2021thermodynamic}, en el cual se plantea el uso de un modelo robusto con el cual se explican los fen\'omenos observados experimentalmente desde la fisicoqu\'imica. Se muestra la capacidad de estos microgeles como encapsuladores de drogas terap\'euticas y se predicen las condiciones \'optimas de laboratorio para dicha tarea.

En el cuarto cap\'itulo, extendemos nuestra investigaci\'on para estudiar c\'omo la estructura que define a los nanogeles polim\'ericos afecta su respuesta a est\'imulos \cite{na10.1021/acs.jpcb.3c07283}. Este estudio explora el impacto de la funcionalizaci\'on de la red polim\'erica y la composici\'on qu\'imica en el comportamiento de nanogeles polim\'ericos sensibles al pH y su adsorci\'on de prote\'inas.

Para cerrar esta tesis, en el cap\'itulo \ref{chapter:mc:soluciones} se aplican todos los conocimientos adquiridos sobre microgeles y nanogeles polim\'ericos para el estudio de soluciones compuestas por part\'iculas con m\'ultiples respuestas a est\'imulos. En este cap\'itulo nos enfocamos en el efecto de la concentraci\'on de nanogeles y sus consecuencias sobre la respuesta a cambios del medio que los contiene.

El prop\'osito de este pr\'ologo es preparar al lector para una inmersi\'on profunda en los temas que se tratar\'an. A trav\'es de este documento, se busca no solo compartir los resultados obtenidos, sino tambi\'en inspirar a futuras investigaciones en el campo de los hidrogeles polim\'ericos y sus aplicaciones.




